%Add in broader background short paragraph.

Since the introduction of the chemoaffinity hypothesis \cite{sperry1963chemoaffinity}, developmental neurobiologists have identified a host of molecular gradients and cues in intermediate and final target regions that help establish orderly connections between projection neurons and their targets.
With this emphasis on source-target correspondence, the field has focused less attention on axons within developing tracts, leaving several questions: what, if any, defined axon arrangements exist prior to target entry; through what molecular environments do axons navigate in their tracts; and what interactions occur between axons and the tract environment, and between cohorts of axons?
Furthermore, is pre-target axon organization a crucial step in the creation of accurate synaptic connections in the final target, or are cues within the target sufficient to establish accurate circuitry, leaving tract order largely incidental?

The idea that axons are pre-sorted in their tracts is not new.
Though his contributions to the field have been credited for the emphasis on chemoaffinity-driven axon-target interactions, Sperry himself suggested that axons are organized, or at least guided, by chemical cues along the pathway: “Not only the details of synaptic association within terminal centers, but also the routes by which the fibers reach their synaptic zones would seem to be subject to regulation during growth by differential chemical affinities” \cite{attardi1963preferential}.
Sperry’s contemporaries, too, hinted at the possibility of pre-target axon guidance cues in the tract, asserting that “mechanical factors may play an insignificant role in determining how and where a growing nerve fiber shall end,” but “more subtle forces are at work, forces whose basic mechanisms are almost totally unknown at the present time” \cite{barnard1956study}.
In the more than half-century since, researchers have identified numerous chemoattractive and chemorepulsive sources and gradients that guide axons within their targets or through intermediate choice points, such as at the central nervous system (CNS) midline in the optic chiasm or spinal cord floorplate. %Edit this sentence - too long
Concurrent with research on intermediate and final targets, some investigators also explored the organization of axons within developing tracts.
Reports from the late 1970s identified pre-target axon order in the developing visual pathway, with Scholes even declaring that “common sense” dictated the importance of fiber-fiber interactions and pre-target sorting in the formation of developing pathways \cite{cook1977multiple,scholes1979nerve}.
The field has since learned much about the order of retinal axons along the optic nerve and tract, and identified axon organization in other sensory and central tracts. %Edit, don't want to overstate how much we know
The question of how axons are arranged in their tracts has gained more traction, and recent studies have utilized advanced molecular genetic and selective labeling techniques to directly probe whether such pre-target axon order is required for accurate innervation of targets.
%Edit next couple sentences to be specific to thesis Intro
In this chapter, we first review what is known about the modes and degrees of pre-target axon organization across sensory, motor, and central neural circuits, before surveying the molecular mechanisms at play as axons course through their tracts.
We then explore recent research on the relationship between tract organization and the specificity of synaptic connectivity within final target regions.
