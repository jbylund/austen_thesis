Olfactory System

Whereas the visual system relays topographically related sensory information between sense organ and brain target, the olfactory system transduces chemical stimuli, which do not require a one-to-one spatial map to represent the sensory information. 
While there are accordingly fundamental differences in the relationship between sense organ and target in the visual and olfactory systems, they both exhibit order among axons in their tracts. Olfactory sensory neurons (OSNs) lining the olfactory epithelium (OE) of the nose send their axons into the brain target, the olfactory bulb (OB). 
In the OB, OSN axons are organized by a “typographic” principle, where OSNs of the same type, i.e. expressing the same olfactory receptor (OR), coalesce into distinct target glomeruli via presumed homotypic axon-axon interactions based on OR expression \cite{feinstein2004contextual}. 
As retinotopy is the hallmark of visual system targeting, OR-based typography is the hallmark of olfactory system targeting.
However, there is a coarse spatial pattern of expression of ORs among OSNs in the OE, as well as some topographic correspondence between the OE and OB. 
Thus, while ultimate axon sorting decisions based on OR expression occur within the glomerular layer of the OB, pre-target ordering of axons in the olfactory nerve may serve a preliminary role in establishing the olfactory sensory circuit \cite{miller2010axon}.

Topographic correspondence between the OE and OB was first identified over a half century ago, using degeneration studies in the rabbit olfactory system \cite{clark1951projection}. 
The findings were confirmed by subsequent anatomical and physiological studies, which describe a correspondence of the dorsal-ventral and medio-lateral axes in the OE to the same axes in the OB, but little to no rostral-caudal correspondence between the two structures \cite{costanzo1978spatially, land1973localized, saucier1986analysis}. 
With the development of more advanced genetic tools, a series of studies showed first that there is zonal patterning in the OE, with OSNs expressing certain ORs restricted to particular zones \cite{ressler1993zonal, vassar1993spatial}; and second, that all OSNs expressing one type of OR project to specific, bilaterally symmetric glomeruli in the OB, creating what Ressler et al. term an “epitope map” of olfactory stimuli in the OB \cite{ressler1994information, vassar1994topographic}. 
Importantly, this glomerular map in the OB is remarkably consistent across animals, demonstrating a highly stereotyped typographic map of ORs in the OB \cite{ressler1994information, vassar1994topographic}. 
None of these studies directly examined axon arrangements in the olfactory nerve, but the degree of order collectively described between the OE and OB provided the first suggestions that organization of axons entering into the target may be important for wiring the circuit. 

Further exploration of the olfactory nerve found that OSN axons undergo organizational events as they grow towards the OB. 
Two subpopulations of OSNs were identified in Xenopus tadpoles based on their relative expression levels of Neuropilin (Nrp) and Plexin, both receptors to Semaphorins (Semas) \cite{satoda1995differential}. 
The two cohorts of axons are intermingled early in the olfactory nerve but become increasingly sorted as they project to the OB. 
In the distal olfactory nerve, there is a sharp border between the two bundles of axons, the segregation of which reflects their final segregation in the OB \cite{satoda1995differential}. 
The position of additional axon cohorts in the olfactory nerve is defined by expression of other molecular markers, with a similar degree of order described in all cases \cite{imai2009pre, miller2010axon}. 
Regardless of the marker examined, OSN axon cohorts exit the OE in overlapping swathes of axons and become more orderly along the nerve. 
There is a very clear order of axon bundles prior to entry into the inner nerve layer of the OB, where axons undergo further OR-based sorting into individual glomeruli \cite{imai2009pre, miller2010axon}.

Similar to investigations in the visual system, studies in the olfactory system aimed to determine if this pre-target axon order arises due to target-derived cues or is driven independently by active organizational mechanisms occurring within the olfactory nerve. 
The fact that pre-target sorting of axons in the olfactory nerve is apparent by E12 in mice, at least 72 hours prior to onset of synaptogenesis in the OB, argues in favor of target-independent organizational mechanisms \cite{miller2010axon}. 
Even stronger evidence that axon sorting in the olfactory nerve occurs independently of OB cues comes from a series of experiments that surgically \cite{graziadei1978regeneration} or genetically ablated the OB \cite{stjohn2003sorting} or subsets of target cells in the OB \cite{bulfone1998olfactory}. 
In all such experiments, OSN axons sort normally in the nerve and form OR-specific glomerulus-like bundles, which invade ectopic brain regions in the absence of a true OB. 
Similar to optic tectum removal experiments in frog \cite{reh1983organization}, these target-ablating experiments provide compelling evidence of active and independent pre-sorting of axons in the olfactory nerve.

Studies examining higher order olfactory targets beyond the OB have noted that central olfactory regions fail to maintain topographic correspondence to the spatial organization of glomeruli in the OB \cite{luskin1982distribution, sosulski2011distinct}. 
Topographic order of OB efferents can only be found in the earliest portion of the lateral olfactory tract (LOT), and axon order reflecting spatial information of the OB is presumably lost along the length of the LOT \cite{price1975observation}. 
A more recent study found that OB efferents are in fact arranged chronotopically in the LOT, with younger axons coursing nearer the pial surface, and older axons more deeply in the tract \cite{yamatani2004chronotopic}, notably similar to chronotopic order in the optic tract. 
This finding again highlights the fact that lack of evidence of one mode of axon organization does not necessarily mean a lack of organization overall.