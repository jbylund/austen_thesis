Auditory System
Relative to the other sensory systems, axon guidance in the auditory system is particularly poorly understood. Tonotopy – the organization of sound frequencies along an axis – is evident in the basilar membrane of the cochlea and all brainstem and central auditory centers (reviewed in Appler & Goodrich, 2011). Tonotopic organization at each auditory brain target develops precisely before the onset of hearing and processing of acoustic information (Appler & Goodrich, 2011; Rubel & Fritzsch, 2002). Though it has yet to be directly tested, preservation of tonotopy amongst axons within the auditory nerve and tracts connecting auditory brain centers could be a means of establishing precise tonotopy in each auditory target early in development. 
At least one portion of the auditory pathway maintains an incredibly precise tonotopic map through the tract. Axons in the crossed dorsal cochlear tract (XDCT) connecting the nucleus magnocellularis (NM) and nucleus laminaris (NL) in the chick auditory brainstem are arranged in strict correspondence with their positions in both NM and NL (Kashima, Rubel, & Seidl, 2013). Multicellular labeling of regions in the NM by electroporation of dextran dyes in two areas along the tonotopic axis shows a clear correspondence between the position of groups of neurons in the NM and their axons in the XDCT. The authors also followed axons from individually labeled NM cells and found a strikingly linear relationship between the position of the soma in the NM and their respective axons in the XDCT (Kashima et al., 2013). To date, this study is one of the highest resolution in any system, tracing individual axons as they project through the tract to their synaptic targets. 
The degree of tonotopy in this brainstem auditory tract appears far more precise than any axon order so far described in the retinotectal or olfactory systems, in which axons are more coarsely organized. A compelling question raised by this work is whether the higher degree of precision found in this auditory tract corresponds to a more precise initial mapping of axon terminations in the target relative to other systems. In other words, does the NL rely less upon successive activity-driven refinement of axon terminals than does, for instance, the dLGN in the visual system? The relationship between degree of pre-target axon order and accuracy of initial target innervation has been discussed in the visual system, where, relative to mammals, fish and frogs display greater order in the nerve and tract and greater initial accuracy of terminals in the target (Simon & O'Leary, 1991). Based on this correlation in the visual system, it is reasonable to expect that the auditory system exhibits a greater degree of precision in its early circuit formation than do other sensory systems. Why some systems may deploy different developmental strategies of circuit formation is unclear and is an area ripe for comparative evolutionary study. 
