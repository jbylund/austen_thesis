Spinal Sensorimotor System

The sensorimotor tracts in the spinal cord are another system in which topography of projections, similar to retinotopy in the visual system, spatially reflects the information being transmitted.
In this case, somatotopy, the representation of the body map, must be contained in both motor efferents projecting to specific muscle fibers and sensory afferents conveying touch and pain information from the periphery.
The somatotopic correspondence of the periphery, neuronal nuclei in the spinal cord, and the sensory and motor cortices has long been known.
Indeed much of the research on the development of the sensorimotor system has focused on understanding mechanisms of neuronal diversification and migration to their correct locations at these discrete points in the system (reviewed in \cite{kania2014spinal}.
However, details of motor and sensory axon tract organization within this system remain less clear.

Early tract tracing experiments revealed an orderly arrangement of ascending and descending axon tracts in the chick spinal cord, where newer axons grow along older ones \cite{nornes1980pattern}, similar to chronotopic layering of axons in the optic and lateral olfactory tracts. 
The authors concluded that positional information of neurons in the spinal cord “can be used as a mechanism for imparting order to the presynaptic components” of the pathway, hypothesizing that contact-mediated guidance and selective fasciculation underlie such pre-target axon order \cite{nornes1980pattern}. 
Another early study in the chick showed that motor neuron axons destined for a specific muscle course together in distinct bundles but do not maintain strict neighbor relationships along the path to the muscle \cite{lance1981pathway}. 
Instead, axons rearrange along the path and sort out within the limb plexus into discrete fascicles bound for specific muscles. 
This finding led the authors to suggest that active mechanisms of pre-target axon sorting form the circuit rather than either passive maintenance of neighbor relationships or exuberant, disorganized growth followed by pruning \cite{lance1981pathway}. 
More recent retrograde labeling studies tracked motor axon fascicles through the peripheral sciatic-tibial nerve in rats and found somatotopy preserved along the length of the nerve \cite{badia2010topographical}. 
Further details of topographic organization of axon bundles in the spinal cord and periphery remain largely undefined.

Other studies have found clear subtype organization (i.e., typographic order) of axons in the spinal cord prior to target innervation.
For instance, motor neuron axons innervating fast and slow muscle fibers in the chick hindlimb fasciculate separately from each other starting in the proximal regions of the limb, well before reaching their target muscles \cite{milner1998selective}.
However, motor axons projecting to different parts of fast muscle fibers remain intermingled en route to the muscle, implying that a second step of axon sorting must occur within the target to organize this subset of inputs \cite{milner1998selective}.
Additionally, mechanosensory and proprioceptive sensory afferents segregate along the medial-lateral axis in the dorsal column of the mouse spinal cord. Notably, axons are arranged somatotopically within this typographic order \cite{niu2013modality}, revealing again the layering of modes of axon organization.
Given that these afferents terminate in a modality-specific manner in the medulla, pre-target typographic order of the axons may be functionally important in establishing modality based targeting patterns \cite{niu2013modality}.

One of the better-understood aspects of axon organization in the periphery is the relationship between motor and sensory axon fascicles, which run alongside each other in peripheral nerves \cite{honig1998spatial}.
During the earliest phases of axon extension, motor neuron axons enter peripheral nerve tracts, displaying a high degree of specificity in their initial trajectory, followed by sensory afferents, which use motor axon fascicles as a tracking system to lead them to their respective targets \cite{huettl2011npn,landmesser1986altered,wang2013axons,wang2014conserved}.
Motor and sensory axons are closely associated but segregated – indeed their segregation can be recapitulated in vitro \cite{gallarda2008segregation} – and have a hierarchical dependence on each other, such that sympathetic efferents rely on sensory afferents, which in turn rely on motor efferents to project to their respective targets \cite{wang2014conserved}.
In sum, the sensory and motor projections from and to the periphery again demonstrate the common principles of topographic, typographic, and chronotopic order, with a necessary layering of axon order modalities within tracts.
