Since the identification of growth cones by Santiago Ram\'on y Cajal and the introduction of the chemoaffinity hypothesis by Roger Sperry \cite{sperry1963chemoaffinity}, developmental neurobiologists have made great strides in understanding how the brain form during embryogenesis and early postnatal development.
A guiding principle across the many sensory systems often used as models to study neural development and axon guidance, has been the correspondence between the organization of neuron cell bodies and their axon projections.
That is, in the visual system, for instance, retinal ganglion cells (RGCs) in the retina relay a map of visual space into the vision-forming targets in the thalamus, which in turn relay this spatial information to the visual cortex.
Even in non-sensory systems, such as the cortico-cortical projections across the corpus callosum and thalamocortical projections, axon terminals are project homotopically in their targets. %cite..see zhou 2013 for ideas, use udin/fawcett 1988, dufour and powell references for tca?

In understanding how axons make the journey to those targets, a host of molecular gradients and cues have been identified in intermediate and final target regions that help establish orderly connections between projection neurons and their targets.
With this emphasis on source-target correspondence, however, the field has focused less attention on axons within developing tracts, leaving several questions: what, if any, defined axon arrangements exist prior to target entry; through what molecular environments do axons navigate in their tracts; and what interactions occur between axons and the tract environment, and between cohorts of axons?
Furthermore, is pre-target axon organization a crucial step in the creation of accurate synaptic connections in the final target, or are cues within the target sufficient to establish accurate circuitry, leaving tract order largely incidental?

Though his contributions to the field have been credited for the emphasis on chemoaffinity-driven axon-target interactions, Sperry himself suggested that axons are organized, or at least guided, by chemical cues along the pathway: ``Not only the details of synaptic association within terminal centers, but also the routes by which the fibers reach their synaptic zones would seem to be subject to regulation during growth by differential chemical affinities'' \cite{attardi1963preferential}.
Sperry`s contemporaries, too, hinted at the possibility of pre-target axon guidance cues in the tract, asserting that ``mechanical factors may play an insignificant role in determining how and where a growing nerve fiber shall end,'' but ``more subtle forces are at work, forces whose basic mechanisms are almost totally unknown at the present time'' \cite{barnard1956study}.
In the more than half-century since this work, researchers have identified numerous chemoattractive and chemorepulsive sources and gradients that guide axons within their targets or through intermediate choice points, such as the central nervous system (CNS) midline at the optic chiasm or spinal cord floorplate.
Concurrent with research on intermediate and final targets, some investigators also explored the organization of axons within developing tracts.
Reports from the late 1970s identified pre-target axon order in the developing visual pathway of fish, and Scholes wrote that ``common sense'' dictated the importance of fiber-fiber interactions and pre-target sorting in the formation of developing pathways \cite{cook1977multiple,scholes1979nerve}.