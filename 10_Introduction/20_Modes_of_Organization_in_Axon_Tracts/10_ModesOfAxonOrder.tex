Since these early hints and speculations, the question of how axons are arranged in their tracts has continued to gain traction.
In the last few decades, three main modes of tract order have been identified in the visual, olfactory, auditory, sensorimotor, callosal, and thalamocortical systems.
These common modes of axon organization fall into one of following three categories, which are illustrated in Figure~\ref{AxonOrgThreeModes}:
\begin{itemize}
\item Topography, which is based on the topographic or spatial arrangement of the neurons.
This is the most common organizational feature found in neural circuits.
\item Typography, in which axon order is based on neuron subtype, or molecularly similar cohorts.
\item Chronotopy, in which axon age is the organizing principle, such that younger axons grow along older axons in a sequentially additive fashion.
\end{itemize}
\begin{figure}[hbtp]
    \begin{center}
    \includegraphics{Figures/AxonOrg_ThreeModes.pdf}
    \caption[Three modes of axon organization in tracts.]
    {Three modes of axon organization in tracts.
    From top to bottom: topography, based on spatial positioning; typography, based on subtype or molecular identity; and chronotopy, or age-related order.
    \label{AxonOrgThreeModes}}
    \end{center}
\end{figure}
These three modes are well conserved across systems.
Recent studies have utilized advanced molecular genetic and selective labeling techniques to directly probe whether pre-target axon order within tracts is required for accurate target innervation.
Experiments in the olfactory system, corpus callosum, and auditory brainstem all provide compelling evidence that maintaining appropriate organization of axons along their paths is a crucial aspect of forming an accurate functional circuit \cite{imai2009pre,zhou2013axon,michalski2013robo3}.
The relationship between pre-target tract organization and targeting will be discussed in more detail later in this chapter, in Section~\ref{sec:TractOrderIndependOfTargeting}.

In the following section, I review the current knowledge of topographic, typographic, and chronotopic organization in the axon tracts of a variety of sensory and central neural circuits.
This paves the way for a discussion in Section~\ref{sec:MoleculesMechanisms} of common molecular mechanisms leading to axon organization in tracts, which are broadly categorized into axon-extrinsic and axon-intrinsic mechanisms.