Since these early hints and speculations, the question of how axons are arranged in their tracts has continued to gain traction. 
In the last few decades, three main modes of tract order have been identified in the visual, olfactory, auditory, sensorimotor, callosal, and thalamocortical systems.
These common modes of axon organization can be categorized as topography, order based on topographic arrangement of the neurons; typography, order based on neuron subtype; and chronotopy, order based axon age (Figure~\ref{AxonOrg_ThreeModes}).
\begin{figure}[hbtp]
	\makebox[\textwidth]{\framebox[5cm]{\rule{0pt}{5cm}}}
	\caption[Three Modes of Axon Organization in Tracts]{Three Modes of Axon Organization in Tracts: topography, based on spatial positioning; typography, based on subtype or molecular identity; and chronotopy, or age-related order. \label{AxonOrg_ThreeModes}}
	\end{figure}
These three modes appear well conserved across the various systems in which they have been studied.
Furthermore, recent studies have utilized advanced molecular genetic and selective labeling techniques to directly probe whether pre-target axon order within tracts is required for accurate target innervation.
Experiments in the olfactory system, corpus callosum, and auditory brainstem all provide compelling evidence that maintaining appropriate organization of axons in their tracts is a crucial step in forming an accurate functional circuit \cite{imai2009pre,zhou2013axon,michalski2013robo3}.

In this chapter, I first review our current knowledge of topography, typography, and chronotopy in the axon tracts of a variety of sensory and central neural circuits.
This paves the way for a discussion of common molecular mechanisms leading to axon organization in tracts, which I have categorized roughly into axon-extrinsic and axon-intrinsic mechanisms.
I will then summarize recent research supporting a role for pre-target axon organization in establishing precise synaptic connectivity within final target regions.