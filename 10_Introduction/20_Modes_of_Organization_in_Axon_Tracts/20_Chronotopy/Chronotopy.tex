Visual
Chronotopy, or age-related order, was first suggested as a mode of axon organization in the frog, where backlabeling a portion of the optic tract results in annular retrograde labeling in the retina, denoting a cohort of RGCs born at the same time \cite{fawcett1984fibre,reh1983organization}.
Axons are topographically ordered within age-related bundles, where dorsal and ventral RGCs extend axons laterally and medially, respectively, while nasal and temporal RGC axons are mixed together across much of the tract \cite{reh1983organization}.
Note that this arrangement differs from topographic order in mammalian tracts.
The authors propose that active reordering mechanisms play a part in the topographic arrangement of fibers, while chronotopic order likely stems from passive mechanisms related to space constraints on incoming axons exiting the optic chiasm.
Furthermore, Reh et al. hypothesize the presence of attractive cues along the dorsal neuraxis that could preferentially act upon ventral RGC axons, providing a tract-specific mechanism of active axon organization that would occur independently of target derived cues \cite{reh1983organization}.
Mammalian optic tracts also display chronotopic axon order (Figure 1B), with younger axons running along the glial endfeet-lined pial surface, displacing older axons progressively deeper in the tract \cite{colello1992observations,reese1987distributionrat,reese1990fibre,reese1997chronotopic,walsh1985age}. %Check Reese 1987 citation
Interestingly, age-related axon order is largely lost within the chiasm itself, though it reappears in the tract \cite{colello1998changing}.

