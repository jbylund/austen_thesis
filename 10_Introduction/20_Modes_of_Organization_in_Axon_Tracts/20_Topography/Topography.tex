Topographic order is the most common organizational principle of neural circuits.
This can be most intuitively appreciated in the visual and sensorimotor systems, in which visual and body space, respectively, are mapped onto the brain in a spatially ordered fashion, providing a logical transmission of these two sensory modalities.
In the visual system, RGCs are the only projection neurons that exit the retina and connect to central brain targets in a hallmark retinotopic manner, with a clear topographic correspondence between RGC position in the retina and terminal targets in the brain \cite{lund1974organization}.
In the sensorimotor system, somatotopy, the representation of the body map, must be contained in both motor efferents projecting to specific muscle fibers and sensory afferents conveying touch and pain information from the periphery, and can be easily appreciated, for example, in the barrel fields representing the vibrissae of rodents \cite{woolsey1970structural}.

Having established topographic correspondence between neuron and target, a handful of researchers in the 1970s turned their attention to the question of whether axons maintain similar topographic organization along the length of their tracts.
Hypotheses on the extent and significance of such organization fell along a continuum between two extremes.
At one extreme, precise maintenance of neighbor relationships would be preserved along the length of a tract, from cell bodies to axon terminations.
In this hypothetical instance, mechanical forces between neighboring axons would theoretically serve a guidance role as axons grew to the correct part of the topographically organized target.
At the other extreme of the hypothetical spectrum, simple chemoaffinity within the target would be sufficient to organize incoming axons, and axon order in the tract would either be non-existent or inconsequential to subsequent targeting.
In other words, as long as axons grossly reached the correct target, any organization within the tract would be irrelevant for targeting decisions.

In support of the first of these hypotheses, RGC axons were observed to bundle into fascicles as they navigate toward the optic disc within the developing chick retina \cite{goldberg1972topographical}.
It was suggested that this preponderance of fasciculating fibers prior to exiting the retina helped maintain order along the remainder of the developing visual pathway. 
When technical advances allowed for closer examination, however, a remarkable extent of splitting and crossing over of axons between fascicles was observed in both the retina \cite{simon1991relationship} and optic stalk \cite{williams1985dispersion,simon1991relationship}.
Thus, one-to-one neighbor relationships are not preserved between RGC axons upon entry into the optic nerve, undermining the simplest interpretation of a hypothetically organized axon tract.

Further studies using classical tracing methods and electrophysiology revealed that RGC axons maintain a rough retinotopic arrangement as they exit the eye and enter the optic nerve.
This coarse retinotopy is gradually, though not entirely, lost as the axons extend within the optic nerve towards the optic chiasm.
Such progressive scattering of retinotopic order along the length of the nerve has been described in frog \cite{montgomery1998organization}, chick \cite{ehrlich1984course}, mouse \cite{chan1999changes,plas2005pretarget}, rat \cite{simon1991relationship,chan1994changes}, ferret \cite{reese1993reestablishment}, cat \cite{horton1979non,naito1986course}, and monkey \cite{naito1994retinogeniculate}.
These studies of retinotopic axon order in the optic nerve reiterate the lack of one-to-one neighbor relationships, but all support a consistent, if coarse, organization within the nerve.
Indeed, despite finding that axon order in the cat optic nerve is not a simple reflection of retinal coordinates, Horton et al. ``do not conclude that the fibres are arranged haphazardly.''
Instead, the authors speculate: ``once the fibres diverge quickly in the initial portion of the nerve, they tend to maintain a more constant disposition, hinting that the scatter is somehow controlled'' \cite{horton1979non}.
Thus, though it may still be unclear how, axon organization in the nerve might serve as a functional prelude to successful navigation of later guidance cues or targeting steps.

Once RGC axons are in the optic chiasm, the midline choice point that sorts out the ipsi- and contralateral retinal projections, topographic order appears largely, though not completely lost.
RGC axons unbundle from their more tightly organized nerve fascicles, spread out across and through the depth of the chiasm, and then enter into the optic tract, as they extend to the dorsal lateral geniculate nucleus (dLGN), superior colliculus (SC), and other non-image-forming targets.
Once in the optic tract, RGC axons re-sort into orderly arrangements, again coarsely reflecting retinotopic order \cite{torrealba1982studies,reh1983organization,reese1993reestablishment,chan1994changes,chan1999changes,plas2005pretarget}.
The details of topographic organization in the optic tract will be discussed in more detail in Section~\ref{sec:RetinogeniculateOrganization}.

Similar to retinotopy in the visual system, a coarse somatotopy has been identified in the sensorimotor pathways \cite{barnard1956study,maslany1991somatotopic,milner1998selective}
Many details of the axon organization within the descending and ascending tracts in the spinal cord remain largely undefined, but what is known recapitulates the general findings from studies of the retinofugal tracts: coarse topography is maintained in tracts, but one-to-one neighbor relationships between axons are not.
In the chick, for instance, motor neuron axons destined for a specific muscle course together in distinct bundles, but do not preserve strict neighbor relationship along their path to the muscle \cite{lance1981pathway}. 
Instead, axons rearrange along the path and sort out within the limb plexus into discrete fascicles bound for specific muscles. 
This finding led the authors to suggest that active mechanisms of pre-target axon sorting form the circuit rather than either passive maintenance of neighbor relationships or exuberant, disorganized growth followed by pruning \cite{lance1981pathway}. %delete?
More recent retrograde labeling studies tracked motor axon fascicles through the peripheral sciatic-tibial nerve in rats and found somatotopy preserved along the length of the nerve \cite{badia2010topographical}, confirming the maintenance of topographic order in motor axon tracts. 
What is perhaps better understood than somatotopic organization in sensory and motor axon tracts is the relationship between the two tracts and modality-specific organization within them, which will be discussed further in the following section (Section~\ref{sec:Typography}).

Topographic organization is a defining factor in the auditory and olfactory systems, as well, despite the fact that these two sensory modalities do not directly relate to three-dimensional space.
In the auditory system, sound frequencies are transduced along the tonotopically organized basilar membrane in the cochlea.
This tonotopic organization is maintained throughout the brainstem and central auditory nuclei (reviewed in \cite{appler2011connecting}.
Tonotopic organization at each auditory brain target develops precisely before the onset of hearing and processing of acoustic information \cite{rubel2002auditory,appler2011connecting}.
Though it has yet to be directly tested, preservation of tonotopy among axons within the auditory nerve and tracts connecting auditory brain centers could be a means of establishing precise tonotopy in each auditory target early in development.

At least one portion of the auditory pathway has been shown to maintain an incredibly precise tonotopic map through its tract. 
Axons in the crossed dorsal cochlear tract (XDCT) connecting the nucleus magnocellularis (NM) and nucleus laminaris (NL) in the chick auditory brainstem are arranged in strict correspondence with their positions in both NM and NL \cite{kashima2013pre}. 
Multicellular labeling of regions in the NM by electroporation of dextran dyes in two areas along the tonotopic axis shows a clear correspondence between the position of groups of neurons in the NM and their axons in the XDCT. 
The authors also followed axons from individually labeled NM cells and found a strikingly linear relationship between the position of the soma in the NM and their respective axons in the XDCT \cite{kashima2013pre}. 
To date, this study is one of the highest resolution in any system, tracing individual axons as they project through the tract to their synaptic targets. 

The olfactory system encodes chemical cues, which lack any inherent spatial or topographic information.
However, even this system utilizes topographic organization in both the sense organ, the olfactory epithelium (OE) and the first olfactory target, the olfactory bulb (OB).
Topographic correspondence between the OE and OB was first identified over a half century ago, using degeneration studies in the rabbit olfactory system \cite{clark1951projection}.
Subsequent anatomical and physiological studies confirmed a correspondence of the dorsal-ventral and medio-lateral axes in the OE to the same axes in the OB, but little to no rostral-caudal correspondence between the two structures \cite{land1973localized,costanzo1978spatially,saucier1986analysis}.
Development of more advanced genetic tools revealed further topographic subdivisions in the OE, with olfactory sensory neurons (OSNs) expressing certain olfactory receptors (ORs), restricted to particular zones within the OE \cite{ressler1993zonal,vassar1993spatial}.
This confluence of topographic and typographic (i.e., based on OR-expression or OSN subtype) organizational principles is a theme common to other systems, which will be discussed further in Section~\ref{sec:Typography}.
Topographic arrangements of OSN axons in the olfactory have not been directly examined, but the typographic order that has been studied may be embedded in a broader topographic organization.
Studies examining higher order olfactory targets beyond the OB have noted that central olfactory regions fail to maintain topographic correspondence to the OB \cite{luskin1982distribution,sosulski2011distinct}.
Topographic order of OB efferents are only found in the earliest portion of the lateral olfactory tract (LOT), and axon order reflecting spatial information of the OB is presumably lost along the length of the LOT \cite{price1975observation}.

Similar to tracts in the spinal, visual, and auditory pathways, topography is evident in all cortical tracts that have been studied. 
In both rodents and primates, there is a topographic correspondence between primary thalamic nuclei and target regions in the cortex \cite{caviness1980tangential,hohl1991topographical}. 
An early study of the thalamocortical tract used anterograde tract labeling in ex vivo slice preparations of mouse brain to examine the entire projection from the ventrobasal complex in the thalamus to the primary sensory cortex \cite{bernardo1987axonal}. 
Though the order of specific subsets of axons within the tract was not examined, the authors found that axons diverged from their neighbors and rejoined other fascicles at numerous points along their path, again demonstrating the principle that one-to-one neighbor relationships are not preserved in axon tracts.
The patterns of axon rearrangements were consistent from animal to animal, indicating that while neighbor relationships between thalamocortical axons (TCAs) are not maintained, there is an underlying orderliness in the fiber bundling patterns along the thalamocortical tract. 
%One particular aspect of the tract described in this paper is a 180° rotation of the entire bundle of axons in the internal capsule (IC), which accounts for a rotated topographic map between the thalamus and sensory cortex \cite{bernardo1987axonal}. 
%The 180° rotation in the mouse optic tract, discussed above, could similarly account for the transposition of the topographic map between retina and SC \cite{plas2005pretarget}. 
Subsequent studies demonstrated that TCAs are topographically ordered as they course through the IC and approach the pallial/subpallial boundary (PSPB), before entering the cortex \cite{molnar1998mechanisms}.

Reciprocal contralaterally projecting corticocortical axons make up another major white matter tract in the cortex, the corpus callosum (CC). 
Structural studies identified an anterior-posterior distribution of fibers within the human CC based on axon diameter \cite{aboitiz1992fiber} and broadly topographic correspondence between a neuron’s regional location in the cortex and its axon position in the CC of mice \cite{ozaki1992prenatal}, cats \cite{nakamura1989topography}, and humans \cite{de1985topography}. 
Diffusion tensor imaging in human subjects has confirmed the topographic correspondence of cortical brain regions across the anterior-posterior axis of the CC in greater detail \cite{hofer2006topography}. 
Histological analyses in both macaques and human subjects also show anterior-posterior order in the CC of axon bundles corresponding to different cortical regions \cite{caminiti2013diameter}.
Evidence from anterograde and retrograde labeling studies in the cat visual cortex shows further that sub-region topographic maps are contained within the gross topography along the rostral-caudal axis of the CC \cite{payne1991visual}. 

The development of genetic tools and advanced labeling techniques has provided higher resolution views of axon order in the CC. 
Axons of Layer II/III sensory and motor cortical neurons in the mouse are topographically ordered as they traverse the CC to innervate the contralateral cortex \cite{zhou2013axon}. 
%Zhou et al. examined whether the topographic relationships between the neurons were preserved in their axons and, further, whether such order might affect their homotopic targeting in the contralateral cortex. 
The authors found a clear correspondence between neuron position in one hemisphere, axon position in the CC, and targeting position in the contralateral hemisphere: specifically, the medial-lateral axis is mirrored between the two hemispheres and maps onto the dorsal-ventral axis of the CC. 
This is true of axons arising from separate sensory and motor cortical areas as well as those axons arising from discrete medial-lateral positions within each cortical area \cite{zhou2013axon}. 
%Transition sentence?