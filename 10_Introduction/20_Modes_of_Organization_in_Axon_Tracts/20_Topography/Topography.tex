%Sperry’s pioneering work underlying the chemoaffinity hypothesis was conducted on retinal ganglion cell axon terminals as they map onto the optic tectum \cite{sperry1950regulative,sperry1963chemoaffinity}.
%Though topographic order of axon terminations in their targets has been characterized in many other systems since the 1960s, the retinofugal pathway has remained one of the dominant classic model systems.
%As such, some of the first steps towards understanding pre-target axon sorting were taken in the visual system.

Topographic order is a common feature
Examples



%%%
Retinal ganglion cells (RGCs) are the only projection neurons that exit the retina and connect to central brain targets, and the hallmark of the visual system is its retinotopy, the topographic correspondence between RGC location in the retina and terminal targets in the brain \cite{lund1974organization}.
During development, RGCs extend axons out of the retina and into the optic nerve to the optic chiasm, the midline choice point or intermediate target region of the retinofugal pathway.
RGC axons then grow into the optic tract, extending to the primary image-processing visual targets in the thalamus and midbrain, the dorsal lateral geniculate nucleus (dLGN) and superior colliculus (SC).
While subsets of RGC axons extend to other non-image-processing targets (e.g., the suprachiasmatic nucleus and ventral lateral geniculate nucleus), tracts projecting to these regions and retinotopy therein have been less well characterized.
Instead, pre-target axon sorting has been examined along two segments of the retino-thalamic pathway: the pre-chiasmatic optic nerve and the post-chiasmatic optic tract.

%Moved to tract-target section: Early hypotheses on the relationship between tract organization and targeting fall along a continuum between two extremes.
%Moved to tract-target section: At one extreme, retinotopy is thought to arise from maintenance of neighbor relationships between RGC axons from the retina to the brain targets; whereas at the other extreme, simple chemoaffinity within the target is thought to be sufficient to organize incoming axons, leaving axon order in the tract either non-existent or inconsequential.
In support of the first of these hypotheses, RGC axons were observed to bundle into fascicles as they navigate toward the optic disc within the developing chick retina \cite{goldberg1972topographical}.
This preponderance of fasciculating fibers prior to exiting the retina was suggested to help maintain order along the remainder of the developing visual pathway. 
When technical advances allowed for closer examination, however, a remarkable extent of splitting and crossing over of axons between fascicles was observed in both the retina \cite{simon1991relationship} and optic stalk \cite{simon1991relationship,williams1985dispersion}.
One-to-one neighbor relationships are not preserved between RGC axons upon entry into the optic nerve, and thus, as suggested by Barnard and Woolsey in 1956, simple mechanical maintenance of neighbor relationships must be insufficient to explain retinotopic mapping in central targets.

Further studies using classical tracing methods and electrophysiology revealed that as RGC axons exit the eye and enter the optic nerve, they maintain a rough retinotopic arrangement that coarsely reflects their neighbor relationships within the retina, but gradually become less ordered as they near the optic chiasm.
This progressive scattering of retinotopic order along the length of the nerve has been described in frog \cite{montgomery1998organization}, chick \cite{ehrlich1984course}, mouse \cite{chan1999changes,plas2005pretarget}, rat \cite{chan1994changes,simon1991relationship}, ferret \cite{reese1993reestablishment}, cat \cite{horton1979non,naito1986course}, and monkey \cite{naito1994retinogeniculate}.
These studies of retinotopic axon order in the optic nerve again demonstrate that one-to-one neighbor relationships are not maintained.
However, despite finding that axon order in the cat optic nerve is not a simple reflection of retinal coordinates, Horton et al. “do not conclude that the fibres are arranged haphazardly.”
Instead, the authors speculate: “once the fibres diverge quickly in the initial portion of the nerve, they tend to maintain a more constant disposition, hinting that the scatter is somehow controlled" \cite{horton1979non}.
Thus, though it may still be unclear how, axon organization in the nerve might serve as a functional prelude to successful navigation of later guidance cues or targeting steps.

At the chiasm itself, several guidance factors have been identified on radial glia and a specialized neuronal population that mediate the decision of retinal axons to project ipsilaterally (reviewed in \cite{petros2008retinal}, and discussed below).
While efforts are ongoing to identify the full symphony of decussation cues in the chiasm, details and mechanisms of topographic axon sorting therein are considerably less clear than decussation cues in the chiasm.
Topographic order appears to be largely, but not completely, lost as axons unbundle from nerve fascicles, spread out across and through the depth of the chiasm, and then re-sort into orderly arrangements as they enter the optic tract \cite{chan1999changes,chan1994changes}.
Indeed, the fascicular composition of RGC axons changes entirely within the optic chiasm, as axons from one eye split into many separate fascicles that cross contralateral axon bundles from the opposite eye at right angles, forming a braid-like configuration at the midline \cite{colello1998changing}.

After navigating the complex array of cues at the optic chiasm, RGC axons enter into the optic tract.
Though reports vary in the details and extent of order, a coarse topography of axons has been reported in the optic tract of many species \cite{chan1999changes,chan1994changes,plas2005pretarget,reese1993reestablishment,reh1983organization,torrealba1982studies}.
The variations between these studies can be attributed to different axon tracing methodologies, ranging from degeneration studies to various retrograde and anterograde axonal tracers, and to disagreement between physiological and anatomical data.
Additionally, it is sometimes difficult to find agreement among published histological analyses because many studies used unconventional planes of section in attempts to take true cross-sections of the tract, a fiber bundle that hugs the curves of the perimeter of the thalamus.

Anterograde labeling of retinal quadrants reveals that dorsal RGC axons course through the medial optic tract and ventral RGC axons through the lateral tract \cite{chan1999changes,chan1994changes,plas2005pretarget,reese1993reestablishment,reese1990fibre,reh1983organization}.
However, in the distal optic tract of the mouse, just prior to axon entry into the dLGN, dorsal RGC axons are found laterally and ventral RGC axons medially, opposite to their position in the proximal tract \cite{plas2005pretarget}. %Double check
This may be a reflection of a twist in the optic tract as a whole, which is evident in wholemount views of the surface of the thalamus, and also corresponds to the eventual dorsal-ventral mapping of RGCs along the lateral-medial axis in the SC \cite{plas2005pretarget}.
The chemical or mechanical factors that create this positional shift of dorsal and ventral axons in the tract are unknown.
It is noteworthy that in studies that directly examined the topographic order of retinal axons in the tract, in both mammalian \cite{chan1994changes,plas2005pretarget,reese1993reestablishment} and non-mammalian species \cite{ehrlich1984course,montgomery1998organization,reh1983organization,thanos1983investigations}, most conclude that the segregation between dorsal and ventral retinal axons is more distinct compared to that between nasal or temporal retinal axons, which are often described as being positioned adjacent, mixed, or otherwise overlapping in the tract.


In summary, there are at least three different types of axon order in the developing optic tract: coarse topography, chronotopy (which may reflect or occur independently of functional specificity), and eye-specificity (which may be considered typographic order based on cohort identity) (Figure 1).
These different modes of order are not easy to fully synthesize together, as they do not always appear in register with one another.
However, early reports of a lack of fiber order in the optic tract \cite{horton1979non} can be reconciled with later findings of axon order by an overlap of different modes of axon order.
One mode (e.g., retinotopy) may be obscured by another (e.g., chronotopy), depending on the methodology used to identify axon order.
In other words, failure to find a certain type of axon order at points along the visual pathway does not necessarily demonstrate a lack of order altogether.
Future studies may identify further contradictions between modes of axon order in the optic tract or even new modes altogether, which may, for instance, involve genetically defined subsets of RGCs.


Olf
Topographic correspondence between the OE and OB was first identified over a half century ago, using degeneration studies in the rabbit olfactory system \cite{clark1951projection}.
The findings were confirmed by subsequent anatomical and physiological studies, which describe a correspondence of the dorsal-ventral and medio-lateral axes in the OE to the same axes in the OB, but little to no rostral-caudal correspondence between the two structures \cite{costanzo1978spatially,land1973localized,saucier1986analysis}.
With the development of more advanced genetic tools, a series of studies showed first that there is zonal patterning in the OE, with OSNs expressing certain ORs restricted to particular zones \cite{ressler1993zonal,vassar1993spatial}; and second, that all OSNs expressing one type of OR project to specific, bilaterally symmetric glomeruli in the OB, creating what Ressler et al. term an “epitope map” of olfactory stimuli in the OB \cite{ressler1994information,vassar1994topographic}.
Importantly, this glomerular map in the OB is remarkably consistent across animals, demonstrating a highly stereotyped typographic map of ORs in the OB \cite{ressler1994information,vassar1994topographic}.
None of these studies directly examined axon arrangements in the olfactory nerve, but the degree of order collectively described between the OE and OB provided the first suggestions that organization of axons entering into the target may be important for wiring the circuit.

Studies examining higher order olfactory targets beyond the OB have noted that central olfactory regions fail to maintain topographic correspondence to the spatial organization of glomeruli in the OB \cite{luskin1982distribution,sosulski2011distinct}.
Topographic order of OB efferents can only be found in the earliest portion of the lateral olfactory tract (LOT), and axon order reflecting spatial information of the OB is presumably lost along the length of the LOT \cite{price1975observation}.


Aud
Relative to the other sensory systems, axon guidance in the auditory system is particularly poorly understood. 
Tonotopy – the organization of sound frequencies along an axis – is evident in the basilar membrane of the cochlea and all brainstem and central auditory centers (reviewed in \cite{appler2011connecting}. 
Tonotopic organization at each auditory brain target develops precisely before the onset of hearing and processing of acoustic information \cite{appler2011connecting,rubel2002auditory}. 
Though it has yet to be directly tested, preservation of tonotopy amongst axons within the auditory nerve and tracts connecting auditory brain centers could be a means of establishing precise tonotopy in each auditory target early in development. 

At least one portion of the auditory pathway maintains an incredibly precise tonotopic map through the tract. 
Axons in the crossed dorsal cochlear tract (XDCT) connecting the nucleus magnocellularis (NM) and nucleus laminaris (NL) in the chick auditory brainstem are arranged in strict correspondence with their positions in both NM and NL \cite{kashima2013pre}. 
Multicellular labeling of regions in the NM by electroporation of dextran dyes in two areas along the tonotopic axis shows a clear correspondence between the position of groups of neurons in the NM and their axons in the XDCT. 
The authors also followed axons from individually labeled NM cells and found a strikingly linear relationship between the position of the soma in the NM and their respective axons in the XDCT \cite{kashima2013pre}. 
To date, this study is one of the highest resolution in any system, tracing individual axons as they project through the tract to their synaptic targets. 


Spinal
The sensorimotor tracts in the spinal cord are another system in which topography of projections, similar to retinotopy in the visual system, spatially reflects the information being transmitted.
In this case, somatotopy, the representation of the body map, must be contained in both motor efferents projecting to specific muscle fibers and sensory afferents conveying touch and pain information from the periphery.
The somatotopic correspondence of the periphery, neuronal nuclei in the spinal cord, and the sensory and motor cortices has long been known.
Indeed much of the research on the development of the sensorimotor system has focused on understanding mechanisms of neuronal diversification and migration to their correct locations at these discrete points in the system (reviewed in \cite{kania2014spinal}.
However, details of motor and sensory axon tract organization within this system remain less clear.

The authors concluded that positional information of neurons in the spinal cord “can be used as a mechanism for imparting order to the presynaptic components” of the pathway, hypothesizing that contact-mediated guidance and selective fasciculation underlie such pre-target axon order \cite{nornes1980pattern}. 
Another early study in the chick showed that motor neuron axons destined for a specific muscle course together in distinct bundles but do not maintain strict neighbor relationships along the path to the muscle \cite{lance1981pathway}. 
Instead, axons rearrange along the path and sort out within the limb plexus into discrete fascicles bound for specific muscles. 
This finding led the authors to suggest that active mechanisms of pre-target axon sorting form the circuit rather than either passive maintenance of neighbor relationships or exuberant, disorganized growth followed by pruning \cite{lance1981pathway}. 
More recent retrograde labeling studies tracked motor axon fascicles through the peripheral sciatic-tibial nerve in rats and found somatotopy preserved along the length of the nerve \cite{badia2010topographical}. 
Further details of topographic organization of axon bundles in the spinal cord and periphery remain largely undefined.


Cortical
Similar to tracts in the spinal, visual, and auditory pathways, topography is evident in all of the cortical tracts described. 
In both rodents and primates, there is a topographic correspondence between primary thalamic nuclei and target regions in the cortex \cite{caviness1980tangential,hohl1991topographical}. 
An early study of the thalamocortical tract used anterograde tract labeling in ex vivo slice preparations of mouse brain to examine the entire projection from the ventrobasal complex in the thalamus to the primary sensory cortex \cite{bernardo1987axonal}. 
Though the order of specific subsets of axons within the tract was not examined, the authors found that axons diverged from their neighbors and rejoined other fascicles at numerous points along their path. 
The patterns of axon rearrangement were consistent from animal to animal, indicating that while neighbor relationships between thalamocortical axons (TCAs) are not maintained, there is an underlying orderliness in the fiber bundling patterns along the thalamocortical tract. 
One particular aspect of the tract described in this paper is a 180° rotation of the entire bundle of axons in the internal capsule (IC), which accounts for a rotated topographic map between the thalamus and sensory cortex \cite{bernardo1987axonal}. 
The 180° rotation in the mouse optic tract, discussed above, could similarly account for the transposition of the topographic map between retina and SC \cite{plas2005pretarget}. 

Subsequent studies sought to describe the axon organization along the thalamocortical pathway in more detail and discover the underlying mechanisms of tract order and target specificity. 
Explants of thalamic tissue innervate cortical slices promiscuously in vitro, regardless of whether the cortical slice is from the appropriate region for the thalamic explant to target \cite{molnar1991lack}, suggesting that chemotropic cues in the cortex are insufficient to organize TCAs. 
Later studies from the same group confirmed the suspicion that TCAs are topographically ordered as they course through the IC and approach the pallial/subpallial boundary (PSPB), before entering the cortex \cite{molnar1998mechanisms}. 
Several cues in this subpallial region have been identified that orient subsets of TCAs, arranging them topographically inside the corridor \cite{dufour2003area,powell2008topography,seibt2003neurogenin2}. 

In addition to the maintenance of topographic order of TCAs in the thalamocortical tract, there are distinct spatial and temporal relationships between the reciprocal projections of TCAs and corticothalamic axons (CTAs) along this pathway. 
Anterograde labeling of small populations of cortical and thalamic projection neurons shows axons from each population running alongside each other in the same fascicles, rather than two separate tracts of TCA and CTA bundles \cite{molnar1998mechanisms}. 
Thus, ascending and descending axons in this tract are both topographically ordered and course in close association with each other. 

Reciprocal contralaterally projecting corticocortical axons make up another major white matter tract in the cortex, the corpus callosum (CC). 
Structural studies identified an anterior-posterior distribution of fibers within the human CC based on axon diameter \cite{aboitiz1992fiber} and broadly topographic correspondence between a neuron’s regional location in the cortex and its axon position in the CC of mice \cite{ozaki1992prenatal}, cats \cite{nakamura1989topography}, and humans \cite{de1985topography}. 
Diffusion tensor imaging in human subjects has confirmed the topographic correspondence of cortical brain regions across the anterior-posterior axis of the CC in greater detail \cite{hofer2006topography}. 
Histological analyses in both macaques and human subjects confirm diffusion tensor imaging findings of anterior-posterior order of axon bundles corresponding to different cortical regions \cite{caminiti2013diameter}.
Evidence from anterograde and retrograde labeling studies in the cat visual cortex shows further that sub-region topographic maps are contained within the gross topography along the rostral-caudal axis of the CC \cite{payne1991visual}. 

The development of genetic tools and advanced labeling techniques has provided higher resolution views of axon order in the CC. 
Axons of Layer II/III sensory and motor cortical neurons in the mouse are topographically ordered as they traverse the CC to innervate the contralateral cortex \cite{zhou2013axon}. 
Zhou et al. examined whether the topographic relationships between the neurons were preserved in their axons and, further, whether such order might affect their homotopic targeting in the contralateral cortex. 
The authors found a clear correspondence between neuron position in one hemisphere, axon position in the CC, and targeting position in the contralateral hemisphere: specifically, the medial-lateral axis is mirrored between the two hemispheres and maps onto the dorsal-ventral axis of the CC. 
This is true of axons arising from separate sensory and motor cortical areas as well as those axons arising from discrete medial-lateral positions within each cortical area \cite{zhou2013axon}. 


