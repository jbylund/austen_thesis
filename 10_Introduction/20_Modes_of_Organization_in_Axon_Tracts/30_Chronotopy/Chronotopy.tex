\label{sec:Chronotopy}
A second commonly identified mode of axon organization in tracts is based on axon age and/or timing of entry into the tract. 
This principle is partially evident in the choreographed timing of corticothalamic and thalamocortical axons at the pallial/subpallial boundary (PSPB), where the arrival time of each axonal cohort into that region is an important step in the formation of a functional thalamocortical circuit (reviewed in \citenoparens{leyva2013and}).

However, chronotopy is more often considered in the way it was first described in the optic tract of the frog.
Backlabelling a portion of the optic tract resulted in an annular retrograde label in the retina, denoting a cohort of RGCs born at the same time \cite{reh1983organization,fawcett1984fibre}.
The annular backlabeling pattern reflects the additive retinogenesis pattern in frogs and fish: as the animal grows, RGCs are added in concentric rings at the periphery of the retina.
Furthermore, axons are topographically ordered within age-related bundles, where dorsal and ventral RGCs extend axons laterally and medially, respectively, while nasal and temporal RGC axons are mixed together across much of the tract.
\citetext{reh1983organization} propose that chronotopic order likely stems from passive mechanisms related to space constraints on incoming axons exiting the optic chiasm, while topographic arrangement of axons are a result of active reordering mechanisms in the optic tract.

The mammalian optic tract also display chronotopic axon order, with younger axons running along the glial endfeet-lined pial surface, displacing older axons progressively deeper in the tract \cite{walsh1985age,reese1987distributionrat,reese1990fibre,colello1992observations,reese1997chronotopic}.
Interestingly, while there appears to be little to no chronotopic order in much of the nerve, growth cones cluster nearer the pial surface as they approach the chiasm.
Age-related axon order is then again largely lost within the chiasm itself, only to reappear in the tract \cite{reese1987distributionrat,colello1992observations,colello1998changing}.

Chronotopic order of axons in the optic tract may reflect segregation of functional subtypes of RGCs.
In the cortex, timing of neuronal differentiation is critical for establishing neuronal subtype identity and laminar position \cite{molyneaux2007neuronal}, a principle that appears to be true of RGCs in the visual system as well.
RGCs of differing soma diameter, an early proxy for identifying RGC subtypes, are generated at different times in the retina \cite{reese1994birthdates,rapaport1995spatiotemporal}, and birthdate of RGC subtypes correlates to target matching strategies in the developing SC and dLGN \cite{osterhout2014birthdate}.
Furthermore, timing is particularly important for the appropriate differentiation of ipsilateral RGCs \cite{bhansali2014delayed}.
These recent studies, therefore, align well with earlier findings of segregation based on axon diameter in the optic tract \cite{guillery1982arrangement,torrealba1982studies,reese1987distributionrat,reese1990fibre}.

\label{sec:LOT}
In the olfactory system, studies failed to find topographic order in all but the earliest portion of the lateral olfactory tract (LOT) \cite{price1975observation}.
Central olfactory regions also fail to maintain a clear topographic correspondence to the spatial organization of glomeruli in the OB \cite{luskin1982distribution,sosulski2011distinct}.
However, chronotopic order has been identified among the OB efferents in the LOT, such that, similar to chronotopy in the optic tract, younger axons exiting the OB course nearer the pial surface, with older axons running more deeply in the tract \cite{yamatani2004chronotopic}. 
This finding highlights the fact that lack of evidence of one mode of axon organization does not necessarily mean a lack of organization overall.
Finally, chronotopy is also found in the chick spinal cord: younger axons, i.e. those that have more recently entered the tract, grow along older ones, contributing to the orderly arrangement of ascending and descending sensorimotor axon tracts \cite{nornes1980pattern}.