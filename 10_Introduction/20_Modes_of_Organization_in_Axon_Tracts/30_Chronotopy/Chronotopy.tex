Visual
Chronotopy, or age-related order, was first suggested as a mode of axon organization in the frog, where backlabeling a portion of the optic tract results in annular retrograde labeling in the retina, denoting a cohort of RGCs born at the same time \cite{fawcett1984fibre,reh1983organization}.
Axons are topographically ordered within age-related bundles, where dorsal and ventral RGCs extend axons laterally and medially, respectively, while nasal and temporal RGC axons are mixed together across much of the tract \cite{reh1983organization}.
Note that this arrangement differs from topographic order in mammalian tracts.
The authors propose that active reordering mechanisms play a part in the topographic arrangement of fibers, while chronotopic order likely stems from passive mechanisms related to space constraints on incoming axons exiting the optic chiasm.
Furthermore, Reh et al. hypothesize the presence of attractive cues along the dorsal neuraxis that could preferentially act upon ventral RGC axons, providing a tract-specific mechanism of active axon organization that would occur independently of target derived cues \cite{reh1983organization}.
Mammalian optic tracts also display chronotopic axon order (Figure 1B), with younger axons running along the glial endfeet-lined pial surface, displacing older axons progressively deeper in the tract \cite{colello1992observations,reese1987distributionrat,reese1990fibre,reese1997chronotopic,walsh1985age}. %Check Reese 1987 citation
Interestingly, age-related axon order is largely lost within the chiasm itself, though it reappears in the tract \cite{colello1998changing}.

Olf
A more recent study found that OB efferents are in fact arranged chronotopically in the LOT, with younger axons coursing nearer the pial surface, and older axons more deeply in the tract \cite{yamatani2004chronotopic}, notably similar to chronotopic order in the optic tract.
This finding again highlights the fact that lack of evidence of one mode of axon organization does not necessarily mean a lack of organization overall.

No evidence in auditory system

sensorimotor
Early tract tracing experiments revealed an orderly arrangement of ascending and descending axon tracts in the chick spinal cord, where newer axons grow along older ones \cite{nornes1980pattern}, similar to chronotopic layering of axons in the optic and lateral olfactory tracts. 

TCA
Additionally, the meeting of CTAs and TCAs at the PSPB is temporally regulated, as CTAs extend into the PSPB first and then wait for the arrival of TCAs (reviewed in \cite{leyva2013and}). 
This interaction is required for the proper guidance of both cohorts and was dubbed the “handshake hypothesis” (\cite{molnar1995thalamic}; discussed in more detail below). 
Similar to the temporal regulation of motor and sensory axons in the spinal cord, the chronological sequence of axon extension is a key element guiding the development of the reciprocal connections between cortex and thalamus. 
