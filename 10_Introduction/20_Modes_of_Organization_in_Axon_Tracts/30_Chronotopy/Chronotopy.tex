\label{sec:Chronotopy}
The third commonly identified mode of axon organization in tracts is based on axon age and/or timing of entry into the tract. 
This principle is partially evident in the choreographed timing of CTA and TCA axons at the PSPB, discussed in the previous section.
That is, the timing of entry of each axon cohort into that portion of the thalamocortical tract is an important step in the formation of a functional thalamocortical circuit.

Chronotopy is, however, more often considered in the way it was first described in the optic tract of the frog.
Backlabelling a portion of the optic tract resulted in an annular retrograde label in the retina, denoting a cohort of RGCs born at the same time \cite{fawcett1984fibre,reh1983organization}.
The annular backlabeling pattern reflects the additive retinogenesis pattern in these species: as the animal grows, RGCs are added in concentric rings at the periphery of the retina.
Furthermore, axons are topographically ordered within age-related bundles, where dorsal and ventral RGCs extend axons laterally and medially, respectively, while nasal and temporal RGC axons are mixed together across much of the tract.
The authors propose that chronotopic order likely stems from passive mechanisms related to space constraints on incoming axons exiting the optic chiasm, while topographic arrangement of axons are a result of active reordering mechanisms in the optic tract \cite{reh1983organization}.
The mammalian optic tract also display chronotopic axon order, with younger axons running along the glial endfeet-lined pial surface, displacing older axons progressively deeper in the tract \cite{colello1992observations,reese1987distributionrat,reese1990fibre,reese1997chronotopic,walsh1985age}.
Interestingly, while there appears to be little to no chronotopic order in much of the nerve, growth cones cluster nearer the pial surface as they approach the chiasm, but age-related axon order is largely lost within the chiasm itself, only to reappear in the tract \cite{colello1998changing,colello1992observations,reese1987distributionrat}.

Returning to the theme of how different modes of axon organization relate to one another, chronotopic order of axons in the optic tract may reflect segregation of functional subtypes of RGCs.
In the cortex, timing of neuronal differentiation is critical for establishing neuronal subtype identity and laminar position \cite{molyneaux2007neuronal}, a principle that appears to be true of RGCs in the retinofugal system as well.
RGCs of differing soma diameter, an early proxy for identifying RGC subtypes, are generated at different times in the retina \cite{rapaport1995spatiotemporal,reese1994birthdates}.
Furthermore, timing is important for the appropriate differentiation of ipsilateral RGCs \cite{bhansali2014delayed}, and birthdate of RGC subtypes correlates to target matching strategies in the developing SC and dLGN \cite{osterhout2014birthdate}.
These recent studies provide support for earlier findings, discussed in the previous section, which found axon segregation based on axon diameter in the optic tracts \cite{reese1987distributionrat,guillery1982arrangement,torrealba1982studies,reese1990fibre}.

\label{sec:LOT}
While studies failed to find topographic order in all but the earliest portion of the LOT \cite{price1975observation}, and central olfactory regions fail to maintain topographic correspondence to the spatial organization of glomeruli in the OB \cite{luskin1982distribution,sosulski2011distinct}, chronotopic order has been identified amongst the OB efferents in this central olfactory tract.
Similar to the chronotopy identified in the optic tract, younger axons exiting the OB course nearer the pial surface, with older axons running more deeply in the tract \cite{yamatani2004chronotopic}. 
This finding highlights the fact that lack of evidence of one mode of axon organization does not necessarily mean a lack of organization overall.
Finally, chronotopy is also found in the chick spinal cord: younger axons, i.e. those have more recently entered the tract, grow along the older ones, contributing to the orderly arrangement of ascending and descending sensorimotor axon tracts \cite{nornes1980pattern}.