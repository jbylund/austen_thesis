\label{sec:Typography}
Typographic order has also been identified in many axon tracts.
In this organizational paradigm, axons arising from functionally and/or molecularly distinct neuronal cohorts are sorted in their tract.
The olfactory system is perhaps the clearest example of this principle. 
In the OB, OSN axon terminals are organized typographically, where OSNs of the same type, i.e. expressing the same OR, coalesce into distinct, bilaterally symmetric glomeruli in the OB, creating what Ressler et al. term an “epitope map” of olfactory stimuli in the OB \cite{ressler1994information,vassar1994topographic}.
This epitope, or glomerular, map in the OB is remarkably consistent across animals, demonstrating a highly stereotyped typographic map of ORs in the OB \cite{ressler1994information,vassar1994topographic}.

It is thought that axon sorting into glomeruli occurs, possibly by homotypic axon-axon interactions based on OR expression \cite{feinstein2004contextual}, within the glomerular layer of the OB.
However, pre-target ordering of axons in the olfactory nerve may serve a preliminary role in establishing the olfactory sensory circuit \cite{miller2010axon}.
%!!See Raper abstract from CSHL meeting - ORs themselves are not necessarily what guide final glomeruli choice.
In \emph{Xenopus} tadpoles, relative expression levels of Neuropilin (Nrp) and Plexin, both receptors to Semaphorins (Semas), label two subpopulations of OSNs \cite{satoda1995differential}.
These two cohorts of axons, while intermingled early in the olfactory nerve, become increasingly sorted as they project to the OB.
In the distal olfactory nerve, there is a sharp border between the two bundles of axons, the segregation of which is reflected in their final segregation within the OB \cite{satoda1995differential}.
Similarly, Nrp1 and Sema3A demarcate subpopulations of OSN axons in the mouse olfactory nerve that exit the OE in overlapping swathes of axons and become increasingly orderly along the nerve, until there is a very clear segregation of axon cohorts prior to entry into the inner nerve layer of the OB \cite{imai2009pre,miller2010axon}.
%Add more discussion, including more things from Raper's work?

In the visual system, RGCs are classified by functional subtypes.
The number of such functional subtypes is growing as newer methods identify increasingly specific functional and genetic programs that define RGC subtypes \cite{rivlin2011transgenic,baden2016functional}.
However, no study to date has examined the organization of axons extending from any of these distinct subtypes in any of the retinofugal tracts.
Before there were reliable genetic tools to classify RGC subtypes, histological studies used axon diameter and cell morphology as proxies of functional identity.
Based on these anatomical criteria, subsets of RGC axons are segregated in the optic tract of rodents \cite{reese1987distributionrat}, cats \cite{guillery1982arrangement,torrealba1982studies}, and monkeys \cite{reese1990fibre}.
Future studies may use new reporter mouse lines to examine the axon organization of genetically-defined RGC subtypes in the optic nerve and tract.
One major classification of RGC subtype is based on axon behavior at the midline choice point of the optic chiasm - i.e., ipsilaterally- and contralaterally-projecting RGCs.
Only one study has examined ipsilateral and contralateral RGC axon position in the optic tract, and only in a limited fashion \cite{godement1984prenatal}.
Briefly, Godement et al. \shortcite{godement1984prenatal} concluded that ipsilateral RGC axons localize to the superficial optic tract, while contralateral RGC axons occupy the entire tract.
I will discuss these findings in more detail in the following chapter, prior to presenting my own findings on ipsi- and contralateral RGC axon organization in the mouse retinogeniculate pathway.

Typographic order is also evident in the subtype organization of axons in the spinal tracts.
Motor neuron axons innervating fast and slow muscle fibers in the chick hindlimb fasciculate separately from each other starting in the proximal regions of the limb, well before reaching their target muscles \cite{milner1998selective}.
However, motor axons projecting to different parts of fast muscle fibers remain intermingled en route to the muscle, suggesting that a second step of axon sorting must occur within the target to organize this subset of inputs \cite{milner1998selective}.
Additionally, mechanosensory and proprioceptive sensory afferents segregate along the medial-lateral axis in the dorsal column of the mouse spinal cord.
Notably, axons are arranged somatotopically within this typographic order \cite{niu2013modality}, revealing a layering of modes of axon organization.
Given that these afferents terminate in a modality-specific manner in the medulla, pre-target typographic order of the axons may be functionally important in establishing modality based targeting patterns \cite{niu2013modality}.

In both the sensorimotor and thalamocortical tracts, consistent typographic order exists between heterotypic axon cohorts.
Take, for instance, the tight relationship between motor and sensory axon fascicles, which run alongside each other in peripheral nerves \cite{honig1998spatial}. 
During the earliest phases of axon extension, motor neuron axons enter peripheral nerve tracts, displaying a high degree of specificity in their initial trajectory, followed by sensory afferents, which use motor axon fascicles as a tracking system to lead them to their respective targets \cite{landmesser1986altered,huettl2011npn,wang2013axons,wang2014conserved}.
Motor and sensory axons are closely associated but segregated –- indeed their segregation can be recapitulated in vitro \cite{gallarda2008segregation} –- and have a hierarchical dependence on each other, such that sympathetic efferents rely on sensory afferents, which in turn rely on motor efferents to project to their respective targets \cite{wang2014conserved}.

Similarly, there are distinct spatial and temporal relationships between the reciprocal projections of TCAs and corticothalamic axons (CTAs) in their tract.
The meeting of CTAs and TCAs at the pallial-subpallial boundary (PSPB) is temporally regulated: CTAs extend into the PSPB first and then wait for the arrival of TCAs (reviewed in \citenoparens{leyva2013and}). 
This interaction is required for the proper guidance of both cohorts and was dubbed the “handshake hypothesis” \cite{molnar1995thalamic}.
Similar to the temporal regulation of motor and sensory axons in the spinal cord, this sequence of axon extension and the ensuing relationship between the two axon cohorts are key elements guiding the development of the reciprocal connections between cortex and thalamus. 
The mechanisms and purpose of the TCA waiting period below the cortex remain unclear, but recent evidence suggests that ascending TCAs require descending CTAs in order to cross the PSPB and innervate the cortex \cite{chen2012evidence}.
The relationship between these reciprocal axon cohorts is quite distinct.
Anterograde labeling of small populations of cortical and thalamic projection neurons shows axons from each population running alongside each other in the same fascicles, rather than two separate tracts of TCA and CTA bundles \cite{molnar1998mechanisms}. 
These data suggest that ascending and descending axons in this tract are both topographically ordered and course in close association with each other, an indication of their cohort-specific organization. 
%Edit this