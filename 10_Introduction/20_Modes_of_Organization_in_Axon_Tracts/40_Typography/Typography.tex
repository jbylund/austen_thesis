Chronotopic order of axons in the optic tract may reflect segregation of functional subtypes of RGCs.
Work in the cortex has shown that timing of neuronal differentiation is critical for establishing neuronal subtype identity and laminar position \cite{molyneaux2007neuronal}, a principle that appears to be true of RGCs in the retinofugal system as well.
RGCs of differing soma diameter, an early proxy for identifying RGC subtypes, are generated at different times in the retina \cite{rapaport1995spatiotemporal,reese1994birthdates}.
Furthermore, timing is important for the appropriate differentiation of ipsilateral RGCs \cite{bhansali2014delayed}, and birthdate of RGC subtypes correlates to target matching strategies in the developing SC and dLGN \cite{osterhout2014birthdate}.
These recent studies provide support for earlier findings, which used axon diameter and cell morphology to show that functional subsets of RGC axons are segregated in the optic tracts of rodents \cite{reese1987distributionrat}, cats \cite{guillery1982arrangement,torrealba1982studies}, and monkeys \cite{reese1990fibre}.


Olf
Whereas the visual system relays topographically related sensory information between sense organ and brain target, the olfactory system transduces chemical stimuli, which do not require a one-to-one spatial map to represent the sensory information.
While there are accordingly fundamental differences in the relationship between sense organ and target in the visual and olfactory systems, they both exhibit order among axons in their tracts. Olfactory sensory neurons (OSNs) lining the olfactory epithelium (OE) of the nose send their axons into the brain target, the olfactory bulb (OB).
In the OB, OSN axons are organized by a “typographic” principle, where OSNs of the same type, i.e. expressing the same olfactory receptor (OR), coalesce into distinct target glomeruli via presumed homotypic axon-axon interactions based on OR expression \cite{feinstein2004contextual}.
As retinotopy is the hallmark of visual system targeting, OR-based typography is the hallmark of olfactory system targeting.
However, there is a coarse spatial pattern of expression of ORs among OSNs in the OE, as well as some topographic correspondence between the OE and OB.
Thus, while ultimate axon sorting decisions based on OR expression occur within the glomerular layer of the OB, pre-target ordering of axons in the olfactory nerve may serve a preliminary role in establishing the olfactory sensory circuit \cite{miller2010axon}.

Further exploration of the olfactory nerve found that OSN axons undergo organizational events as they grow towards the OB.
Two subpopulations of OSNs were identified in Xenopus tadpoles based on their relative expression levels of Neuropilin (Nrp) and Plexin, both receptors to Semaphorins (Semas) \cite{satoda1995differential}.
The two cohorts of axons are intermingled early in the olfactory nerve but become increasingly sorted as they project to the OB.
In the distal olfactory nerve, there is a sharp border between the two bundles of axons, the segregation of which reflects their final segregation in the OB \cite{satoda1995differential}.
The position of additional axon cohorts in the olfactory nerve is defined by expression of other molecular markers, with a similar degree of order described in all cases \cite{imai2009pre,miller2010axon}.
Regardless of the marker examined, OSN axon cohorts exit the OE in overlapping swathes of axons and become more orderly along the nerve.
There is a very clear order of axon bundles prior to entry into the inner nerve layer of the OB, where axons undergo further OR-based sorting into individual glomeruli \cite{imai2009pre,miller2010axon}.

Spinal
Other studies have found clear subtype organization (i.e., typographic order) of axons in the spinal cord prior to target innervation.
For instance, motor neuron axons innervating fast and slow muscle fibers in the chick hindlimb fasciculate separately from each other starting in the proximal regions of the limb, well before reaching their target muscles \cite{milner1998selective}.
However, motor axons projecting to different parts of fast muscle fibers remain intermingled en route to the muscle, implying that a second step of axon sorting must occur within the target to organize this subset of inputs \cite{milner1998selective}.
Additionally, mechanosensory and proprioceptive sensory afferents segregate along the medial-lateral axis in the dorsal column of the mouse spinal cord. Notably, axons are arranged somatotopically within this typographic order \cite{niu2013modality}, revealing again the layering of modes of axon organization.
Given that these afferents terminate in a modality-specific manner in the medulla, pre-target typographic order of the axons may be functionally important in establishing modality based targeting patterns \cite{niu2013modality}.

One of the better-understood aspects of axon organization in the periphery is the relationship between motor and sensory axon fascicles, which run alongside each other in peripheral nerves \cite{honig1998spatial}.
During the earliest phases of axon extension, motor neuron axons enter peripheral nerve tracts, displaying a high degree of specificity in their initial trajectory, followed by sensory afferents, which use motor axon fascicles as a tracking system to lead them to their respective targets \cite{huettl2011npn,landmesser1986altered,wang2013axons,wang2014conserved}.
Motor and sensory axons are closely associated but segregated – indeed their segregation can be recapitulated in vitro \cite{gallarda2008segregation} – and have a hierarchical dependence on each other, such that sympathetic efferents rely on sensory afferents, which in turn rely on motor efferents to project to their respective targets \cite{wang2014conserved}.
