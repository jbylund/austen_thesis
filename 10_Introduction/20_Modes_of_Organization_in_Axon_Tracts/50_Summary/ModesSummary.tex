In this section, I have described the three common modes of axon order that have been identified in various central and peripheral axon tracts.
While strict neighbor relationships do not persist in axon tracts (with the possible exception of the auditory brainstem XDCT \cite{kashima2013pre}), coarse topography has been identified in nearly all axon tracts studied.
Additionally, typography is notable in many systems, and chronotopy as well appears in some axon tracts.
These different organizational modes can be challenging to fully synthesize for a couple of reasons.
Not all axon organizational modes are apparent in all systems - this could reflect either a true lack of certain organizational modes in certain tracts, or simply a lack of evidence of such mode in a particular system.
One organizational mode might be obscured by another, making it more difficult to identify in that particular tract.
Similarly, failure to find a certain type of axon order at points along a pathway does not necessarily demonstrate a lack of order altogether.
This is demonstrated well in the LOT, which was thought to lack order altogether \cite{price1975observation}, but in fact contains chronotopic axon order \cite{yamatani2004chronotopic}.
This raises the possibility that within chrontopically ordered axon bundles in the LOT, topographic or typographic organization may still occur.
Perhaps there is a hierarchy among the organizational principles, with some modes serving a greater purpose in a certain tract, and others arising in a more incidental fashion.
For instance, as Reh et al. (1983) propose, chronotopic axon order may be a relatively simple consequence of space constraints as new axons enter a developing tract, while topographic order, which is crucial to preserve within the target, might be maintained in the tract by more active organizational elements.
To this end, most of the molecular mechanisms that have been identified in organizing axons in their tracts pertain to topography or typography, and less so to chronotopy.
The following section summarizes some common mechanisms underlying the organization within axon tracts.