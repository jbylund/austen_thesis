In this section, I described three common modes of axon order in various central and peripheral axon tracts.
While strict neighbor relationships do not persist in axon tracts (with the possible exception of the auditory brainstem XDCT, see \citenoparens{kashima2013pre}), coarse topography has been identified in nearly all axon tracts studied.
Additionally, typography is notable in many systems, and chronotopy also appears in some axon tracts.
These different organizational modes can be challenging to fully synthesize into a comprehensive overview of tract organization.
For one, not all axon organizational modes are apparent in all systems --- reflecting either a true lack of certain organizational modes in certain tracts, or simply a lack of evidence of such modes in a particular system.
One organizational mode might be obscured by another, making it more difficult to identify in that particular tract.
Similarly, failure to find a certain type of axon order at points along a pathway does not necessarily demonstrate a lack of order altogether.
This is demonstrated well in the LOT, which was thought to lack order altogether \cite{price1975observation}, but in fact displays chronotopic axon order \cite{yamatani2004chronotopic}.
It is possible that within chrontopically ordered axon bundles in the LOT, topographic or typographic organization may occur, but remain unidentified.

Secondly, it is unclear if there is a hierarchy among the organizational principles, with some modes serving a greater purpose in a certain tract, and others arising in a more incidental fashion.
For instance, as Reh et al. \shortcite{reh1983organization} propose, chronotopic axon order may be a relatively simple consequence of space constraints as new axons enter a developing tract.
Topographic order, which must be preserved within the target for appropriate function of the circuit, might be maintained in the tract by more active organizational elements \cite{reh1983organization}.
To this end, more molecular mechanisms have been identified in organizing axons based on topography or typography, while fewer mechanisms have been identified that establish chronotopic order.
The following section summarizes common mechanisms underlying the organization within axon tracts.