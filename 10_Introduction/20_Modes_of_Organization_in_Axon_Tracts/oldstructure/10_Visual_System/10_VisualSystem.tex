Visual System
Sperry’s pioneering work underlying the chemoaffinity hypothesis was conducted on retinal ganglion cell axon terminals as they map onto the optic tectum \cite{sperry1950regulative,sperry1963chemoaffinity}.
Though topographic order of axon terminations in their targets has been characterized in many other systems since the 1960s, the retinofugal pathway has remained one of the dominant classic model systems.
As such, many of the first steps towards understanding pre-target axon sorting were taken in the visual system, with a multitude of studies characterizing axon order in the optic nerve and tract. %Edit
Retinal ganglion cells (RGCs) are the only projection neurons that exit the retina and connect to central brain targets, and the hallmark of the visual system is its retinotopy, the topographic correspondence between RGC location in the retina and terminal targets in the brain \cite{lund1974organization}.
During development, RGCs extend axons out of the retina and into the optic nerve to the optic chiasm, the midline choice point or intermediate target region of the retinofugal pathway.
RGC axons then grow into the optic tract, extending to the primary image-processing visual targets in the thalamus and midbrain, the dorsal lateral geniculate nucleus (dLGN) and superior colliculus (SC).
While subsets of RGC axons extend to other non-image-processing targets (e.g., the suprachiasmatic nucleus and ventral lateral geniculate nucleus), tracts projecting to these regions and retinotopy therein have been less well characterized.
Instead, pre-target axon sorting has been examined along two segments of the retino-thalamic pathway: the pre-chiasmatic optic nerve and the post-chiasmatic optic tract.

Early hypotheses on the relationship between tract organization and targeting fall along a continuum between two extremes.
At one extreme, retinotopy is thought to arise from maintenance of neighbor relationships between RGC axons from the retina to the brain targets; whereas at the other extreme, simple chemoaffinity within the target is thought to be sufficient to organize incoming axons, leaving axon order in the tract either non-existent or inconsequential.
In support of the first of these hypotheses, RGC axons were observed to bundle into fascicles as they navigate toward the optic disc within the developing chick retina \cite{goldberg1972topographical}.
This preponderance of fasciculating fibers prior to exiting the retina was suggested to help maintain order along the remainder of the developing visual pathway. 
When technical advances allowed for closer examination, however, a remarkable extent of splitting and crossing over of axons between fascicles was observed in both the retina \cite{simon1991relationship} and optic stalk \cite{simon1991relationship,williams1985dispersion}.
One-to-one neighbor relationships are not preserved between RGC axons upon entry into the optic nerve, and thus, as suggested by Barnard and Woolsey in 1956, simple mechanical maintenance of neighbor relationships must be insufficient to explain retinotopic mapping in central targets.

Further studies using classical tracing methods and electrophysiology revealed that as RGC axons exit the eye and enter the optic nerve, they maintain a rough retinotopic arrangement that coarsely reflects their neighbor relationships within the retina, but gradually become less ordered as they near the optic chiasm.
This progressive scattering of retinotopic order along the length of the nerve has been described in frog \cite{montgomery1998organization}, chick \cite{ehrlich1984course}, mouse \cite{chan1999changes,plas2005pretarget}, rat \cite{chan1994changes,simon1991relationship}, ferret \cite{reese1993reestablishment}, cat \cite{horton1979non,naito1986course}, and monkey \cite{naito1994retinogeniculate}.
These studies of retinotopic axon order in the optic nerve again demonstrate that one-to-one neighbor relationships are not maintained.
However, despite finding that axon order in the cat optic nerve is not a simple reflection of retinal coordinates, Horton et al. “do not conclude that the fibres are arranged haphazardly.”
Instead, the authors speculate: “once the fibres diverge quickly in the initial portion of the nerve, they tend to maintain a more constant disposition, hinting that the scatter is somehow controlled" \cite{horton1979non}.
Thus, though it may still be unclear how, axon organization in the nerve might serve as a functional prelude to successful navigation of later guidance cues or targeting steps.

Retinofugal axons next navigate the optic chiasm, the midline choice point in binocular visual systems, where ipsi- and contralaterally projecting RGC axons from each eye diverge into the optic tract of the appropriate hemisphere.
Although degree of binocularity varies across species, axon reorganization at the chiasm is a key step in the development of the visual circuit and raises the question of how ipsi- and contralateral RGC axons are organized in the optic nerve prior to reaching the chiasm.
By backlabeling retinal axons from the optic tract, studies in rodents concluded that, similar to retinotopy in the nerve, some order exists early in the nerve between future ipsi- and contralaterally projecting RGC axons, but is progressively lost as the axons near the chiasm \cite{baker1989distribution,colello1990early}.
More recent approaches utilizing a genetic marker of ipsilateral identity confirm the segregation of ipsilateral axons from their contralateral neighbors in the nerve, but suggest that a greater degree of order is retained in the distal optic nerve than previously thought (Sitko and Mason, unpublished).

At the chiasm itself, several guidance factors have been identified on radial glia and a specialized neuronal population that mediate the decision of retinal axons to project ipsilaterally (reviewed in \cite{petros2008retinal}, and discussed below).
While efforts are ongoing to identify the full symphony of decussation cues in the chiasm, details and mechanisms of topographic axon sorting therein are considerably less clear than decussation cues in the chiasm.
Topographic order appears to be largely, but not completely, lost as axons unbundle from nerve fascicles, spread out across and through the depth of the chiasm, and then re-sort into orderly arrangements as they enter the optic tract \cite{chan1999changes,chan1994changes}.
Indeed, the fascicular composition of RGC axons changes entirely within the optic chiasm, as axons from one eye split into many separate fascicles that cross contralateral axon bundles from the opposite eye at right angles, forming a braid-like configuration at the midline \cite{colello1998changing}.

After navigating the complex array of cues at the optic chiasm, RGC axons enter into the optic tract.
Though reports vary in the details and extent of order, a coarse topography of axons has been reported in the optic tract of many species \cite{chan1999changes,chan1994changes,plas2005pretarget,reese1993reestablishment,reh1983organization,torrealba1982studies}.
The variations between these studies can be attributed to different axon tracing methodologies, ranging from degeneration studies to various retrograde and anterograde axonal tracers, and to disagreement between physiological and anatomical data.
Additionally, it is sometimes difficult to find agreement among published histological analyses because many studies used unconventional planes of section in attempts to take true cross-sections of the tract, a fiber bundle that hugs the curves of the perimeter of the thalamus.
Regardless of apparent inconsistencies across these studies, three broad conclusions surface from this work, which are expanded upon in the following paragraphs and summarized in Figure 1.
First, there is a coarse topographic order amongst retinal axons in the optic tract (Figure 1A).
Second, evidence supports an age-related, or chronotopic, order of retinal axons in the tract (Figure 1B), and finally, several studies argue for order among functional subclasses of RGC axons, including between ipsi- and contralateral axons (Figure 1C).

Anterograde labeling of retinal quadrants reveals that dorsal RGC axons course through the medial optic tract and ventral RGC axons through the lateral tract \cite{chan1999changes,chan1994changes,plas2005pretarget,reese1993reestablishment,reese1990fibre,reh1983organization}.
However, in the distal optic tract of the mouse, just prior to axon entry into the dLGN, dorsal RGC axons are found laterally and ventral RGC axons medially, opposite to their position in the proximal tract \cite{plas2005pretarget}. %Double check
This may be a reflection of a twist in the optic tract as a whole, which is evident in wholemount views of the surface of the thalamus, and also corresponds to the eventual dorsal-ventral mapping of RGCs along the lateral-medial axis in the SC \cite{plas2005pretarget}.
The chemical or mechanical factors that create this positional shift of dorsal and ventral axons in the tract are unknown.
It is noteworthy that in studies that directly examined the topographic order of retinal axons in the tract, in both mammalian \cite{chan1994changes,plas2005pretarget,reese1993reestablishment} and non-mammalian species \cite{ehrlich1984course,montgomery1998organization,reh1983organization,thanos1983investigations}, most conclude that the segregation between dorsal and ventral retinal axons is more distinct compared to that between nasal or temporal retinal axons, which are often described as being positioned adjacent, mixed, or otherwise overlapping in the tract.



Chronotopic order of axons in the optic tract may reflect segregation of functional subtypes of RGCs.
Work in the cortex has shown that timing of neuronal differentiation is critical for establishing neuronal subtype identity and laminar position \cite{molyneaux2007neuronal}, a principle that appears to be true of RGCs in the retinofugal system as well.
RGCs of differing soma diameter, an early proxy for identifying RGC subtypes, are generated at different times in the retina \cite{rapaport1995spatiotemporal,reese1994birthdates}.
Furthermore, timing is important for the appropriate differentiation of ipsilateral RGCs \cite{bhansali2014delayed}, and birthdate of RGC subtypes correlates to target matching strategies in the developing SC and dLGN \cite{osterhout2014birthdate}.
These recent studies provide support for earlier findings, which used axon diameter and cell morphology to show that functional subsets of RGC axons are segregated in the optic tracts of rodents \cite{reese1987distributionrat}, cats \cite{guillery1982arrangement,torrealba1982studies}, and monkeys \cite{reese1990fibre}.

The development of genetic tools and identification of RGC subtype-specific markers (e.g. \cite{blackshaw2004genomic,dhande2014retinal}) will afford even more precise analyses of the axon organization of molecularly defined RGC subtypes.
In the meantime, two subtypes of RGC subtypes can be defined based on their trajectory at the chiasm of binocular animals: those projecting ipsi- or contralaterally.
A moderate degree of order between ipsi- and contralateral axon cohorts, or “eye-specific” order, has been described in the optic tract of cats \cite{torrealba1982studies}.
In neonatal mice, contralateral axons are spread across the entire optic tract, with ipsilateral axons constrained to the anterior-lateral edge \cite{godement1984prenatal} (Figure 1C). %Edit all this
The earliest-born RGCs in the mouse extend axons from the dorsocentral retina and project either ipsi- or contralaterally.
Even though the early ipsilateral RGCs arising from this region are transient, their axons segregate in the optic tract from those of their contralaterally-projecting counterparts, underscoring the presence of eye-specific order in the tract \cite{soares2015transient}.

A major question arising from this body of work is whether pre-target axon organization is functionally important in the formation of the retinofugal pathway.
Though most of the studies in the optic nerve and tract to date are descriptive, they argue that pre-target axon organization is governed by a consistent logic and is likely an important step in the formation of the visual circuit.
For instance, in the cichlid fish optic tract, topographic order is maintained amongst retinal axon fascicles until very near the optic tectum, at which point the fascicles rearrange as they enter the appropriate tectal region \cite{scholes1979nerve}.
Scholes, like Reh et al. (1983) and Horton et al. (1979), infers that the consistent maintenance of axon order along the length of the tract is indicative of target-independent cues in the tract; that is, short-range fiber-fiber interactions may provide organization in the tract prior to reaching target-derived guidance cues \cite{scholes1979nerve}.
Furthermore, retinotopy in the frog optic nerve and tract is largely normal following surgical removal of the optic tectum early in development \cite{reh1983organization}.
Axon order in these tectum-less frogs provides even stronger evidence that fiber-fiber interactions and/or environmental cues in the tract organize axons en route to the target.

In summary, there are at least three different types of axon order in the developing optic tract: coarse topography, chronotopy (which may reflect or occur independently of functional specificity), and eye-specificity (which may be considered typographic order based on cohort identity) (Figure 1).
These different modes of order are not easy to fully synthesize together, as they do not always appear in register with one another.
However, early reports of a lack of fiber order in the optic tract \cite{horton1979non} can be reconciled with later findings of axon order by an overlap of different modes of axon order.
One mode (e.g., retinotopy) may be obscured by another (e.g., chronotopy), depending on the methodology used to identify axon order.
In other words, failure to find a certain type of axon order at points along the visual pathway does not necessarily demonstrate a lack of order altogether.
Future studies may identify further contradictions between modes of axon order in the optic tract or even new modes altogether, which may, for instance, involve genetically defined subsets of RGCs.
