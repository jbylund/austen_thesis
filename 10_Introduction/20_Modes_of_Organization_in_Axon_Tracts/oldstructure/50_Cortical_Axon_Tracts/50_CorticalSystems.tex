Central Axon Tracts: To, From, and Within the Cortex

While the sensory systems, especially the visual system and spinal circuitry, have dominated axon guidance study for the last century, the last few decades have seen increased interest in examining the development of central neural circuits. 
The thalamocortical projection in particular has proven a useful model for understanding complex axon guidance and axon organizational principles. 
This section will survey work describing the axon tracts entering into (thalamocortical), exiting from (corticothalamic), and coursing within (corpus callosum) the cortex. 
While there are other corticofugal tracts, the corticothalamic tract is the best studied with regards to axon trajectories and organization, and as such, the discussion in this section will reflect that focus. 

In both rodents and primates, there is a topographic correspondence between primary thalamic nuclei and target regions in the cortex \cite{caviness1980tangential,hohl1991topographical}. 
An early study of the thalamocortical tract used anterograde tract labeling in ex vivo slice preparations of mouse brain to examine the entire projection from the ventrobasal complex in the thalamus to the primary sensory cortex \cite{bernardo1987axonal}. 
Though the order of specific subsets of axons within the tract was not examined, the authors found that axons diverged from their neighbors and rejoined other fascicles at numerous points along their path. 
The patterns of axon rearrangement were consistent from animal to animal, indicating that while neighbor relationships between thalamocortical axons (TCAs) are not maintained, there is an underlying orderliness in the fiber bundling patterns along the thalamocortical tract. 
One particular aspect of the tract described in this paper is a 180° rotation of the entire bundle of axons in the internal capsule (IC), which accounts for a rotated topographic map between the thalamus and sensory cortex \cite{bernardo1987axonal}. 
The 180° rotation in the mouse optic tract, discussed above, could similarly account for the transposition of the topographic map between retina and SC \cite{plas2005pretarget}. 

Subsequent studies sought to describe the axon organization along the thalamocortical pathway in more detail and discover the underlying mechanisms of tract order and target specificity. 
Explants of thalamic tissue innervate cortical slices promiscuously in vitro, regardless of whether the cortical slice is from the appropriate region for the thalamic explant to target \cite{molnar1991lack}, suggesting that chemotropic cues in the cortex are insufficient to organize TCAs. 
Later studies from the same group confirmed the suspicion that TCAs are topographically ordered as they course through the IC and approach the pallial/subpallial boundary (PSPB), before entering the cortex \cite{molnar1998mechanisms}. 
Several cues in this subpallial region have been identified that orient subsets of TCAs, arranging them topographically inside the corridor \cite{dufour2003area,powell2008topography,seibt2003neurogenin2}. 

In addition to the maintenance of topographic order of TCAs in the thalamocortical tract, there are distinct spatial and temporal relationships between the reciprocal projections of TCAs and corticothalamic axons (CTAs) along this pathway. 
Anterograde labeling of small populations of cortical and thalamic projection neurons shows axons from each population running alongside each other in the same fascicles, rather than two separate tracts of TCA and CTA bundles \cite{molnar1998mechanisms}. 
Thus, ascending and descending axons in this tract are both topographically ordered and course in close association with each other. 
Additionally, the meeting of CTAs and TCAs at the PSPB is temporally regulated, as CTAs extend into the PSPB first and then wait for the arrival of TCAs (reviewed in \cite{leyva2013and}). 
This interaction is required for the proper guidance of both cohorts and was dubbed the “handshake hypothesis” (\cite{molnar1995thalamic}; discussed in more detail below). 
Similar to the temporal regulation of motor and sensory axons in the spinal cord, the chronological sequence of axon extension is a key element guiding the development of the reciprocal connections between cortex and thalamus. 

Reciprocal contralaterally projecting corticocortical axons make up another major white matter tract in the cortex, the corpus callosum (CC). 
Structural studies identified an anterior-posterior distribution of fibers within the human CC based on axon diameter \cite{aboitiz1992fiber} and broadly topographic correspondence between a neuron’s regional location in the cortex and its axon position in the CC of mice \cite{ozaki1992prenatal}, cats \cite{nakamura1989topography}, and humans \cite{de1985topography}. 
Diffusion tensor imaging in human subjects has confirmed the topographic correspondence of cortical brain regions across the anterior-posterior axis of the CC in greater detail \cite{hofer2006topography}. 
Histological analyses in both macaques and human subjects confirm diffusion tensor imaging findings of anterior-posterior order of axon bundles corresponding to different cortical regions \cite{caminiti2013diameter}.
Evidence from anterograde and retrograde labeling studies in the cat visual cortex shows further that sub-region topographic maps are contained within the gross topography along the rostral-caudal axis of the CC \cite{payne1991visual}. 

The development of genetic tools and advanced labeling techniques has provided higher resolution views of axon order in the CC. 
Axons of Layer II/III sensory and motor cortical neurons in the mouse are topographically ordered as they traverse the CC to innervate the contralateral cortex \cite{zhou2013axon}. 
Zhou et al. examined whether the topographic relationships between the neurons were preserved in their axons and, further, whether such order might affect their homotopic targeting in the contralateral cortex. 
The authors found a clear correspondence between neuron position in one hemisphere, axon position in the CC, and targeting position in the contralateral hemisphere: specifically, the medial-lateral axis is mirrored between the two hemispheres and maps onto the dorsal-ventral axis of the CC. 
This is true of axons arising from separate sensory and motor cortical areas as well as those axons arising from discrete medial-lateral positions within each cortical area \cite{zhou2013axon}. 

The multiple tracts coursing into, away from, and within the cortex demonstrate the principles of axon organization that have been identified in other tracts discussed so far. 
Similar to tracts in the spinal, visual, and auditory pathways, topography is evident in all of the cortical tracts described. 
Like sensory and motor axons in the spinal cord, CTAs and TCAs in the thalamocortical tract engage in heterotypic interactions that are integral to the formation of their reciprocal connections, while maintaining segregation of axon cohorts according to their origin. 
The reciprocal CTA and TCA projections are also temporally regulated in a similar manner to sensory and motor axons in the spinal cord. 
Because sensory modalities are organized into topographically defined subregions of the cortex, typographic order overlays topographic arrangements in this system. 

Despite the differences evident across the systems discussed in this section, three major themes of axon organization are evident: topography, chronotopy or temporal regulation, and subtype organization or typography (schematized in the visual system in Figure 1). 
Additionally, in all but the auditory system, axons in tracts make numerous rearrangements along their paths, which, while disrupting neighbor-neighbor relationships, appear orderly and consistent. 
Pre-target axon order is likely governed by many common mechanisms across systems in order to achieve the combinatorial axon order of the three general organizational modes discussed in this first section. 