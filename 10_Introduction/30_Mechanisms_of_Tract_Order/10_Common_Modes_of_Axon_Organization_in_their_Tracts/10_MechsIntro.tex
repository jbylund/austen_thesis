Molecules & Mechanisms Guiding Pre-Target Axon Organization

Despite the gaps remaining in our knowledge of the modes and degree of axon pre-sorting in all white matter tracts, mechanisms guiding such order have begun to be identified. 
This section explores known and hypothesized mechanisms of pre-target axon organization in developing circuits, divided into two broad categories: first, cues extrinsic to the axons, e.g., glia and choice points or intermediate targets, and second, axon-intrinsic (i.e., axon-axon) mechanisms, such as fasciculation, repulsion, and other axon-axon interactions (summarized in Figure 2).

Curiously, the majority of progress made in unraveling tract-specific mechanisms of axon organization has occurred in tracts that have been less well described than the retinofugal pathway. 
This is because while research in the visual system led the way in anatomical characterization of axon organization in developing tracts, it has lagged behind other systems in terms of studying the molecular mechanisms guiding that organization. 
The exception to this rule is the midline choice point, the optic chiasm, which has been well characterized. As such, we begin our discussion in this section with the role of intermediate targets and choice points in organizing axon tracts, starting with the optic chiasm. 
The remainder of the organizational mechanisms discussed throughout this section will come from studies conducted in other systems. 
