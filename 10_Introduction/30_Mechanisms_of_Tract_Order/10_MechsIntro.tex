\label{sec:MoleculesMechanisms}
The known and hypothesized mechanisms guiding pre-target axon organization in developing axon tracts can be broadly divided into axon-extrinsic and axon-intrinsic mechanisms.
These two broad categorizations can be further subdivided as follows.
The first set of axon-extrinsic mechanisms are guidepost cells, intermediate targets, and/or choice points along axon tracts.
Midline choice points, particularly the optic chiasm, have been well studied over the last few decades, and we have learned a great deal about growth cone dynamics and behavior as they navigate a variety of choice points.
The second set of axon-extrinsic mechanism are cues arising from the extracellular matrix and glia positioned in and around developing tracts.

Fasciculation is the most classic axon-intrinsic mechanism, although it can also be mediated by extrinsic cues creating surround-repulsion.
Despite this caveat, I will consider both heterotypic and homotypic fasciculation and axon-axon interactions in the first subdivision of axon-intrinsic organizational mechanisms.
Finally, neural activity is known to play a role in axon guidance, and there is evidence that it can also affect axon organization in tracts.
These axon-extrinsic and intrinsic mechanisms that I will discuss in this section are summarized in Figure~\ref{AxonOrg_Mechs}.
\begin{figure}[hbtp]
    \begin{center}
    \includegraphics{Figures/AxonOrg_Mechs.pdf}
    \caption[Common mechanisms of axon organization in tracts.]
    {Common mechanisms of axon organization in tracts.
    Combinations of axon-extrinsic and axon-intrinsic mechanisms establish axon organization in tracts.
    Axon-extrinsic mechanisms include (A) molecular gradients in a tract or within a choice point or intermediate target region, and/or (B) glia and extracellular matrix cues in and around the tract.
    Axon-intrinsic mechanisms include (C) homotypic, (D) grossly heterotypic, and (E) finely heterotypic fasciculation and axon-axon interactions.}
    \label{AxonOrgMechs}
    \end{center}
\end{figure}
