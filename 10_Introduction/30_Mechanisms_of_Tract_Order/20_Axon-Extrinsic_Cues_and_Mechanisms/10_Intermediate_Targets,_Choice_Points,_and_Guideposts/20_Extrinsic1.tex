A common feature of developing axon tracts is the presence of choice points and/or intermediate target zones that axons traverse as they navigate towards their final targets.
Choice points and intermediate targets provide crucial cues as subsets of axons need to make appropriate decisions to, for instance, cross or not cross the midline, or branch off from the main tract.
The structure and cues provided by many such choice points have been described in several systems, especially at the CNS midline.

Midline radial glia in the ventral diencephalon and a raft of early-born neurons caudal to the midline glial palisade provide guidance cues at the optic chiasm (reviewed in \citenoparens{erskine2014connecting,petros2008retinal}) that direct the crossing of contralateral RGC axons \cite{charron2003morphogen,erskine2011vegf,kuwajima2012optic,williams2006role} and the repulsion of ipsilateral RGC axons \cite{williams2003ephrin,petros2010ephrin,petros2009specificity}. 
Similarly, a transient neuronal population at the cortical midline called the subcallosal sling is important for guiding axons across the callosal midline during CC development \cite{suarez2014evolution}. 
Subcallosal sling neurons express Sema3C, which attracts Nrp1-expressing pioneer cingulate cortical axons \cite{niquille2009transient,piper2009neuropilin}. 
In addition to their guidance role, these two molecules are also important for establishing the dorsal-ventral organization of callosal axons within the CC \cite{zhou2013axon}. 

In another midline system, the developing ventral spinal cord, a host of molecular mechanisms guide axons to, through, and out of the floor plate, which contains radial neuroepithelial cells (reviewed in \citenoparens{neuhaus2015crossing}).
In the chick floorplate, crossing the midline leads to a change in surface cell adhesion molecule (CAM) expression on commissural and motor neuron axons, from TAG-1 to L1 \cite{dodd1988spatial}.
This change in surface protein expression on axons and their growth cones is thought to be a common mechanism guiding midline-crossing axons through and beyond the choice point into the continuation of their tracts.
Midline-induced changes in surface CAM expression could lead to specific bundling and organizational behaviors of post-midline axons in their developing tracts.
This has yet to be directly tested in the visual system, but a change in surface CAM expression on ipsilateral and/or contralateral RGC axons could affect interactions between the two axon cohorts or between one group of axons and the tract environment. 

In systems where axon tracts navigate across a midline choice point, it is challenging to disentangle the molecular mechanisms at the choice point from later organizational mechanisms at play within the tract or target.
In other words, it is difficult to identify whether post-midline organizational events are consequences of events occurring at the choice point (e.g., changes in CAM expression on axons), or are independently tract-specific.
This question is experimentally challenging, given the difficulty in isolating molecular events in the tract from those at the midline or at the target, especially in systems where the same guidance molecules are present at multiple steps in the pathway. 

Work in the thalamocortical system demonstrates how a non-midline intermediate target region can influence axon order in a tract. 
As described earlier, TCAs are topographically ordered as they pass through the developing striatum and IC \cite{garel2014inputs}.
A population of migrating guidepost cells creates a permissive corridor for extending TCAs in an otherwise growth-prohibitive environment \cite{bielle2011emergent,lopez2006tangential}. 
Perturbing the development of the corridor region leads to disordered TCA topography, and subsequent failure of a subset of TCAs to reach the cortex \cite{garel2002early}. 
The evidence that topography of incoming TCAs relies on cues in the corridor region spurred a series of studies that has defined many cues present along the thalamocortical pathway in the subpallium. 

Graded expression of Slit1 by corridor cells contributes to the organization of intermediate and rostral TCAs in the subpallium \cite{bielle2011emergent}. 
In addition to Slit1-expressing corridor cells, there are several gradients of guidance cues across the rostral-caudal axis of the striatum at the time of TCA extension through the region. 
Subsets of thalamic projection neurons express specific combinations of guidance receptors and CAMs that correspondingly interact with gradients of guidance cues in the subpallium, leading to topographic organization of TCAs (reviewed in \citenoparens{garel2014inputs,molnar2012mechanisms}). 
Differential expression of EphA3, 4, and 7 on TCAs mediates their interaction with a rostral-low to caudal-high ephrin-A5 gradient in the corridor and subpallium \cite{dufour2003area}, and TCAs expressing EphB1 and B2 receptors are repelled from the ventrolateral telencephalon by ephrin-B1 \cite{robichaux2014ephb}.
Responsiveness to the rostral-high caudal-low netrin-1 gradient in the developing striatum is mediated by differential TCA expression of DCC and Unc5 \cite{powell2008topography}, Robo1, and its coreceptor FLRT3 \cite{leyva2014flrt3}.
Finally, neurogenin2 guides TCA responsiveness to subpallium cues \cite{seibt2003neurogenin2}; and NrCAM mediates responsiveness to Sema3F in the subpallium \cite{demyanenko2011nrcam}. 

\label{sec:CTA_waiting}
The CTA waiting period in the subpallium, a necessary step in the ``handshake'' with TCAs, required for mutual guidance, is regulated by transient expression of PlexinD1 on pioneer CTAs. 
PlexinD1 binds to Sema3E on radial glia near the PSPB in the intermediate target region of the pathway, mediating a growth-inhibition response \cite{deck2013pathfinding}. 
This waiting period is essential for the correct projection of CTAs because it temporally positions them to engage in necessary interactions with reciprocal TCAs. 
A waiting period has also been described for TCAs, which cluster and pause below the cortical plate before entering the cortex and navigating to the appropriate cortical regions \cite{ghosh1992pathfinding,leyva2013and,lund1977development}.