\label{sec:GliaECM}
Glia and extracellular matrix (ECM) elements have been described in and surrounding many axon tracts, and evidence for their role in axon organization within tracts is growing. 
At midline choice points, glia provide cues for decussation and other axon segregation events, and they, along with transient neuronal populations, also create physical and chemorepellent barriers that are essential for the overall formation of the midline tracts \cite{raper2010cellular,suarez2014evolution}. 
For instance, as discussed above, radial glia at the optic chiasm direct midline decussation of RGCs while a transient neuronal population provides a barrier behind the chiasm \cite{petros2008retinal,raper2010cellular}.
%It is unclear whether, and if so how early-born midline neurons contribute to axon order at the chiasm.

Likewise, glia are essential in the development of the CC, where a glial wedge provides a physical and chemorepellent boundary at the ventromedial boundary, indusium griseum glia form a dorsomedial boundary, and midline zipper glia provide a ventromedial boundary (reviewed in \citenoparens{suarez2014evolution}).
Although glial structures and the transient neuronal populations in the subcallosal sling of the CC are important in the formation of the midline tracts and the guidance of axons across the midline \cite{gobius2016astroglial}, it is not known whether these glial populations contribute to axon organization within the CC.

Outside of midline choice points, glia and ECM elements support early axon outgrowth. 
They specify boundaries by surrounding developing tracts \cite{marcus1995expression} and provide growth-permissive substrates essential for pioneer axon extension (reviewed in \citenoparens{raper2010cellular}). 
For example, expression of growth cone-collapsing glycosaminoglycans in the chick hindlimb delineates growth-permissive corridors for axons extending into the limb \cite{tosney1985development}. 

Molecules expressed by glia or in the ECM may be integral for organizing axons within tracts. 
In the ferret visual system, for instance, radial glia-associated chondroitin sulfate proteoglycans (CSPGs) are distributed asymmetrically in the optic tract.
CSPGs are more densely localized to the deep part of the tract, opposite the superficial aspect where the youngest RGC axons chronotopically extend \cite{reese1997chronotopic}.
Enzymatic removal of CSPGs abolishes chronotopy, as young axons, no longer inhibited by the CSPG border, run throughout the entire width of the tract \cite{leung2003enzymatic}.
Heparan sulfate proteoglycans (HSPGs) also contribute extra-axonal cues to organize axons in their tracts. 
In the zebrafish retinotectal system, HSPGs act non-cell-autonomously on RGC axons to correct naturally occurring pathfinding errors prior to axon entry into the tectum \cite{poulain2013proteoglycan}. 

Other evidence supporting a role for glia in organizing axons within tracts remains largely circumstantial. 
For example, in ferrets, interfascicular glia predominate in the optic nerve, radial glia in the optic chiasm, and in the post-chiasm optic tract, glia cluster along the pia \cite{guillery1987changing,colello1992observations}.
The changing glial profiles are presumed to be involved in the different modes of axon organization at each step of the pathway. 
Guillery and Walsh \shortcite{guillery1987changing} suggest that age-related order in the mouse optic tract arises from radial glia at the chiasm guiding growth cones to the pial surface, whereas the relative lack of age-related axon order in the optic nerve may be due to the preponderance of interfascicular glia found there. 
Similarly, in fish, the optic nerve/optic tract boundary is demarcated by a sharp change in glia morphology and molecular profile: glia in the nerve express a vimentin-like protein and those in the tract express glial fibrillary acidic protein (GFAP).
This change in glial profile spatially coincides with a marked re-sorting of retinal axons into their final topographic order, again providing circumstantial evidence that the change in glial environment is involved these axon rearrangements \cite{maggs1986glial}.
Thus, though it has yet to be directly demonstrated, glia are well positioned to be providing organizational cues to axons within their tracts.

Additionally, in the antennal lobe of the moth \emph{Manduca sexta}, glia play a role in organizing OSN axons.
Direct effects of sorting zone glia on OSN axons have been demonstrated \invitro{} \cite{tucker2004vitro}, and electron microscopy analysis shows a close association between OSN axon growth cones and glia in the glial-rich sorting region where axons rearrange prior to glomerulus entry \cite{oland1998targeted}.
Subsequent experiments demonstrated the necessity of glia in this sorting region, as OSN axons in glia-deficient moths fail to correctly fasciculate or target glomeruli appropriately \cite{rossler1999development}. 

In the vertebrate olfactory system, olfactory ensheathing glia (OEG) are positioned favorably to play a similar organizational role as glia do in the insect antennal lobe (reviewed in \citenoparens{tolbert2004bidirectional}).
OEG surround OSN axons and extend cytoplasmic processes in between axon fascicles, heterogeneously expressing several guidance molecules, including ephrin-B2, NCAM, GAP-43, laminin, HSPGs, Sema3A, and a host of growth factors (reviewed in \citenoparens{chuah2002cellular}).
Furthermore, OEG precede OSN axon fascicles growing into the OB, and could thus provide guidance or organizational cues to leading OSN growth cones \cite{chuah2002cellular,tolbert2004bidirectional}.

Loss of function studies in Sema3A mutant mice specifically implicate OEG in pre-target organization of olfactory axons.
Sema3A is expressed by OEG in the developing mouse olfactory system, and Nrp1-expressing OSN axons misroute and mistarget in Sema3A knockout mice \cite{schwarting2000semaphorin}.
Axon-axon interactions are also implicated in this scenario, as Nrp1 negative axons, i.e., those that do not respond to Sema3A, also suffer misrouting defects in the OB in Sema3A mutants \cite{schwarting2000semaphorin}.

Recently, microglia have been implicated in tract formation and organization, raising the possibility of an entirely new and non-canonical function for these immune cells.
Perturbing microglia activity leads to marked defasciculation of callosal axons and subsequent disturbance in the dorsal-ventral order in the CC \cite{pont2014microglia}.
Strikingly, microglia in the embryonic mouse brain at E14.5 cluster at decision points of axon tracts, displaying a distinctly different distribution than at E12.5 or postnatally, when microglia are more evenly spread throughout the brain \cite{squarzoni2014microglia}.
Furthermore, eliminating or immune-inactivating microglia in the embryonic brain produces wiring defects in dopaminergic axon outgrowth and positioning of neocortical interneurons \cite{squarzoni2014microglia}.
These two studies provide compelling evidence for a novel role of microglia in tract formation and axon organization.

The evidence argues that ECM and glia could provide far more organizational cues to axons in their tracts than have so far been identified.
In many tracts, the expression profile of glia within and surrounding the tract is completely unknown, and similarly little is known about the ECM of many tracts.
Additionally, we have yet to fully separate effects of guidance molecules within choice points/intermediate targets and within tracts themselves.
More knowledge of the molecular expression profiles of cells in choice points and within tracts will help us probe how those cells affect axon organization at different stages along their navigation to their targets.