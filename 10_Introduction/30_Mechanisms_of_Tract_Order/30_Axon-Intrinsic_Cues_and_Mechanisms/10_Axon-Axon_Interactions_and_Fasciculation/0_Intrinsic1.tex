Axon-intrinsic mechanisms contributing to pre-target organization of axons in their tracts include fasciculation, or axon-axon interactions, and neural activity.
While it has largely been studied in the context of synaptic refinement, neural activity also contributes to axon guidance and, at least in the few cases studied, to organization of axons in their tracts.
I will address these findings in Section~\ref{sec:Intrinsic2}.
First, I will explore the different types of fasciculation involved in axon organization in the next two sub-sections.

Despite the fact that fasciculation is considered a fundamental mechanism of neural circuit development (reviewed in \citenoparens{raper2010cellular,wang2013axons}), our understanding of the mechanisms involved in axon-axon interactions, and especially how they contribute to sorting of axons in tracts, remains limited.
In a review of axon-axon interactions, Wang and Marquardt \shortcite{wang2013axons} offer an insightful distinction between homotypic and heterotypic axon fasciculation, which I will adopt for this overview of axon-axon interactions involved in pre-target axon organization. 
Homotypic fasciculation of like axons, i.e., those of the same source or subset, can be mediated by adhesive forces joining axons together and/or surround repulsion corralling homotypic axons together (Section~\ref{sec:HomotypicFascic}).
Heterotypic fasciculation can occur between two different axon cohorts arising from separate sources (grossly heterotypic), or between cohorts of axons arising from subsets of neurons in the same source (finely heterotypic).
Both of these scenarios are discussed in Section~\ref{sec:HeterotypicFascic}.