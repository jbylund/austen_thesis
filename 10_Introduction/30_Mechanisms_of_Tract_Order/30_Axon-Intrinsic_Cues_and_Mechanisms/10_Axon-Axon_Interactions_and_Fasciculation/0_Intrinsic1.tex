Axon-intrinsic mechanisms contributing to pre-target organization of axons in their tracts include fasciculation, or axon-axon interactions, and neural activity.
While it has largely been studied in the context of synaptic refinement, neural activity also contributes to axon guidance and, at least in the few cases studied, to organization of axons in their tracts.
I will address these findings in Section~\ref{sec:Intrinsic2}.
First, I will explore the different types of fasciculation involved in axon organization in the next two sub-sections.

Despite the fact that fasciculation is considered a fundamental mechanism of neural circuit development (reviewed in \citenoparens{raper2010cellular,wang2013axons}), our understanding of the mechanisms involved in axon-axon interactions, and especially how they contribute to sorting of axons in tracts, remains limited.
In a review of axon-axon interactions, \citetext{wang2013axons} offer an insightful distinction between homotypic and heterotypic axon fasciculation, which I will adopt for this overview of axon-axon interactions involved in pre-target axon organization. 
Homotypic fasciculation of like axons, i.e., those of the same source or subset, can be mediated by adhesive forces joining axons together and/or surround repulsion corralling homotypic axons together.
Heterotypic fasciculation can occur between two different axon cohorts arising from separate sources (grossly heterotypic), or between cohorts of axons arising from subsets of neurons in the same source (finely heterotypic).

%\label{sec:HomotypicFascic}
\noindent\textbf{Homotypic Fasciculation}\newline
\indent Fasciculation of homotypic axons has been studied in the pioneer-follower model of early tract formation, the classic paradigm in which younger axons fasciculate along pioneer axons that trailblazed the initial pathway (e.g., \citenoparens{raper1983pathfinding,tosney1985development}, and reviewed in \citenoparens{raper2010cellular,wang2013axons}).
Pioneer-follower fasciculation and other instances of homotypic fasciculation can be mediated by adhesive interactions between axons \cite{van1998adhesion}.
Indeed, early studies of neurite-neurite interactions made important progress in identifying CAMs involved in fasciculation and axon guidance (e.g., \citenoparens{rutishauser1978adhesion,rathjen1987membrane}).
Additionally, several early in vitro experiments demonstrated that axons need a substrate on which to extend \cite{hammarback1988growth}, and that guidance decisions are made at the growth cone, which interpret attractive and repulsive signals \cite{raper1990temporal,fan1995localized}.

Homotypic fasciculation of follower axons along pioneers is an important step in neural circuit formation.
Loss of pioneer axons often disturbs tract formation, sometimes even resulting in complete failure of the tract to form (reviewed in \citenoparens{raper2010cellular}).
An elegant set of experiments in the developing zebrafish retinotectal system provides a case in which pioneer RGC axons are both necessary and sufficient for leading later-born RGC axons along the correct path to the optic tectum \cite{pittman2008pathfinding}.
The results of this study also imply that pioneer-follower RGC axon-axon interactions trump extra-axonal cues along the tract.
Transplanting mutant pioneer RGCs from zebrafish lacking the Slit receptor Robo2 into wild-type retinae caused wild-type axons to make routing errors alongside transplanted mutant pioneers.
Conversely, transplanting wild-type pioneer RGCs into mutant retinae rescued many host axons from making pathfinding errors \cite{pittman2008pathfinding}.
Slit and Robo are also involved in motor axon fasciculation in the mouse, where Slit2 directly promotes fasciculation via interactions with Robo1 and Robo2 by an autocrine/juxtaparacrine mechanism \cite{jaworski2012autocrine}.

However, in many experiments that disrupt the generation of pioneer axons, other axons are capable of taking on the pioneering role and navigating, sometimes with errors or delays, to the target (reviewed in \citenoparens{raper2010cellular}).
Thus, it is not necessarily that pioneer axons are endowed with a special ability or identity, but rather that following axons are capable of utilizing the first-extending axons as a scaffold, highlighting the importance of direct axon-axon interactions in generating organized axon tracts.
Indeed, axons prefer to grow along other axons both in vivo \cite{raper1983pathfinding,tosney1985development} and in vitro \cite{bonhoeffer1985position}.

Surround repulsion can also lead to homotypic fasciculation by corralling like axons together away from external repulsive signals.
%Despite the fact that such surround-repulsion scenarios result from extra-axonal cues, I consider them in this section because homotypic fasciculation is the end result and the response to external repulsive cues is contingent on selective expression of appropriate receptors by the axons.
For instance, motor axons expressing Nrp1 fasciculate via surround repulsion from Sema3A in the mouse forelimb.
In the absence of Sema3A, motor axons defasciculate and subsequently make errors in dorsal-ventral pathfinding in the limb plexus region \cite{huber2005distinct}.
On the other hand, Sema3F-Nrp2 interactions do not affect fasciculation of motor axons in the limb \cite{huber2005distinct}.
Sema-Nrp interactions do, however, drive fasciculation of vomeronasal axons in the murine accessory olfactory system, although they are not involved in pathfinding in this system \cite{cloutier2002neuropilin}.

The Eph/ephrin family of cell surface proteins is also involved in mediating selective homotypic fasciculation of axon cohorts via selective surround repulsion.
EphB2-expressing sensory and motor axons fasciculate via surround repulsion from ephrin-B1 produced in the limb bud mesenchyme \cite{luxey2013eph}.
Ephrin-B1 expression on sensory axons indicates a more complicated interaction may be involved in addition to the surround-repulsion from the mesenchyme \cite{luxey2013eph}.
Loss of ephrin-B1 in mice leads to defasciculation of axon tracts, especially the oculomotor nerve \cite{davy2004ephrin}; and in double knockout mutants lacking both EphB2 and EphB3, axons extending from the habenular nuclei extensively defasciculate \cite{orioli1996sek4}.
Fasciculation is normal, however, in each single knockout mutant \cite{orioli1996sek4}.
Whether appropriate maintenance of homotypic fasciculation is necessary for appropriate innervation and synaptogenesis in the target remains to be more fully elucidated, but in several cases, defasciculation can lead to aberrant pathfinding and targeting.

It is important to note that surround repulsion mechanisms of tract fasciculation are not unique to homotypic fasciculation scenarios.
Indeed, surround repulsion is critical for the formation of tracts, defining boundaries and corridors by hemming heterotypic populations of axons into a defined region.
For instance, the optic chiasm in Slit1/2 double knockout mice is defasciculated and a secondary chiasm structure forms anterior to the normal chiasm \cite{plump2002slit1}.
Slit1 and 2 normally define zones surrounding the optic chiasm that are inhibitory to RGC axon growth, and in their absence, RGC axons are no longer penned into the appropriate area as they navigate the midline \cite{plump2002slit1}.
Other examples of this type of surround repulsion that affects heterotypic groups of axons were discussed previously in the context organizational cues from glia or ECM (Section~\ref{sec:GliaECM}).
As such, the following section discussing heterotypic fasciculation will not discuss surround repulsion much further.

%\label{sec:HeterotypicFascic}
\noindent\textbf{Heterotypic Fasciculation}\newline
\indent Heterotypic fasciculation occurs between axon cohorts that are grossly or finely heterotypic.
This section will first consider grossly heterotypic axon-axon interactions, where two cohorts from different sources run segregated from each other in a tract containing reciprocal projections.
Prime examples of these sorts of reciprocal projections include TCAs and CTAs in the thalamocortical tract, and sensory and motor spinal axons in the periphery.
Finely heterotypic axon-axon interactions are those occurring between subsets of axons arising from the same source but representing different molecularly defined neuronal subtypes; these cases will be considered at the end of this section.
Common guidance molecules participate in both types of heterotypic axon-axon interactions, including the Eph:ephrin family, Semas, Plexins, and Neuropilins.

In the peripheral nerve tracts, motor axons extend before sensory axons, which in turn rely on motor axons to form correct patterns of connections \cite{wang2013axons}.
While these two heterotypic axon cohorts run closely together, sensorimotor function requires appropriate segregation of sensory and motor axons.
In the axial nerves, this segregation depends on repulsive trans-axonal signaling between motor axon EphA3/EphA4 receptors and sensory axon ephrin-A ligands.
Perturbing ephrinA-EphA forward signaling results in severe wiring defects in the peripheral nerves \cite{gallarda2008segregation}.
EphA3/4 and ephrin-As are also required for sensory axons to track along preformed motor axon pathways in the limb \cite{wang2011anatomical}.
This dual action of EphAs/ephrin-As relies on the forward and reverse signaling capabilities of the binding partners; namely, segregation of motor and sensory axons results from forward signaling and sensory axon tracking along motor axons utilizes reverse signaling \cite{wang2011anatomical}.
The reliance on motor axons for sensory axon pathfinding was further demonstrated by assessing the sensory axon routes taken in two mouse strains that display motor axon pathfinding errors (EphA4 and Ret mutants).
In both cases, as well as in analogous zebrafish mutants, sensory axons mimicked the pathfinding errors made by motor axons \cite{wang2014conserved}.

The reciprocal projections comprising the thalamocortical pathway are another prime example of grossly heterotypic axon-axon interactions in developing tracts.
Recent work demonstrates a mutual reliance of TCAs and CTAs on each other for successful tract organization and navigation to their respective targets (e.g., \citenoparens{chen2012evidence,deck2013pathfinding}), and some of the molecules involved in these axon-axon interactions have been identified.
The CTA waiting period, discussed previously (see Section~\ref{sec:CTAwaiting}), is governed by Sema3E/Plexin-D1 interactions.
If Sema3E/Plexin-D1 signaling is perturbed, all corticofugal axons pass the pallial/subpallial boundary (PSPB) prematurely, bypassing their waiting period and therefor also bypassing their encounter with TCAs.
CTAs subsequently take a cortico-subcerebral trajectory, rather than dividing into the appropriate corticothalamic and cortico-subcerebral paths \cite{deck2013pathfinding}.
This finding highlights the necessity of the subpallial ``handshake'' between thalamic and cortical axons.

In the converse scenario, when CTA outgrowth is prevented, TCAs are rendered incapable of navigating across the PSPB and into the cortex \cite{chen2012evidence}.
This mutual reliance is suggestive of heterotypic axon-axon interactions or fasciculation occurring in the PSPB and along the tract.
Curiously, TCAs and CTAs collapse upon contact with each other in vitro, and thus are not intrinsically inclined to co-fasciculate \cite{bagnard2001axonal}.
Thus, the ``handshake'' between TCAs and CTAs is likely mediated by axon-extrinsic cues in the surrounding ventral telencephalon, which may in turn affect direct axon-axon interactions between the two axon populations.
The intimate association between thalamic and cortical axons in the internal capsule (IC) and along the tract \cite{molnar1998mechanisms} suggests cofasciculation between these heterotypic axon cohorts.
Likewise, their codependency on each other for guidance, and the fact that a subset of CTAs track along misrouted TCAs in EphB1/2 double knockout mice \cite{robichaux2014ephb}, provides further support to a model in which direct axon-axon interactions are involved in organization and guidance of these axonal cohorts.

The Eph/ephrin family may be involved in such an interaction.
In the primary olfactory nerve of the moth \emph{Manduca sexta}, Eph receptors and ephrins are differentially expressed among glomeruli and are candidates for mediating OSN axon sorting prior to glomerulus innervation \cite{kaneko2003interaxonal}.
Importantly, there is no detectable Eph or ephrin expression intrinsic to the OSN axons' target, the antennal lobe, signifying that Eph-ephrin repulsive signaling involved in sorting axons would necessarily occur via direct axon-axon interactions and not via axon-target signaling \cite{kaneko2003interaxonal}.

Eph/ephrin mediated sorting of moth OSN axons involves a different kind of heterotypy than has been considered thus far.
As opposed to the grossly heterotypic interactions of reciprocal projections in the spinal cord or thalamocortical tract, finely heterotypic axon-axon interactions mediate the segregation of axons originating from the same source into appropriate sub-fascicles representing different neuronal subtypes or different sub-regions of a target.
Sorting of axons into sub-fascicles in a tract is thought to occur by selective inhibition of fasciculation, such that individual cohorts can separate from a larger group and select the correct trajectory to the appropriate target.
An example comes from the chick spinal cord marginal zone, where intermediate and medial longitudinal commissural (ILc and MLc) axons run parallel in the lateral and ventral funiculi, respectively \cite{sakai2012axon}.
ILc axons express Robo, which is thought to inhibit N-cadherin-driven fasciculation in order to separate ILc axons away from the MLc fascicle.
Accordingly, loss of Robo function leads to ILc axons aberrantly coursing alongside MLc axons in the ventral funiculus and subsequently mistargeting in the brain \cite{sakai2012axon}.

Semas and their Nrp and Plexin receptors are also involved in sorting finely heterotypic axons within tracts.
In the mouse OE, different populations of OSN axons express Sema3A and Nrp1 in a roughly regionally defined manner \cite{imai2009pre}.
These OSN axons are likewise segregated from each other in the olfactory nerve.
Selectively deleting either Nrp1 or Sema3A in OSNs leads to a blurring of pre-target axon order and shifts in glomerular position in the OB \cite{imai2009pre}.
Because these effects are found in conditional mutants that only affect subsets of OSNs, they occur independently of Sema3A surround expression by OEG.
Given this result, axon-axon interactions are critical for establishing pre-target axon organization in the olfactory nerve \cite{imai2009pre}.

Sema3A and Nrp1 are also implicated in establishing the dorsal-ventral segregation of axons in the CC \cite{zhou2013axon}.
This order reflects a medial-high to lateral-low gradient of Nrp1 expression in the cortex and a reciprocal, though broader, expression of Sema3A, which is slightly higher in the lateral cortex.
Disrupting these molecular gradients perturbs both axon position in the CC and the position of axon terminations in the contralateral cortex.
The authors propose that both axon-extrinsic and axon-axon interactions are at play in the CC \cite{zhou2013axon}.
It is intriguing to speculate that Sema/Nrp interactions may be a conserved mechanism in mediating axon sorting across other systems, as they are a recurring molecular theme in studies of pre-target axon sorting in multiple systems.