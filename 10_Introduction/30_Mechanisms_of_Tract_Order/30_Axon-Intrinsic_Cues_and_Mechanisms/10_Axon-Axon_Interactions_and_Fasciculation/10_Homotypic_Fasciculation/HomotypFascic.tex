\label{sec:HomotypicFascic}
Fasciculation of homotypic axons has been studied most commonly in the pioneer-follower model of early tract formation.
In this classic paradigm, pioneer axons provide a path along which younger axons follow.
Loss of pioneer axons often disturbs tract formation, sometimes even resulting in complete failure of the tract to form (reviewed in \cite{raper2010cellular}).
An elegant set of experiments in the developing zebrafish retinotectal system provides a case in which pioneer RGC axons are both necessary and sufficient for leading later-born RGC axons along the correct path to the optic tectum \cite{pittman2008pathfinding}.
The results of this study also imply that pioneer-follower RGC axon-axon interactions trump extra-axonal cues along the tract.
Transplanting mutant pioneer RGCs from zebrafish lacking the Slit receptor Robo2 into wild type (wt) retinae caused wt axons to make routing errors alongside transplanted mutant pioneers.
Conversely, transplanting wt pioneer RGCs into mutant retinae rescued many host axons from making pathfinding errors \cite{pittman2008pathfinding}.
However, in many experiments disrupting the generation of pioneer axons, other axons are capable of taking on the pioneering role and navigating, sometimes with errors or delays, successfully to the target (reviewed in \cite{raper2010cellular}).
Thus, it is not necessarily a special ability inherent to pioneer axons that is crucial, but rather the ability for following axons to utilize the first-extending axons as a scaffold, highlighting the importance of direct axon-axon interactions in generating organized vertebrate axon tracts.
Slit and Robo are also involved in motor axon fasciculation in the mouse, where Slit2 directly promotes fasciculation via interactions with Robo1 and Robo2 by an autocrine/juxtaparacrine mechanism \cite{jaworski2012autocrine}. 
%Add more detail?

Surround repulsion can also lead to homotypic fasciculation by corralling like axons together away from external repulsive signals.
Despite the fact that such surround-repulsion scenarios result from extra-axonal cues, I consider them in this section because homotypic fasciculation is the end result and the response to external repulsive cues is contingent on selective expression of appropriate receptors by the axons.
For instance, motor axons expressing Nrp1 fasciculate via surround repulsion from Sema3A in the mouse forelimb.
In the absence of Sema3A, motor axons defasciculate and subsequently make errors in dorsal-ventral pathfinding in the limb plexus region \cite{huber2005distinct}.
On the other hand, Sema3F-Nrp2 interactions do not affect fasciculation of motor axons in the limb \cite{huber2005distinct}.
They do, however, drive fasciculation of vomeronasal axons in the murine accessory olfactory system, although they are not involved in pathfinding in this system \cite{cloutier2002neuropilin}.

The Eph:ephrin family of cell surface proteins is also involved in mediating selective homotypic fasciculation of axon cohorts via selective surround repulsion.
EphB2-expressing sensory and motor axons fasciculate via surround repulsion from ephrin-B1 produced in the limb bud mesenchyme.
Ephrin-B1 expression on sensory axons indicates a more complicated interaction may be involved in addition to the surround-repulsion from the mesenchyme \cite{luxey2013eph}.
Loss of ephrin-B1 in mice leads to defasciculation of axon tracts, especially the oculomotor nerve \cite{davy2004ephrin}; and in double knockout mutants lacking both EphB2 and EphB3, axons extending from the habenular nuclei extensively defasciculate.
Fasciculation is normal, however, in each single knockout mutant \cite{orioli1996sek4}.
Whether appropriate maintenance of homotypic fasciculation is necessary for appropriate innervation and synaptogenesis in the target remains to be more fully elucidated, but in several cases, defascicultion can lead to aberrant pathfinding and targeting.