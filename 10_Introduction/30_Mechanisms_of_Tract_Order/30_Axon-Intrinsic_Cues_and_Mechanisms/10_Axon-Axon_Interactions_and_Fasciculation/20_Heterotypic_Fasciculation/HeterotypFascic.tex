\label{sec:HeterotypicFascic}
Heterotypic fasciculation occurs between axon cohorts that are either grossly or finely heterotypic.
This section will first consider grossly heterotypic axon-axon interactions, where two cohorts from different sources run segregated from each other in a tract containing reciprocal projections.
Prime examples of these sorts of reciprocal projections include TCAs and CTAs in the thalamocortical tract, and sensory and motor spinal axons in the periphery.
Finely heterotypic axon-axon interactions are those occurring between subsets of axons arising from the same source but representing different molecularly defined neuronal subtypes; these cases will be considered at the end of this section.
Common guidance molecules participate in both types of heterotypic axon-axon interactions, including the Eph:ephrin family, Semas, Plexins, and Neuropilins. 

In the peripheral nerve tracts, motor axons extend before sensory axons, which in turn rely on motor axons to form correct patterns of connections \cite{wang2013axons}.
While these two heterotypic axon cohorts run closely together, sensorimotor function requires appropriate segregation of sensory and motor axons.
In the axial nerves, this segregation depends on repulsive trans-axonal signaling between motor axon EphA3/EphA4 receptors and sensory axon ephrin-A ligands.
Perturbing ephrinA-EphA forward signaling results in severe wiring defects in the peripheral nerves \cite{gallarda2008segregation}.
EphA3/4 and ephrin-As are also required for sensory axons to track along preformed motor axon pathways in the limb \cite{wang2011anatomical}.
This dual action of EphAs/ephrin-As relies on the forward and reverse signaling capabilities of the binding partners; namely, segregation of motor and sensory axons results from forward signaling and sensory axon tracking along motor axons utilizes reverse signaling \cite{wang2011anatomical}.
The reliance on motor axons for sensory axon pathfinding was further demonstrated by assessing the sensory axon routes taken in two mouse strains that display motor axon pathfinding errors (EphA4 and Ret mutants).
In both cases, as well as in analogous zebrafish mutants, sensory axons mimicked the pathfinding errors made by motor axons \cite{wang2014conserved}.

The reciprocal projections comprising the thalamocortical pathway are another prime example of grossly heterotypic axon-axon interactions in developing tracts.
Recent work demonstrates a mutual reliance of TCAs and CTAs on each other for successful tract organization and navigation to their respective targets (e.g. \citenoparens{chen2012evidence,deck2013pathfinding}), and some of the molecules involved in these axon-axon interactions have been identified.
The CTA waiting period, as discussed previously, is governed by Sema3E/Plexin-D1 (see Section~\ref{sec:CTA_waiting}).
If Sema3E/Plexin-D1 signaling is perturbed, all corticofugal axons pass the pallial/subpallial boundary (PSPB) prematurely, bypassing their waiting period and therefor also bypassing their encounter with TCAs.
CTAs subsequently take a cortico-subcerebral trajectory, rather than dividing into the appropriate corticothalamic and cortico-subcerebral paths \cite{deck2013pathfinding}.
This finding highlights the necessity of the subpallial ``handshake'' between thalamic and cortical axons.

In the converse scenario, when CTA outgrowth is prevented, TCAs are rendered incapable of navigating across the PSPB and into the cortex \cite{chen2012evidence}.
This mutual reliance is suggestive of heterotypic axon-axon interactions or fasciculation occurring in the PSPB and along the tract.
Curiously, TCAs and CTAs collapse upon contact with each other in vitro, and thus are not intrinsically inclined to co-fasciculate \cite{bagnard2001axonal}.
Thus, the ``handshake'' between TCAs and CTAs is likely mediated by axon-extrinsic cues in the surrounding ventral telencephalon, which may in turn affect direct axon-axon interactions between the two axon populations.
The intimate association between thalamic and cortical axons in the internal capsule (IC) and along the tract \cite{molnar1998mechanisms} suggests cofasciculation between these heterotypic axon cohorts.
Likewise, their codependency on each other for guidance, and the fact that a subset of CTAs track along misrouted TCAs in EphB1/2 double knockout mice \cite{robichaux2014ephb}, provides further support to a model in which direct axon-axon interactions are involved in organization and guidance of these axonal cohorts.

The Eph:ephrin family may be involved in such an interaction.
In the primary olfactory nerve of the moth \emph{Manduca sexta}, Eph receptors and ephrins are differentially expressed among glomeruli and are candidates for mediating OSN axon sorting prior to glomerulus innervation \cite{kaneko2003interaxonal}.
Importantly, there is no detectable Eph or ephrin expression intrinsic to the OSN axons’ target, the antennal lobe, signifying that Eph-ephrin repulsive signaling involved in sorting axons would necessarily occur via direct axon-axon interactions and not via axon-target signaling \cite{kaneko2003interaxonal}.

Eph-ephrin mediated sorting of moth OSN axons involves a different kind of heterotypy than has been considered thus far.
As opposed to the grossly heterotypic interactions of reciprocal projections in the spinal cord or thalamocortical tract, finely heterotypic axon-axon interactions mediate the segregation of axons originating from the same source into appropriate sub-fascicles representing different neuronal subtypes or different sub-regions of a target.
Sorting of axons into sub-fascicles in a tract is thought to occur by selective inhibition of fasciculation, such that individual cohorts can separate from a larger group and select the correct trajectory to the appropriate target.
An example comes from the chick spinal cord marginal zone, where intermediate and medial longitudinal commissural (ILc and MLc) axons run parallel in the lateral and ventral funiculi, respectively \cite{sakai2012axon}.
ILc axons express Robo, which is thought to inhibit N-cadherin-driven fasciculation in order to separate ILc axons away from the MLc fascicle.
Accordingly, loss of Robo function leads to ILc axons aberrantly coursing alongside MLc axons in the ventral funiculus and subsequently mistargeting in the brain \cite{sakai2012axon}.

Semas and their Nrp and Plexin receptors are also involved in sorting finely heterotypic axons within tracts.
In the mouse OE, different populations of OSN axons express Sema3A and Nrp1 in a roughly regionally defined manner \cite{imai2009pre}.
These OSN axons are likewise segregated from each other in the olfactory nerve. 
Selectively deleting either Nrp1 or Sema3A in OSNs leads to a blurring of pre-target axon order and shifts in glomerular position in the OB \cite{imai2009pre}.
Because these effects are found in conditional mutants that only affect subsets of OSNs, they occur independently of Sema3A surround expression by OEG.
Given this result, the authors conclude that axon-axon interactions are critical for establishing pre-target axon organization in the olfactory nerve \cite{imai2009pre}.

Sema3A and Nrp1 are also implicated in establishing the dorsal-ventral segregation of axons in the CC \cite{zhou2013axon}.
This order reflects a medial-high to lateral-low gradient of Nrp1 expression in the cortex and a reciprocal, though broader, expression of Sema3A, which is slightly higher in the lateral cortex.
Disrupting these molecular gradients perturbs both axon position in the CC and the position of axon terminations in the contralateral cortex.
The authors propose that both axon-extrinsic and axon-axon interactions are at play in this situation \cite{zhou2013axon}.
It is intriguing to speculate that Sema/Nrp interactions may be a conserved mechanism in mediating axon sorting across other systems, as they are a recurring molecular theme in studies of pre-target axon sorting in multiple systems.