\label{sec:Intrinsic2}
Neural activity, both spontaneous and activity-dependent, is a known factor in synaptic maturation and refinement of axonal projections in their targets \cite[reviewed in][]{zhang2001electrical}, but it may also play a role in the development and organization of axon tracts.
In vitro experiments demonstrate that neuronal firing patterns regulate expression levels of CAMs in dorsal root ganglion neurons \cite{itoh1997activity}.
In the olfactory system, single-cell microarrays of OSNs showed that OR-driven cyclic adenosine monophosphate (cAMP) levels complementarily regulate Nrp1 and Sema3A expression \cite{imai2006odorant}, two key players in modulating OSN axon organization in the olfactory nerve \cite{imai2009pre}.
Furthermore, both pre-target axon order in the olfactory nerve and glomerular patterning in the OB are perturbed in mice lacking adenylyl cyclase 3, a key enzyme involved in the production of cAMP \cite{miller2010axon}.

A series of studies in the chick spinal cord demonstrated the relationship between spontaneous bursting activity and both pathfinding decisions and axon fasciculation \cite[reviewed in][]{hanson2008spontaneous}.
Pharmacological blockade of neurotransmitters governing early bursting activity leads to dorsal-ventral pathfinding errors of spinal motor axons, while slight increases in bursting rate disrupt motor neuron pool specific fasciculation and targeting choices.
Again, these effects are likely mediated by activity-dependent changes in guidance molecule expression patterns.
While blocking evoked synaptic transmission during development affects neither the pre-target axon topography nor subsequent targeting of TCAs \cite{molnar2002normal}, spontaneous activity in thalamic neurons mediates TCA growth rates via transcriptional regulation of Robo1 \cite{mire2012spontaneous}.
Thus, different types of neuronal activity may have highly specific effects on the expression of guidance molecules and thereby influence the organization of axon tracts.