\label{sec:TractOrderIndependOfTargeting}
With the growing body work demonstrating that axons are consistently organized in their tracts and revealing some conserved mechanisms used to create such order, some major questions are raised.
Is tract organization an independent, tract-specific phenomenon, or merely a result of axons responding to long-range signals from their eventual targets?
Perhaps more importantly, does tract organization matter for the development of a functional neural circuit?
Or, so long as axons grossly reach their target, is any organization along the way purely incidental?
While the consistency found in pre-target axon organization argues indirectly for the meaningfulness of axon organization in tracts - rarely is a conserved biological phenomenon irrelevant to the function of an organism - a handful of studies have attempted to more directly address these questions.
Collectively, these studies provide evidence that axons are organized in their tracts independently of long-range target-derived cues, and that maintaining proper organization in the tract is crucial for making correct targeting decisions.

One way of testing whether axon organization in a tract occurs independently of the target is to remove the target early in development and see if the axon tract maintains its organizational structure.
This type of experiment has been conducted in several ways in the olfactory system, where a coarse topographic and typographic organization occurs among axons in the olfactory nerve.
Whether the OB, the target of growing OSN axons, is surgically removed \cite{graziadei1978regeneration}, genetically ablated \cite{stjohn2003sorting}, or lack a subset of target cells in the OB \cite{bulfone1998olfactory}, OSN axons still sort normally within the olfactory nerve and form OR-specific glomerulus-like bundles.
These target-ablation experiments provide compelling evidence of active and independent pre-sorting mechanisms within the olfactory nerve.
Additionally, in a normally developing mouse olfactory system, pre-target sorting of axons in the olfactory nerve is apparent by E12, at least 72 hours prior to onset of synaptogenesis in the OB \cite{miller2010axon}.
This, too, argues in favor of target-independent organizational mechanisms, rather than long-distance sorting cues originating from the target.

Similar to OB-ablation experiments, surgical removal of the \emph{Xenopus} optic tectum early in development does not affect the retinotopy found in the optic nerve or optic tract \cite{reh1983organization}.
Other evidence that tract organization occurs independently of target-derived cues in the visual system is more circumstantial.
For instance, in the cichlid fish optic tract, topographic order is maintained amongst retinal axon fascicles until very near the optic tectum, at which point the fascicles rearrange as they enter the appropriate tectal region \cite{scholes1979nerve}.
Presumably the axons maintain an order based on fiber-fiber interactions and/or environmental cues in the tract and only respond to target-derived cues once in close proximity and/or contact with the target.

Along these lines, the remarkably precise tonotopy maintained in the auditory brainstem tract \cite{kashima2013pre} corresponds to a precise tonotopic wiring in central auditory nuclei that is evident prior to onset of hearing and acoustic processing \cite{appler2011connecting}.
The degree of tonotopy in this brainstem auditory tract appears far more precise than any axon order so far described in the retinotectal or olfactory systems, in which axons are generally more coarsely organized.
This raises the compelling question of whether the higher degree of precision found in this auditory tract corresponds to a more precise initial mapping of axon terminations in the target relative to other systems.
In other words, does the NL rely less upon successive activity-driven refinement of axon terminals than does, for instance, the dLGN in the visual system? 
There is a relationship between degree of pre-target axon order and accuracy of initial target innervation in the visual system, where, relative to mammals, fish and frogs display greater order in the nerve and tract and greater initial accuracy of terminals in the target \cite{simon1991relationship}.
Based on this correlation in the visual system, it is reasonable to expect that the auditory system exhibits a greater degree of precision in its early circuit formation than do other sensory systems.
%Why some systems may deploy different developmental strategies of circuit formation is unclear and is an area ripe for comparative evolutionary study. 
More importantly, this raises the possibility that there is a linear relationship between degree of tract organization and precision of initial targeting, which certainly suggests a key role for axon organization in tracts in the development of neural circuits.

In vitro experiments have also added to the mounting evidence that tract organization occurs independently of target-derived cues, and that tract organization is important for axons to find their correct targets.
For instance, thalamic explants innervate cortical slices promiscuously in vitro, regardless of whether the cortical slice is from the appropriate region for the thalamic explant to target \cite{molnar1991lack}.
This suggests that chemotropic cues in the cortex are insufficient to organize TCAs. 
As such, the robust topographic organization of TCAs as they course through the IC and approach the PSPB \cite{molnar1998mechanisms}, may be a necessary step in TCAs entering the appropriate region of the cortex. 

The OB and optic tectum ablation experiments, correlation between precision of tract order and precision of initial targeting, and in vitro evidence from thalamic-cortical co-cultures all argue that pre-target axon order is occurring independently of target-derived cues.
%Many studies, including OB \cite{stjohn2003sorting} or optic tectum removal experiments \cite{reh1983organization}, indicate that pre-target axon organization occurs largely independently of target-derived cues.
That is, axons are actively ordered within their tracts, and organization occurs without any cues from their targets.
However, the functional relevance of pre-target tract organization in circuit formation is still unclear, specifically whether or not tract order is necessary for appropriately specific synaptic connections of terminals in their targets.
Experiments in the olfactory system and CC have attempted to directly address that issue.

In the olfactory system, selectively perturbing Nrp1 or Sema3A expression in a subset of OSNs disrupts axon order in consistent directions along the central-peripheral axis of the olfactory nerve and produces concomitant shifts in glomerular position in the OB \cite{imai2009pre}.
Similarly, in the CC, constitutive knockout of Sema3A or selective knockout of Nrp1 in the motor cortex both lead to a blurring of dorsal-ventral axon segregation in the CC \cite{zhou2013axon}.
More strikingly, there is a linear relationship between dorsal-ventral position of an axon in the CC and its subsequent medial-lateral position in the target contralateral cortex.
This linear relationship between tract position and targeting is true both in the wt context as well as in the experimental manipulations of CC axon order, leading the authors to conclude that axon position in the tract is a determining factor in establishing correct terminations within the target.
Importantly, postnatal refinement is largely unable to correct mistargeted axons in the contralateral cortex, underscoring the significance of early axon targeting decisions even in systems with known refinement mechanisms at play \cite{zhou2013axon}.

Findings from other systems also support the importance of pre-target axon order for fidelity of axon targeting.
Blocking Robo1 signaling in chick spinocerebellar neurons causes their axons to aberrantly fasciculate with other ascending spinal axons in the medial rather than lateral funiculus of the spinal cord marginal zone.
This incorrect positioning in the tract results in the axons bypassing the cerebellum and misprojecting instead to other hindbrain targets \cite{sakai2012axon}.
In the chick retinotectal system, blocking NCAM expression in focal subsets of RGCs causes affected axons to run in ectopically diffuse positions in the optic nerve and tract.
While roughly 15\% of these ectopic axons are able to make dramatic course corrections to reach the appropriate location once in the tectum, the rest project incorrectly \cite{thanos1984fiber}.
More direct evidence comes from the thalamocortical tract, where many cues within the subpallium organize ascending TCAs.
Selectively perturbing subpallium development blurs the topography of extending TCAs, leading to similarly blurred topography of the axon terminals in the somatosensory cortex.
An elegant set of control experiments substantiates the conclusion that the targeting defects are a direct result of defective pre-target axon order, rather than aberrant cues in either the thalamus or cortex \cite{lokmane2013sensory}.
All of these experiments provide very compelling evidence that axon organization in a tract, which includes axon position in the tract and the appropriate fasciculation or bundling partners, is critical, and perhaps instructive, for successful innervation of the final target.

On the other hand, not all studies corroborate the importance of axon pre-sorting for accuracy of target innervation.
For one, plasticity is a hallmark of developing neural circuits, so pre-target errors may be correctable once axons invade the target.
Disrupted HSPG synthesis in the zebrafish optic tract, for instance, leads to axon disorganization in the optic tract, but RGC axons still appear to innervate the optic tectum in topographically correct positions \cite{lee2004axon}.
In the mouse periphery, loss of ephrin-B1 leads to defasciculation of motor and sensory axons, but motor axons innervate the limb in grossly normal patterns \cite{luxey2013eph}.
Further, while much evidence in the thalamocortical system supports a role for axon pre-sorting in establishing accurate connections, cues within the cortex also drive guidance and targeting behavior (reviewed in \cite{garel2014inputs}).
When the arealization of the cortex is perturbed, for example, TCAs retain a normal topography in the subpallium but undergo extensive rearrangements within the cortex in order to innervate the appropriate target region \cite{shimogori2005fibroblast}.
It is worth noting, however, that this study assessed gross targeting outcomes but not maturation of synaptic connections.
Thus, while the TCAs innervating a disorganized cortex grossly find their correct targets, it is possible that subtle defects persist at the synaptic level.

Recent work in the auditory system demonstrated the degree of subtlety in downstream effects of pathfinding perturbations \cite{michalski2013robo3}.
Conditional knockout of Robo3 from contralaterally projecting ventral cochlear nucleus neurons causes their axons to misroute ipsilaterally. 
The misrouted axons innervate their appropriate target, the medial nucleus of the trapezoid body, but with incorrect laterality. 
While axon targeting appeared grossly normal, aside from the altered laterality, closer structural and physiological examination revealed failures in synaptic maturation.
It is unlikely that Robo3 is directly involved in synaptic development in this case, as it is downregulated immediately after the axons decussate in wt mice. 
Further, temporally controlled conditional knockdown of Robo3 after decussation showed no effect on synaptic development \cite{michalski2013robo3}.
The authors conclude that the act of decussation at the midline conditions the axons of ventral cochlear nucleus neurons for appropriate synaptic development.
Hence, while this study does not examine axon order in the post-decussation tract, it underscores the idea that individual events along a pathway, such as interaction with the choice point or intermediate target, can have subtle and surprising effects on downstream steps in the developing circuit (e.g., targeting or, theoretically, tract order).
Aberrant navigation of the midline decussation point leads to subtle but critical functional defects of subsequent synaptic connections, apparently independently of target-derived cues \cite{michalski2013robo3}.