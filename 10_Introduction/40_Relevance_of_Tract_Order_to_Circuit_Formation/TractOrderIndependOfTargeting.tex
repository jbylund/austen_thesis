\label{sec:TractOrderIndependOfTargeting}
The authors concluded that positional information of neurons in the spinal cord “can be used as a mechanism for imparting order to the presynaptic components” of the pathway, hypothesizing that contact-mediated guidance and selective fasciculation underlie such pre-target axon order \cite{nornes1980pattern}. 


Studies examining higher order olfactory targets beyond the OB have noted that central olfactory regions fail to maintain topographic correspondence to the spatial organization of glomeruli in the OB \cite{luskin1982distribution,sosulski2011distinct}.
Topographic order of OB efferents can only be found in the earliest portion of the lateral olfactory tract (LOT), and axon order reflecting spatial information of the OB is presumably lost along the length of the LOT \cite{price1975observation}.


A major question arising from this body of work is whether pre-target axon organization is functionally important in the formation of the retinofugal pathway.
Though most of the studies in the optic nerve and tract to date are descriptive, they argue that pre-target axon organization is governed by a consistent logic and is likely an important step in the formation of the visual circuit.
For instance, in the cichlid fish optic tract, topographic order is maintained amongst retinal axon fascicles until very near the optic tectum, at which point the fascicles rearrange as they enter the appropriate tectal region \cite{scholes1979nerve}.
Scholes, like Reh et al. (1983) and Horton et al. (1979), infers that the consistent maintenance of axon order along the length of the tract is indicative of target-independent cues in the tract; that is, short-range fiber-fiber interactions may provide organization in the tract prior to reaching target-derived guidance cues \cite{scholes1979nerve}.
Furthermore, retinotopy in the frog optic nerve and tract is largely normal following surgical removal of the optic tectum early in development \cite{reh1983organization}.
Axon order in these tectum-less frogs provides even stronger evidence that fiber-fiber interactions and/or environmental cues in the tract organize axons en route to the target.

The degree of tonotopy in this brainstem auditory tract appears far more precise than any axon order so far described in the retinotectal or olfactory systems, in which axons are more coarsely organized. 
A compelling question raised by this work is whether the higher degree of precision found in this auditory tract corresponds to a more precise initial mapping of axon terminations in the target relative to other systems. 
In other words, does the NL rely less upon successive activity-driven refinement of axon terminals than does, for instance, the dLGN in the visual system? 
The relationship between degree of pre-target axon order and accuracy of initial target innervation has been discussed in the visual system, where, relative to mammals, fish and frogs display greater order in the nerve and tract and greater initial accuracy of terminals in the target \cite{simon1991relationship}. 
Based on this correlation in the visual system, it is reasonable to expect that the auditory system exhibits a greater degree of precision in its early circuit formation than do other sensory systems. 
Why some systems may deploy different developmental strategies of circuit formation is unclear and is an area ripe for comparative evolutionary study. 


Explants of thalamic tissue innervate cortical slices promiscuously in vitro, regardless of whether the cortical slice is from the appropriate region for the thalamic explant to target \cite{molnar1991lack}, suggesting that chemotropic cues in the cortex are insufficient to organize TCAs. 
%Later studies from the same group confirmed the suspicion that TCAs are topographically ordered as they course through the IC and approach the pallial/subpallial boundary (PSPB), before entering the cortex \cite{molnar1998mechanisms}. 
Several cues in this subpallial region have been identified that orient subsets of TCAs, arranging them topographically inside the corridor \cite{dufour2003area,powell2008topography,seibt2003neurogenin2}. 
