The previous sections have demonstrated that in many central and peripheral axon tracts, axons adhere to common organizational modes, which are governed by both extinsic and intrinsic cues.
Organization inside of axon tracts occurs independently of the target and is also important for axons to find their correct target and form functional synapses there.
My thesis project centers around the organization of axons in the mouse retinogeniculate pathway, specifically examining the organization of ipsi- and contralateral RGC axons in the optic nerve and tract.

The mouse visual system is a long-standing classic model system for studying axon guidance and neural development.
Even prior to Sperry's seminal work in the frog visual system formalizing chemoaffinity \cite{sperry1963chemoaffinity}, Santiago Ram\'on y Cajal proposed the foundational ideas of chemoaffinity and growth cones after studying, among many other systems, the visual system.
The binocular visual system, in particular, has proven incredibly useful for studying midline choice and the molecules and growth cone dynamics involved in steering axons through or away from the midline optic chiasm.

In the binocular visual system, transcriptional programs underlying ipsilateral and contralateral identity have been partially worked out, with \emph{Zic2} guiding the ipsilateral program and \emph{Isl2} guiding the contralateral program \cite{herrera2003zic2,garcia2008zic2,pak2004magnitude}.
Other transcription factors are likely involved and work is ongoing to fully identify the ipsilateral and contralateral transcriptional programs (reviewed in part by \citenoparens{erskine2014connecting}).
The optic chiasm, the midline choice point of the retinogeniculate pathway, has garnered a significant amount attention, resulting in great progress in dissecting the molecular mechanisms of RGC axon behavior at the midline (reviewed in \citenoparens{erskine2014connecting,petros2008retinal}).
The receptor tyrosine kinase EphB1, and its repulsive interaction with ephrin-B2 at the optic chiasm is largely responsible for mediating the ipsilateral turn away from the chiasm \cite{nakagawa2000ephrin,williams2003ephrin,petros2009specificity}.
Sonic hedgehog (Shh) signaling with its receptor Boc, expressed by ipsilateral RGCs is also involved in ipsilateral navigation of the chiasm \cite{fabre2010segregation}.
Meanwhile, NrCAM and PlexinA1 interactions with NrCAM, Sema6D, and PlexinA1; and Nrp1 interactions with VEGF-A at the midline guide contralateral RGC axons across the optic chiasm \cite{williams2006role,kuwajima2012optic,erskine2011vegf}.

Finally, the molecular patterning and refinement mechanisms underlying targeting within the two thalamic targets, the dLGN and superior colliculus (SC), have been another major area of focus.
Ephs and ephrins are largely responsible for creating the topographic map in the dLGN and SC (reviewed in \citenoparens{erskine2014connecting,feldheim2010visual}), but it is still unclear what attracts ipsilateral RGC axons to their target zone within the dLGN and SC.
Even though activity-dependent synaptic refinement is considered largely responsible for the refinement of eye-specific zones in the targets (reviewed in \citenoparens{feller2009retinal}), it appears that ipsilateral RGC axons roughly navigate to their dorsal-medial location in the developing dLGN early on, before refinement occurs in the target \cite{jaubert2005structural}.
Indeed, even when activity-based synaptic refinement is selectively blocked in ipsilateral RGCs, ipsilateral terminals still cluster together in the correct zone in the dLGN \cite{koch2011pathway}.
In this conditional mutant, the ipsilateral terminals fail to block the invasion of contralateral terminals in the ipsi zone of the dLGN, but this faulty refinement does not grossly perturb their overall position or clustering together \cite{koch2011pathway}.
Thus, ipsilateral RGC axons appear to navigate to and cluster together within their zone independently of glutamate based activity-dependent refinement.
The binocular projections to the dLGN have been a useful tool for probing specific aspects of synaptic refinement and recently this system has been a springboard for the growing understanding of the role of glia in neural circuit development and synaptic refinement (e.g. \citenoparens{chung2013astrocytes,schafer2012microglia}).

What has been less well characterized, however, is the eye-specific organization of axons as they project from the retina to their thalamic targets.
Chapter 2 presents the first part of my thesis, in which I sought to clearly define the organization of ipsilateral and contralateral RGC axons in the developing mouse optic nerve and tract.
I also compare eye-specific, or typographic (i.e., ipsilateral and contralateral) axon order in the nerve and tract to topographic order, as previous reports on topography in the tract suggested a conflict between these two modes of axon organization.
In Chapter 3, I explore homotypic fasciculation as a mechanism of either creating or maintaining typographic order between ipsilateral and contralateral RGC axons.
To do this, I adapted a retinal explant culture system and established a reliable method of quantifying subtle differences in fasciculation between ipsilateral and contralateral neurites.
Chapter 4 presents in vivo and in vitro work on the EphB1 mutant mouse, which I use a model to address two questions: is EphB1 involved in ipsilateral homotypic fasciculation; and what is the relationship between midline choice, tract organization, and targeting?
The results and conclusions from these three portions of my thesis work will be discussed further in Chapter 5, where I will also consider future directions for this work to take and its implications for the field.

I have included in the appendices work that is not directly a part of my thesis project.
The first Appendix includes my characterization of the retinal projections and eye-specific axon organization in the optic tracts of several mutants, including a mutant for the LRR receptor Islr2, which was part of a publication in 2015 in \emph{Neural Development}.
I assisted a former postdoctoral fellow in the lab, Dr. Takaaki Kuwajima, in his development of a tissue clearing method, Clear\textsuperscript{\emph{T}}, which was published in \emph{Development} in 2013.
This is briefly described in Appendix 2.
The last appendix summarizes some early characterization of astrocytes and microglia in and along the optic tract, to better understand whether or not they are positioned favorably to provide organizational cues to ipsilateral or contralateral RGC axons.