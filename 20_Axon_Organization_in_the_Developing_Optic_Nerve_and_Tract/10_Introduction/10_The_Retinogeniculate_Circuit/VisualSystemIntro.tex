Retinal ganglion cells (RGCs) are the only retinal neurons that extend axons out of the retina.
RGC axons navigate to the optic disc at the back of the retina and then out of the retina into the optic nerve.
After extending along the length of the optic nerve, RGC axons next navigate through the ventral diencephalon to form the optic chiasm, the midline choice point of the visual system.
In primates, whose eyes are located frontally on the head, roughly half of the RGC axons cross the midline through the optic chiasm, and the other half are repelled away from the midline and project ipsilaterally.
This decussation in the optic chiasm is the basis for binocular vision --- each hemisphere of the brain receives input from both eyes.
In animals with more laterally positioned eyes, and therefore less binocular vision, more RGC axons project contralaterally at the optic chiasm.
In mice, which have relatively poor binocular vision, roughly 3-5\% of RGCs project ipsilaterally at the midline, leaving the vast majority to cross the midline and project contralaterally (reviewed in \citenoparens{petros2008retinal}).

\begin{figure}[hbtp]
    \begin{center}
        \includegraphics{Figures/RGP_Schema.pdf}
        \caption[The mouse retinogeniculate system.]
        {The mouse retinogeniculate system.
        Schematic is shown in the frontal plane, with the cortex in dashed lines for context.
        Contralateral RGC axons (contra, magenta) cross at the optic chiasm and ipsilateral RGC axons (ipsi, green), which arise in the ventrotemporal (VT) retina only, turn away from the chiasm.}
        \label{Figures/RGPSchema}
    \end{center}
\end{figure}
After navigating the optic chiasm, ipsilateral (ipsi) and contralateral (contra) RGC axons run in the optic tract and project into the dorsal lateral geniculate nucleus (dLGN), and further dorsally and caudally, into the superior colliculus (SC).
Relay neurons in the dLGN and SC project to the visual cortex for higher order visual processing.
In addition to the dLGN and SC, subsets of RGC axons extend to non-vision-forming targets.
My thesis focused only on the projection of RGC axons to the dLGN, which comprises the majority of the optic tract.
Additionally, the axon projections to non-vision-forming are poorly characterized.
The retinogeniculate pathway is schematized in Figure~\ref{Figures/RGPSchema}, with ipsi RGC axons colored in green and contra in magenta.
This color scheme will be used throughout the thesis for consistency.
While eye-specific zones in primates and carnivorous mammals are organized into laminae, the mouse dLGN has a center-surround organization of ipsi and contra zones.
These eye-specific zones are refined from coarsely targeted axon terminals in the first two weeks of postnatal life, largely by activity-dependent mechanisms (reviewed in \citenoparens{huberman2008mechanisms,feldheim2010visual}).
\begin{figure}[hbtp]
    \begin{center}
        \includegraphics{Figures/RGP_DevSeries.pdf}
        \caption[Three phases of axon outgrowth from the mouse retina.]
        {Three phases of axon outgrowth from the mouse retina.
        In the early phase of RGC axon outgrowth, at left, ipsi and contra RGCs arise from the central retina.
        RGC differentiation continues in a center-to-periphery wave.
        During the peak phase of axon outgrowth, the ventrotemporal (VT) retina produces predominantly ipsi RGCs, with contra RGCs arising predominantly from non-VT retina.
        Finally, at right, during the late phase of axon outgrowth, VT retina produces both ipsi and contra RGCs.}
        \label{Figures/RGPDevSeries}
    \end{center}
\end{figure}

As shown in Figure~\ref{Figures/RGPDevSeries}, RGC axon outgrowth from the retina occurs in three phases during embryogenesis.
In the early phase, from embryonic day (E) 11 or 12 to E14, the central retina produces both ipsi and contra RGCs.
This population of early-born ipsi RGCs, however, is transient \cite{drager1985birth,colello1990early,soares2015transient}.
E14-16 marks the peak phase of RGC axon outgrowth from the retina \cite{drager1985birth,petros2008retinal}.
While the VT retina produces predominantly ipsi RGCs during this period, the extent of heterogeneity of ipsi/contra RGC differentiation in the VT retina is still unclear, and details are currently being more closely examined (Marcucci and Mason, unpublished).
In the late phase of axon outgrowth, from E17 to birth, the VT retina produces both ipsi and contra RGCs \cite{drager1985birth,petros2008retinal}.
The late-born VT contra RGCs target the dorsal tip of the dLGN (not shown in Figure~\ref{Figures/RGPDevSeries}), in a markedly different position than their VT ipsi RGC counterparts, which all cluster in the central zone of the dLGN.