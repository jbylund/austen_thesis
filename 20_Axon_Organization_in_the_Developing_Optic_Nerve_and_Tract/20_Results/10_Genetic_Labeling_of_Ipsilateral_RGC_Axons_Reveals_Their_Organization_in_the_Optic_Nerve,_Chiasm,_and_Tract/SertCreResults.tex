\label{sec:SertCreResults}
Given the identification of ipsilateral RGC-specific markers, one of the best ways, in theory, to trace ipsi axons in the mouse retinogeniculate pathway would be using a genetic reporter line.
Because the transcription factor Zic2 regulates the ipsilateral RGC projection \cite{herrera2003zic2,williams2003ephrin}, a Zic2 reporter mouse would be ideal to genetically trace the ipsi RGC axon cohort.
A BAC-transgenic Zic2 reporter mouse exists (Tg(Zic2-EGFP)HT146Gsat), but analysis of this mouse by a former lab member indicated that the green fluorescent protein (GFP) signal was not faithful to endogenous \emph{Zic2} expression \cite{wang2013neuronal}.
Specifically, GFP is expressed by some but not all Zic2\textsuperscript{+} cells, and is also expressed by some Zic2\textsuperscript{-} cells \cite{wang2013neuronal}, rendering it unreliable for labeling the ipsi RGC population and its axonal projections.
%add marcucci 2016 when avail

Zic2, however, directly regulates expression of the serotonin transporter (SERT) (Figure~\ref{Figures/SertRetina}A) \cite{garcia2010zic2}.
SERT is important for eye-specific segregation in the dLGN and SC, but is not involved in the midline choice of ipsi RGC axons at the optic chiasm \cite{upton1999excess,salichon2001excessive,garcia2010zic2}.
A report using the ET33 SERT-Cre line from GENSAT to study eye-specific axon refinement in the dLGN confirmed the specific expression of Cre in ipsi RGC axon terminals in the dLGN \cite{koch2011pathway}.
This mouse line is useful not only because of the reliability and specificity afforded by genetics, but because it provides the clearest view of ipsilateral axon organization in the optic nerve.
Anterograde labeling cannot distinguish ipsi and contra axons in the nerve, prior to their sorting at the midline optic chiasm; and retrograde labeling experiments \cite{colello1990early}, may not reliably label the entire ipsilateral cohort.
We therefore acquired ET33 SERT-Cre:zsgreen mice to follow ipsi RGC axons through the optic nerve, chiasm, and tract.
This section will present my analysis of ipsi RGC axon organization in the retinogeniculate pathway utilizing zsgreen expression in SERT-Cre:zsgreen mice.

\begin{figure}[hbtp]
    \begin{center}
        \includegraphics{Figures/Sert_Retina.pdf}
        \caption[SERT is a marker of ipsi RGCs.]
        {SERT is a marker of ipsi RGCs.
        A) Known (solid lines) and hypothesized (dashed lines) transcriptional regulation in ipsi (green) and contra (magenta) RGCs.
        The ipsi-specific transcription factor Zic2 drives expression of the serotonin transporter SERT \cite{garcia2010zic2}.
        SERT is a direct target of Zic2, and regulation of SERT by Zic2 occurs independently of EphB1.
        B) A flat-mounted retina from a postnatal day (P)0 ET33 SERT-Cre:zsgreen mouse shows zsgreen\textsuperscript{+} cells largely restricted to the VT retina, as described by \citenoparens{garcia2010zic2,koch2011pathway}.
        D=dorsal, V=ventral, T=temporal, N=nasal.
        Scale=500$\mu$m}
        \label{Figures/SertRetina}
    \end{center}
\end{figure}

SERT-Cre::zsgreen mice have robust zsgreen expression restricted to the ventrotemporal (VT) retina, the source of ipsi RGCs (Figure~\ref{Figures/SertRetina}B).
This expression pattern is consistent with the SERT-Cre expression pattern shown by Koch et al., \shortcite{koch2011pathway} and SERT mRNA expression shown by Garc\'ia-Frigola \& Herrera \shortcite{garcia2010zic2}, which corresponds to Zic2 mRNA expression.
The expression of zsgreen, seen in a frontal section of the retina, is restricted to the RGC layer, and zsgreen\textsuperscript{+} RGC axons can be traced as they exit the retina and enter the ventral portion of the optic nerve (Figure~\ref{Figures/SertNerveToTract}A).
Once in the optic nerve, zsgreen\textsuperscript{+} axons cluster predominantly in a ventrolateral position, although there is a secondary cluster of zsgreen\textsuperscript{+} axons in the medial optic nerve (Figure~\ref{Figures/SertNerveToTract}B\textsuperscript{1}).
This second cluster of zsgreen\textsuperscript{+} axons appears more loosely bundled than the larger set of zsgreen\textsuperscript{+} axons in the ventrolateral nerve.
As the optic nerve nears the optic chiasm, labeled axons appear progressively less well ordered and less tightly bundled together (Figure~\ref{Figures/SertNerveToTract}B\textsuperscript{2-4}).
This observation is in line with previous reports based on retrograde labeling from the optic tract \cite{colello1990early}.

\begin{figure}[hbtp]
    \begin{center}
        \includegraphics[width=\textwidth]{Figures/Sert_NerveToTract.pdf}
        \caption[SERT-Cre:zsgreen axons in the optic nerve, chiasm, and tract.]
        {SERT-Cre:zsgreen axons in the optic nerve, chiasm, and tract.
        25$\mu$m frontal cryosections in a P0 SERT-Cre:zsgreen mouse, immunostained for zsgreen (green) to label SERT-Cre:zsgreen axons, and neurofilament (NF, magenta) to label all axons.
        Sections shown are representative of four samples collected from two litters. %Check how many samples, how many litters!
        A) Zsgreen\textsuperscript{+} cells are visible in the RGC layer of a frontal section taken through the temporal retina (top).
        Higher magnification at the optic nerve head (bottom) shows the zsgreen\textsuperscript{+} axons (arrow) exiting the retina.
        Bracket indicates full width of the optic nerve
		Zsgreen\textsuperscript{+} axons exit the retina in a ventral position.
        B-D) Postion indicated at left (in $\mu$m) specifies the position of the section relative to the beginning of the optic chiasm (0$\mu$m).
        Negative values are rostral to the chiasm (nearer the retina) and positive values are caudal to the chiasm, entering into the thalamus.
        B) Zsgreen\textsuperscript{+} axons are clustered in the proximal optic nerve (B\textsuperscript{1-2}), with a primary bundle in the ventral-lateral optic nerve and a smaller, more loosely organized bundle in the medial nerve.
        The axons become progressively less well bundled as they approach the optic chiasm (B\textsuperscript{3-4}).
        C) At the midline, zsgreen\textsuperscript{+} axons interact with most of the dorsal-ventral extent of the midline (indicated by vertical lines).
        D) As zsgreen\textsuperscript{+} axons enter the optic tract caudal to the optic chiasm, they become rebundled and segregate to the lateral tract.
        D=dorsal, V=ventral, M=medial, L=lateral.
        All scale bars=200$\mu$m}
        \label{Figures/SertNerveToTract}
    \end{center}
\end{figure}

Within the optic chiasm, zsgreen\textsuperscript{+} axons can be seen across nearly the full dorsal-ventral extent of the chiasm (vertical lines in Figure~\ref{Figures/SertNerveToTract}C).
The images shown in Figure~\ref{Figures/SertNerveToTract} are from a P0 mouse.
The majority of ipsi RGC axons have turned away from and extended beyond the optic chiasm at this age, hence the predominant zsgreen signal at either side of the midline.
However, the axons from the latest-born ipsi RGCs are still navigating the chiasm at P0, and these are likely the zsgreen\textsuperscript{+} axons seen more sparsely in the middle of the micrographs in Figure~\ref{Figures/SertNerveToTract}C.

Zsgreen\textsuperscript{+} RGC axons in the optic tract are loosely bundled just caudal to the chiasm (Figure~\ref{Figures/SertNerveToTract}D\textsuperscript{1}) and become increasingly well bundled the farther they extend into the tract (Figure~\ref{Figures/SertNerveToTract}D\textsuperscript{2-3}).
This progression in extent of bundling is similar to that seen in the optic nerve (Figure~\ref{Figures/SertNerveToTract}B\textsuperscript{1-4}), but in reverse (i.e., from less to more bundled along the rostral-caudal axis).
Farther distally, near the first thalamic target, the dLGN, zsgreen\textsuperscript{+} RGC axons remain clustered to the lateral edge of the optic tract (Figure~\ref{Figures/SertdLGN}).
The clustering of zsgreen\textsuperscript{+} axons to the lateral edge of the optic tract does not appear to change along the length of the tract, even as axons approach the dLGN.
%Given that the more caudal of the two sections shown in Figure~\ref{Figures/SertdLGN} (B) contains more zsgreen+ axons than its neighboring prior section (A), it is possible that this cohort of axons are biased to the caudal optic tract.

\begin{figure}[hbtp]
    \begin{center}
        \includegraphics{Figures/Sert_dLGN.pdf}
        \caption[SERT-Cre:zsgreen axons in the optic tract and dLGN.]
        {SERT-Cre:zsgreen axons in the optic tract and dLGN.
        25$\mu$m frontal cryosections from the same P0 SERT-Cre:zsgreen mouse as shown in Figure~\ref{Figures/SertNerveToTract}, immunostained for zsgreen (green) to label SERT-Cre:zsgreen axons, and neurofilament (NF, magenta) to label all axons.
        Sections shown are representative of four samples collected from two litters. %Check how many samples, how many litters!
        A-B) Zsgreen\textsuperscript{+} axons cluster to the lateral of edge of the optic tract along its full extent.
        A is 1250$\mu$m caudal to the optic chiasm and B is another 100$\mu$m caudal to A.
        C) Zsgreen\textsuperscript{+} axons in the dLGN are shown magnified in C\textsuperscript{1} and with NF label in C\textsuperscript{2}.
        D=dorsal, V=ventral, M=medial, L=lateral.
		Scale=200$\mu$m}
        \label{Figures/SertdLGN}
    \end{center}
\end{figure}
