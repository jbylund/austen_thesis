The results from the VT/VT anterograde labeling experiment raise the question of whether the eye-specific typographic map is out of register with the topographic map in the mouse optic tract.
While the literature on topographic organization of RGC axons in the optic nerve and tract is fairly extensive, it is also hindered by disagreements regarding the extent and details of organization among retinofugal axons.
Only a few of these reports studied the mouse, more often favoring the visual systems of the cat, ferret, hamster, rat, or marsupials.
Furthermore, none of the studies directly compared topographic and eye-specific order in the nerve and tract.
Thus, to directly address the extent to which the topographic (i.e., retinotopic) and eye-specific maps in the optic tract are in register, and to clarify some of the inconsistencies in the existing literature, I performed a series of DiI/DiD two-color labeling experiments on different combinations of topographic coordinates in the retina at P0.

In the first set of these experiments, I labeled the dorsal (D) and ventral (V), or the nasal (N) and temporal (T) hemiretinae of fixed P0 WT pups.
The results from this anterograde tracing, shown in Figure~\ref{Figures/SerialSectionsP0Topog}, clearly show topographic organization within the optic tract.
Dorsal and ventral axons run through the caudal and rostral aspects of the optic tract, respectively, with dorsal axons biased medially and ventral axons biased slightly laterally (Figure~\ref{Figures/SerialSectionsP0Topog}B\textsuperscript{1}).
Temporal and nasal axons also appear organized across the rostral-caudal axis of the optic tract, with temporal axons coursing through the caudal tract and nasal axons in the rostral tract (Figure~\ref{Figures/SerialSectionsP0Topog}C\textsuperscript{1}).
Unlike the dorsal and ventral axons, the nasal and temporal cohorts do not appear to be particularly well segregated along the medial-lateral axis of the optic tract.
\begin{figure}[hbtp]
    \begin{center}
        \includegraphics[width=\textwidth]{Figures/SerialSectionsP0_Topog.pdf}
        \caption[Topographic organization of RGC axons in the optic tract.]
        {Topographic organization of RGC axons in the optic tract.
        A) Cartoon depicting the labeling scheme.
        One retina in a fixed P0 WT head was labeled with DiI and DiD in opposite retinal poles: D/V or N/T, and sectioned horizontally 75$\mu$m thick.
        B\textsuperscript{1}-C\textsuperscript{1}) Every third serial section from a single optic tract contralateral to the labeled retina.
        The sections begin ventrally at right, nearest the optic chiasm, and progress dorsally leftwards, towards the dLGN and SC.
        Samples shown are representative of six or more pups from three or more litters, per labeling scheme. %check numbers
        B) Dorsal hemiretina labeled with DiD (magenta), ventral hemiretina labeled with DiI (green).
        B\textsuperscript{1}) Dorsal RGC axons course through the medial-caudal optic tract and ventral axons course through the rostral tract, with a slight bias laterally relative to dorsal axons.
        C) Temporal hemiretina labeled with DiD (yellow), nasal hemiretina labeled with DiI (cyan).
        C\textsuperscript{1}) Temporal RGC axons are positioned caudally, and nasal axons rostrally in the optic.
        D=dorsal, V=ventral, N=nasal, T=temporal, R=rostral, C=caudal, M=medial, L=lateral.
        Scale=500$\mu$m for B, C; Scale=100$\mu$m for B\textsuperscript{1}, C\textsuperscript{1}.}
        \label{Figures/SerialSectionsP0Topog}
    \end{center}
\end{figure}

%The hemiretina labels provided a clear indication of the topographic map in the optic tract, but they lack detail.
%Some of the axons labeled by a dorsal label in one sample would also be labeled by both the temporal and nasal labels in another sample.
%Therefore, to acquire a more detailed view of the topography, which is necessary in order to directly probe how it relates to the ipsi/contra order in the tract, I performed another series of anterograde labeling experiments using very small amounts of DiI and DiD.
%In this set of samples, I attempted to label small swathes of RGCs in each of the topographic poles and directly compare their projections in the optic tracts both ipsilateral and contralateral to the labeled retina.
I next performed similar but more focal labeling of retinal quadrants in P0 samples (Figure~\ref{Figures/TopogIpsiContra}).
In these samples, I imaged both hemispheres of horizontal sections through the tract in order to directly compare the ipsi and contra projections from each retinal region.
Figure~\ref{Figures/TopogIpsiContra}A shows a dorsal (magenta) and ventral (green) label, with noticeably fewer axons labeled than in the corresponding hemiretina sample shown in Figure~\ref{Figures/SerialSectionsP0Topog}.
(Note that while a large swathe of the peripheral retina is labeled Figure~\ref{Figures/TopogIpsiContra}A, the fraction of axons labeled is best judged by viewing the axon bundles nearer the optic disc.)
In the contralateral tract, as before, the dorsal axons run in the caudal tract, largely segregated from the ventral axons, which run in the rostral tract.
A direct comparison between the ipsi and contra tracts of the ventral retina label demonstrates a lateral shift in the ipsi population relative to the contra cohort.
Moreover, an area of weaker fluorescent signal is evident in the lateral portion of the ventral axon cohort on the contralateral side, corresponding to the location of the ventral axons in the ipsi tract.
\begin{figure}[hbtp]
    \begin{center}
        \includegraphics{Figures/TopogIpsiContra.pdf}
        \caption[Comparison of ipsi and contra projections from each topographic quadrant.]
        {Comparison of ipsi and contra projections from each topographic quadrant.
        Focal DiI (magenta) and DiD (green) labeling performed in the right retina of fixed P0 WT pups.
        For each case, the labeled retina is shown at left and a representative 75$\mu$m-thick horizontal section taken ~150$\mu$m dorsal to the chiasm.
        Sections include the optic tract contralateral (left) and ipsilateral (right) to the labeled eye.
        The midline  and lateral edge of the thalamus are indicated by dashed lines for each section.
        A) D/V labeling confirms the segregation of the two populations to the medial-caudal and anterior-lateral aspects of the contralateral tract.
        Ipsi ventral axons appear to be positioned farther laterally than contra ventral axons.
        B) N/T labeling shows a rostral to caudal segregation in the contralateral tract.
        Ipsi axons from the temporal retina are positioned caudally and laterally to the contra temporal axons.
        C) Labeling neighboring ventral and temporal retinal quadrants shows organization along a roughly rostral-lateral to caudal-medial axis in the contralateral tract.
        They are similarly segregated in the ipsilateral tract, again favoring the lateral side of the tract.
        D=dorsal, V=ventral, N=nasal, T=temporal, R=rostral, C=caudal.
        Scale=500$\mu$m.}
        \label{Figures/TopogIpsiContra}
    \end{center}
\end{figure}

Similarly, in Figure~\ref{Figures/TopogIpsiContra}B, axons arising from the nasal (magenta) and temporal (green) retinal fields are segregated along a roughly rostral-caudal axis in the tract.
Temporal axons in the ipsi tract appear in largely the same region as temporal axons in the contra tract, again with a slight lateral bias.
Finally, labeling the neighboring ventral and temporal retinal quadrants (Figure~\ref{Figures/TopogIpsiContra}C) reveals a rostal-lateral to caudal-medial segregation of ventral (magenta) and temporal (green) axons in the tract.
The ipsi projection from the same regions is largely in register with the topographic map seen on the contra side, but with a small lateral shift.
Similar to Figure~\ref{Figures/TopogIpsiContra}A, there is an apparent ``gap'' in the ventral label in the contra tract - presumably where ipsi ventral axons run.
There is a less obvious area of lower signal in the temporal cohort in the contra tract, suggesting that the lateral shift of ipsi axons in the tract corresponds mostly to the ventral topographic aspect and less so to the temporal topographic aspect.
Figure~\ref{Figures/TopographySummary} summarizes the results presented in this section of the relationship between the eye-specific and topographic maps in the optic tract.
Dorsal axons (magenta) are largely segregated to the medial, slightly caudal aspect of the tract.
Nasal axons (represented in orange) run in the rostral and slightly medial portion of the tract.
Ventral axons (green) are positioned in the lateral and slightly rostral tract, while temporal axons (cyan) are in the lateral-caudal tract.
The ipsilateral cohort of ventral and temporal axons (lighter shades of green and cyan, respectively) are in register with but shifted laterally and slightly caudally relative to their contralateral topographic partners.
\begin{figure}[hbtp]
    \begin{center}
        \includegraphics{Figures/TopographySummary.pdf}
        \caption[Summary of eye-specific and topographic mapping in the optic tract.]
        {Summary of eye-specific and topographic mapping in the optic tract.
        Summary cartoon drawn from comparisons of two-color labeling experiments, overlaying ipsi and contra axon projections.
        Dorsal RGC axons (magenta) are positioned in the medial, slightly caudal tract; ventral axons (green) in the lateral tract; nasal axons (orange) in the rostral tract; temporal axons (cyan) in the caudal-lateral tract.
        Ipsi axons from the ventral and temporal retina (light green and light cyan, respectively) are in register with their contra topographic counterparts, but shifted slightly laterally and caudally.
        R=rostral, C=caudal, M=medial, L=lateral.}
        \label{Figures/TopographySummary}
    \end{center}
\end{figure}
