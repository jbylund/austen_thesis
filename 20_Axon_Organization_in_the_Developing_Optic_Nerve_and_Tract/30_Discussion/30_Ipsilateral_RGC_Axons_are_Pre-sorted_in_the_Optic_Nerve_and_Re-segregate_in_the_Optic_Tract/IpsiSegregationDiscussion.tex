Ipsilateral RGC axon organization appears dynamic along the length of the retinogeniculate pathway (Figure~\ref{Figures/SertNerveToTract}).
Using SERT expression as a marker of ipsi identity in the SERT-Cre:zsgreen mouse, zsgreen\textsuperscript{+} axons can be seen exiting the retina and entering the ventral optic nerve.
They continue to be relatively well bundled through the proximal optic nerve and progressively splay apart as the nerve approaches the optic chiasm.
This is in line with previous reports indicating a progressive loss of topopgraphic or eye-specific axon order in the nerve \cite{colello1990early,chan1994changes,chan1999changes}, although my results indicate a greater degree of organization in the nerve than previously thought \cite{baker1989distribution,colello1990early}.
The presence of the secondary, more loosely bundled set of zsgreen\textsuperscript{+} axons in the medial optic nerve is surprising, given the robust bundling of the majority of zsgreen\textsuperscript{+} axons to the ventrolateral aspect of the nerve.
This secondary bundle is consistently observable in all of the samples analyzed.
Given that the heterogeneity of the ipsi RGC population is still unclear, this secondary medial bundle of zsgreen\textsuperscript{+} axons could represent a specific subset of ipsi RGCs.
Alternatively, if SERT is not as specific to ipsi RGCs as has been thought (see Section~\ref{sec:TechnicalSERT}), this secondary bundle could be yet a different subset of RGC axons altogether.
If a Zic2 reporter mouse line is made, it could used to trace the ipsi RGC projection more reliably.
Such a study would clarify the questions surrounding the appearance of a primary and secondary zsgreen\textsuperscript{+} axon bundle in the SERT-Cre:zsgreen line used here.

Ipsi RGC axons are defasciculated and spread out within the optic chiasm (Figure~\ref{Figures/SertNerveToTract}).
Previous reports \cite{colello1997changing} and unpublished observations (K.Y. Chung and C.A. Mason) suggested that growth cones of ipsi axons only interact with the ventral-most portion of the midline chiasm region.
My observations of zsgreen\textsuperscript{+} axons in the SERT-Cre:zsgreen optic chiasm instead suggest that ipsi axons scatter along the majority of the ventral-dorsal aspect of the chiasm (Figure~\ref{Figures/SertNerveToTract}C) as they interact with midline radial glia that provide cues for ipsi axons to turn away from the midline.
Higher resolution microscopy of genetically-labeled ipsi RGC axons in the optic chiasm should further extend our understanding of the distribution of RGC axon cohorts within the complex chiasm region.
Past studies have used either electron microscopy (EM) of small regions of the chiasm or epifluorescent microscopy of sections or semi-intact preparations of the chiasm regions. %cite these? ugh
Current advances in both tissue clearing techniques \cite{tainaka2015chemical} and high-resolution microscopy for large intact samples, like light-sheet microscopy \cite{keller2015visualizing}, will soon afford more complete views of axons in this region than have previously been possible to acquire.

After exiting the chiasm region, ipsi RGC axons regroup in the early tract and are robustly bundled in the lateral aspect of the tract for the rest of their journey to the dLGN and SC.
Two-color anterograde labeling shows a clear segregation of ipsi axons away from contra axons in the tract (Figures~\ref{Figures/DiIDiDWT}, \ref{Figures/DiIDiDWTSerial}).
Postnatally, some contra axons appear to invade the lateral portion of the tract (the ipsi zone), but the two populations remain largely separate along the lateral-medial axis of the optic tract.
The position of ipsi axons in the lateral optic tract, identified here either via SERT-Cre:zsgreen expression or with DiI and DiD, corresponds to the early report by Godement et al. \shortcite{godement1984prenatal}.
Direct comparison between ipsi and contra RGC axons presented in this chapter, however, provides greater detail than was previously observed.
While contra axons extend through much of the optic tract, they are not uniformly distributed, as was concluded from single-eye HRP labeling \cite{godement1984prenatal}.
Rather, contra axons seem to avoid the lateral region of the tract, where ipsi axons are bundled.
This is especially apparent in late embryonic and newborn samples, where there is a clear gap in the contra label corresponding to the highest density of ipsi label (Figure~\ref{Figures/DiIDiDWT}B).

Furthermore, eye-specific axon organization is reasonably stable throughout development, with ipsi axons consistently positioned in the lateral optic tract.
Early in tract development (E15-16), a small number of ipsi axons are loosely spread in the medial tract (Figure~\ref{Figures/DiIDiDWT}B).
These axons are no longer visible by P0 and presumably extend from the transient population of early-born dorsocentral (DC) RGCs \cite{drager1985birth,soares2015transient}.
A recent study from our lab showed that axons from the transient early-born ipsi RGCs are positioned in the medial aspect of the tract \cite{soares2015transient}.
Thus the early-born DC ipsi RGC axons are in a markedly different position in the tract than the permanent ipsi axons that arise from RGCs in the VT retina and are situated in the lateral optic tract.
This is not necessarily surprising, given that the early-born transient ipsi population is thought to be unique in other ways.
For instance, they do not interact with the chiasm like the permanent ipsi axons, but instead navigate far from the midline region, turning from the distal optic nerve into the nascent ipsilateral tract \cite{godement1987study,godement1990retinal,marcus1995first}.
Evidence from frogs and zebrafish led to the hypothesis that early-born DC RGC axons are pioneer axons \cite{holt1984does,pittman2008pathfinding}, but the fact that the early DC RGC ipsi and contra axons segregate from each other in the optic tract \cite{soares2015transient} suggests that the DC ipsi axons may not in fact serve as a pioneering substrate for contra axons.
The segregation noted between DC ipsi and contra RGC axons underscores the principle of typographic (eye-specific) organization of RGC axons in the mouse optic tract.
The position of DC ipsi RGC axons in the medial tract also raises questions about the relationship between this transient ipsi RGC population and the permanent ipsi RGCs in the VT retina.
Perhaps their position in the medial tract relates to their eventual failure to form permanent connections with the SC \cite{soares2015transient}.
%Expand on this at all
%Mentioning that DC ipsi axons don't penetrate the dLGN, but do a little bit the SC? Different timing of axon refinement in the SC between ipsi and contras (dhande paper) but not so in the dLGN?

While VT ipsi RGC axons remain restricted to the lateral portion of the optic tract throughout development, an increasing proportion of contra axons partially overlap the ipsi zone of the tract postnatally (Figure~\ref{Figures/DiIDiDWT}B, P2).
It is possible that this subpopulation of contra axons running partially in the lateral optic tract arises from the VT retina.
The late-born population of VT contra RGCs appears to be unique in some ways relative to the rest of the contra population \cite{williams2006role}, and as such they could also behave differently from other contra axons in the tract.
However, two-color anterograde tracing from the VT region of both retinae (Figure~\ref{Figures/DiIDiDWTVTVT} shows that ipsi and contra RGC axons arising from the same topographic region (i.e., the VT retina) are largely segregated.
This underscores the existence of both a topographic and typographic map in the mouse optic tract, which will be discussed further in the next section.
Of course, the segregation of ipsi and contra axons is not complete - i.e., there are not two distinct bundles in the tract, one of only ipsi axons and one of only contra axons.
Rather, the tract is better thought of as a spectrum, with ipsi axons more heavily distributed at the lateral end and contra axons more heavily distributed to the medial end of a medial-lateral axis.
%Could add in a frontal section figure and explain here that the medial-lateral axis is not that simple, as the tract curves around the thalamus
Ipsi/contra distribution and segregation is best appreciated in the high-resolution images of the cleared wholemount optic tract shown in Figure~\ref{Figures/HighResIpsiContra}B-C.
In these images it is obvious that there is a combination of ipsi- and contra-specific fascicular organization alongside generally loose bundling of both cohorts, with some intermingling between the two.
This spectrum-like distribution is suggestive of the presence of a chemoattractive or chemorepulsive gradient within the optic tract.
In this regard, however, the optic tract remains a black box -- we have a very limited understanding of the cells within and around the tract and almost no information on the molecular expression of those cells.

Once RGC axons reach the dLGN, ipsi axons appear to remain in the tract and, relative to contra axons, only a few ipsi fibers enter into the dLGN (Figures~\ref{Figures/SertdLGN} \& ~\ref{Figures/DiIDiDWTSerial}).
This is in line with what has been seen in single-eye anterograde tracings in the E18.5 mouse \cite{godement1987study}, and raises the question of whether the ipsi and contra RGC populations innervate the dLGN in different time courses.
In cats, ferrets, and hamsters, ipsi RGC axons arrive at the dLGN after contra axons \cite{frost1979postnatal,linden1981dorsal,shatz1983prenatal}.
However, a recent study tracing single RGC axons in the mouse indicates that while there is a lag in ipsi axon arborization relative to contra axons in the SC, the two populations appear to elaborate their axon arbors along a similar timeline within the dLGN \cite{dhande2011development}.
Thus, what my experiments show and what has been reported in the mouse dLGN previously \cite{godement1987study} may reflect the large difference in population size of ipsi and contra RGCs, rather than a delay in the ipsi population reaching the dLGN.
That is, while few ipsi axons appear to have innervated the dLGN at P0, it may be proportionally equivalent to the fraction of contra axons that have entered the target at the same time.
