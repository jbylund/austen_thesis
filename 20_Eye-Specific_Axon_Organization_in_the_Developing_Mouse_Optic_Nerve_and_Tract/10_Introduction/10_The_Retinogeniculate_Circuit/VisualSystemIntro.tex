Retinal ganglion cells (RGCs) are the only retinal neurons that send axons out of the eyes.
RGC axons navigate to the optic disc at the back of the retina and then out of the eye into the optic nerve.
After extending along the length of the optic nerve, RGC axons next navigate the optic chiasm, the midline choice point of the visual system.
In primates, whose eyes are located more frontally on the head, roughly half of the RGC axons cross the midline through the optic chiasm, and the other half project ipsilaterally after being repelled away from the chiasm.
This choice at the optic chiasm is what underlies binocular vision - each hemisphere of the brain receives input from both eyes.
In animals with more laterally positioned eyes, and therefore less binocular vision, more RGC axons project contralaterally at the optic chiasm.
In mice specifically, which have relatively poor binocular vision, roughly 3-5\% of RGCs project ipsilaterally at the midline, leaving the vast majority to cross the midline and project contralaterally \cite{petros2008retinal}.

After navigating the optic chiasm, ipsilateral and contralateral RGC axons run in the optic tract and project into the dorsal lateral geniculate nucleus (dLGN), and further dorsally and caudally, into the superior colliculus (SC).
Relay neurons in the dLGN and SC then project to the visual cortex for higher order visual processing.
For the purposes of my thesis, I have focused only on the projections of RGC axons to the dLGN, as this includes the majority of the optic tract.
The retinogeniculate pathway is schematized in Figure~\ref{figures/RGP_Schema}, with ipsilateral RGC axons colored in green and contralatal axons in magenta.
I will use this color scheme throughout the thesis for consistency.

\begin{figure}[H]
\begin{center}
\includegraphics[width=\textwidth]{figures/RGP_Schema.svg}
\caption{The mouse retinogeniculate system. Contralateral RGC axons (magenta) cross at the optic chiasm and ipsilateral RGC axons (green), which arise in the ventrotemporal retina only, turn away from the chiasm. The cortex is shown with dashed lines for context. The schematic is shown in a frontal view.}
\label{figures/RGP_Schema}
\end{center}
\end{figure}