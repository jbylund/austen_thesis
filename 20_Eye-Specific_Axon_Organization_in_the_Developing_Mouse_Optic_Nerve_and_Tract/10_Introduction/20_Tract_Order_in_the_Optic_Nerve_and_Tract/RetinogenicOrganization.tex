\label{sec:RetinogeniculateOrganization}
%Sperry’s pioneering work underlying the chemoaffinity hypothesis was conducted on retinal ganglion cell axon terminals as they map onto the optic tectum \cite{sperry1950regulative,sperry1963chemoaffinity}.
%Though topographic order of axon terminations in their targets has been characterized in many other systems since the 1960s, the retinofugal pathway has remained one of the dominant classic model systems.
%As such, some of the first steps towards understanding pre-target axon sorting were taken in the visual system.

During development, RGCs extend axons out of the retina and into the optic nerve to the optic chiasm, the midline choice point or intermediate target region of the retinofugal pathway.
RGC axons then grow into the optic tract, extending to the primary image-processing visual targets in the thalamus and midbrain, the dorsal lateral geniculate nucleus (dLGN) and superior colliculus (SC).
While subsets of RGC axons extend to other non-image-processing targets (e.g., the suprachiasmatic nucleus and ventral lateral geniculate nucleus), tracts projecting to these regions and retinotopy therein have been less well characterized.
Instead, pre-target axon sorting has been examined along two segments of the retino-thalamic pathway: the pre-chiasmatic optic nerve and the post-chiasmatic optic tract.

At the chiasm itself, several guidance factors have been identified on radial glia and a specialized neuronal population that mediate the decision of retinal axons to project ipsilaterally (reviewed in \cite{petros2008retinal}, and discussed below).
While efforts are ongoing to identify the full symphony of decussation cues in the chiasm, details and mechanisms of topographic axon sorting therein are considerably less clear than decussation cues in the chiasm.

The fascicular composition of RGC axons changes entirely within the optic chiasm, as axons from one eye split into many separate fascicles that cross contralateral axon bundles from the opposite eye at right angles, forming a braid-like configuration at the midline \cite{colello1998changing}.


%After navigating the complex array of cues at the optic chiasm, RGC axons enter into the optic tract.
Though reports vary in the details and extent of order, %a coarse topography of axons has been reported in the optic tract of many species \cite{chan1999changes,chan1994changes,plas2005pretarget,reese1993reestablishment,reh1983organization,torrealba1982studies}.
The variations between these studies can be attributed to different axon tracing methodologies, ranging from degeneration studies to various retrograde and anterograde axonal tracers, and to disagreement between physiological and anatomical data.
Additionally, it is sometimes difficult to find agreement among published histological analyses because many studies used unconventional planes of section in attempts to take true cross-sections of the tract, a fiber bundle that hugs the curves of the perimeter of the thalamus.

Anterograde labeling of retinal quadrants reveals that dorsal RGC axons course through the medial optic tract and ventral RGC axons through the lateral tract \cite{chan1999changes,chan1994changes,plas2005pretarget,reese1993reestablishment,reese1990fibre,reh1983organization}.
However, in the distal optic tract of the mouse, just prior to axon entry into the dLGN, dorsal RGC axons are found laterally and ventral RGC axons medially, opposite to their position in the proximal tract \cite{plas2005pretarget}. %Double check
This may be a reflection of a twist in the optic tract as a whole, which is evident in wholemount views of the surface of the thalamus, and also corresponds to the eventual dorsal-ventral mapping of RGCs along the lateral-medial axis in the SC \cite{plas2005pretarget}.
The chemical or mechanical factors that create this positional shift of dorsal and ventral axons in the tract are unknown.
It is noteworthy that in studies that directly examined the topographic order of retinal axons in the tract, in both mammalian \cite{chan1994changes,plas2005pretarget,reese1993reestablishment} and non-mammalian species \cite{ehrlich1984course,montgomery1998organization,reh1983organization,thanos1983investigations}, most conclude that the segregation between dorsal and ventral retinal axons is more distinct compared to that between nasal or temporal retinal axons, which are often described as being positioned adjacent, mixed, or otherwise overlapping in the tract.

Retinofugal axons next navigate the optic chiasm, the midline choice point in binocular visual systems, where ipsi- and contralaterally projecting RGC axons from each eye diverge into the optic tract of the appropriate hemisphere.
Although degree of binocularity varies across species, axon reorganization at the chiasm is a key step in the development of the visual circuit and raises the question of how ipsi- and contralateral RGC axons are organized in the optic nerve prior to reaching the chiasm.
By backlabeling retinal axons from the optic tract, studies in rodents concluded that, similar to retinotopy in the nerve, some order exists early in the nerve between future ipsi- and contralaterally projecting RGC axons, but is progressively lost as the axons near the chiasm \cite{baker1989distribution,colello1990early}.
More recent approaches utilizing a genetic marker of ipsilateral identity confirm the segregation of ipsilateral axons from their contralateral neighbors in the nerve, but suggest that a greater degree of order is retained in the distal optic nerve than previously thought (Sitko and Mason, unpublished).

The development of genetic tools and identification of RGC subtype-specific markers (e.g. \cite{blackshaw2004genomic,dhande2014retinal}) will afford even more precise analyses of the axon organization of molecularly defined RGC subtypes.
In the meantime, two subtypes of RGC subtypes can be defined based on their trajectory at the chiasm of binocular animals: those projecting ipsi- or contralaterally.
A moderate degree of order between ipsi- and contralateral axon cohorts, or “eye-specific” order, has been described in the optic tract of cats \cite{torrealba1982studies}.
In neonatal mice, contralateral axons are spread across the entire optic tract, with ipsilateral axons constrained to the anterior-lateral edge \cite{godement1984prenatal} (Figure 1C). %Edit all this
The earliest-born RGCs in the mouse extend axons from the dorsocentral retina and project either ipsi- or contralaterally.
Even though the early ipsilateral RGCs arising from this region are transient, their axons segregate in the optic tract from those of their contralaterally-projecting counterparts, underscoring the presence of eye-specific order in the tract \cite{soares2015transient}.

%Note that this arrangement differs from topographic order in mammalian tracts. (move into retinogeniculate org section?)
Furthermore, Reh et al. hypothesize the presence of attractive cues along the dorsal neuraxis that could preferentially act upon ventral RGC axons, providing a tract-specific mechanism of active axon organization that would occur independently of target derived cues \cite{reh1983organization}.