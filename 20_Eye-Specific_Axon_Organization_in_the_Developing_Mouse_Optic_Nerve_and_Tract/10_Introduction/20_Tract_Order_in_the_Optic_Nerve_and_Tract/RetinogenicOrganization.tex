Retinofugal axons next navigate the optic chiasm, the midline choice point in binocular visual systems, where ipsi- and contralaterally projecting RGC axons from each eye diverge into the optic tract of the appropriate hemisphere.
Although degree of binocularity varies across species, axon reorganization at the chiasm is a key step in the development of the visual circuit and raises the question of how ipsi- and contralateral RGC axons are organized in the optic nerve prior to reaching the chiasm.
By backlabeling retinal axons from the optic tract, studies in rodents concluded that, similar to retinotopy in the nerve, some order exists early in the nerve between future ipsi- and contralaterally projecting RGC axons, but is progressively lost as the axons near the chiasm \cite{baker1989distribution,colello1990early}.
More recent approaches utilizing a genetic marker of ipsilateral identity confirm the segregation of ipsilateral axons from their contralateral neighbors in the nerve, but suggest that a greater degree of order is retained in the distal optic nerve than previously thought (Sitko and Mason, unpublished).

The development of genetic tools and identification of RGC subtype-specific markers (e.g. \cite{blackshaw2004genomic,dhande2014retinal}) will afford even more precise analyses of the axon organization of molecularly defined RGC subtypes.
In the meantime, two subtypes of RGC subtypes can be defined based on their trajectory at the chiasm of binocular animals: those projecting ipsi- or contralaterally.
A moderate degree of order between ipsi- and contralateral axon cohorts, or “eye-specific” order, has been described in the optic tract of cats \cite{torrealba1982studies}.
In neonatal mice, contralateral axons are spread across the entire optic tract, with ipsilateral axons constrained to the anterior-lateral edge \cite{godement1984prenatal} (Figure 1C). %Edit all this
The earliest-born RGCs in the mouse extend axons from the dorsocentral retina and project either ipsi- or contralaterally.
Even though the early ipsilateral RGCs arising from this region are transient, their axons segregate in the optic tract from those of their contralaterally-projecting counterparts, underscoring the presence of eye-specific order in the tract \cite{soares2015transient}.
