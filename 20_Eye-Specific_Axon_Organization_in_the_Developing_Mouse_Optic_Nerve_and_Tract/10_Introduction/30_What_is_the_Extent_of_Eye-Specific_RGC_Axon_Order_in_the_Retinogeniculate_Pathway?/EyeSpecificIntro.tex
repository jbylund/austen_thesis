Chronotopic organization in the optic tract may be a reflection of some elements of RGC subtype organization.
As discussed earlier, RGC soma and axon diameter were used as early proxies of RGC subtype, and different sized RGCs are produced at different times during retinogenesis \cite{rapaport1995spatiotemporal,reese1994birthdates}.
There are newer genetic tools that allow highly specific identification of RGC subtype-specific markers (e.g. \cite{blackshaw2004genomic,dhande2014retinal,baden2016functional,rivlin2011transgenic}), which could be used to more directly test whether or not RGC subtypes are typographically organized in the optic nerve and tract.%decide on best refs to use here

In the meantime, one of the simplest classifications of RGC subtype is based on trajectory choice at the optic chiasm in binocular animals.
Relative to topographic and chronotopic axon order, typographic organization between ipsi and contra RGC axons, also referred to as eye-specific order, has not been well examined.
Backlabeling RGC axons from the optic tract in adult rats and hamsters shows that some order exists early in the nerve between future ipsi- and contralaterally projecting RGC axons, but is progressively lost as the axons near the chiasm \cite{baker1989distribution}.
A similar study in embryonic mice also showed that axons in the optic nerve ipsilateral to the optic tract backfill were clustered to the lateral portion of the early optic nerve, but more scattered closer to the chiasm \cite{colello1990early}.
The authors note that despite gross clustering of ipsilateral axons in the optic nerve closest to the retina, there are ipsilateral axons in many different fascicles, including in the medial nerve \cite{colello1990early}.
Thus, what ipsilateral RGC axon sorting does occur prior to the optic chiasm, is relatively coarse.

The picture of eye-specific order in the optic tract, after axons exit the chiasm, is even less clear.
One study in cats found a very moderate degree of order between ipsi- and contralateral axon cohorts in the optic tract, using a combination of degeneration and dye tracing experiments \cite{torrealba1982studies}.
The authors note that only the most dorsal-medial (in a frontal view) portion of the tract is free of ipsilateral fibers, and ipsilateral fibers are scattered throughout the rest of the tract \cite{torrealba1982studies}.
However, the authors propose that the retinotopic and eye-specific maps in the tract are mainly a result of age-related entry into the tract, rather than active topographic or typographic organization.
Whether or not that conclusion is accurate, versus a model in which topography, chronotopy, and typography are all present relatively equivalently, is still unclear.
%Furthermore, it is also not clear to degree to which these observations in the cat apply to the mouse visual system, which has a much smaller ipsilateral population.

Only one study has examined post-chiasmatic eye-specific axon order in the murine optic tract \cite{godement1984prenatal}.
In this study, intraocular HRP injections were performed in one eye of either late embryonic or early neonatal mice in order to assess the projection patterns of ipsi and contra RGCs into the dLGN and SC.
While the focus of this paper was mostly on the timing of ipsi and contra axon ingrowth to the dLGN and SC, some observations on the distribution of ipsi and contra fibers in the optic tract were also made.
Specifically, contra axons were observed spread across the entire optic tract, while ipsi axons appeared more constrained to the anterior-lateral edge of the tract, but appear more uniformly spread through the tract more dorsally, near the dLGN \cite{godement1984prenatal}.
This study provided a good description of the developmental time course of eye-specific RGC axon innervation of the thalamic targets, but was somewhat limited by the fact that it only labeled a single eye, rendering direct ipsi/contra comparisons within the same hemisphere impossible.
Two-color tracing studies have since built upon these early findings and improved our understanding of the time course of ipsi/contra axon ingrowth and refinement in the dLGN and SC (e.g., \cite{jaubert2005structural}).
Likewise, our understanding of the eye-specific organization of RGC axons in the mouse optic tract can be improved and extended.

In this chapter, I have aimed to provide a more detailed assessment of the eye-specific RGC axon organization in both the optic nerve and tract of the developing mouse.
A clearer understanding of how eye-specific axons are ordered relative to each other, and relative to the topographic map in the optic nerve and tract, is necessary in order to study organizational mechanisms and explore the relationship between tract organization and targeting.
While this question has been addressed by the few studies listed above, our view of axon organization in the developing mouse retinogeniculate pathway is still very limited, largely due to the limitations of the technical and imaging approaches available at the time.
Thus, I will first describe my use of a genetic reporter line to trace ipsilateral RGC axons in the optic nerve, chiasm, and tract.
In the second section, I will present my findings using more classical labeling approaches, aided by higher resolution imaging, to directly compare ipsi and contra RGC axon cohorts in the optic tract.
Finally, I will report on a series of labeling experiments I conducted to directly compare topographic organization in the optic tract with the eye-specific map identified in the previous experiments.