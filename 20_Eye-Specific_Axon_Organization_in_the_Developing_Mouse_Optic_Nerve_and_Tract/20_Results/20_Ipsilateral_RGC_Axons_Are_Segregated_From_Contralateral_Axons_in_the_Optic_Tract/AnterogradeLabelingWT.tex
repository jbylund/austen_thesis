In order to more directly compare the position of ipsi and contra RGC axons in the optic tract, I used classic anterograde tracing with the carbocyanine lipophilic dye tracer DiI and its red-shifted analog DiD (Molecular Probes, Inc., more details in Methods Section).
%add sec ref for methods
DiI and DiD, when placed directly onto the optic nerve head of a fixed head with the eyecup removed, completely or near-completely label the RGC projection from the labeled eye.
Thus, by labeling the right optic nerve head with DiI and the left optic nerve head with DiD, I acquired two-color images of the optic tract to analyze the relative position of ipsi and contra RGC axons in the tract (schematized in Figure~\ref{Figures/DiIDiDWT}A).

\begin{figure}[hbtp]
    \begin{center}
        \includegraphics{Figures/DiIDiD_WT.pdf}
        \caption[Ipsi and contra RGC axons are segregated in the developing optic tract.]
        {Ipsi and contra RGC axons are segregated in the developing optic tract.
        A) Cartoon depicting the labeling scheme.
        DiI was placed onto the optic nerve head of the right eye, DiD likewise in the left eye of fixed embryonic or postnatal heads.
        After incubation to allow the transport of the lipophilic dye tracers, horizontal sections were taken through the tract.
        B) Horizontal sections taken approximately 150$\mu$m dorsal to the optic chiasm illustrate ipsi/contra RGC axon segregation in the optic tract across embryonic and early postnatal development.
        Ipsi axons are consistently found in the lateral edge of the optic tract, an area that contra axons largely avoid.
        Ipsi RGC axons are labeled with DiD (pseudocolored green) and contra RGC axons are labeled with DiI (pseudocolored magenta).
        R=rostral, C=caudal, M=medial, L=lateral. Scale=100$\mu$m}
        \label{Figures/DiIDiDWT}
    \end{center}
\end{figure}

I performed two-color labeling on a series of embryonic and early postnatal samples from wild-type C57BL/6J mice and collected 75$\mu$m thick sections through the entire optic tract in both frontal and horizontal sections.
Figure~\ref{Figures/DiIDiDWT} shows representative sections in the horizontal plane of the optic tract from E15, E16, E18, P0, and P2 mice.
Horizontal sections most clearly illustrate the eye-specific axon segregation, though the segregation can also be appreciated in the frontal view (data not shown). %Add frontal view figure?
At E15 and E16, ipsi axons from the VT retina have just begun to enter the tract and cluster to the lateral edge of the tract.
Other ipsi RGC axons are more sparsely positioned in the medial optic tract.
These spares, medially-positioned axons disappear by later embryonic ages are likely from the earliest-born ipsi RGCs in the dorsal-central retina, a transient population \cite{drager1985birth} whose axons course in the medial optic tract \cite{soares2015transient}.

Later, at E18, the crispness of ipsi/contra segregation is more apparent, as increasing numbers of VT-ipsi RGC axons make their way into the tract.
The segregation of ipsi axons to the lateral tract, and the avoidance of the lateral tract by contra axons is most striking at P0, and continues through P2, when most, if not all, RGC axons have reached the optic tract.
Moreover, this organization is maintained throughout the full length of the tract, even as the tract as a whole changes shape, widening and flattening slightly as it nears the geniculate (Figures~\ref{Figures/DiIDiDWT_Serial} \& ~\ref{DiIDiD_WholeTract}).
Figure~\ref{Figures/DiIDiDWT_Serial} shows every third serial section through a P0 tract, illustrating the consistency of eye-specific axon segregation through the extent of the tract.
In the nascent dLGN itself (second and third sections from the left in Figure~\ref{Figures/DiIDiDWT_Serial}), ipsi axons only barely start to enter into the target, while proportionally more contra axons have done so.
Figure~\ref{DiIDiD_WholeTract} shows a tiled image of a cleared, slightly flattened wholemounted P0 optic tract.
The wholemount is from a hemisected brain with the cortex removed (see Figure~\ref{Figures/HighResIpsiContra}A for a diagram of the wholemount preparation).
The full structure of the optic tract as it curves around the perimeter of the thalamus can be appreciated in Figure~\ref{DiIDiD_WholeTract}.
The curvature in this view is less pronounced than it is in actuality, because the wholemount was partially flattened for imaging.
Figure~\ref{DiIDiD_WholeTract}A shows the maximum projection of the image stack, acquired from the lateral to medial aspect.
The optic is positioned to the right of the image (not visible), and the fibers of the tract can be seen spreading progressively wider as they near and spread around the outside of the dLGN.
Figure~\ref{DiIDiD_WholeTract}B shows the individual optical sections from the maximum projection image, in which it is clear that the ipsi RGC axons (green) are situated in the lateral aspect of the tract.
Additionally, the ipsi axons are most well bundled in the middle of the rostral-caudal axis of the tract, with some less tightly bundled ipsi axons along the caudal portion of the tract.
\begin{figure}[hbtp]
    \begin{center}
        \includegraphics[width=\textwidth]{Figures/DiIDiD_WT_Serial.pdf}
        \caption[Ipsi and contra RGC axons are segregated throughout the optic tract.]
        {Ipsi and contra RGC axons are segregated throughout the optic tract.
        Ipsi RGC axons maintain their position in the lateral edge of the optic tract as they extend from the ventral aspect (near the optic chiasm, at right) dorsally through the dLGN (at left).
        As the tract nears the LGN, it widens along the lateral side of the thalamus, and ipsi axons continue to cluster in the lateral-most aspect, with a slight anterior bias.
        Every third section through the optic tract of a P0 WT C57BL/6J mouse is shown, cute 75$\mu$m thick in the horizontal plane, same as shown in Figure~\ref{Figures/DiIDiDWT}.
        Ipsi RGC axons are labeled with DiD (pseudocolored green) and contra RGC axons are labeled with DiI (pseudocolored magenta).
        R=rostral, C=caudal, M=medial, L=lateral.
        Scale=100$\mu$m.}
        \label{Figures/DiIDiDWTSerial}
    \end{center}
\end{figure}
\begin{figure}[hbtp]
    \begin{center}
        \includegraphics{Figures/DiIDiD_WholeTract.pdf}
        \caption[Wholemounted intact optic tract.]
        {Wholemounted intact optic tract.
        Low resolution tiled confocal image stack of a cleared, flattened wholemount of the optic tract labeled with DiI (contra RGC axons, magenta) and DiD (ipsi RGC axons, green).
        A) Maximum projection of the full image stack.
        B) Single optical sections of the image stack (optical slice=9.11$\mu$m), progressing from medial (1) to lateral (16).
        16 of 24 total optical sections shown.
        R=rostral, C=caudal, D=dorsal, V=ventral.
        dLGN=dorsal lateral geniculate nucleus.
        Scale=200$\mu$m.
        }
        \label{Figures/DiIDiDWholeTract}
    \end{center}
\end{figure}

I acquired high resolution images of an intact, wholemounted optic tract at P0, labeled as before with DiI in the right retina and DiD in the left retina (Figure~\ref{Figures/HighResIpsiContra}).
Using a multiphoton confocal microscope with a 40X objective (see Methods Section for more detail) to image cleared wholemount samples allowed a tremendously detailed view of the optic tract.
%add Methods sec ref
In the single optical slices shown in Figure~\ref{Figures/HighResIpsiContra}B-C of the proximal and distal optic tract, respectively, ipsi and contra axons can be seen bundling together to varying extents, with some large bundles of each type visible, but also single axons of different types mingling in close proximity to each other.
The orthogonal views provide an overview of the tract for each image.
The change in shape of the tract from the proximal to distal aspect can be seen here, with the distal tract flattening out (Figure~\ref{Figures/HighResIpsiContra}C) as it nears the dLGN, compared with the more rounded proximal tract (Figure~\ref{Figures/HighRes_IpsiContra}B).
The important element to appreciate in the orthogonal views is the overall degree of segregation between ipsi and contra RGC axons: while instances of individual ipsi and contra fibers commingling are apparent in the optical slices shown, on the population level (seen in the orthogonal views), the two axon cohorts are well segregated.
Furthermore, there appear to distinct fascicles, of varying sizes, that are ipsi- and contra-specific - these can be seen as different sized puncta in the orthogonal images.
\begin{figure}[hbtp]
    \begin{center}
        \includegraphics{Figures/HighRes_IpsiContra.pdf}
        \caption[High resolution wholemount views of ipsi and contra RGC axons in the P0 optic tract.]
        {High resolution wholemount views of ipsi and contra RGC axons in the P0 optic tract.
        A) Cartoon showing the wholemount tract preparation.
        After labeling with DiI and DiD, the brain is removed and then the cortex is gently peeled away from the thalamus.
        The cortex-free sample is either hemisected or kept intact and cleared to enhance imaging.
        The wholemount sample is positioned such that the microscope objective is viewing the tract directly along the perimeter of the thalamus.
        The z-dimension progresses from lateral (superficial) to medial (deep).
        B-C) One optical slice and corresponding orthogonal view of the proximal (B) and distal (C) optic tract.
        Contra RGC axons are magenta, ipsi axons are green.
        V=ventral, D=dorsal, R=rostral, C=caudal, M=medial, L=lateral.
        Scale=100$\mu$m.
        }
        \label{Figures/HighResIpsiContra}
    \end{center}
\end{figure}

Given that ipsi RGCs originate from a topographically restricted area in the retina (the VT retinal crescent), one could argue that the segregation of ipsi and contra axons is merely a result of their topographic origins in the retina, rather than true typographic organization.
However, the developmental progression of the retina provides a way to directly distinguish between topographic and typographic order.
As shown previously in Figure~\ref{Figures/RGP_DevSeries}, there are three phases of retinal development.
In the late stage of RGC axon outgrowth, from approximately E17 until birth, the VT retina produces both ipsi and contra RGCs \cite{drager1985birth}.
Thus, both ipsi and contra RGCs can be labeled within the same restricted topographic zone, allowing us to determine if a typographic map exists within the topographic map.
If the results shown in Figures~\ref{Figures/DiIDiDWT} \&~\ref{Figures/DiIDiDWTSerial} were merely a reflection of topography in the optic tract, then labeling only the VT retina of both eyes would theoretically lead to overlapping labeling in the tract.
If however, a typographic map of eye-specific RGC projections exists independently of or within a topographic map in the tract, a VT-specific label in both retinae would lead to largely non-overlapping labeling in the optic tract.
This experimental design and its results are shown in Figure~\ref{Figures/DiIDiDWT_VTVT}A and B-C, respectively.
At P0 and P2, most if not all of the late-born VT contra and VT ipsi axons will have entered the optic tract.
%The overlap between ipsi and contra axons arising from a small DiI and DiD label in the right and left retinas, respectively, can be seen as the white area in the merged image at both ages.
While there is a small degree of overlap between the ipsi and contra VT labels, the two populations are largely segregated from each other, demonstrating, as did the whole-eye labeling, the lateral position of ipsi RGC axons in the optic tract.
Because focal anterograde labeling of the retina is inherently variable, I show in Figure~\ref{Figures/DiIDiDWT_VTVT}C the labeled retinae and both hemispheres of optic tract from another P0 sample for further examination.
Segregation of ipsi VT axons from contra VT axons is apparent in each hemisphere and when comparing the two hemispheres in each single-channel image.
\begin{figure}[hbtp]
    \begin{center}
        \includegraphics{Figures/DiIDiD_WT_VTVT.pdf}
        \caption[Ipsi and contra RGC axons arising from the same topographic region are segregated throughout the optic tract.]
        {Ipsi and contra RGC axons arising from the same topographic region are segregated throughout the optic tract.
        A) Cartoon depicting the labeling scheme.
        A small amount of DiI and DiD was placed onto the ventrotemporal (VT) retina of right and left intact retinas, respectively, of fixed embryonic or postnatal heads.
        After incubation to allow the transport of the lipophilic dye tracers, horizontal sections were taken through the tract.
        B) Horizontal sections taken approximately 150$\mu$m dorsal to the optic chiasm illustrate the segregation of VT-ipsi and VT-contra RGC axons in the optic tract.
        At P0 and P2, when the late-born contingent of VT contra RGCs has had time to cross the chiasm and enter the tract, it is apparent that ipsi/contra segregation exists, with ipsi axons biased laterally and VT-contra axons biased more medially.
        Ipsi RGC axons are labeled with DiD (pseudocolored green) and contra RGC axons are labeled with DiI (pseudocolored magenta).
        C) Another sample showing the ipsi and contra segregation of axons from VT retina at P0.
        Labeled retinas are shown at top, followed by a section approximately 150$\mu$m dorsal to the optic chiasm showing the ipsi and contra projections from both retinae.
        R=rostral, C=caudal, M=medial, L=lateral, D=dorsal, V=ventral, T=temporal.
        Scale=100$\mu$m in B, 500$\mu$m in C.}
        \label{Figures/DiIDiDWTVTVT}
    \end{center}
\end{figure}
