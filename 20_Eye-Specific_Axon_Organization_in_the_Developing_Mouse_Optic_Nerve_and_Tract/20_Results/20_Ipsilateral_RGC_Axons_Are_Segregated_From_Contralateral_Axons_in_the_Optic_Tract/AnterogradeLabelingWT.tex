In order to more directly compare the position of ipsi and contra RGC axons in the optic tract, I used the classic anterograd tracing approach with the carbocyanine lipophilic dye tracer DiI and its red-shifted analog DiD.
DiI and DiD, when placed directly onto the optic nerve head of a fixed head with the eyecup removed, completely or near-completely label the RGC projection from the labeled eye.
Thus, by labeling the right optic nerve head with DiI and the left optic nerve head with DiD, I could acquire two-color images of the optic tract to analyze the relative position of ipsi and contra RGC axons in the tract (schematized in Figure~\ref{Figures/DiIDiD_WT}A).

\begin{figure}[hbtp]
    \begin{center}
        \includegraphics{Figures/DiIDiD_WT.pdf}
        \caption[Ipsi and contra RGC axons are segregated in the developing optic tract.]
        {Ipsi and contra RGC axons are segregated in the developing optic tract.
        A) Cartoon depicting the labeling scheme.
        DiI was placed onto the optic nerve head of the right eye, DiD likewise in the left eye of fixed embryonic or postnatal heads.
        After incubation to allow the transport of the lipophilic dye tracers, horizontal sections were taken through the tract.
        B) Horizontal sections taken approximately 150$\mu$m dorsal to the optic chiasm illustrate ipsi/contra RGC axon segregation in the optic tract across embryonic and early postnatal development.
        Ipsi axons are consistently found in the lateral edge of the optic tract, an area that contra axons largely avoid.
        Ipsi RGC axons are labeled with DiD (pseudocolored green) and contra RGC axons are labeled with DiI (pseudocolored magenta).
        R=rostral, C=caudal, M=medial, L=lateral. Scale=100$\mu$m}
        \label{Figures/DiIDiD_WT}
    \end{center}
\end{figure}

I performed this labeling on a series of embryonic and early postnatal samples from wild type (WT) C57BL/6J mice and collected 75$\mu$m thick sections through the entire optic tract in both frontal and horizontal sections.
Figure~\ref{Figures/DiIDiD_WT} shows representative sections in the horizontal plane of the optic tract from E15, E16, E18, P0, and P2 mice.
Horizontal sections most clearly illustrate the eye-specific axon segregation, though the segregation can also be appreciated in the frontal view (data not shown). %Add frontal view figure?
At E15 and E16, ipsi axons from the VT retina have just begun to enter the tract and cluster to the lateral edge of the tract.
Visible in these sections are other ipsi RGC axons more sparsely running through the medial optic tract.
These disappear by later embryonic ages are likely the axons of the earliest-born ipsi RGCs in the dorsal-central retina, which are a transient population \cite{drager1985birth} and have been shown to course in the medial optic tract \cite{soares2015transient}.

Later, at E18, the crispness of ipsi/contra segregation is more apparent, as increasing numbers of VT-ipsi RGC axons make their way past the chiasm and into the tract.
The segregation of ipsi axons to the lateral tract, as well as the avoidance of the same area by contra axons is most striking at P0, and continues through P2, when most, if not all, RGC axons have reached the optic tract.
Moreover, this organization is maintained throughout the full length of the tract, even while the tract as a whole changes shape and flattens out as it nears the geniculate.
Figure~\ref{Figures/DiIDiD_WT_Serial} shows every third serial section through a P0 tract, illustrating the consistency of eye-specific axon segregation through the extent of the tract.
In the nascent dLGN itself (second and third sections from the left in Figure~\ref{Figures/DiIDiD_WT_Serial}), ipsi axons only barely start to enter into the target, and, as seen in Figure~\ref{Figures/Sert_dLGN}, remain largely unramified.

\begin{figure}[hbtp]
    \begin{center}
        \includegraphics[width=\textwidth]{Figures/DiIDiD_WT_Serial.pdf}
        \caption[Ipsi and contra RGC axons are segregated throughout the optic tract.]
        {Ipsi and contra RGC axons are segregated throughout the optic tract.
        Ipsi RGC axons maintain their position in the lateral edge of the optic tract as they extend from the ventral aspect (near the optic chiasm, at right) dorsally through the dLGN (at left).
        As the tract nears the LGN, it widens along the lateral side of the thalamus, and ipsi axons continue to cluster in the lateral-most aspect, with a slight anterior bias.
        Every third section through the optic tract of a P0 WT C57BL/6J mouse is shown, cute 75$\mu$m thick in the horizontal plane, same as shown in Figure~\ref{Figures/DiIDiD_WT}.
        Ipsi RGC axons are labeled with DiD (pseudocolored green) and contra RGC axons are labeled with DiI (pseudocolored magenta).
        R=rostral, C=caudal, M=medial, L=lateral. Scale=100$\mu$m}
        \label{Figures/DiIDiD_WT_Serial}
    \end{center}
\end{figure}

Given that ipsi RGCs originate from a topographically restricted area in the retina (the VT retinal crescent), one could argue that the segregation of ipsi and contra axons is merely a result of their topographic origins in the retina, rather than a true typographic organization.
However, the developmental progression of the retina provides a way to directly distinguish between topographic and typographic order.
As shown previously in Figure~\ref{Figures/RGP_DevSeries}, there are three phases of retinal development.
In the late stage of RGC axon outgrowth, from approximately E17 until birth, the VT retina produces both ipsi and contra RGCs \cite{drager1985birth}.
This means that both ipsi and contra RGCs can be labeled within the same restricted topographic zone, allowing us to see if a typographic map exists within the topographic map.
If the results shown in Figures~\ref{Figures/DiIDiD_WT}\&~\ref{Figures/DiIDiD_WT_Serial} were merely a reflection of topography in the optic tract, then labeling only the VT retina of both eyes would theoretically lead to overlapping labeling in the tract.
If however, a typographic map of eye-specific RGC projections exists independently of or within a topographic map in the tract, a VT-specific label in the retina would lead to largely non-overlapping labeling in the tract.
This experimental setup and its results are shown in Figure~\ref{Figures/DiIDiD_WT_VTVT}A and B-C, respectively.
At P0 and P2, most if not all of the late-born VT contra and VT ipsi axons will have entered the optic tract.
The overlap between ipsi and contra axons arising from a small DiI and DiD label in the right and left retinas, respectively, can be seen as the white area in the merged image at both ages.
While there is a small degree of overlap between the ipsi and contra VT labels, the two populations are still largely segregated from each other, again reflecting the lateral position of ipsi RGC axons in the optic tract.
Given the inherent variability in anterograde labeling of the retina, Figure~\ref{Figures/DiIDiD_WT_VTVT}C shows the labeled retinae and both hemispheres of optic tract from another P0 sample.
The segregation of ipsi VT axons from contra VT axons is apparent in each hemisphere, and also when comparing the two hemispheres in each single-channel image.

\begin{figure}[hbtp]
    \begin{center}
        \includegraphics{Figures/DiIDiD_WT_VTVT.pdf}
        \caption[Ipsi and contra RGC axons arising from the same topographic region are segregated throughout the optic tract.]
        {Ipsi and contra RGC axons arising from the same topographic region are segregated throughout the optic tract.
        A) Cartoon depicting the labeling scheme.
        A small amount of DiI and DiD was placed onto the ventrotemporal (VT) retina of right and left intact retinas, respectively, of fixed embryonic or postnatal heads.
        After incubation to allow the transport of the lipophilic dye tracers, horizontal sections were taken through the tract.
        B) Horizontal sections taken approximately 150$\mu$m dorsal to the optic chiasm illustrate the segregation of VT-ipsi and VT-contra RGC axons in the optic tract.
        At P0 and P2, when the late-born contingent of VT contra RGCs has had time to cross the chiasm and enter the tract, it is apparent that ipsi/contra segregation exists, with ipsi axons biased laterally and VT-contra axons biased more medially.
        Ipsi RGC axons are labeled with DiD (pseudocolored green) and contra RGC axons are labeled with DiI (pseudocolored magenta).
        C) Another sample showing the ipsi and contra segregation of axons from VT retina at P0.
        Labeled retinas are shown at top, followed by a section approximately 150$\mu$m dorsal to the optic chiasm showing the ipsi and contra projections from both retinae.
        R=rostral, C=caudal, M=medial, L=lateral, D=dorsal, V=ventral, T=temporal. Scale=100$\mu$m in B, 500$\mu$m in C.}
        \label{Figures/DiIDiD_WT_VTVT}
    \end{center}
\end{figure}
