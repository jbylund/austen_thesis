Ipsilateral RGC axon organization appears to be dynamic along the length of the retinogeniculate pathway.
Using SERT expression as a marker of ipsi identity in the SERT-Cre:zsgreen mouse, zsgreen+ axons can be seen exiting the retina and entering the ventral optic nerve.
They continue to be relatively well bundled through the proximal optic nerve and progressively splay apart as the nerve approaches the optic chiasm.
This is in line with previous reports indicating a progressive loss of topopgraphic or eye-specific axon order in the nerve \cite{colello1990early,chan1994changes,chan1999changes}, although my results indicate a greater degree of organization in the nerve than previously thought \cite{baker1989distribution}.
The presence of the secondary, more loosely bundled set of zsgreen+ axons in the medial optic nerve is surprising, given the robust bundling of the majoriting of zsgreen+ axons to the ventral-lateral aspect of the nerve.
This secondary bundle is consistently observable in all of the samples I analyzed.
Given that the heterogeneity of the ipsi RGC population is still unclear, this secondary medial bundle of zsgreen+ axons could represent a particular subset of ipsi RGCs.
Alternatively, if SERT is not as specific to ipsi RGCs as has been thought (see Section~\ref{sec:TechnicalSERT}), this secondary bundle could be yet a different subset of RGC axons altogether.
If a Zic2 reporter mouse line is made, it could used to trace the ipsi RGC projection more reliably.
Such a study would clarify the questions surrounding the appearance of a primary and secondary zsgreen+ axon bundle in the SERT-Cre:zsgreen line used here.

Ipsi RGC axons are defasciculated and spread out within the optic chiasm, but once in the tract, they regroup in the early tract and are robustly bundled in the lateral aspect of the tract for the rest of their journey to the dLGN and SC.
Two-color anterograde labeling shows a clear segregation of ipsi axons away from contra axons.
Postnatally, some contra axons appear to invade the lateral portion of the tract (the ipsi zone), but the two populations remain largely separate along the lateral-medial axis of the optic tract.
The position of ipsi axons in the lateral optic tract, identified here either via SERT-Cre:zsgreen expression or with DiI and DiD, corresponds to the early report by Godement et al. (1984).
The direct comparison between ipsi and contra RGC axons presented in this chapter affords far greater detail than was previously observed \cite{godement1984prenatal}.
Furthermore, I also tracked the eye-specific axon organization through late embryonic and early postnatal development, which had not yet been done.
While 

%



The earliest-born RGCs in the mouse extend axons from the dorsocentral retina and project either ipsi- or contralaterally.
Even though the early ipsilateral RGCs arising from this region are transient, their axons segregate in the optic tract from those of their contralaterally-projecting counterparts, underscoring the presence of eye-specific order in the tract \cite{soares2015transient}.


Do ipsi axons arrive later to the dLGN than contras? 
They do in cats \cite{shatz1983prenatal} and probably also in syrian hamsters \cite{frost1979postnatal} and ferrets \cite{linden1981dorsal}
but maybe not in mice \cite{dhande2011development}
Thus what I see may be a reflection more of the proportion of ipsi/contra entering the dLGN at P0, and it's not as though I see heavily ramified contra axons in there anyway



This late-born population of VT contra RGCs appears to be unique in some ways relative to the rest of the contra population \cite{williams2006role}.


apparently Ipsi/contra are not segregated in ON, according to \cite{baker1989distribution}