\label{sec:TechnicalSERT}
%Both the genetic and anterograde labeling techniques I used in this chapter have limitations that, while theoretically minor, should still be addressed.
I used genetic labeling of SERT to track the ipsi RGC population through the optic nerve and tract based on the robust evidence and generally accepted conclusion that SERT is specific to ipsi RGCs \cite{garcia2010zic2,koch2011pathway,upton1999excess}.
However, the possibility remains that a small subset of contra RGCs also express SERT.
This possibility is raised in part by the expression pattern of SERT in the retina, which extends farther dorsally and nasally than the ipsi-specific zone is thought to extend \cite{upton1999excess}.
Whether a subset of contra RGCs, either within or outside of the VT retina, also express SERT has yet to be directly tested, and as such, I cannot definitively state that all of the zsgreen+ axons analyzed in Section~\ref{sec:SertCreResults} are indeed only ipsilateral RGC axons.

A former student in the Mason Lab used \emph{SERT} along with \emph{Zic2} as ipsi-specific control genes in a microarray comparing ipsi and contra RGCs at E16.5. %Cite Wang when accepted
Her results show a strong correlation between \emph{Zic2} and \emph{SERT} enrichment in the ipsi RGC subpopulation, which was collected by performing a rapid backfill from the optic tract, followed by fluorescence-activated cell sorting (FACS). %delete abbrev if not used again later
Additionally, there is relatively no enrichment of either \emph{Zic2} and \emph{SERT} in the contra RGC population, arguing that SERT is indeed ipsi-specific.
However, because these experiments were conducted at E16.5, they do not contain any of the late-born VT contra RGCs in their analysis.
As such, it is possibile that a subset of later-born contra RGCs express SERT.
In a similar vein, it is also unclear if SERT is expressed by 100\% of ipsi RGCs, or some fraction of the ipsi population.
It is generally thought that the majority of ipsi RGCs express SERT \cite{garcia2010zic2,koch2011pathway}, but the fraction of ipsi RGCs that do not express SERT is not known, nor is the significance of such a possible heterogeneity in SERT expression among ipsi RGCs.

The theoretically ideal way to directly assess what proportion of ipsi, and possibly contra, RGCs express SERT is to backlabel RGC axons from the optic tract and analyze the overlap of SERT expression in the retina of a SERT reporter mouse and the backfilled RGCs in the ipsi and contra retinae.
This experiment has yet to be reported in the literature, likely because it is significantly hampered by technical concerns pertaining to the onset of reporter fluorescence and the time at which a retrograde tracing can be successfully performed from the optic tract.
\emph{SERT} mRNA is detectable in RGCs as early as E14.5 \cite{garcia2010zic2} and SERT protein is detectable in axons as early as E15.5 \cite{upton1999excess}, but Cre-recombinase mediated expression of zsgreen creates a lag time between onset of \emph{SERT} expression and onset of zsgreen expression.
Specifically, we find that only a very small fraction of RGCs express zsgreen earlier than E17.5 (data not shown), and zsgreen expression is not robust until E18.5-P0, whereas backlabeling RGCs from the optic tract tends to be unsuccessful when performed later than E17.5.
(I attempted, unsuccessfully, to retrogradely trace RGC axons in E18.5 SERT-Cre:zsgreen embryos.)
These two technical limitations make it incredibly challenging to reliably determine if retrogradely labeled contra RGCs express \emph{SERT}, or what fraction of the total ipsi RGC population expresses \emph{SERT}.
One way of avoiding the timing issue between fluorescent reporter onset and the time of backfill would be to use immunohistochemistry to label SERT.
However, this too presents a technical challenge, as all three anti-SERT antibodies (two commercial, one non-commercial) I tested do not work well in embryonic or neonatal tissue.

While it is reasonable to use \emph{SERT} as a reliable marker for ipsi RGCs, and indeed other studies have done so (e.g., \cite{koch2011pathway}, Wang 2016), it is still important to interpret the results presented in Figures~\ref{Figures/Sert_NerveToTract} and ~\ref{Figures/Sert_dLGN} in the context of the caveats discussed above. 
%Add Qing's paper when out
Specifically, the axons labeled by zsgreen shown in those images may contain a small number of contra RGC axons and/or may not label all ipsi RGC axons.
That being said, if the zsgreen+ axons in the optic nerve and tract contain both ipsi and some contra RGC axons, the ipsi fibers likely far outnumber the contras; likewise, if zsgreen+ axons do not represent the full ipsi population, it provides a reliable indication of the tendency of the ipsi population.
Thus, the observations and conclusions drawn from these experiments are, at worst, a good approximation of ipsi RGC axon distribution in the retinogeniculate pathway.
If SERT is indeed expressed in a subset of contra RGCs, it is intriguing to speculate that the smaller, more loosely bundled cohort of zsgreen+ axons in the medial optic nerve (Figure~\ref{Figures/Sert_NerveToTract}B).
The more that is learned about the transcriptional regulation and genetic identity of ipsi and contra RGCs, the better equipped we will be to address some of these issues and gain an even more detailed view of not only ipsi/contra axon organization, but also the organization of other RGC subtypes.
%again cite Qing, also some of the subtype stuff?

The rest of the experiments presented in this chapter utilized two-color anterograde tracing with carbocyanine lipophilic dye tracers, which have been workhorses of anatomical analyses for decades \cite{godement1987study,honig1989dil,gan2000multicolor}.
These dyes are very reliable, and anterograde labeling from the optic nerve head labels the entire RGC axonal projection from that retina.
Where variability is introduced is in the focal labeling experiments tracing the distribution of topographically-defined axonal cohorts.
There is no stereotaxic system for the retina, and as such the results of these labeling experiments are limited by experimenter variability.
In order to reduce that variability, I performed several rounds of the topographic and VT/VT labeling and found my results to be internally consistent.
This experimenter variability in dye placement should be a minor caveat, but bears mentioning.
In addition to other limitations mentioned previously regarding the existing literature on topographic axon organization in the optic nerve and tract (e.g., differing planes of section, different methods of tracing), consistency between research groups in how the dorsal-ventral and nasal-temporal axes are defined could be additional source of discrepancy between various reports.
%While slight inconsistencies or human error in dye placement for anterograde tracing is an acceptable source of variability for axon tracing experiments, I address it here for one specific reason.
%A useful extension of the experiments illustrated in Figure~\ref{Figures/DiIDiD_WT_VTVT}C and Figure~\ref{TopogIpsiContra} would be to create a 