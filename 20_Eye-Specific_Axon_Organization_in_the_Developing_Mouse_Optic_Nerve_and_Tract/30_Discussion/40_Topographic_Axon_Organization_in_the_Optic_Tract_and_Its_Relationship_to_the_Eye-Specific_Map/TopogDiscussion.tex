I undertook a reassessment of the topographic organization of axons in the optic tract to clarify the relationship between the topographic and typographic (eye-specific) maps in the mouse optic tract.
It was challenging to make this comparison between my data on eye-specific axon organization and existing reports on topographic axon organization because of general inconsistencies and ambiguities within the literature describing topographic order in the tract.
As such, I first performed hemiretina labeling of each topographic pole (Figure~\ref{Figures/SerialSectionsP0_Topog}) followed by smaller labeling in opposite (D/V, N/T) and adjacent (V/T) retinal quadrants, in which I directly compared the ipsi and contra projections from each topographic region (Figure~\ref{Figures/TopogIpsiContra}).
This is the first such direct comparison between the eye-specific and topographic projections of RGCs, providing a relatively comprehensive view of axon organization in the optic tract, which shows that the eye-specific map is shifted slightly laterally relative to the topographic map (Figure~\ref{Figures/TopographySummary}).

Previous studies by and large concluded that there was an absence of organization between nasal and temporal RGC axons in the optic tract \cite{chan1994changes,plas2005pretarget,reese1993reestablishment}.
My results, however, argue otherwise, indicating instead that there is a coarse rostral-medial to caudal-lateral organization of nasal and temporal axons.
Previous reports may have overlooked this organization by analyzing other planes of section, in which the rostal-caudal axis is less easily appreciated.
My observations generally agree with broad conclusions of dorsal and ventral RGC axon position in the tract from previous reports, which found a dorsal RGC axon to medial tract and ventral RGC axon to lateral tract organization \cite{chan1999changes,chan1994changes,plas2005pretarget,reese1993reestablishment,reese1990fibre,reh1983organization,torrealba1982studies}.%Check torrealba reference for specifics, review all
I, too, observe this medial-lateral organization, and additionally identify a slight caudal-rostral segregation between dorsal and ventral axons in the tract.
The optic tract is a challenging structure in which to define axon organization, as descriptions of rostal-caudal and medial-lateral axes must take into account both the curvature of the entire tract around the outer limit of the thalamus, as well as the slight crescent shape of the tract itself (when viewed in a horizontal plane).
As such, much of the ambiguity in the previous literature is understandable, especially as researchers often took unusual planes of section through the tract.
As high-resolution imaging becomes more readily available for large intact samples, three-dimensional reconstructions of the whole tract can be performed with different retinal labels to provide a more fullsome view of the axon distribution within the tract.

Since some of the first reports on topographic axon organization in the optic nerve and tract, researchers have speculated about the mechanisms giving rise to such order (e.g., \cite{cook1977multiple,reh1983organization}), but little headway has been made in this area.
The favored hypothesis very early on was based on mechanical forces and axon-axon interactions that maintained neighboring relationships between axons \cite{cook1977multiple}, and some evidence was given suggesting that extracellular tunnels and other structural alleys and paths were responsible for the orderly arrangement of axons exiting the retina \cite{silver1980mechanism}.
However, as increasing numbers of studies disagreed about the degree of orderliness in the optic nerve and tract \cite{martin1983role}, alternative explanations based on diffusable chemoattractive and repulsive cues became more popular \cite{reh1983organization}.
Reh et al. (1983) hypothesize that in the frog, putative attractive cues along the dorsal neuraxis could preferentially act upon ventral RGC axons, thus leading to their orderly arrangement.
This has yet to be proven one way or the other.

Some studies have speculated that axons lack order in the nerve and tract, only to regain retinotopic order near the dLGN, presumably based on local chemoaffinity in the dLGN \cite{horton1979non}.
While local cues in and near the targets are surely involved in the specificity of early target innervation, findings of axon organization in the pathway \cite{martin1983role,chan1999changes,chan1994changes,plas2005pretarget,reese1993reestablishment}, including those presented here, suggest that tract-specific cues organize axons prior to their arrival at the dLGN.
Specifically, the re-establishment of both retinotopic and eye-specific axon organization shortly after the optic chiasm suggests that there are active cues - either inter-axonal or diffusable cues inside the tract, or both - leading to and maintaining RGC axon organization in the tract.
My data show SERT-Cre:zsgreen+ axons bundled in the early optic nerve, followed by a span of loose bundling or unraveling through the chiasm, and subsequent regrouping into the lateral optic tract (Figure~\ref{Figures/Sert_NerveToTract}).
Likewise, many reports on the topographic axon order within the optic nerve describe a progressive loss of axon order as axons approach the chiasm, followed by a reestablishment of retinotopy within the tract \cite{chan1999changes,plas2005pretarget,chan1994changes,simon1991relationship,reese1993reestablishment,horton1979non,naito1986course,naito1994retinogeniculate,colello1998changing,montgomery1998organization,ehrlich1984course}. 
%add my frontal view sections of topography showing the optic nerve order?
All of these findings point to an active management of axon organization as they grow towards their targets, rather than a passive maintenance of organization, or an inconsequential or random order that gets sorted out upon arrival to the target zones.

The next chapter addresses one potential organizational mechanism of eye-specific axon order in the retinogeniculate pathway, specifically axon-axon interactions or fasciculation.
While, as Reh et al. (1983) postulate, there are likely cues within the tract that act preferentially upon different topographic, eye-specific, or functional subtypes of RGC axons, we first need both a better understanding of the molecular profiles of axon cohorts in the tract and of the tract itself, neither of which have been well studied.
A recently-published microarray screen of E16.5 ipsi and contra RGCs represents significant progress in our understanding of ipsi and contra gene profiles.  %cite Qing
This will serve as a good place to begin probing for candidate binding partners between RGC subtypes and cues in the tract.
To most strategically approach this question, gene profiling of the cells in and around the tract should also be done.
The radial glia and immature neurons in the optic chiasm region that provide guidance cues to ipsi and contra RGC axons have been explored (reviewed in \cite{erskine2014connecting,herrera2008genetics,petros2008retinal}), but very little is known about the glia and other cells within the nerve tract \cite{colello1992observations,guillery1987changing}.
Given our very limited knowledge of the cells and putative cues within the nerve and tract, as well as anecdotal evidence in our lab regarding selective fasciculation behavior between ipsi and contra RGC axons, I chose to examine whether the two cohorts of axons have different fasciculation behavior, and whether this behavior might contribute in part to their organization in the pathway.