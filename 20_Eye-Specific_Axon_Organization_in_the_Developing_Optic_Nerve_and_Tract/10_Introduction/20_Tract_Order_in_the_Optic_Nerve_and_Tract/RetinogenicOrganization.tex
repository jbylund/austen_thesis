\label{sec:RetinogeniculateOrganization}
As discussed in Chapter 1, the pre-target organization of retinal axons projecting from the eye to their thalamic targets, the dLGN and SC has been studied in a variety of species.
While subsets of RGC axons extend to other non-image-processing targets (e.g., the suprachiasmatic nucleus and ventral lateral geniculate nucleus), tracts projecting to these regions and retinotopy therein have been largely unstudied.
Instead, pre-target axon sorting has been examined along two segments of the retino-thalamic pathway: the pre-chiasmatic optic nerve and the post-chiasmatic optic tract.
This work has almost exclusively focused on topographic and chronotopic organization, which I will summarize below.

The literature on axon organization in the optic nerve, chiasm, and tract, while extensive, is often in disagreement.
In a review of the literature in 1983, disagreement was already apparent, with varying claims about the degree of retinotopy in the axon tracts \cite{martin1983role}.
Unfortunately, the inconsistencies have largely persisted in studies published since 1983.
This can be accounted for in part by the fact that studies used a wide variety of model organisms, from monkeys to sheep, cats, ferrets, rabbits, and a variety of rodents, including hyraxes, rats, and mice \cite{martin1983role}. 
But most of the disagreement in the literature likely stems from the diverse technical approaches employed to assess the degree of retinotopic axon order in the optic nerve and tract: silver stains of cut and degenerating axons, anterograde and retrograde labeling with horseradish peroxidase (HRP) or one of several other tracers, and electrophysiological recordings from axons were all utilized.
Furthermore, most of these studies were performed in adult animals, and as such, cannot shed light on the developmental processes leading to axon organization.
Perhaps most importantly, differing and often unconventional planes of section were used by different groups, rendering comparisons across studies challenging at best.
The medley of section planes reflects an effort to obtain true cross-section samples of the optic nerves and tracts, structures that hug the pronounced curve of the lateral perimeter of the thalamus.
As such, what one group refers to as a rostral-caudal segregation may or may not agree with what another group refers to as ventral-dorsal segregation. %Double check refs and make more specific

Despite these caveats, some general, if fuzzy, conclusions emerge from the literature.
As mentioned in Chapter 1, RGC axons appear to maintain a coarse topography in the proximal optic nerve of both mammalian \cite{chan1999changes,plas2005pretarget,chan1994changes,simon1991relationship,reese1993reestablishment,horton1979non,naito1986course,naito1994retinogeniculate} and non-mammalian species \cite{montgomery1998organization,ehrlich1984course}, but this order is blurred as axons approach the optic chiasm.
The fascicular composition of RGC axons then changes entirely within the optic chiasm, as axons from one eye split into many separate fascicles that cross contralateral axon bundles from the opposite eye at right angles, forming a braid-like configuration at the midline \cite{colello1998changing}.

While several cues determining whether or not an axon crosses the midline have been identified \cite{erskine2014connecting}, much of the logic underlying RGC axon organization in the chiasm remains unclear, especially with regard to topography (i.e., retinotopy).
In both rats and mice, anterograde labeling of retinal quadrants shows considerable blurring of retinotopy within the chiasm region itself, followed by reappearance of topographic order in the optic tract \cite{chan1994changes,chan1999changes}.
The mechanisms responsible for reestablishing topographic organization in the optic tract are unknown, but coarse topographic order is found in the optic tract of many species.
It is entirely possible that a conserved and logical organization of axons exists within the chiasm region, but has so far eluded detection.
As demonstrated by work in the lateral olfactory tract (see Section~\ref{sec:LOT} and \citenoparens{yamatani2004chronotopic}), one mode of organization may obscure another mode in a particular tract or region, leading investigators to erroneously conclude that order does not exist.
Therefore, technical approaches used thus far may lack a level of sophistication and precision necessary to detect axon organization in complex areas like the optic chiasm.

Once in the optic tract, topographic organization of RGC axons, including both ipsi and contra populations, becomes apparent again.
Anterograde labeling of retinal quadrants reveals that dorsal RGC axons course through the medial optic tract and ventral RGC axons through the lateral tract \cite{chan1999changes,chan1994changes,plas2005pretarget,reese1993reestablishment,reese1990fibre,reh1983organization,torrealba1982studies}.%Check torrealba reference for specifics, review all
However, in the distal optic tract of the mouse, just prior to axon entry into the dLGN, dorsal RGC axons are positioned laterally and ventral RGC axons medially, opposite to their position in the proximal tract \cite{plas2005pretarget}. %Double check
This may be due to a twist of the optic tract, which is evident in wholemount views of the surface of the thalamus, and corresponds to the eventual dorsal-ventral mapping of RGCs along the lateral-medial axis in the SC \cite{plas2005pretarget}.
The chemical or mechanical factors that create this positional shift of dorsal and ventral axons in the tract are unknown.
It is noteworthy that in studies that directly examined the topographic order of retinal axons in the tract, in both mammalian \cite{chan1994changes,plas2005pretarget,reese1993reestablishment} and non-mammalian species \cite{ehrlich1984course,montgomery1998organization,reh1983organization,thanos1983investigations}, most conclude that the segregation between dorsal and ventral retinal axons is more distinct compared to that between nasal or temporal retinal axons, which are often described as being positioned adjacent, mixed, or otherwise overlapping in the tract.

In fish and frogs, which continue to add new RGCs as the animal grows, topographic order can be found within the chronotopically ordered axon bundles in the optic tract.
Within the age-related axon bundles, dorsal and ventral RGCs extend axons laterally and medially, respectively, while nasal and temporal RGC axons are mixed together across much of the tract \cite{reh1983organization}.
Interestingly, this arrangement differs from the topographic order in mammalian tracts described above, in which dorsal and ventral RGC axons run in the medial and lateral tracts, respectively.
%Review the details of these studies. 
As mentioned in Chapter 1, RGC axons are also chronotopically ordered in the mammalian optic tract, as younger axons are added to the lateral edge of the tract, alongside the pia \cite{colello1992observations,reese1987distributionrat,reese1990fibre,reese1997chronotopic,walsh1985age}. %Check Reese 1987 citation
Again, as with retinotopic order, attempts to find age-related order within the optic chiasm have failed \cite{colello1998changing}.
%Is there chronotopy in the nerve? Check this.