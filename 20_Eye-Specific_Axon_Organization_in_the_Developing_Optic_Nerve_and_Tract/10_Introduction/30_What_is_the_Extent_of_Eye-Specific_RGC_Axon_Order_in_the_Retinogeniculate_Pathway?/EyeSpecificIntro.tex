Chronotopic organization in the optic tract may reflect elements of RGC subtype identity and organization.
As discussed in Chapter 1 (Sections~\ref{sec:Chronotopy} and ~\ref{sec:Typography}), RGC soma and axon diameter were used as early proxies of RGC subtype, and different sized RGCs are produced at different times during retinogenesis \cite{rapaport1995spatiotemporal,reese1994birthdates}.
Newer genetic tools afford highly specific identification of RGC subtype-specific markers (e.g. \citenoparens{blackshaw2004genomic,baden2016functional,rivlin2011transgenic,sanes2015types}), which could be used to more directly test whether or not RGC subtypes are typographically organized in the optic nerve and tract.
%decide on best refs to use here

Perhaps the simplest classifications of RGC subtype in binocular organisms is based on trajectory choice at the optic chiasm.
Relative to topographic and chronotopic axon order, typographic organization between ipsi and contra RGC axons, also referred to as eye-specific order, has not been well examined.
Retrogradely labeling RGC axons from the optic tract in adult rats and hamsters showed that order exists early in the nerve between future ipsi- and contralaterally projecting RGC axons, but is progressively lost as the axons near the chiasm \cite{baker1989distribution}.
A similar study in embryonic mice also showed that axons in the optic nerve ipsilateral to the labeled optic tract clustered in the lateral portion of the early optic nerve, but became more scattered closer to the chiasm \cite{colello1990early}.
The authors note that despite gross clustering of ipsilateral axons in the optic nerve closest to the retina, ipsilateral axons are in many different fascicles, including some positioned in the medial nerve \cite{colello1990early}.
Thus, ipsilateral RGC axon sorting prior to the optic chiasm was determined to be relatively coarse.

Whether eye-specific order exists in the optic tract is even less clear.
One study in cats found a moderate segregation of ipsi- and contralateral axon cohorts in the optic tract using a combination of degeneration and dye tracing experiments \cite{torrealba1982studies}.
Only the most dorsal-medial (in a frontal view) portion of the tract is free of ipsilateral fibers, and ipsilateral fibers are scattered throughout the rest of the tract \cite{torrealba1982studies}.
However, the authors propose that the retinotopic and eye-specific maps in the tract are mainly a result of age-related entry into the tract, rather than active topographic or typographic organization.
Whether or not this conclusion is accurate, rather than a model in which topography, chronotopy, and typography participate together more equally in tract organization, is still unclear.
%Furthermore, it is also not clear to degree to which these observations in the cat apply to the mouse visual system, which has a much smaller ipsilateral population.
An analysis using HRP intraocular labeling and autoradiography in the developing ferret reported slightly more clear eye-specific axon organization in the optic tract \cite{linden1981dorsal} than that reported by Torrealba et al. (1982) in the cat.
Specifically, ipsi/contra axon organization in the ferret optic tract increases during early postnatal development, with a crescent of the medial optic tract free of ipsi axons \cite{linden1981dorsal}.

Only one study has examined post-chiasmatic eye-specific axon order in the murine optic tract \cite{godement1984prenatal}.
Intraocular HRP injections were performed in one eye of late embryonic and early neonatal mice in order to assess the projection patterns of ipsi and contra RGCs into the dLGN and SC.
While the focus of this paper was the timing of ipsi and contra axon ingrowth to the dLGN and SC, some observations on the distribution of ipsi and contra fibers in the optic tract were also made: contra axons spread across the entire optic tract, while ipsi axons are more constrained to the anterior-lateral edge of the tract, but become more uniformly spread through the tract nearer the dLGN \cite{godement1984prenatal}.
This study describes the developmental time course of eye-specific RGC axon innervation of the thalamic targets, but is limited by the fact that one eye was labeled, rendering direct ipsi/contra comparisons within the same hemisphere impossible.
Two-color tracing studies have since built upon these early findings and improved our understanding of the time course of ipsi/contra axon ingrowth and refinement in the dLGN and SC (e.g., \citenoparens{jaubert2005structural}).
%Likewise, our understanding of the eye-specific organization of RGC axons in the mouse optic tract can be improved and extended.

In this chapter, I aim to provide a more detailed assessment of the eye-specific RGC axon organization in the optic nerve and tract of the developing mouse.
Only one section of the optic tract at two postnatal ages following intraocular injection of HRP was shown by Godement et al. \shortcite{godement1984prenatal}.
Thus our understanding of eye-specific axon organization in the tract during development and throughout the full extent of the tract remain incomplete.
A clearer view of how eye-specific axons are ordered relative to each other, and to the topographic map in the optic nerve and tract, is necessary in order to study organizational mechanisms and explore the relationship between tract organization and targeting.
%While this question has been addressed by the few studies listed above, our view of axon organization in the developing mouse retinogeniculate pathway is still very limited, largely due to the limitations of the technical and imaging approaches available at the time.
Thus, I first describe my use of a genetic reporter line to trace ipsilateral RGC axons in the optic nerve, chiasm, and tract.
In the second section, I present my findings using more classical labeling approaches, aided by higher resolution imaging, to directly compare ipsi and contra RGC axon cohorts in the optic tract.
Finally, I report on a series of labeling experiments to directly compare topographic organization in the optic tract with the eye-specific map identified in the first two sets of experiments.