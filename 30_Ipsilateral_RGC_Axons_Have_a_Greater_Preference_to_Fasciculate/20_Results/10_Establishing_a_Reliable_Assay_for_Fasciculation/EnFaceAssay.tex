In order to test the hypothesis that ipsi RGC axons self-fasciculate to a greater degree than contra RGC axons, I wanted to develop an in vitro assay that could reliably identify potentially subtle differences in fasciculation between the two populations.
Retinal explant culture systems have been used to address questions regarding axon outgrowth, target selection, and midline choice of retinal neurites \cite[e.g.][]{kuwajima2012optic,wang1996chemosuppression,bonhoeffer1985position}.
Our lab has optimized conditions for serum-free culturing of E14.5 retinal explants in order to study neurite outgrowth \cite[e.g.][]{kuwajima2012optic,petros2010ephrin,wang1996chemosuppression}.
While the retinal explant system does not provide a pure population of RGCs in vitro, RGCs are the only retinal cells that extend long neurites at this developmental stage, and therefore we assume the neurite outgrowth being studied is largely, if not completely, stemming from RGCs.
%clean up citations here, add a citation for RGCs being only axon extending cells at this age?

In this chapter, I will present three ways that I adapted this retinal explant system to assess differential fasciculation behaviors and axon-axon interactions of ipsi and contra retinal neurites.
In all of the experimental conditions in this chapter, I used E14.5 ventrotemporal (VT) and dorsotemporal (DT) retinal explants and refer to these, respectively, as ipsi and contra explants.
While VT retina is likely not 100\% ipsilateral at any point in development, E14.5 represents the peak divergence of ipsi and contra production -- i.e., VT retina is not yet producing the late-born VT-contra RGCs and appears to be primarily, if not exclusively, producing ipsi RGCs \cite{drager1985birth}.
Thus, while VT retina may not be a pure population of ipsi retinal cells, it should be sufficiently ipsi in its identity to make sound ipsi/contra comparisons between VT and DT explants.
While the use of VT and DT explants to represent ipsi and contra RGCs in culture is standard practice \cite[e.g.][]{kuwajima2012optic,petros2010ephrin}, I also performed a control experiment to validate ipsi identity in VT explants.
Immunostaining VT and DT explants at 4, 24, and 48 hours in vitro for the RGC marker, Islet1/2, and the ipsi RGC marker, Zic2, showed sufficiently high expression of Zic2 in the VT explants and nearly zero Zic2 expression in DT explants (see Methods for more details).
%Insert figure reference here

I will first describe an early attempt at creating an assay to study axon-axon interactions between like and unlike pairs of ipsi and contra neurites in vitro.
I refer to this assay as the \emph{en face} assay; the experimental design is summarized in Figure~\ref{Figures/EnFaceAssayDesign}.
The next two sections will cover two subsequent assays I developed, one of which, the bundle width assay, has been successful and the second, a novel Y-assay, is still in development.

\begin{figure}[hbtp]
    \begin{center}
        \includegraphics{Figures/EnFaceAssayDesign.svg}
        \caption[Experimental design for the \emph{en face} assay of axon-axon interaction.]
        {Experimental design for the \emph{en face} assay of axon-axon interaction.
		E14.5 embryos were harvested and decapitated.
		Either GFP or mCherry plasmid was injected via micropipet into either the dorsal or ventral retina, in the subretinal space.
		Heads are electroporated with the +paddle on the ventral side for a dorsal injection, or on the dorsal side for a ventral injection, so that the DNA is pulled into the retina from the subretinal space.
		The labeled eyes are dissected out with a ventral cut to indicate orientation, the lens is removed, and eyecups are segregated into left and right for both GFP and mCherry labels, before being incubated in serum free media (SFM) overnight.
		On the second day, dorsotemporal (DT) and ventrotemporal (VT) explants are dissected from labeled hemiretinas and plated in like (VT/VT or DT/DT) or unlike (VT/DT) pairs 1.5mm apart on 8$\mu$g/ml laminin, and cultured overnight.
		The next day, explants are fixed with warm 4\% paraformaldehyde (PFA) and immunostained for neurofilament (NF, to label all neurites), GFP and mCherry.
		}
        \label{Figures/EnFaceAssayDesign}
    \end{center}
\end{figure}
The rationale behind the \emph{en face} assay was to directly assess axon-axon interactions in a system with relatively low axon outgrowth, so as to allow for more detailed analyses than those normally afforded by explant cultures with robust outgrowth.
While robust neurite outgrowth allows for qualitative and often subjective assessments of neurite bundling or fasciculation, my goal was to analyze and quantify individual axon-axon interactions.
To do this, I used a lower concentration of laminin as a permissive growth substrate than our lab typically uses (8$\mu$g/ml laminin instead of the typical 10$\mu$g/ml).
I utilized \emph{ex vivo} retinal electroporation \cite{petros2009utero} to label the ventral or dorsal hemiretina at E14.5 with either GFP or mCherry.
After electroporation, each electroporated retina, with lens removed, was dissected from the head and incubated overnight in serum free media (SFM) (see Methods and Figure~\ref{Figures/EnFaceAssayDesign}).
%Add references for methods section and Figure when in place
The following day, VT or DT explants were dissected from the labeled hemiretinae and plated on 8$\mu$g/ml laminin in like (VT/VT or DT/DT) or unlike (VT/DT) pairs, with the center of each explant 1.5mm apart from each other.
Unlike pairs were balanced between VT/DT placement and DT/VT placement, to control for any possible effect of sidedness.
GFP and mCherry were similarly counterbalanced for side of placement and also for region - with an equal number of dorsal and ventral injections for each color label.
Explant pairs were incubated overnight in SFM+0.4\% methylcellulose and on the following day, fixed with warm 4\% paraformaldehyde (PFA) and immunostained for neurofilament (NF) and GFP and mCherry to amplify the signal of the electroporated fluorescent proteins.

The space between the two explants was then imaged and analyzed.
I excluded explant pairs that were +/-250$\mu$m away from being 1.5mm apart.
The area between the two explants was subsequently limited to the area between two parallel lines drawn 100$\mu$m away from the edge of the explant body (see Figure~\ref{Figures/EnFaceData}A).
%Check that it's actually 100 and not 150 or 200
I excluded the region closest to the explant body from analysis in this way because the neurites are so dense nearest the explant body, it makes quantification challenging and unreliable.
Thus, my analysis was limited to an area where neurite growth was less dense and I would be able to more reliably discern individual fasciculation events.

\begin{figure}[hbtp]
    \begin{center}
        \includegraphics{Figures/EnFaceData.svg}
        \caption[Ipsi neurites self-fasciculate more frequently than contra neurites.]
        {Ipsi neurites self-fasciculate more frequently than contra neurites.
		A) Screenshot from Neurolucida demonstrating the analysis method.
		A dorsotemporal (DT, contra) explant is on the left, and a ventrotemporal (VT, ipsi) explant is on the right.
		Explants have been immunostained for neurofilament (NF, in white), and for mCherry (the DT explant) and GFP (the VT explant) to amplify the electroporated fluorescent signal.
		Two parallel orange lines are drawn 100$\mu$m away from the explant bodies and demarcate the area of analysis.
		Neurite tips are marked with x for the left explant and + for the right explant, in this case a contra and ipsi unlike pair.
		(The same set-up is used for like pairs of VT/VT and DT/DT.)
		Cyan, pink, and orange lines are traced lengths of neurite-neurite interactions: intra-explant interactions in cyan and pink, and inter-explant interactions in orange (the latter has fewer instances, all of which are for short distances).
		B) VT explants have a higher frequency of fasciculation events than DT explants, as measured by the average number of fasciculation events per neurite tip.
		n=25 explants for DT, n=19 explants for VT, from four rounds of experiments. Calculation combines measures taken of intra-explant neurite-neurite interactions in both like and unlike pairs.
		Data split between the two pair conditions are similar (not shown).
		**=p<0.01, 1 tailed t-test.
		C-D) Axon bundling index (ABI) for neurite-neurite interactions within explants (C) and between explants (D).
		ABI measures the average length of neurite-neurite interactions.
		C) When grown in like pairs, the intra-explant fasciculation events of DT and VT explants are of similar lengths (left two columns).
		When grown in unlike pairs, the intra-explant fasciculation events of VT explants are longer than those of DT explants (right two columns).
		The reduction in ABI between DT explants grown in like to unlike pairs is not statistically significant (p=0.06, 1-tailed t-test).
		n=18 explants (9 pairs) for DT/DT, n=12 explants (6 pairs) for VT/VT, n=7 explants each for VT and DT in DT/VT pairs (7 pairs).
		*=p<0.05, 1-tailed t-test.
		D) Inter-explant fasciculation events, as measured by ABI, are relatively short (note the difference in scale of Y axis in C and D) in all conditions, with no statistical differences between DT/DT, VT/VT, and DT/VT pairs.
		n=8 explant pairs for DT/DT, n=5 explant pairs for VT/VT, n=5 explant pairs for DT/VT.
		One pair from each like pairing, and two pairs from unlike pairings had no inter-explant neurite interactions and thus were not included in the analysis (although the intra-explant measurements were included from those pairs).
		Error bars are standard error of the mean.
		}
        \label{Figures/EnFaceData}
    \end{center}
\end{figure}
The original intention was to only analyze the neurite-neurite interactions between explant pairs, to assess relative frequency and length of fasciculation events between neurites from the same or different type.
I hypothesized that VT/VT (ipsi/ipsi) neurites would fasciculate more (both in frequency of fasciculation events and length of fasciculation association) than DT/DT (contra/contra) neurite pairs, and both of those pairs would fasciculate more than the unlike pairing between VT and DT explants.
However, in all cases, fasciculation events between explant pairs (inter-explant interactions) occurred infrequently and in many cases, the neurites from the two explants avoided each other altogether.
This is not surprising, as the \emph{en face} setup forces growth cones to approach each other head-on, which is an unfavorable interaction for growing neurites and does not recapitulate a typical in vivo fasciculation event, in which growth cones are coursing in a similar trajectory.
However, given the infrequency of inter-explant neurite interactions, I adjusted my analysis to also include intra-explant neurite interactions - that is, within each explant of the pair, I analyzed the frequency and length of neurite-neurite interactions, in addition to the inter-explant fasciculatione events, if there were any.
Neurite interactions were quantified by manual tracing of fasciculation events in Neurolucida (v11, MBF Biosciences, Williston, VT, USA) and marking neurite tips.
A fasciculation event was defined as any time two neurites were observed joining together - the length of the interaction was traced until the neurites could be seen splitting apart, or if the neurites ended.
All discernable neurite endings, or tips, were marked.
Thus, I was able to calculate two aspects of both intra- and inter-explant neurite-neurite interactions: the frequency of fasciculation events (the number of fasciculation lengths measured divided by the number of neurite tips), and the axon bundling index (ABI) \cite[after][]{barry2010polarized}, a measure of the average length of fasciculation events (see Figure~\ref{Figures/EnFaceData}).

%Present the data.

The data from the \emph{en face} assay proved to be very noisy.
I re-analyzed the data categorizing fasciculation event length into three categories: crossings (x-x$\mu$m), short (x-x$\mu$m), and long (x-x$\mu$m) interactions.
This analysis, shown in Figure~\ref{} did not show any differences and failed to add any clarity to the data.
Additionally, the possibile experimenter variability could be very high in this analysis - tracing neurite interactions by hand might be inconsistent between experimenters.
While I do not have direct evidence to support that concern (I was only one to perform these analyses), it was enough of a concern that I next aimed to establish a more objective and more readily reproducible assay.