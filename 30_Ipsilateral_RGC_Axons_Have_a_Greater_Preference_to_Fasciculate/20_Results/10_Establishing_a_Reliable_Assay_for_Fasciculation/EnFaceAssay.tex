In order to test the hypothesis that ipsi RGC axons self-fasciculate to a greater degree than contra RGC axons, I wanted to develop an in vitro assay that could reliably identify potentially subtle differences in fasciculation between the two populations.
Retinal explant culture systems have been used to address questions regarding axon outgrowth, target selection, and midline choice of retinal neurites \cite[e.g.][]{kuwajima2012optic,wang1996chemosuppression,bonhoeffer1985position}.
Our lab has optimized conditions for serum-free culturing of E14.5 retinal explants in order to study neurite outgrowth \cite[e.g.][]{kuwajima2012optic,petros2010ephrin,wang1996chemosuppression}.
While the retinal explant system does not provide a pure population of RGCs in vitro, RGCs are the only retinal cells that extend long neurites at this developmental stage, and therefore we assume the neurite outgrowth being studied is largely, if not completely, stemming from RGCs.
%clean up citations here, add a citation for RGCs being only axon extending cells at this age?

In this chapter, I will present three ways that I adapted this retinal explant system to assess differential fasciculation behaviors and axon-axon interactions of ipsi and contra retinal neurites.
In all of the experimental conditions in this chapter, I used E14.5 ventrotemporal (VT) and dorsotemporal (DT) retinal explants and refer to these, respectively, as ipsi and contra explants.
While VT retina is likely not 100\% ipsilateral at any point in development, E14.5 represents the peak divergence of ipsi and contra production -- i.e., VT retina is not yet producing the late-born VT-contra RGCs and appears to be primarily, if not exclusively, producing ipsi RGCs \cite{drager1985birth}.
Thus, while VT retina may not be a pure population of ipsi retinal cells, it should be sufficiently ipsi in its identity to make sound ipsi/contra comparisons between VT and DT explants.
While the use of VT and DT explants to represent ipsi and contra RGCs in culture is standard practice \cite[e.g.][]{kuwajima2012optic,petros2010ephrin}, I also performed a control experiment to validate ipsi identity in VT explants.
Immunostaining VT and DT explants at 4, 24, and 48 hours in vitro for the RGC marker, Islet1/2, and the ipsi RGC marker, Zic2, showed sufficiently high expression of Zic2 in the VT explants and nearly zero Zic2 expression in DT explants (see Methods for more details).
%Insert figure reference here

I will first describe an early attempt at creating an assay to study axon-axon interactions between like and unlike pairs of ipsi and contra neurites in vitro.
I refer to this assay as the \emph{en face} assay; the experimental design is summarized in Figure~\ref{}.
%Add figure ref
The next two sections will cover two subsequent assays I developed, one of which, the bundle width assay, has been successful and the second, a novel Y-assay, is still in development.



The rationale behind the \emph{en face} assay was to directly assess axon-axon interactions in a system with relatively low axon outgrowth, so as to allow for more detailed analyses than those normally afforded by explant cultures with robust outgrowth.
While robust neurite outgrowth allows for qualitative and often subjective assessments of neurite bundling or fasciculation, my goal was to analyze and quantify individual axon-axon interactions.
To do this, I used a lower concentration of laminin as a permissive growth substrate than our lab typically uses (8$\mu$g/ml laminin instead of the typical 10$\mu$g/ml).
I utilized \emph{ex vivo} retinal electroporation \cite{petros2009utero} to label the ventral or dorsal hemiretina at E14.5 with either GFP or mCherry.
After electroporation, each electroporated retina, with lens removed, was dissected from the head and incubated overnight in serum free media (SFM) (see Methods and Figure~\ref{}).
%Add references for methods section and Figure when in place
The following day, VT or DT explants were dissected from the labeled hemiretinae and plated on 8$\mu$g/ml laminin in like (VT/VT or DT/DT) or unlike (VT/DT) pairs, with the center of each explant 1.5mm apart from each other.
Explant pairs were incubated overnight in SFM+0.4\% methylcellulose and on the following day, fixed with warm 4\% paraformaldehyde (PFA) and immunostained for neurofilament (NF) and GFP and mCherry to amplify the signal of the electroporated fluorescent proteins.




% in order to quantify how neurites of one type (ipsi or contra) interacted with neurites of the same or different type.
