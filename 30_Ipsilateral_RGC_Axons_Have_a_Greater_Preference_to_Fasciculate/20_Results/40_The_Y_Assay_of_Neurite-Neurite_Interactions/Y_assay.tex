Having reliably established an intrinsic preference for ipsi (VT) neurites to fasciculate more frequently (as shown by the \emph{en face} assay) and to a greater extent (as shown by the bundle width assay) than contra (DT) neurites, I wanted to establish a better way of assessing axon-axon interactions between axon types.
The limitation in doing this with the \emph{en face} assay was that having growth cones approach each other head-on was not conducive to fasciculation.
Thus, I thought of adapting the classic `Y' setup used by Bonhoeffer for testing target choice \cite{bonhoeffer1985position} for the purposes of instead assaying fasciculation and axon-axon interactions in the absence of a target.
The conceptualization of this experiment is illustrated in Figure~\ref{Figures/Y_Assay_Design}.
Dr. Franco Weth and his student Dr. Felix Fiederling, part of Dr. Martin Bastmeyer's group at Karlsruher Institut f\"ur Technologie, collaborated with us on this endeavor.
Microcontact printing is a method regularly used in the Bastmeyer group to produce patterned substrates of various proteins for the assessment of axon outgrowth and navigation, often using chick RGCs or retinal explants \cite{von2006microcontact}.
After some consultation, Drs. Weth and Fiederling produced silicone masters of a Y pattern to my specifications (Figure~\ref{Figures/Y_Assay_Design}A) and sent me several identical polydimethylsiloxane (PDMS) stamps to create the Y pattern in a laminin substrate.

The microcontact printing method is explained in detail by von Philipsborn et al. (2006), but I will provide a brief summary here.
I incubated 40$\mu$g/ml laminin on the PDMS stamps for 1-3 hr, rinsed the laminin solution off by dipping into distilled water (dH_20), and quickly dried the stamp with high purity nitrogen.
The stamp was immediately placed face-down onto a prepared glass coverslip (plasma cleaned and epoxysilanized (see Methods)) long enough to mark the pattern on the backside of the coverslip with a fine-tip permanent marker.
The stamp was then removed and the coverslip covered with warm SFM+0.4\% methylcellulose.
E14.5 VT or DT explants were plated in the two arms of the Y and allowed to grow for two nights (approximately 36-40hr), with fresh warm SFM+0.4\% methylcellulose added after one night in vitro.
Explants were fixed with a specialized fixative mix (see Methods) that aided maintenance of neurite adhesion to the coverslip, and cultures were immunostained for laminin and NF.
I then imaged and analyzed neurites in the stem of the Y, the lane in which neurites from each explant meet and navigate fasciculation choices.
\begin{figure}[hbtp]
    \begin{center}
        \includegraphics{Figures/Y_Assay_Design.pdf}
        \caption[Experimental design and hypothetical outcomes of the Y assay for fasciculation.]
        {Experimental design and hypothetical outcomes of the Y assay for fasciculation.
		A) Design of the Y pattern and summary of experimental steps.
		The black lines indicate the negative space on the PDMS stamps, i.e., the area which does not transfer laminin to the coverslip, creating the borders to a Y shape where laminin is transferred.
		B) Hypothesized outcomes of different explant pairs in the Y assay.
		Neurites arising from DT/DT pairs (contra) will likely fasciculate to a lesser extent than those from VT/VT (ipsi) pairs.
		Unlike pairs of DT/VT explants (contra/ipsi) will likely present a mix of responses, with VT neurites hyperfasciculating and DT neurites avoiding VT neurites by either defasciculating or forming a separate fascicle.
		VT=ventrotemporal (ipsilateral) retinal explant.
		DT=dorsotemporal (contralateral) retinal explant.
		}
        \label{Figures/Y_Assay_Design}
    \end{center}
\end{figure}
Based on my findings in the previously discussed assays, I anticipated that neurites from a DT/DT pair of explants would fasciculate, but to a lesser extent than neurites from a VT/VT pair (Figure~\ref{Figures/Y_Assay_Design}B).
In the unlike pair, I expected hyperfasciculation of the VT neurites and defasciculation or repulsion of the DT neurites.

The microcontact printing proved to be technically challenging, and many rounds of troubleshooting were necessary to successfully execute the experiment.
Aside from the mundane technical issues surrounding establishing a new technique (e.g., coverslip preparation, purity grade of our nitrogen source, both of which were different from our normal setup), three technical limitations remain, though to a lesser extent.
The first of these is the efficient transfer of laminin via microcontact printing.
In the bundle width assay, I used 20$\mu$g/ml laminin as a growth permissive substrate for the retinal explant cultures.
Here, too, I needed only a growth-permissive substrate, but applied via microcontact printing rather than incubating the coverslips with the laminin solution.
When I used microcontact printing to form the patterned laminin substrate at a concentration of 20$\mu$g/ml, explants consistently failed to effectively adhere to the coverslip, and those few that did often failed to extend more than few short neurites.
After adjusting several other parameters and continuing to find the same problem, I suspected that the microcontact printing approach was failing to transfer the laminin with 100\% efficiency.
With this suspicion in mind, I tested other laminin concentrations and found explant adhesion and outgrowth to both be much more robust and reliable at a concentration of 40$\mu$g/ml compared to 20$\mu$g/ml.

A second technical issue occured during the fixation step.
Using warm 4\% PFA to fix the cultures, as I and the rest of the lab does to fix other explant cultures, I found that neurites in the lane of the Y often peeled off the coverslip and folded up between the two explants.
This is likely due to the high levels of fasciculation in the lane of the Y, such that the entire fascicle is adhered to a very small overall surface area.
Having such limited surface area for neurite-substrate adhesion makes the cultures much more sensitive to any physical force, such as the flow of media or fixative across the coverslip.
Using a fixative mix suggested by Dr. Fiederling has lessened this issue.
Compared with 4\% PFA, a cocktail of PFA and glutaraldehyde in a 10\% sucrose solution better matches the viscosity of the growth media, which helps reduce the stress on neurite-substrate adhesion during the fixation step.

Finally, the paradigm in and of itself leads to greater inconsistency and a relatively low yield of successful explant pairs compared to the previous two assays discussed in this chapter.
The assay is dependent on the explant pairs adhering in exactly the position they have been placed during the plating step.
That is, if one or both of the explants shifts position before it is able to extend neurites, it will fail to extend neurites into the rest of the Y pattern.
Explants often shift tens of microns as they settle into their final positions and extend neurites, forming more points of contact with the coverslip and stabilizing their position on the coverslip.
Because of this, some fraction of explant pairs inevitably fail to grow into the Y, but instead sometimes around the outer perimeter

That being said, this approach appears to be less consistent overall than our traditional retinal explant setup
, and I have found these experiments to be relatively low yield

Figure~\ref{Figures/Y_Data_1ExampleEach} shows one example each for each explant pair type in the Y assay.
\begin{figure}[hbtp]
    \begin{center}
        \includegraphics{Figures/Y_Data_1ExampleEach.pdf}
        \caption[Preliminary results of the Y assay for fasciculation.]
        {Preliminary results of the Y assay for fasciculation.
		One example is shown of each explant pair type.
		From top to bottom: VT/VT (ipsi like pair), DT/DT (contra like pair), DT/VT (unlike pair).
		The left shows cultures immunostained for laminin (magenta) and neurofilament (NF, green) to label all neurites.
		The right-hand column shows the same cultures, zoomed in on the lane of the Y, and only showing the NF staining.
		The Y border is marked by a white dashed line.
		VT=ventrotemporal (ipsilateral) retinal explant.
		DT=dorsotemporal (contralateral) retinal explant.
		NF=neurofilament.
		Scale=250$\mu$m.
		}
        \label{Figures/Y_Data_1ExampleEach}
    \end{center}
\end{figure}