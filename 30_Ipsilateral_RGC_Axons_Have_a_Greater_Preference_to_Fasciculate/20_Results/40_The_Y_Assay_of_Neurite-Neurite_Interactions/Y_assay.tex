Having reliably established an intrinsic preference for ipsi (VT) neurites to fasciculate more frequently (as shown by the \emph{en face} assay) and to a greater extent (as shown by the bundle width assay) than contra (DT) neurites, I wanted to establish a better way of assessing axon-axon interactions between axon types.
The limitation in doing this with the \emph{en face} assay was that having growth cones approach each other head-on was not conducive to fasciculation.
Thus, I thought of adapting the classic `Y' setup used by Bonhoeffer for testing target choice \cite{bonhoeffer1985position} for the purposes of instead assaying fasciculation and axon-axon interactions in the absence of a target.
The conceptualization of this experiment is illustrated in Figure~\ref{Figures/Y_Assay_Design}.
Dr. Franco Weth and his student Dr. Felix Fiederling, part of Dr. Martin Bastmeyer's group at Karlsruher Institut f\"ur Technologie, collaborated with us on this endeavor.
Microcontact printing is a method regularly used in the Bastmeyer group to produce patterned substrates of various proteins for the assessment of axon outgrowth and navigation, often using chick RGCs or retinal explants \cite{von2006microcontact}.
After some consultation, Drs. Weth and Fiederling produced silicone masters of a Y pattern to my specifications (Figure~\ref{Figures/Y_Assay_Design}A) and sent me several identical polydimethylsiloxane (PDMS) stamps to create the Y pattern in a laminin substrate.

The microcontact printing method is explained in detail by von Philipsborn et al. (2006), but I will provide a brief summary here.
I incubated 40$\mu$g/ml laminin on the PDMS stamps for 1-3 hr, rinsed the laminin solution off by dipping into distilled water (dH_20), and quickly dried the stamp with high purity nitrogen.
The stamp was immediately placed face-down onto a prepared glass coverslip (plasma cleaned and epoxysilanized (see Methods)) long enough to mark the pattern on the backside of the coverslip with a fine-tip permanent marker.
The stamp was then removed and the coverslip covered with warm SFM+0.4\% methylcellulose.
E14.5 VT or DT explants were plated in the two arms of the Y and allowed to grow for two nights (approximately 36-40hr), with fresh warm SFM+0.4\% methylcellulose added after one night in vitro.
Explants were fixed with a specialized fixative mix (see Methods) that aided maintenance of neurite adhesion to the coverslip, and cultures were immunostained for laminin and NF.
I then imaged and analyzed neurites in the stem of the Y, the lane in which neurites from each explant meet and navigate fasciculation choices.
\begin{figure}[hbtp]
    \begin{center}
        \includegraphics{Figures/Y_Assay_Design.pdf}
        \caption[Short caption]
        {Long caption}
        \label{Figures/Y_Assay_Design}
    \end{center}
\end{figure}
I hypothesized outcomes based on explant pairing, illustrated in Figure~\ref{Figures/Y_Assay_Design}B.
Based on my findings in the previously discussed assays, I anticipated that neurites from a DT/DT pair of explants would fasciculate, but to a lesser extent than neurites from a VT/VT pair.
In the unlike pair, I expected hyperfasciculation of the VT neurites and defasciculation or repulsion of the DT neurites.



\begin{figure}[hbtp]
    \begin{center}
        \includegraphics{Figures/Y_Data_1ExampleEach.pdf}
        \caption[Short caption]
        {Long caption}
        \label{Figures/Y_Data_1ExampleEach}
    \end{center}
\end{figure}