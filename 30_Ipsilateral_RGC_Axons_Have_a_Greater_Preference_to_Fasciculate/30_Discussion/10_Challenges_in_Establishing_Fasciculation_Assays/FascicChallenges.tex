Assaying fasciculation in vitro provides a way to directly test intrinsic fasciculation behaviors of different cohorts of axons.
Especially in a tract like the murine optic tract, in which we know very little about the cells and molecular cues present within the tract, it is important to understand the intrinsic fasciculation behaviors of axon cohorts.
This knowledge is necessary for eventually probing the extrinsic molecular cues within the tract, so as to build a full understanding of the symphony of intrinsic and extrinsic cues that both guide and organization axon cohorts in their tracts.

There are some general challenges to all in vitro assays of fasciculation, which are important to keep in mind when interpreting results from such assays.
One issue, which I have tried to address with the assays presented in this chapter, is that of objective and reliable quantification of fasciculation that can detect subtle differences in fasciculation behavior.
Some fasciculation assays use on qualitative assessment of gross differences in degree of fasciculation {what to cite}, while others use quantification approaches that detect very large differences (e.g., \citenoparens{jaworski2012autocrine}), but would likely fail to detect more subtle differences, which, while smaller in magnitude, may be as meaningful as more easily detectable differences.
Understanding subtle differences in intrinsic fasciculation behaviors may be key for understanding how axons organize within a tract.
Ipsi and contra RGCs, for instance, while distinct cohorts, are members of the same neuronal type (RGCs). 
Thus, one would expect any differences in fasciculation behavior to be smaller within a neuronal type (e.g., ipsi and contra RGCs) than between two neuronal types (e.g., RGCs and OSNs).
Therefore, as we probe further into the organization of axon tracts and the molecular mechanisms creating such order, it will be imperative to use assays that have a high detection resolution of potentially small but meaningful differences between different cohorts and subclasses of axons.

It was with this goal of discerning potentially subtle differences in ipsi and contra RGC fasciculation that I established one of the central parameters in the \emph{en face} assay: a low concentration of laminin substrate to reduce neurite outgrowth so as to facilitate tracking individual fasciculation events.
However, this reduction in neurite outgrowth, and concomitantly fewer overall fasciculation events, might have contributed to the very large amount of variability in my data from this assay.
Furthermore, manually tracking fasciculation events introduces experimenter variability and an undesirable amount of subjectivity to the analysis.
The final strike against the \emph{en face} assay was the fact that it failed to effectively test neurite-neurite interactions between two explants, which was one of the original goals of creating a fasciculation assay.
Thus, the \emph{en face} assay failed to fulfill the goal of establishing an objective and reliable fasciculation assay that could detect subtle differences.
This effort, however, highlights the challenges surrounding development of such an assay - some of the ways in which I tried to make the assay more precise instead led to greater overall variability.

The bundle width assay successfully overcame some of the limitations of the \emph{en face} assay, but at the cost of limiting the type of data it can produce.
Namely, I sacrificed the ability to compare neurite-neurite interactions between like and unlike pairs of explants for greater reliability in the readout of a simpler metric - the neurite bundle width within single explants grown in isolation.
This was a reasonable approach to take, given that the \emph{en face} assay largely failed at producing a reliable readout of inter-explant neurite interactions anyway.
Furthermore, the ABI measurement from the \emph{en face} assay provided data on average length of a fasciculation event predominantly from intra-explant neurite interactions, given the small number of inter-explant interactions.
But because fasciculation is likely a highly dynamic process, it is unclear how informative fasciculation length measurements are in a static, fixed culture.
Bundle width might be a more reliable measure of degree of fasciculation than fascicle length, and because the ABI measurements were mostly intra-explant in the \emph{en face} assay, the bundle width assay, which only assess fasciculation within single explants, seems a better option overall.

One point to note is that the \emph{en face} assay showed statistically significant differences between VT and DT ABI only when explants were plated in unlike pairs, but not so when they were plated in like pairs.
This suggests that the context an explant grows in affects its fasciculation behavior, at least as measured by the average length of a fasciculation event.
Therefore, VT and DT explants and/or their neurites might communicate with each other via contact-mediated or diffusible cues to affect each other's self-fasciculation and/or repulsion from unlike neurites.
However, in order to directly test this hypothesis, data from the bundle width assay is likely a necessary first step.
That is, a clearer understanding of the intrinsic fasciculation behaviors of each explant type in isolation is necessary in order to interpret results from co-explant experiments.
Overall, the bundle width assay provides more objective and reliable data, which also lend themselves to more straightforward interpretations.

Finally, the Y assay attempts to overcome the shortcomings of the \emph{en face} assay in regards to measuring neurite-neurite interactions between two explants.
While the \emph{en face} assay presented two explants head-to-head, a situation in which opposing growth cones are more likely to collapse or turn away than fasciculate together, the Y assay presented two explants in such a way that they could meet in a lane growing nearly in parallel with each other.
The limitations of the Y assay are largely technical in nature, mainly centering around the technical minutiae of microcontact printing.
Discussed in the final results section of this chapter, these technical limitations predominantly hinge on the efficient transfer of laminin as a growth-permissive substrate onto the coverslip used to culture the explants.
Because microcontact printing is not as consistent or efficient a route of application compared with direct incubation of the coverslip with a laminin solution, there is greater overall variability in the adhesive and growth-permissive nature of the laminin substrate laid down via microcontact printing.
This leads to a greater frequency of shifting in explant position, which renders several explant pairs unsuitable for analysis (as one or both explants shift outside of the border of the Y pattern and subsequently fail to send their neurites into the Y lane).
Additionally, I have found a greater amount of variation in extent of neurite outgrowth in the microcontact printed cultures compared with the traditional culture system.

The final way in which variability in the laminin substrate leads to overall greater variability in the assay as well as a low yield of successful experiments occurs during the fixation step.
Using warm 4\% PFA to fix the cultures, neurites in the lane of the Y often peeled off the coverslip and folded up between the two explants.
This is likely due to the high levels of fasciculation in the lane of the Y, such that the entire fascicle is adhered to a small overall surface area.
This limited surface area for neurite-substrate adhesion combined with variability in the laminin substrate, renders the neurites highly sensitive to physical forces, such as the flow of media or fixative across the coverslip.
Dr. Fiederling suggested a fixative mix of PFA and glutaraldehyde in a 10\% sucrose solution, which has greatly ameliorated this problem and helps maintain the integrity of the explant culture through fixation and immunostaining.