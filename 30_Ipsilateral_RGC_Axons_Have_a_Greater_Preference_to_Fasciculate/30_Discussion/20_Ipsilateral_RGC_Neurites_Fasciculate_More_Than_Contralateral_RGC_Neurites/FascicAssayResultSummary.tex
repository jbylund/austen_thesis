Despite the limitations of the \emph{en face} assay discussed above and in the results section, it did reliably indicate that E14.5 VT (ipsi) neurites fasciculate more frequently than DT (contra) neurites (Figure~\ref{Figures/EnFaceData}B).
This confirms, in part, the hypothesis that ipsi axons fasciculate to a greater extent than contra axons.
The bundle width assay then showed very clearly that VT (ipsi) neurites form thicker fascicles than DT (contra) neurites when grown as single explants (Figures~\ref{Figures/BundleWidth}B).
These two separate measures of fasciculation taken together provide robust support for the conclusion that ipsi neurites, as assessed by E14.5 VT explants, fasciculate to a greater degree than contra neurites (E14.5 DT explants), both in terms of frequency of fasciculation events and bundle width.


%
Relate to in vivo data from sert labeling - dynamic fasciculation

Then discuss the chiasm cells experiment
and generally what might be going on in the chiasm and after the chiasm

