Fasciculation, the bundling together of axons, is an important mechanism in axon guidance and circuit formation, as nearly all axons grow along other axons as they extend towards their targets.
The exception to this is pioneer axons, the first axons to traverse a pathway and set the stage for subsequent axons to follow along specific paths \cite{raper2010cellular,wang2013axons}.
There have been many elegant experiments on the pioneer-follower paradigm of fasciculation and the role and mechanisms of pioneer axons.
These studies have provided a great deal of knowledge regarding isotypic axon-axon adhesion and led to the discovery of many cell adhesion molecules (CAMs) \cite{raper2010cellular,wang2013axons}.
These canonical studies have been nicely reviewed by both Raper \& Mason (2010) and Wang \& Marquardt (2013), and make up the bulk of research on fasciculation.
However, selective fasciculation is also involved in organizing developing axon tracts.
It is this latter function of fasciculation that is pertinent to the research presented in this dissertation.
I will first review some of the work on fasciculation and axon-axon interactions within developing tracts, outside of the pioneer-follower paradigm, before introducting my work studying the fasciculation behaviors of ipsi and contra RGC axons.

Given the variety of forces involved in axon bundling, fasciculation can be a challenge to study.
First among these forces is axon-axon adhesion mediated by CAMs \cite{wang2013axons,van1998adhesion}.
%It can be difficult sometimes to experimentally differentiate between axon-axon attraction and axon-substrate repulsion.
Recognition among axons also factors into fasciculation.
Axons can recognize similar and unlike axons in culture, leading to growth cone collapse upon meeting a neurite from an unlike explant, and forming tight fascicles when near many unlike processes \cite{kapfhammer1986selective}.
Axons can also recognize self and other identity within a given subtype - this specific self-identification is what leads to precise tiling of axonal (and dendritic) arbors \cite{grueber2010self}.
Recognition of molecules on other axons or on intermediate target or guidepost cells or other surrounding substrata can induce signaling cascades within the axon \cite{bashaw2010signaling,wang2013axons}.
These signaling cascades can in turn mediate changes in the axon's response to surrounding signals.
Finally, external cues, including the permissiveness of the environment to growth, number of optional paths for the axon to choose amongst, and glial interactions can all factor into the extent and nature of axon fasciculation \cite{wang2013axons}.

Additionally, the extent to which fasciculation between axons occurs predominantly at the level of the growth cone or along the axon shaft itself is unknown.
Early work focused exclusively on growth cone responses to neurites of the same or different type \cite{nakajima1965selectivity}, but the dynamics of fasciculation remain poorly understood and it is likely that axons expression CAMs and other molecules along the shaft as well as in the growth cone. 
%Eventually could cite the zippering paper here?
Further examination of axon-axon interactions along axon shafts distal to the growth cone will require better labels for known guidance molecules and high resolution microscopy.