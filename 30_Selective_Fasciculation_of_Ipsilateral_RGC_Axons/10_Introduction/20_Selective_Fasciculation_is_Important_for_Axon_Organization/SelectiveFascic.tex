Studies on the fasciculation between sensory and motor axons in the peripheral nerves and limb buds have identified guidance molecules that act both as guidance factors and effectors of fasciculation.
Motor and sensory axons closely associate with each other but form separate axon bundles as they extend in reciprocal directions - this segregation is even recapitulated in vitro \cite{gallarda2008segregation}.
Semaphorins (Semas) and Neuropilin (Nrp) are classic molecules involved in axon guidance, which can also mediate fasciculation and defasciculation \cite{raper2000semaphorins,bashaw2010signaling,kolodkin2011mechanisms}.
Sema3A in the early limb bud drives fasciculation of Nrp1-expressing lateral motor column (LMC) motor neuron (MN) axons by surround repulsion \cite{huber2005distinct}.
Perturbing Nrp1-Sema3A signaling causes the the motor axons to defasciculate, which in turn appears to lead to misrouting in the limb \cite{huber2005distinct}.
In addition to the fasciculation of MN axons by surround repulsion from a local environmental cue, MN and sensory axons also communicate with each other to mediate their organization.
Both motor and sensory axons express Nrp1, and selective ablation of Nrp1 from either population highlights the interdependence of the two populations on each other \cite{huettl2011npn}.
Removal of Nrp1 from motor axons leads to motor axon defasciculation and pathfinding errors, but largely spares sensory axons, which appear to need minimal scaffolding from motor axons.
However, removal of Nrp1 from sensory axons leads to defasciculation of both sensory and motor axons \cite{huettl2011npn}.

Nrp and Semas are also involved in axonal fasciculation and organization in the olfactory system.
Loss of Nrp2 causes vomeronasal sensory axons to defasciculate, likely through a Sema3F mediated interaction \cite{cloutier2002neuropilin}.
These defasciculated axons inappropriately enter the main olfactory bulb, rather than their correct target, the accessory olfactory bulb \cite{cloutier2002neuropilin}.
Nrp1 and Sema3A are also involved in axon-axon interactions between subsets of OSN axons in the olfactory nerve \cite{imai2009pre}.
Conditional deletion of one molecule or the other in specific OSN subsets perturbs the center-periphery organization of those OSN axons in the olfactory nerve, and leads to concomitant shifts in the glomerular positioning in the OB \cite{imai2009pre}.

Ephs and ephrins, also canonical guidance molecules \cite{bashaw2010signaling,kolodkin2011mechanisms} are involved in heterotypic axon fasciculation, in both the peripheral nerves and the corpus callosum (CC).
Trans-axonal interactions between sensory axons expressing ephrin-A2 -A5, and medial motor column (MMC) MN axons expressing EphA3 and A4, lead to their segregated fasciculation in the developing limb \cite{gallarda2008segregation}.
Perturbing ephrin-A:EphA signaling leads to marked defasciculation and misrouting of motor axons into proximal sensory pathways \cite{gallarda2008segregation}.
Furthermore, properly patterned motor axons are necessary for the appropriate extension of sensory axons, interactions also mediated by ephrin-A:EphA signaling between sensory and motor axons \cite{wang2011anatomical}.
EphA3 is also involved in the dorsal-ventral segregation of medial and lateral cortical axons within the corpus callosum \cite{nishikimi2011segregation}.
This is presumably via a direct inter-axon signaling mechanism, as blocking EphA3 in co-cultured medial and lateral cortical explants in vitro disrupted their segregation \cite{nishikimi2011segregation}.

Class B Ephs and ephrins are also involved fasciculation and organization of developing sensory and motor axons \cite{luxey2013eph}.
Ephrin-B1 is expressed in the limb bud mesenchyme and in sensory axons, while both sensory and motor axons express EphB2.
A series of experiments demonstrated that ephrin-B1 mediates fasciculation of the two axonal cohorts via surround repulsion, although motor axons still made correct dorsal-ventral guidance choices \cite{luxey2013eph}.
Defasciculation was more robust, however, in a full ephrin-B1 knocokout compared to a tissue-specific knockout of ephrin-B1 in the limb bud mesenchyme, suggesting that there may be an axon-axon interaction mediating selective fasciculation, in addition to surround repulsion \cite{luxey2013eph}.

Finally, Slit/Robo signaling controls fasciculation of motor axons \cite{jaworski2012autocrine}.
Experiments in vitro showed that Slit2 is produced by the motor axons themselves and, via Robo1 and 2, leads to their fasciculation, in an autocrine and/or juxtaparacrine fashion \cite{jaworski2012autocrine}.
The authors speculate that Slit/Robo signaling may affect cadherin expression, thus mediating motor axon fasciculation in an indirect way as well, but this has yet to be demonstrated.