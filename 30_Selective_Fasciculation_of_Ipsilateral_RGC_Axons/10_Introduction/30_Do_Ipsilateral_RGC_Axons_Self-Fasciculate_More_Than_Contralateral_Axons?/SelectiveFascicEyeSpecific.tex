The studies described above represent some of the relatively recent progress made in understanding selective fasciculation within axon tracts.
Noticeably, a large fraction of that work has focused on the role of fasciculation in maintaining heterotypic axon cohorts in reciprocal projections (e.g., sensory and motor axons in the peripheral nerves).
However, far less is known about the mechanics of and molecules involved in selective homotypic or finely heterotypic fasciculation within a tract.
It has been suggested in both the visual \cite{weth2014chemoaffinity} and olfactory systems \cite{ebrahimi2000olfactory,imai2011axon}, that, in addition to axon-target interactions, axon-axon interactions are also critical for appropriate targeting.
Indeed, several studies \invitro{} show that neurites from temporal and nasal retinal explants have clear preferences to grow along already-extended neurites of the same type \cite{bonhoeffer1985position}, and growth cones translate recognition into attraction or repulsion \cite{raper1990temporal,fan1995localized}. 
Perhaps axon-axon interactions within the tract also contribute to pre-target organization of axons, and subsequently, to their early targeting decisions.

In Chapter 2, I showed that ipsi RGC axons segregate from contra RGC axons in the optic nerve and tract.
This organization is likely mediated by the combination of several molecular mechanisms, including ipsi/contra axon repulsion, possible gradients of attractive and repulsive cues with the tract, and selective fasciculation of ipsi and contra axon cohorts.
Qualitative anecdotes of greater ipsi versus contra RGC axon fasciculation have been passed down in the oral history of the Mason Lab, but nobody had yet directly tested or quantified this observation.
Furthermore, a microarray of ipsi and contra RGCs performed in our lab revealed over a dozen CAMs that are more highly expressed by ipsi RGCs, as well as differential expression of other known axon guidance molecules \cite{wang2016ipsilateral}.
These expression differences are in line with the hypothesis that ipsi axons selectively self-fasciculate more than contra axons.

In order to test the hypothesis that different cohorts of axons self-organize within their tract via, at least in part, selective fasciculation behaviors, I first needed to establish a reliable fasciculation assay.
Many fasciculation studies are relatively qualitative in nature and those that have quantified the fasciculation of neurites \invitro{} are often comparing very large differences in fasciculation behavior in different conditions.
For example, Jaworski and Tessier-Lavigne (2012) tested Slit/Robo-mediated fasciculation of neurites by quantifying the relative percent of neurite bundles thicker than 8$\mu$m compared to control conditions \cite{jaworski2012autocrine}.
This approach lacks the detail necessary to discern more subtle differences in fasciculation, and it also fails to present a comprehensive picture of the nature of fasciculation in a given condition. 
Thus, in order to assess whether ipsi and contra RGC axons have different fasciculation behaviors, I established three \invitro{} assays of ipsi and contra RGC fasciculation.
If there are differences between the ipsi and contra cohorts, these \invitro{} assays can subsequently serve as platforms to test candidate molecules involved selective fasciculation, such as CAMs upregulated in ipsi RGCs \cite{wang2016ipsilateral}.