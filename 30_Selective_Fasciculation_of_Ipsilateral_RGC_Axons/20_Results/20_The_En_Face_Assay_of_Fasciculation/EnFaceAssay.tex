\label{sec:EnFace}
The rationale behind the \emph{en face} assay was to directly assess axon-axon interactions in a system with relatively low axon outgrowth, so as to allow for more detailed analyses than those afforded by explant cultures with robust outgrowth.
While robust neurite outgrowth allows for qualitative and often subjective assessments of neurite bundling or fasciculation, my goal was to analyze and quantify individual axon-axon interactions.
To do this, I used a lower concentration of laminin as a permissive growth substrate than our lab typically employs (8 $\mu$g/ml laminin instead of the typical 10-20 $\mu$g/ml).
I utilized \emph{\exvivo{}} retinal electroporation \cite{petros2009utero} to label the ventral or dorsal hemiretina at E14.5 with either GFP or mCherry.
After electroporation, each electroporated retina, with lens removed, was dissected from the head and incubated overnight in serum free media (SFM) (see Methods and Figure~\ref{Figures/EnFaceAssayDesign}).
%Add references for methods section and Figure when in place
The following day, VT or DT explants were dissected from the labeled hemiretinae and plated on 8 $\mu$g/ml laminin in like (VT/VT or DT/DT) or unlike (VT/DT) pairs, with the center of each explant 1.5mm apart from each other.
Unlike pairs were balanced between VT/DT placement and DT/VT placement, to control for any possible effect of sidedness.
GFP and mCherry were similarly counterbalanced for side of placement and also for region --- with an equal number of dorsal and ventral injections for each fluorescent protein.
Explant pairs were incubated overnight in SFM+0.4\% methylcellulose and on the following day, fixed with warm 4\% paraformaldehyde (PFA) and immunostained for neurofilament (NF), GFP, and mCherry to label all neurites and amplify the signal of the electroporated fluorescent proteins.

\begin{figure}[hbtp]
    \begin{center}
        \includegraphics{Figures/EnFaceAssayDesign.pdf}
        \caption[Experimental design for the \emph{en face} assay of neurite-neurite interaction.]
        {Experimental design for the \emph{en face} assay of neurite-neurite interaction.
        E14.5 embryos were harvested and decapitated.
        GFP or mCherry plasmid was injected via glass micropipette into the subretinal space of either the dorsal or ventral retina, and heads were electroporated.
        Labeled eyes were dissected out, the lens removed, and segregated into left and right for both GFP and mCherry labels, before incubating in serum free media (SFM) overnight.
        Dorsotemporal (DT) and ventrotemporal (VT) explants were dissected from labeled hemiretinae and plated in like (VT/VT or DT/DT) or unlike (VT/DT) pairs 1.5mm apart on 8 $\mu$g/ml laminin, cultured overnight, before final processing.
        %The next day, explants were fixed with warm 4\% paraformaldehyde (PFA) and immunostained for neurofilament (NF, to label all neurites), GFP and mCherry.
        }
        \label{Figures/EnFaceAssayDesign}
    \end{center}
\end{figure}
The space between the two explants was then imaged and analyzed.
I excluded explant pairs that were $\pm$ 250 $\mu$m away from being 1.5mm apart from each other.
The area between the two explants was subsequently limited to the area between two parallel lines drawn 100 $\mu$m away from the edge of the explant body (see Figure~\ref{EnFaceDataScreenshot}).
%Check that it's actually 100 and not 150 or 200
I excluded the region closest to the explant body from analysis in this way because the extreme density of neurite outgrowth and fasciculation nearest the explant body renders quantification challenging and unreliable.
Thus, my analysis was limited to an area where neurite growth was less dense, allowing me to more reliably discern individual fasciculation events.
\begin{figure}[hbtp]
    \begin{center}
        \includegraphics{Figures/EnFaceData_Screenshot.pdf}
        \caption[Analysis approach for the \emph{en face} assay.]
        {Analysis approach for the \emph{en face} assay.
        Screenshot from Neurolucida demonstrating the analysis method, with dorsotemporal (DT, contra) explant at left and ventrotemporal (VT, ipsi) explant at right.
        Explants were immunostained for neurofilament (NF, white), mCherry (DT explant in this case) and GFP (VT explant) to amplify electroporated fluorescent protein signal.
        Two parallel orange lines 100 $\mu$m away from each explant body demarcate the area of analysis.
        %Confirm 100micron
        Neurite tips are marked with x for the left explant and + for the right explant, in this case a contra and ipsi unlike pair.
        The same set-up is used for like pairs of VT/VT and DT/DT.
        Cyan, pink, and orange lines are traced lengths of neurite-neurite interactions: intra-explant interactions in cyan and pink, and inter-explant interactions in orange (the latter has fewer instances, all of which are for short distances).}
        \label{EnFaceDataScreenshot}
    \end{center}
\end{figure}

The original intention was to only analyze the neurite-neurite interactions between explant pairs and assess relative frequency and length of fasciculation events between neurites from the same or different type.
I hypothesized that VT/VT (ipsi/ipsi) neurites would fasciculate more (both in frequency and length of fasciculation events) than DT/DT (contra/contra) neurite pairs, and both of those pairs would fasciculate more than the unlike pairing between VT and DT explants.
However, in all cases, fasciculation events between explant pairs (inter-explant interactions) occurred infrequently.
In a few cases (specifically, one pair from each like pairing and two pairs from unlike pairings), the neurites from the two explants avoided each other altogether.
This is not surprising, as the \emph{en face} setup forces growth cones to approach each other head-on, which is an unfavorable interaction for growing neurites.
Furthermore, it does not recapitulate a typical \invivo{} fasciculation event, in which growth cones meet as they course along a similar trajectory and choose whether or not to fasciculate as they grow relatively parallel to each other.

Given the infrequency of inter-explant neurite interactions, I adjusted my analysis to include intra-explant neurite interactions --- that is, within each explant of the pair, I analyzed the frequency and length of neurite-neurite interactions, in addition to the inter-explant fasciculation events, if there were any.
Neurite interactions were quantified by manual tracing of fasciculation events and marking of neurite tips in Neurolucida (v11, MBF Biosciences, Williston, VT, USA).
A fasciculation event was defined as any time two neurites were observed joining together, and the length of the interaction was traced until the neurites could be seen splitting apart or one or both neurites ended.
All discernable neurite endings, or tips, were marked.
Thus, I was able to calculate two aspects of both intra- and inter-explant neurite-neurite interactions: the frequency of fasciculation events within an explant (the number of intra-explant fasciculation lengths measured divided by the number of neurite tips), and the axon bundling index (ABI) (after \citenoparens{barry2010polarized}), a measure of the average length of fasciculation events (Figure~\ref{Figures/EnFaceData}).
\begin{figure}[hbtp]
    \begin{center}
        \includegraphics{Figures/EnFaceData.pdf}
        \caption[Ipsi neurites self-fasciculate more frequently than contra neurites.]
        {Ipsi neurites self-fasciculate more frequently than contra neurites.
        A) Intra-explant fasciculation events are more frequent in VT than DT explants, measured by average number of fasciculation events per neurite tip.
        n=25 explants (DT), 19 explants (VT), from 4 experiments.
        Calculation combines measures of intra-explant neurite-neurite interactions in both like and unlike pairs (see Figure~\ref{Figures/EnFaceDataFrequencyIntxnSplit} for segregated data).
        **=p<0.01, 1 tailed t-test.
        B-C) Axon bundling index (ABI) for neurite-neurite interactions within (B) and between (C) explants.
        ABI measures average length of fasciculation event by dividing total length of fasciculation events by number of neurite tips.
        B) In both like and unlike pairs, intra-explant fasciculation events of DT and VT explants are similar (failed to reach statistical significance, Kruskal-Wallis Test).
        n=18 explants (9 pairs) for DT/DT, 12 explants (6 pairs) for VT/VT, 7 explants each for VT and DT in DT/VT pairs (7 pairs).
        C) Inter-explant fasciculation events, measured by ABI, are short (note difference in Y axis in B and C) in all conditions, with no statistical differences between pair types (Kruskal-Wallis Test).
        n=8 explant pairs for DT/DT, 5 explant pairs for VT/VT, 5 explant pairs for DT/VT.
        One pair from each like pairing, and two pairs from unlike pairings had no inter-explant neurite interactions and were excluded from the analysis (intra-explant measurements were included from those pairs).
        Error bars are standard error of the mean.
        }
        \label{Figures/EnFaceData}
    \end{center}
\end{figure}

Figure~\ref{EnFaceDataScreenshot} shows a representative example of a pair of explants in the \emph{en face} assay, in this case an unlike pairing between DT (left) and VT (right).
The image is a screenshot showing the analysis tracings in Neurolucida.
I found that the frequency of (intra-explant) fasciculation events was significantly higher for VT explants as compared to DT explants.
Figure~\ref{Figures/EnFaceData}A shows the aggregate data of both pair conditions, and the data are shown split by pair condition in Figure~\ref{Figures/EnFaceDataFrequencyIntxnSplit}.
Fasciculation frequency is significantly greater in VT explants grown in like pairs, compared to DT explants grown in like pairs.
However, the difference failed to reach significance for the unlike pairings.
Overall, though, these data support my hypothesis that ipsi RGC axons fasciculate more than contra RGC axons, as measured by frequency of fasciculation events.
\begin{figure}[hbtp]
    \begin{center}
        \includegraphics{Figures/EnFaceData_FrequencyIntxnSplit.pdf}
        \caption[VT neurites fasciculate more frequently than DT neurites.]
        {
        VT neurites fasciculate more frequently than DT neurites.
        Data from Figure~\ref{Figures/EnFaceData}A is shown here segregated by pair condition: like (left) and unlike pairs (right).
        Fasciculation frequency is measured as average number of neurite interactions per neurite tip.
        VT (ipsi) neurites self-fasciculate more frequently than DT neurites when grown in like pairs (n=12 DT explants, n=18 VT explants; *=p<0.05, 1-tailed t-test).
        When grown in unlike pairs, the difference between VT and DT fasciculation frequency failed to reach significance (n=7 DT explants, n=7 VT explants, p=0.16, 1-tailed t-test).
        Data shown as mean $\pm$ standard error of the mean.
        }
        \label{Figures/EnFaceDataFrequencyIntxnSplit}
    \end{center}
\end{figure}

Using the ABI measurement, I assessed average length of fasciculation events in both intra- and inter-explant situations.
There was no difference in intra-explant ABI for DT and VT neurites when they were grown in like pairs (Figure~\ref{Figures/EnFaceData}B, left two columns), nor when grown in unlike pairs (Figure~\ref{Figures/EnFaceData}B, right two columns).
In the unlike pairings, a divergence in ABI appeared between VT and DT explants, but failed to reach statistical significance (Kruskal-Wallis Test).
%The difference in intra-explant ABI between DT neurites grown in like versus unlike pairs also failed to reach statistical significance (p=0.06), and there was no difference in ABI between VT neurites grown in like versus unlike pairs.
Finally, inter-explant ABI measurements (Figure~\ref{Figures/EnFaceData}C) show that neurite interactions between explants were shorter on average than intra-explant interactions (Figure~\ref{Figures/EnFaceData}B).
Within these shorter neurite interactions, however, no difference was observed between pair type.

Overall, the data from the \emph{en face} assay had an unsatisfactory signal to noise ratio.
Additionally, the ABI measurement, which only shows the average length of a neurite interaction might not provide an accurate picture of the data, especially in light of the variability I observed in the raw data.
In order to address this concern, I categorized fasciculation events into three categories based on length: crossings (<10 $\mu$m), interactions (10-100 $\mu$m), and long interactions (>100 $\mu$m), in order to gain a more detailed representation of the data.
This analysis did not show any differences in intra-explant fasciculation events between ipsi and contra neurites (Figure~\ref{Figures/EnFaceDataLengths}A).

However, binning the data provided slightly more clarity to the inter-explant fasciculation events (Figure~\ref{Figures/EnFaceDataLengths}B).
Binning the fasciculation data based on length showed that inter-explant interactions between unlike pairs tended to be shorter than those between either of the like pairs --- i.e., when unlike neurites met in between the two explants, they more often crossed each other for short distances rather than fasciculate together.
Likewise, both like pairings had significantly more intermediate interactions (those between 10-100 $\mu$m in length) than the unlike pairs.
Binning the data into even smaller bins could potentially reveal more differences between the various conditions, but the data are challenging to confidently analyze given the few occurrences of inter-explant interactions.
\begin{figure}[hbtp]
    \begin{center}
        \includegraphics{Figures/EnFaceData_Lengths.pdf}
        \caption[Fasciculation events categorized by length.]
        {Fasciculation events categorized by length.
        A) Binned data of intra-explant fasciculation events based on length (crossings = <10 $\mu$m, interactions = 10-100 $\mu$m, long interactions = >100 $\mu$m).
        At left is the combined data from like and unlike pairs.
        At right, data from like and unlike pairs are shown separately.
        No statistically significant differences exist between VT and DT explants in either condition, nor in the combined data.
        n=18 explants (9 pairs) for DT/DT, n=12 explants (6 pairs) for VT/VT, n=7 explants each for VT and DT in DT/VT pairs (7 pairs).
        B) Binned data of inter-explant fasciculation events based on length.
        No significant differences in any length category between DT/DT and VT/VT pairings.
        There were significantly more inter-explant crossings (the shortest fasciculation event) between unlike pairs (orange) than either like pair conditions (Kruskal-Wallis test with Dunn's post-test).
        Concurrently, there were significantly more inter-explant interactions 10-100 $\mu$m long in the like pairs than in the unlike pairs (Kruskal-Wallis test with Dunn's post-test).
        n=8 explant pairs for DT/DT, n=5 explant pairs for VT/VT, n=5 explant pairs for DT/VT.
        }
        \label{Figures/EnFaceDataLengths}
    \end{center}
\end{figure}

Finally, I had concerns about the experimenter variability possible in manually identifying and tracing fasciculation events in this assay.
Tracing neurite interactions by hand might be inconsistent between experimenters, something which I was unable to directly text because I performed all the tracings and analyses independently.
Regardless, this concern provided motivation to establish a more reliable and objective assay that bypassed the other main shortcoming of the \emph{en face} assay, namely the unfavorable conditions for assessing inter-explant fasciculation.
The next two sections present two assays that largely overcome the weaknesses of the \emph{en face} assay.
