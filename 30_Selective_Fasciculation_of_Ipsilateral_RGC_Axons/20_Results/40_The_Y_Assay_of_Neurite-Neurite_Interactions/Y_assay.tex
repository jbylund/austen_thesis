Having reliably established an intrinsic preference for ipsi (VT) neurites to fasciculate more frequently (as shown by the \emph{en face} assay) and to a greater extent (as shown by the bundle width assay) than contra (DT) neurites, I wanted to establish a better way of assessing axon-axon interactions between axon types.
The limitation in doing this with the \emph{en face} assay was that having growth cones approach each other head-on was not conducive to fasciculation.
Thus, I thought of adapting the classic `Y' setup used by Bonhoeffer for testing target choice \cite{bonhoeffer1985position} for the purposes of instead assaying fasciculation and axon-axon interactions in the absence of a target.
The conceptualization of this experiment is illustrated in Figure~\ref{Figures/YAssayDesign}.
Dr. Franco Weth and his student Dr. Felix Fiederling, part of Dr. Martin Bastmeyer's group at Karlsruher Institut f\"ur Technologie, collaborated with us on this endeavor.
Microcontact printing is a method regularly used in the Bastmeyer group to produce patterned substrates of various proteins for the assessment of axon outgrowth and navigation, often using chick RGCs or retinal explants \cite{von2006microcontact}.
After some consultation, Drs. Weth and Fiederling produced silicone masters of a Y pattern to my specifications (Figure~\ref{Figures/YAssayDesign}A) and sent me several identical polydimethylsiloxane (PDMS) stamps to create the Y pattern in a laminin substrate.

The microcontact printing method is explained in detail by von Philipsborn et al. (2006), but I will provide a brief summary here.
I incubated 40$\mu$g/ml laminin on the PDMS stamps for 1-3 hr, rinsed the laminin solution off by dipping into distilled water (dH$_2$O), and quickly dried the stamp with high purity nitrogen.
The stamp was immediately placed face-down onto a prepared glass coverslip (plasma cleaned and epoxysilanized (see Methods)) long enough to mark the pattern on the backside of the coverslip with a fine-tip permanent marker.
The stamp was then removed and the coverslip covered with warm SFM+0.4\% methylcellulose.
E14.5 VT or DT explants were plated in the two arms of the Y and allowed to grow for two nights (approximately 36-40hr), with fresh warm SFM+0.4\% methylcellulose added after one night in vitro.
Explants were fixed with a specialized fixative mix (see Methods) that aided maintenance of neurite adhesion to the coverslip, and cultures were immunostained for laminin and NF.
I then imaged and analyzed neurites in the stem of the Y, the lane in which neurites from each explant meet and navigate fasciculation choices.
\begin{figure}[hbtp]
    \begin{center}
        \includegraphics{Figures/Y_Assay_Design.pdf}
        \caption[Experimental design and hypothetical outcomes of the Y assay for fasciculation.]
        {Experimental design and hypothetical outcomes of the Y assay for fasciculation.
        A) Design of the Y pattern and summary of experimental steps.
        The black lines indicate the negative space on the PDMS stamps, i.e., the area which does not transfer laminin to the coverslip, creating the borders to a Y shape where laminin is transferred.
        B) Hypothesized outcomes of different explant pairs in the Y assay.
        Neurites arising from DT/DT pairs (contra) will likely fasciculate to a lesser extent than those from VT/VT (ipsi) pairs.
        Unlike pairs of DT/VT explants (contra/ipsi) will likely present a mix of responses, with VT neurites hyperfasciculating and DT neurites avoiding VT neurites by either defasciculating or forming a separate fascicle.
        VT=ventrotemporal (ipsilateral) retinal explant.
        DT=dorsotemporal (contralateral) retinal explant.
        }
        \label{Figures/YAssayDesign}
    \end{center}
\end{figure}
Based on my findings in the previously discussed assays, I anticipated that neurites from a DT/DT pair of explants would fasciculate, but to a lesser extent than neurites from a VT/VT pair (Figure~\ref{Figures/YAssayDesign}B).
In the unlike pair, I expected hyperfasciculation of the VT neurites and defasciculation or repulsion of the DT neurites.

Figure~\ref{Figures/YDataAllExamples} shows examples of each explant pair type in the Y assay.
In total, I have collected four successful DT/DT pairs, three VT/VT pairs, two DT/VT pairs, and no successful VT/DT pairs.
In each experiment, I plate an equal number of DT/DT, VT/VT, and unlike pairs - the unlike pairs are split equally between VT/DT and DT/VT.
Counterbalancing the placement of VT and DT explants in the unlike pairings will be important, given that in all cases, the nuerite growth in the lane of the Y appears to have a rightward bias.
\begin{figure}[hbtp]
    \begin{center}
        \includegraphics{Figures/Y_Data_AllExamples.pdf}
        \caption[Preliminary results of the Y assay for fasciculation.]
        {Preliminary results of the Y assay for fasciculation.
        All examples collected to date are shown for each explant pair type.
        From top to bottom: (A) DT/DT (contra like pair, n=4 pairs from 1 experiment), (B) VT/VT (ipsi like pair, n=3 pairs from 2 experiments), (C) DT/VT (unlike pair, n=2 pairs from 2 experiments).
        For each pair type, the top row shows cultures immunostained for laminin (magenta) and neurofilament (NF, green) to label all neurites, and the bottom row shows the same cultures, zoomed in on the lane of the Y, showing NF staining only.
        The Y border is marked by a white dashed line in the NF-only images.
        VT=ventrotemporal (ipsilateral) retinal explant.
        DT=dorsotemporal (contralateral) retinal explant.
        NF=neurofilament.
        Scale=250$\mu$m.
        }
        \label{Figures/YDataAllExamples}
    \end{center}
\end{figure}

I have shown all successful cases collected so far in Figure~\ref{Figures/YDataAllExamples} to demonstrate the variability in this assay.
One source of variability arises from amount of neurite outgrowth - some explants extend robust neurite halos, while others appear relatively stunted.
This variability is likely due to one of the ongoing technical limitations with the assay, namely the efficient and consistent transfer of laminin via microcontact printing.
In the bundle width assay, I used 20$\mu$g/ml laminin as a growth permissive substrate for the retinal explant cultures.
When the same concentration of laminin was used with microcontact printing, explants consistently failed to effectively adhere to the coverslip, and those few that did often failed to extend more than a few short neurites.
After adjusting several parameters and continuing to find the same problem, I suspected that the microcontact printing approach was not transferring laminin with 100\% efficiency.
With this suspicion in mind, I tested other laminin concentrations and found explant adhesion and outgrowth to be much more robust and reliable at a concentration of 40$\mu$g/ml compared to 20$\mu$g/ml.
However, variability remains across stamped coverslips in terms of amount of explant adhesion and neurite outgrowth.
This also gives rise to another factor responsible for the low yield of successful cultures, which is evident in the right-hand explant of the DT/VT pair in Figure~\ref{Figures/YDataAllExamples}C\textsuperscript{2}: explants frequently shift by tens of microns before they are able to extend neurites and sometimes end outside of the Y branch they were originally placed in.
These relatively small movements that occur as the explants settle into position render a number of explant pairs unsuitable for analysis in each experiment.
This has been a significant factor in the low yield of successful cases so far.

Given the technical obstacles in getting this assay established, data collection is ongoing and no final conclusions can be drawn.
In some cases, the qualitative results align with my hypotheses.
For instance, the fasciculation of DT/DT neurites in Figure~\ref{Figures/YDataAllExamples}A\textsuperscript{2} and A\textsuperscript{3} is moderate, but visibly less pronounced than in VT/VT neurites in Figure~\ref{Figures/YDataAllExamples}B\textsuperscript{1} and B\textsuperscript{3}.
The VT/VT pair in Figure~\ref{Figures/YDataAllExamples}B\textsuperscript{2} is unreliable for analysis because the righthand explant has markedly stunted outgrowth and neurites from both explants have made little progress into the length of the Y lane.
Similarly, the unlike explant pair in Figure~\ref{Figures/YDataAllExamples}C\textsuperscript{2} suffered from a shift in position of the VT explant (on the right), rendering the interactions between the two neurite types in the Y lane unreliable.
However, the DT/VT explant pair shown in Figure~\ref{Figures/YDataAllExamples}C\textsuperscript{1} qualitatively matches my hypothesis that VT neurites (on the right) hyperfasciculate and DT neurites (on the left) unravel from the bundle.
This is inconclusive, however, as without differential labeling of neurites arising from each explant, it is unclear what fraction of the large fascicle is composed of VT and DT neurites.

Because of the variability in amount of outgrowth - a factor in all explant culture experiments, but perhaps more pronounced in the Y assay due to the additional source of variability in laminin transfer via microcontact printing - quantitative analysis of this assay will need to be able to take that variability into account.
I plan to use a vertex analysis (e.g., Analyze Skeleton in Fiji \cite{arganda20103d}) as an objective quantitative readout of complexity of the neurite bundles in the lane of the Y.
A vertex analysis will describe the overall structure of the neurites in the Y lane, treating them as a branched structure - this will serve as a readout of connectivity complexity.
The more complex the connectivity and more points of visible neurite crossings and vertices, the less fasciculated the overall outgrowth is.
On the other hand, the less complex the connectivity and the fewer visible points of neurite crossings, the more fasciculated the outgrowth.
This measure may still suffer from a lack of normalization to amount of neurite outgrowth, in which case I will implement additional ways of either excluding low-outgrowth samples (based on neurite outgrowth in the branches of the Y rather than the stem) or otherwise accounting for outgrowth variation.
%
%The microcontact printing proved to be technically challenging, and many rounds of troubleshooting were necessary to successfully execute the experiment.
%Aside from the mundane technical issues surrounding establishing a new technique (e.g., coverslip preparation, purity grade of our nitrogen source, both of which were different from our normal setup), two technical limitations remain, though to a lesser extent.
%The first of these is the efficient transfer of laminin via microcontact printing.
%
%These technical limitations are worth discussing because they are relevant to the
%
%Finally, the paradigm in and of itself leads to greater inconsistency and a relatively low yield of successful explant pairs compared to the previous two assays discussed in this chapter.
%The assay is dependent on the explant pairs adhering in exactly the position they have been placed during the plating step.
%That is, if one or both of the explants shifts position before it is able to extend neurites, it will fail to extend neurites into the rest of the Y pattern.
%Explants often shift tens of microns as they settle into their final positions and extend neurites, forming more points of contact with the coverslip and stabilizing their position on the coverslip.
%Because of this, some fraction of explant pairs inevitably fail to grow into the Y, but instead sometimes around the outer perimeter
