This chapter presented three in vitro assays of fasciculation that I have used to assess intrinsic differences in fasciculation behavior of ipsi and contra RGC axons.
I began with a hypothesis, based in part on anecdotal observations from previous lab members, that ipsi axons fasciculate to a greater extent than contra axons.
Data from two of the assays presented here (the \emph{en face} and bundle width assays) support this hypothesis by showing that ipsi neurites (i.e., those from E14.5 VT retinal explants) fasciculate more frequently than contra neurites (those from DT explants), and ipsi neurites tend to form thicker fascicles than contra neurites.
The third assay presented, the Y assay, has potential to provide greater insight into neurite-neurite interactions as neurites from two types of retinal explants meet and interact.
The bundle width and Y assays will be particularly useful as researchers in our lab and others probe molecules that are differentially expressed by ipsi and contra RGCs (e.g., \citenoparens{wang2016ipsilateral}) and may be involved in ipsi/contra axon sorting and axon-axon interactions.

In this section, I will first discuss the challenges of establishing a reliable fasciculation assay, particularly one which is capable of detecting subtle differences in fasciculation behavior.
The results presented in the chapter will then be discussed, especially in the context of the in vivo data shown in Chapter 2 regarding ipsi and contra RGC axon organization in the optic nerve and tract.
Finally, I will elaborate on the future directions and potential of assaying fasciculation and neurite-neurite interactions in vitro.