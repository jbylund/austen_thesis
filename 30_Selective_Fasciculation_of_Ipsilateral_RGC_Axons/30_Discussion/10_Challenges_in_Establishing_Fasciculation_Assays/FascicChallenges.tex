Assaying fasciculation in vitro provides a way to directly test intrinsic fasciculation behaviors of different cohorts of axons.
Especially in a tract like the murine optic tract, in which we know very little about the cells and molecular cues present within the tract, it is important to understand the intrinsic fasciculation behaviors of axon cohorts.
This knowledge is necessary for eventually probing the extrinsic molecular cues within the tract, so as to build a full understanding of the symphony of intrinsic and extrinsic cues that both guide and organization axon cohorts in their tracts.

There are some general challenges to all in vitro assays of fasciculation, which are important to keep in mind when interpreting results from such assays.
One issue, which I have tried to address with the assays presented in this chapter, is that of objective and reliable quantification of fasciculation that can detect subtle differences in fasciculation behavior.
Some fasciculation assays use on qualitative assessment of gross differences in degree of fasciculation (e.g., \citenoparens{what to cite}), while others use quantification approaches that detect very large differences (e.g., \citenoparens{jaworski2012autocrine}), but would likely fail to detect more subtle differences.
It is important to be able to detect smaller differences because they may be as meaningful as larger, more easily detectable differences.
Indeed, understanding subtle differences in intrinsic fasciculation behaviors may be key for understanding how axons organize within a tract.
Ipsi and contra RGCs, for instance, while distinct cohorts, are members of the same neuronal type (RGCs). 
Thus, one would expect any differences in fasciculation behavior to be smaller within a neuronal type (e.g., ipsi and contra RGCs) than between two neuronal types (e.g., RGCs and OSNs).
Therefore, as we probe further into the organization of axon tracts and the molecular mechanisms creating such order, it will be imperative to use assays that have a high detection resolution of potentially small but meaningful differences between different cohorts and subclasses of axons.

It was with this goal of discerning potentially subtle differences in ipsi and contra RGC fasciculation that I established one of the central parameters in the \emph{en face} assay: a low concentration of laminin substrate to reduce neurite outgrowth so as to facilitate tracking individual fasciculation events.
However, this reduction in neurite outgrowth, and concomitantly fewer overall fasciculation events, might have contributed to the large variability in my data from this assay.
Furthermore, manually tracking fasciculation events introduces experimenter variability and an undesirable amount of subjectivity to the analysis.
The final strike against the \emph{en face} assay was the fact that it failed to effectively test neurite-neurite interactions between two explants, which was one of the original goals of creating a fasciculation assay.
Thus, the \emph{en face} assay failed to fulfill the goal of establishing an objective and reliable fasciculation assay that could detect subtle differences.
%This effort, however, highlights the challenges surrounding development of such an assay - some of the ways in which I tried to make the assay more precise instead led to greater overall variability.

The bundle width assay successfully overcame some of the limitations of the \emph{en face} assay, but at the cost of limiting the type of data it can produce.
Namely, I sacrificed the ability to compare neurite-neurite interactions between like and unlike pairs of explants for greater reliability in the readout of a simpler metric - neurite bundle width within single explants grown in isolation.
This was a reasonable approach to take, given that the \emph{en face} assay largely failed at producing a reliable readout of inter-explant neurite interactions anyway.
Instead, the ABI measurement from the \emph{en face} assay provided data on average length of a fasciculation event predominantly from intra-explant neurite interactions.
Because fasciculation is likely a highly dynamic process, it is unclear how informative fasciculation length measurements are in a static, fixed culture.
Thus, using the bundle width assay to assess intra-explant fascicle size is a more reliable and likely more meaningful measure than fascicle length in the \emph{en face} assay.
%Bundle width might be a more reliable measure of degree of fasciculation than fascicle length, and because the ABI measurements were mostly intra-explant in the \emph{en face} assay, the bundle width assay, which only assess fasciculation within single explants, seems a better option overall.

One point to note is that the \emph{en face} assay showed statistically significant differences between VT and DT ABI only when explants were plated in unlike pairs, but not so when they were plated in like pairs.
This suggests that the context an explant grows in affects its fasciculation behavior, at least as measured by the average length of a fasciculation event.
Therefore, VT and DT explants and/or their neurites might communicate with each other via contact-mediated or diffusible cues to affect each other's self-fasciculation and/or repulsion from unlike neurites.
However, in order to directly test this hypothesis, data from the bundle width assay is a necessary first step: a clearer understanding of the intrinsic fasciculation behaviors of each explant type in isolation is necessary in order to interpret results from co-explant experiments.
Overall, the bundle width assay provides more objective and reliable data, which also lend themselves to more straightforward interpretations.

Finally, the Y assay attempts to overcome the shortcomings of the \emph{en face} assay in regards to measuring neurite-neurite interactions between two explants.
While the \emph{en face} assay presented two explants head-to-head, a situation in which opposing growth cones are more likely to collapse or turn away than fasciculate together, the Y assay presented two explants in such a way that they could meet in a lane growing nearly in a similar trajectory with each other.
The limitations of the Y assay are largely technical in nature, mainly centering around the technical minutiae of microcontact printing.
Discussed in the final results section of this chapter, these technical limitations predominantly hinge on the efficient transfer of laminin as a growth-permissive substrate onto the coverslip used to culture the explants.
Microcontact printing is neither as consistent nor efficient a route of application as direct incubation of the coverslip with a laminin solution.
This leads to greater variability in explant adhesion and neurite outgrowth in the microcontact printed Y assay compared to the standard culture conditions, which renders a large fraction of explant pairs unsuitable for analysis.
Another way in which variability in the laminin substrate leads to greater variability and a low yield of successful experiments occurs during the fixation step.
Using warm 4\% PFA to fix the cultures, neurites in the lane of the Y often peeled off the coverslip and folded up between the two explants.
This may be due to the high levels of fasciculation in the lane of the Y, such that the entire fascicle is adhered to a small overall surface area.
This limited surface area for neurite-substrate adhesion combined with variability in the laminin substrate renders the neurites highly sensitive to physical forces, such as the flow of media or fixative across the coverslip.
Dr. Fiederling suggested a fixative mix of PFA and glutaraldehyde in a 10\% sucrose solution, which has greatly ameliorated this problem and helps maintain the integrity of the explant culture through fixation and immunostaining.

Beyond these assay-specific technical challenges and limitations, there are other considerations that pertain more broadly to retinal explant assays and fasciculation assays.
One set of issues was briefly discussed earlier in this chapter and pertains to the retinal explant approach itself: namely, the heterogeneity of retinal explants in terms of cell types and specifically for the VT retina, in terms of ipsi RGC specificity.
While retinal explant cultures are a standard experimental approach (e.g., \citenoparens{bonhoeffer1985position,wang1996chemosuppression,kuwajima2012optic}), they contain RGCs but also the other cell types in the retina.
It is generally assumed that at E14.5, RGCs are the main, if not only, cell type extending neurite processes, which means the retinal explant approach is a generally reliable method for assaying RGC axon outgrowth and fasciculation.
Alternatively, pure RGCs could be obtained via immunopanning \cite{barres1988immunological}.
This could allow for future experimentation at an even more specific level.
Immunopanning or FACS purification of ipsi and contra RGCs (as in \citenoparens{wang2016ipsilateral}) could also provide pure ipsi and contra specific RGC populations to assay in vitro.
However, the extra labor involved in these approaches would be unlikely to significantly further the findings presented in this chapter.
Furthermore, especially with FACS purification, the extra stress placed on the cells would likely render the RGCs less robust in culture than they are as retinal explants, thus impairing the ability to effectively assay neurite growth and fasciculation.

A new possibility for addressing the issue of VT retina heterogeneity in culture is to use the SERT-Cre reporter line described and used in Chapter 2.
Using this reporter (or an as yet unreleased Zic2 reporter line), true ipsi neurites could be distinguished in culture via fluorescent signal of the reporter.
The downside to this approach is the time-course of Cre recombination and activation of the fluorescent reporter.
In our hands, the SERT-Cre:zsgreen line does not robustly express zsgreen until E17, whereas our culture conditions are tailored specifically to E14.5 retinal explants.
Explants taken from older retinae grow very poorly in our culture system, and after E15.5 almost uniformly fail to adhere to the coverslip or extend neurites.
Our colleagues, however, have recently shared culture conditions for E16.5 retinal explants (personal communication, J. Peng and F. Charron, 2016), so that use of the SERT reporter line for retinal explant experiments may indeed be feasible in the future.

Another consideration and arguable shortcoming of two-dimensional (2D) explant cultures as used here is that they involve both neurite-neurite and neurite-substrate interactions and potentially competition between the two interactions.
Laminin was identified as a robust growth substrate for RGCs in vitro in the 1980s \cite{smalheiser1984laminin,cohen1985retinal}, and I have used it here as a permissive growth substrate for my experiments.
Laminin is also expressed along the basement membrane in the retina and by astrocytes in the optic nerve in chick and rodents and is important for the extension of RGC axons in vivo \cite{cohen1987role,liesi1988astrocyte,sarthy1990localization,morissette1995laminin}.
Thus, while it is standard practice to use laminin or similar growth-permissive substrates in vitro, use of such substrates introduces the variable of neurite-substrate adhesion competing with neurite-neurite adhesion.
In order to more directly test neurite-neurite interactions without interference of neurite-substrate interactions, three-dimensional (3D) culture systems could be used in the future.
I chose not to begin with a 3D culture system for my experiments because one of the principle goals was to identify a method with reliable quantification of fasciculation, a goal which was much more feasible in a 2D system.
Any neurite-substrate interactions that might interfere with or otherwise affect fasciculation in the experiments presented in this chapter would be present in equal measure across explant type.
Thus, comparisons within the culture conditions remain valid, although it will be interesting to see if fasciculation is more or less pronounced, or unchanged, in a 3D system.

One final variation on the assays presented in this chapter, which could provide even more insight into fasciculation dynamics of ipsi and contra neurites, would be to use live imaging. 
As mentioned previously, the dynamics of fasciculation remain relatively poorly understood, but axons can theoretically fasciculate and defasciculate dynamically along the length of the shaft, as well as at the growth cone.
As such, it would be informative to use live or time-lapse imaging of the explants, especially in the Y assay paradigm, to track neurite-neurite interactions and gain a better understanding of their dynamics and the stability of fasciculation events within each cohort of neurites.