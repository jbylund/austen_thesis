Despite the limitations of the \emph{en face} assay discussed above and in the results section, it reliably indicated that E14.5 VT (ipsi) neurites fasciculate more frequently than DT (contra) neurites (Figure~\ref{Figures/EnFaceData}A).
The bundle width assay then showed very clearly that VT (ipsi) neurites form thicker fascicles than DT (contra) neurites when grown as single explants (Figures~\ref{Figures/BundleWidth}B).
These two results confirm the hypothesis that ipsi axons fasciculate to a greater extent than contra axons --- both in terms of fasciculation frequency and width.
Using two separate measures provides robust support of this conclusion.

What is the significance of this difference in ipsi and contra fasciculation behavior?
\Invivo{} data presented in Chapter 2 (Figure~\ref{Figures/SertNerveToTract}) suggests that ipsi RGC axons undergo dynamic changes in fasciculation as they extend to, through, and beyond the optic chiasm.
Specifically, ipsi RGC axons appear well bundled in the ventrolateral portion of the proximal optic nerve and progressively become more scattered as they approach the chiasm (Figure~\ref{Figures/SertNerveToTract}B).
After turning away from the chiasm, ipsi axons resegregate into a clear laterally-positioned bundle in the optic tract (Figures~\ref{Figures/SertNerveToTract}C and ~\ref{Figures/SertdLGN}A-B).
The \invitro{} data demonstrating an intrinsic preference of ipsi neurites to self-fasciculate compared with contra neurites suggest that the \invivo{} observations can be explained in part by selective fasciculation.
Furthermore, the ABI data from unlike explants in the \emph{en face} assay suggest, albeit inconclusively, that growing near or in direct contact with contra neurites, ipsi neurites fasciculate to an even greater extent than when they are only with other ipsi neurites (Figure~\ref{Figures/EnFaceData}C).
Thus, a reasonable working hypothesis is that ipsi RGC axons self-fasciculate to a greater extent than contra RGC axons, and that in the optic nerve and tract, where the two populations are in close proximity, this difference in selective fasciculation is further magnified.
This may involved repulsive interactions between contra and ipsi neurites.
The Y assay is a platform to more directly test the effects of like and unlike neighboring neurites on ipsi fasciculation behavior.

The bundle width assay on a laminin substrate tested the intrinsic fasciculation behavior of ipsi and contra neurites that were na\"ive to midline chiasm cues, effectively a reductionist recapitulation of the optic nerve, rather than the optic tract.
Because RGC axons are thought to undergo a change in surface molecule expression after interacting with the chiasm (as occurs in the spinal cord, see \citenoparens{dodd1988spatial}), it is likely that axons in the tract may behave differently than in the nerve.
For this reason, I repeated the bundle width assay experiments with low density dissociated chiasm cells.
The presence of chiasm cells would more closely recapitulate axon behavior in the optic tract, and would also demonstrate the robustness the intrinsic fasciculation differences between ipsi and contra neurites --- would addition of extrinsic cues abolish, enhance, or have no effect on the intrinsic differences seen in the chiasm-na\"ive cultures?
Co-culturing retinal explants with dissociated chiasm cells does not, of course, truly recapitulate the optic tract.
Axons in the tract are no longer actively interacting with chiasm cues and therefore likely have stabilized any changes in surface molecule expression that occurred within the chiasm. 
Additionally, axons in the tract interact with a new set of tract-specific cues.
However, we have almost no knowledge of tract-specific cues, and understanding how chiasm cues affect the intrinsic fasciculation behavior of RGC axons is necessary for probing subsequent tract-specific fasciculation behavior.
Furthermore, these experiments also provide insight into RGC axon behavior within the chiasm.

The bundle width assay experiments with chiasm cells demonstrated that chiasm cues encourage both ipsi and contra neurites to bundle more thickly than when grown in a chiasm-na\"ive paradigm.
The intrinsic differences between ipsi and contra fasciculation, however, remain robust even in the presence of these extrinsic cues, suggesting that throughout the retinogeniculate pathway, ipsi axons maintain a greater level of self-fasciculation than contra axons, potentially contributing to ipsi/contra axon segregation.

Previous ultrastructural studies of the optic chiasm showed that RGC axons form an intricate braid-like weave of fascicles as they navigate through the molecularly complex midline region \cite{colello1998changing}.
Chiasm cues enhance neurite fascicle thickness of both ipsi and contra neurites \invitro{} (Figures~\ref{Figures/BundleWidth} and ~\ref{Figures/BundleWidthmoreGraphs}), suggesting that the braiding together of RGC axon fascicles in the chiasm may be mediated in part by enhanced fasciculation.
This is a possible mechanism for RGC axons making correct choices in the chiasm, as enhanced selective fasciculation may contribute to maintaining appropriately segregatation of the ipsi and contra populations.

The chiasm co-culture experiments in the bundle width assay are clearly a coarse way of testing the effects of extrinsic cues on ipsi and contra fasciculation.
Dissociated chiasm cells are highly heterogeneous, including a number of different cell types expressing a number of different molecules, many of which have yet to be identified and characterized.
Thus, this approach provides a broad picture of the effects of extrinsic midline cues on selective fasciculation of ipsi and contra neurites.
Future experiments can build on these important findings to test individual cues present at the chiasm.