The data summarized above represent an important step forward in understanding cohort-specific fasciculation behaviors that may affect axonal organization along the retinogeniculate pathway.
In addition to these findings, the fasciculation assays I described in this chapter are valuable tools in the continuing work of identifying and better understanding the molecular mechanisms of fasciculation, axon-axon interactions, and axon organization.
Part of the motivation for establishing such assays arose from recent work by a former student in the lab, Dr. Qing Wang, which identified a host of differentially expressed genes between ipsi and contra RGCs at E16.5 \cite{wang2016ipsilateral}.
A reliable in vitro assay was desirable for efficient testing of candidate genes that may be involved in fasciculation and axon-axon interactions.
The bundle width assay presents such an assay, which provides a reliable, objective, and straightforward readout of fasciculation within a single explant.
The Y assay, especially in conjunction with live imaging, will provide further insight into neurite-neurite interactions and the fasciculation dynamics between different types of neurites.

Of particular interest in Dr. Wang's microarray screen are the approximately dozen cell adhesion molecules (CAMs) that are more highly expressed in ipsi RGCs compared to contra RGCs (at least 2-fold difference in expression, at a p-value of less than 0.05, \citenoparens{wang2016ipsilateral}).
A subset of these CAMs are of particular interest and would be interesting to study in the bundle width and Y assays, either by selectively knocking down expression in ipsi explants or adding function-blocking antibodies to the culture.
Disabled-1 (Dab1) is a key regulator of reelin, and its expression in ipsi RGCs is more than two-fold higher than in contra RGCs.
Dab1 and reelin are involved in the wiring of retinal interneurons in the inner plexiform layer \cite{rice2001reelin}, and Dab mutants phenocopy reelin mutant defects in the mouse hippocampus \cite{borrell2007reelin}.
Borrell et al. \shortcite{borrell2007reelin} demonstrated that Dab1 is involved in axon extension, branching, and refinement, and when treated with reelin, neurites grown in vitro increased their fasciculation.

Another candidate arising from the microarray screen that has been shown to be involved in fasciculation in other systems is the Fat-cadherin family member Fat1.
Expression of Fat1 is nearly three-fold higher in ipsi RGCs compared to contra RGCs at E16.5 \cite{wang2016ipsilateral}.
Fat1 is an actin cytoskeleton regulating protein that is highly localized to cell-cell boundaries, including filopodial tips \cite{tanoue2004mammalian}.
The \emph{Caenorhabditis elegans} Fat-like cadherin-4 (CDH-4) closely resembles both \emph{Drosophila} and mouse Fat1.
CDH-4 \emph{C. elegans} mutants had pronounced axon guidance defects and marked defasciculation along the major axon tracts \cite{schmitz2008fat}.
The localization of Fat1 to filopodial tips and at areas of cell-cell adhesion, along with its demonstrated role in fasciculation in \emph{C. elegans} axon tracts makes it a compelling candidate as a mediator of ipsi-specific RGC fasciculation.

Moesin (Msn), another protein that interacts with the actin cytoskeleton, was found to be expressed nearly four-fold higher in ipsi RGCs compared with contra RGCs at E16.5 \cite{wang2016ipsilateral}.
Msn is part of the ezrin-radixin-moesin (ERM) family of membrane-cytoskeleton linking proteins.
Disrupting ERM-cytoskeleton signaling affects axon morphology \cite{dickson2002functional,marsick2012activation}.
ERM proteins bind to L1, an important CAM in axon guidance \cite{dickson2002functional,marsick2012activation} and Msn and radixin are localized to primary neuron growth cones in vitro and are important for modulating neurite formation and extension \cite{paglini1998suppression}.
Furthermore, neurotrophin and nerve growth factor (NGF) increase active levels of ERM proteins in the growth cone, which is important for actin reorganization in the growth cone, but perhaps more interestingly, a redistribution of adhesion molecules like L1 \cite{marsick2012activation}.
Thus, while Msn has not been shown to have a direct role in fasciculation, its differential expression between ipsi and contra RGCs and cellular position and function in growth cones and axon guidance makes it another compelling candidate for studying in terms of ipsi-specific fasciculation.

Finally, the microarray screen revealed differential expression of some classic axon guidance molecules between ipsi and contra RGCs.
Semaphorin (Sema) 5b, for instance, is more highly expressed by ipsi RGCs, while contra RGCs express both Sema3e and Sema7a more than two-fold higher than ipsi RGCs \cite{wang2016ipsilateral}.
Semaphorins are a diverse and multifaceted family of proteins and are known players in axon guidance, surround repulsion-mediated bundling of axon tracts, fasciculation, defasciculation, and axon sorting \cite{kuwajima2012optic,tran2007semaphorin,kolodkin2011mechanisms,zhou2013axon,imai2009pre,raper2000semaphorins}.
Class 3 Semas in particular play a role in fasciculation and axon sorting \cite{raper2000semaphorins,zhou2013axon,imai2009pre}, so the expression of Sema3e by contra RGC axons could potentially influence fasciculation of ipsi axons and segregation of ipsi and contra RGC axons by a repulsive mechanism.
It will be interesting to test this hypothesis and study the role of these three Semas in fasciculation and more broadly in the organization of ipsi and contra RGC axons in the retinogeniculate pathway.

In addition to testing candidate molecules identified by the microarray screen \cite{wang2016ipsilateral}, the bundle width and Y assays will be useful for testing whether glia from the nerve or tract differentially affect ipsi and contra fasciculation.
Glial organization changes along the retinogeniculate pathway in mice \cite{colello1992observations} and ferrets \cite{guillery1987changing}, and glia are also found in striking associations with RGC axon bundles in the mouse optic chiasm \cite{colello1998changing}.
More recently, microglia have been shown to be involved in axon guidance, wiring, and fasciculation \cite{squarzoni2014microglia,pont2014microglia}.
Thus, it could be particularly fruitful to co-culture retinal explants in the bundle width and Y assays with glia from the optic nerve or tract.
Direct co-culture experiments or the addition of glial-conditioned media to retinal explant cultures could demonstrate whether glia provide contact-mediated or diffusible signals to RGC neurites that modulate their fasciculation behaviors.
%
%The following chapter uses the bundle width assay as part of a larger assessment of RGC axon organization and ipsi axon fasciculation in the EphB1 mutant.
%EphB1, while not mentioned above, is a known mediator of the ipsilateral choice of RGC axons at the optic chiasm midline \cite{williams2003ephrin,petros2009specificity}.