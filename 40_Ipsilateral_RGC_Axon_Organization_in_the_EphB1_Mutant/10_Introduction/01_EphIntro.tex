Eph proteins are receptor tyrosine kinase (RTK) molecules that play important roles in neural development (reviewed in \citenoparens{kania2016mechanisms,klein2014ephrin}).
Eph RTKs are classified into two families, EphAs and EphBs, based on their ability to interact with ephrin-A or ephrin-B ligands, respectively.
There are nine EphA proteins and five EphBs; five ephrin-As and three ephrin-Bs \cite{lisabeth2013eph,kania2016mechanisms}.
EphA/ephrin-A and EphB/ephrin-B binding partners interact promiscuously within their respective classes, although EphA4 and EphB2 also bind across class boundaries \cite{lisabeth2013eph,kania2016mechanisms}.
The designation of Eph receptor and ephrin ligand is somewhat of a misnomer, however: Ephs acting as receptors to ephrin ligands is known as forward signaling, but Ephs can act as ligands to ephrins, known as reverse signaling, and each molecule can simultaneously act as receptor and ligand in bidirectional signaling \cite{kania2016mechanisms,egea2007bidirectional}.
Adding even more versatility to the Eph/ephrin functional repetoire, Eph and ephrin can act in \emph{cis}, which appears to attenuate Eph signalling within the same cell \cite{marquardt2005coexpressed,kao2011ephrin,carvalho2006silencing}.

This large class of RTKs and their receptors, with their versatile binding and signaling mechanisms, play a variety of roles in neural development.
Eph/ephrin signaling contributes to short-range cell-cell communication mediating axon guidance at choice points, formation of topographic maps in target regions, axon fasciculation, synapse formation and stability, neurogenesis, and neural migration \cite{kania2016mechanisms,klein2014ephrin}.
In the visual system in particular, Ephs and ephrins were first identified for their involvement in retinotectal topographic map formation \cite{cheng1995complementary,drescher1995vitro}.
Historically, this finding grew out of the original chemoaffinity hypothesis, which proposed that graded molecular cues guided axons to the correct target region \cite{attardi1963preferential,sperry1963chemoaffinity}.
A series of elegant in vitro assays subsequently demonstrated that such cues are cell surface bound proteins \cite{walter1987recognition,walter1987avoidance,walter1990axonal}, setting the stage for the molecular and biochemical studies that showed the nasal-temporal and dorsal-ventral axes of the retinotectal map were guided by complementary graded expression of EphAs/ephrin-As and EphBs/ephrin-Bs, respectively (reviewed in \citenoparens{triplett2012eph}).
Relative, rather than absolute, levels of EphA expression guide map formation \cite{brown2000topographic}, a finding that introduced the concept of axon competition into the formation of robust topographic maps \cite{feldheim2010visual,triplett2012eph}.
Eph/ephrin interactions are thus involved in both axon-target chemoaffinity and axon-axon chemoaffinity \cite{weth2014chemoaffinity}.
%The versatility of Ephs and ephrins makes them interesting, albeit often challenging to study, players in neural circuit development.
%In this chapter, I have built upon previous work from the Mason Lab on the role of EphB1 in the ipsilateral RGC population, to study the tract organization in EphB1 mutants and explore a possible role for EphB1 in ipsi RGC axon selective fasciculation.