Eph/ephrin involvement in axon-axon interactions is not limited to competition-based refinement in targets, such as those described in the visual system (e.g., \citenoparens{weth2014chemoaffinity,suetterlin2014target,}).


EphA5 involved in fasciculation of cortical neurons in vitro, soluble ephrin-A5 blocked fasciculation in vitro \cite{winslow1995cloning}

EphB2/EphB3 dko mice have defasciculated habenular-interpeduncular tract, independently of pathfinding functions \cite{orioli1996sek4}
also somewhat in optic nerve \cite{birgbauer2000kinase}

EphA3 mediated sorting of medial and lateral cortical axons in CC (in vivo and in vitro) (EphA3 preferentially expressed in lateral cortex); possibly by interaction with ephrin-A5 which is expressed kinda complementarily in medial cortex, but other molecules probably also involved \cite{nishikimi2011segregation}

Axial nerve; motor axon expressing EphA3/EphA4, sensory expressing ephrin-As; repulsive, trans-axonal EphA/ephrin-A signaling important for heterotypic axon segregation, screwing it up has functional consequences (segregation of motor from sensory by forward signaling) \cite{gallarda2008segregation}
then asked what about tracking of sensory axons along motor axons - also involved EphA3/4 and ephrin-A signaling, but different signaling mech than in gallarda paper \cite{wang2011anatomical}

sensory/motor innervation of the limb - ephrin-B1 expressed in sensory axons and limb bud mesenchyme, EphB2 expressed in sensory and motor axons; again independent of pathfinding roles; ephrin-B1 surround repulsion controls fasciculation of motor and sensory axons
\cite{luxey2013eph}


\cite{coate2012otic}

\cite{kaneko2003interaxonal}