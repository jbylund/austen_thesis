\label{sec:EphFascic}
In addition to the axon-target and axon-axon interactions mediating topographic mapping of retinal axon terminals in the visual system (e.g., \citenoparens{weth2014chemoaffinity,suetterlin2014target}), Ephs and ephrins have been implicated in fasciculation and axon sorting in several systems.
Early \invitro{} experiments with primary cortical neurons first demonstrated a role for Ephs in fasciculation, showing that EphA5 mediates neurite fasciculation in culture \cite{winslow1995cloning}.
Adding a soluble form of the cognate receptor, ephrin-A5 to these cultures significantly blocked neurite fasciculation \cite{winslow1995cloning}.
This finding led to other explorations of Eph and ephrin function in fasciculation, which can occur inter-axonally and/or by surround repulsion mechanisms.

Retinal axons in EphB2/EphB3 double knockout mutants suffer some defasciculation intra-retinally \cite{birgbauer2000kinase}.
These mutants also display marked defasciculation in the habenular-interpeduncular axon tract, although pathfinding of these axons appears grossly unperturbed \cite{orioli1996sek4}.
This latter finding highlights the diverse functions of classic guidance molecules such as Ephs and ephrins, in that they have discrete roles in pathfinding and fasciculation.
In the case of Ephs and ephrins, these different functions are thought to be driven by the different signaling mechanisms (e.g., forward and reverse signaling, or parallel and antiparallel bidirectional signaling) \cite{egea2007bidirectional,kania2016mechanisms}.

In longitudinal axon tracts of \emph{Drosophila}, homologues of Eph and ephrin (DEph and Dephrin, respectively) contribute to axon fasciculation by surround repulsion.
Dephrin is expressed along the tracts and repels axonal DEph, keeping axons appropriately bundled together \cite{bossing2002dephrin}.
Perturbing either DEph or Dephrin leads to marked defasciculation and premature exit from the tract, resulting in striking disorganization of the circuit \cite{bossing2002dephrin}.
In the mammalian inner ear, EphA4 in otic mesenchyme cells mediates fasciculation of spiral ganglion (SGN) axons bundles, likely via repulsive interactions with ephrin-B2 on SGN axons \cite{coate2012otic}.

Ephs and ephrins maintain axon tracts via surround repulsion in the mammalian limb bud.
EphB2 is expressed in both sensory and motor axons in the developing limb, while ephrin-B1 is expressed in sensory axons and the limb bud mesenchyme.
Loss of ephrin-B1 from the mesenchyme alone results in defasciculation of sensory and motor axons, indicating that ephrin-B1 surround repulsion mediates the bundling of this reciprocal tract \cite{luxey2013eph}.
Interestingly, constitutive loss of ephrin-B1 has a slightly more robust defasciculation phenotype, suggesting that inter-axonal Eph/ephrin interactions may also be involved in the organization and overall fasciculation of sensory and motor axons in the limb \cite{luxey2013eph}.

EphAs are also involved in motor and sensory axon sorting in the axial nerve.
Repulsive trans-axonal EphA/ephrin-A signaling mediates the precise heterotypic organization of the two axonal cohorts, with EphA3 and 4 expressed by motor axons, and ephrin-As by sensory axons \cite{gallarda2008segregation}.
Perturbing EphA3/4 forward signaling results in anatomical and functional disorganization of the reciprocal projections \cite{gallarda2008segregation}.
A different mode of EphA3/4 signaling with ephrin-As is also required for sensory axons to appropriately track along the preceding motor axons, which is important for their targeting and function \cite{wang2011anatomical}.

EphA3 similarly functions in axon sorting in the corpus callosum (CC) by mediating the sorting of medial and lateral cortical callosal axons \cite{nishikimi2011segregation}.
EphA3 is preferentially expressed in the lateral cortex, and ephrin-A5 is expressed in an incompletely complementary fashion in the medial cortex (i.e., additional molecule(s) are likely involved in the interaction with EphA3).
\Invitro{}, attenuation of EphA3 disrupts the ability of medial and lateral cortical explants to segregate from one another, indicating that direct inter-axonal signaling sorts the two axonal cohorts, although surround repulsion mechanisms may also be involved \invivo{} \cite{nishikimi2011segregation}.
Inter-axonal signaling has also been implicated in sorting olfactory sensory neurons in the moth, \emph{Manduca sexta} \cite{kaneko2003interaxonal}.
%Need transition here?