One particular member of the RTK family of Eph receptors, EphB1, plays a significant role in formation of the binocular visual circuit prior to formation of the retinotopic map in the thalamic targets \cite{williams2003ephrin,petros2008retinal,lee2008zic2}.
Identification of ephrin-B in the optic chiasm midline of organisms with an ipsilateral RGC projection (mice and post-metamorphic frogs), but not in the chiasm of organisms with a purely contralateral RGC projection (fish and chicks), suggested EphB receptors as candidates for mediating the ipsilateral choice of RGC axons at the midline \cite{nakagawa2000ephrin}.

In the mouse, ephrin-B2 is expressed by radial glia at the optic chiasm midline at the time when ipsi RGC axons from VT retina are navigating the midline \cite{williams2003ephrin}.
Concurrently, EphB1 is expressed exclusively in the VT retina at the time when VT retina is producing ipsi RGCs \cite{williams2003ephrin}, and is downstream of the ipsi-specific transcription factor Zic2 (\citenoparens{lee2008zic2,garcia2008zic2}, see Figure~\ref{Figures/SertRetina}A).
EphB1 is also expressed earlier in the dorsocentral retina, where the early-born transient ipsi population of RGCs arises \cite{drager1985birth,williams2003ephrin}.
\Invitro{} experiments and analyses of mutant mice confirmed that ipsi RGC EphB1 receptors interacting with ephrin-B2 at the chiasm midline produces the ipsilateral component of the retinal projection in mice \cite{williams2003ephrin,petros2009specificity,petros2010ephrin,chenaux2011forward}.

Ectopic expression of \emph{EphB1} in the retina drives RGC axons that would normally project contralaterally to instead be repelled from the chiasm and project ipsilaterally \cite{petros2009specificity}.
Knocking out \emph{EphB1}, on the other hand, reduces the ipsi RGC population by approximately half, as VT RGCs fail to recognize the repulsive ephrin-B2 cue expressed by midline radial glia, and instead cross the midline and project contralaterally \cite{williams2003ephrin}.
The targeting of these aberrantly crossing axons, however, remains grossly normal \cite{rebsam2009switching}.
Late-born contra RGCs in the VT retina normally target the dorsal tip of the dLGN, while VT ipsi RGC axons normally target the ipsi-recipient zone in the dLGN, situated dorso-medially and surrounded by contra projections \cite{pfeiffenberger2006ephrin,rebsam2009switching}.
In the EphB1\textsuperscript{-/-} mutant, aberrantly crossing VT RGC axons do not project to the normal contralateral-recipient zone (the dorsal tip of the dLGN), but instead target the ipsi-recipient zone, only in the contralateral thalamus \cite{rebsam2009switching}.

However, these aberrantly decussated axons refine in an activity-dependent manner away from both the remaining ipsi axons and the normal contra axons in the dLGN, forming a segregated ectopic patch of contra axons in the ipsi-recipient zone of the dLGN.
A summary of the decussation and targeting findings from the experiments in the EphB1\textsuperscript{-/-} mutant is shown in Figure~\ref{EphB1schematic}.
Additionally, the overall refinement of ipsi- and contra-recipient zones of the dLGN is impaired in these mutants \cite{rebsam2009switching}.
Therefore, gross eye-specific targeting does not necessarily depend on axons making the correct decision at the chiasm midline, although aberrantly decussated fibers are segregated from normal ipsi and contra fibers.
The converse experiment confirms this relative independence of targeting from midline choice: ectopic expression of \emph{EphB1} drives RGC axons to project ipsilaterally, but these aberrantly uncrossed axons still target the contra zone of the dLGN \cite{rebsam2009switching}.
\begin{figure}[hbtp]
    \begin{center}
        \includegraphics{Figures/EphB1_schematic.pdf}
        \caption[Summary of decussation and targeting phenotype in the EphB1\textsuperscript{-/-} mutant.]
        {Summary of decussation and targeting phenotype in the EphB1\textsuperscript{-/-} mutant.
        In the wild-type (WT) retinogeniculate pathway, ipsi RGC axons target the dorso-medial dLGN, surrounded by contra RGC axons.
        Late-born VT contra RGCs target the dorsal tip of the contra dLGN (right side in green).
        In the EphB1\textsuperscript{-/-} mutant retinogeniculate pathway, a significant portion of VT ipsi RGC axons aberrantly decussates the optic chiasm midline, leading to a reduced ipsi-recipient zone in the dLGN.
        Aberrantly crossed axons target the ipsi-recipient zone, but segregate from normal ipsi axons.
        They do not target the dorsal tip, like the normal population of VT contra RGC axons.
        D=dorsal, V=ventral, N=nasal, T=temporal, dLGN=dorsal lateral geniculate nucleus.}
        \label{EphB1schematic}
    \end{center}
\end{figure}