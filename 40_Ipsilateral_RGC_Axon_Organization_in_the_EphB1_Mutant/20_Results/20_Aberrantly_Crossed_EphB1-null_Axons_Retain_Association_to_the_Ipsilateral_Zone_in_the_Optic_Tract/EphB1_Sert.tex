In order to better assess whether ipsi RGC axons are truly defasciculated in the EphB1\textsuperscript{-/-} optic tract and, moreover, to assess the relationship between midline choice, tract order, and targeting, I needed to be able to track the aberrantly crossing axons - i.e., those that lost expression of EphB1 and subsequently crossed the midline.
The VT retina normally produces both ipsi and contra RGCs in later stages of retinal development in the wild-type (see Figure~\ref{RGPDevSeries})
Therefore, anterograde labeling of RGC axons cannot distinguish between axons that are ``true'' contras (i.e., genetically contra and cross at the midline) and those that are aberrant contras (i.e., genetically ipsi but cross at the midline because they lost \emph{EphB1} expression).
Even performing focal anterograde tracing from just the VT retina (as in Figure~\ref{DiIDiD_WT_VTVT}) would fail to distinguish between true and aberrant contra axons.

Instead, I turned to a genetic approach, using the SERT-Cre:zsgreen reporter line discussed in Chapter 2 (see Section~\ref{sec:SertCreResults}).
SERT is expressed by ipsi RGCs, under direct transcriptional regulation by the ipsi RGC specific transcription factor Zic2 \cite{garcia2010zic2}.
Zic2 also regulates expression of \emph{EphB1}, although it is unclear if this is a direct or indirect regulation \cite{garcia2010zic2}.
The ipsi RGC specific transcription factor therefore regulates expression of both \emph{EphB1} and \emph{SERT}, but, importantly, it does so in parallel transcriptional pathways (see Figure~\ref{Figures/SertRetina}).
Thus, \emph{SERT} expression should be unaffected by knocking out \emph{EphB1}, and would provide a way of tracking all (genetically) ipsi axons, including those that aberrantly cross the midline in the EphB1\textsuperscript{-/-} mutant.

With this line of thinking, I crossed our EphB1\textsuperscript{-/-} line (a homozygous knockout line) with the SERT-Cre:zsgreen reporter line, and bred littermate pairs of SERT-Cre:zsgreen;EphB1\textsuperscript{-/-} mutants and SERT-Cre:zsgreen;EphB1\textsuperscript{+/+} wild-type controls.
To control for the possibility that loss of \emph{EphB1} might affect \emph{SERT} expression, I first compared SERT-Cre:zsgreen expression in the retinae of both SERT-Cre:zsgreen;EphB1\textsuperscript{-/-} and SERT-Cre:zsgreen;EphB1\textsuperscript{+/+} genotypes.
In immunostained wholemount retina preps (as shown for a SERT-Cre:zsgreen sample in Figure~\ref{Figures/SertRetina}B), I counted the total number of zsgreen\textsuperscript{+} cells in littermate paired samples, blinded to genotype.
These cell counts showed no significant difference across genotypes (including SERT-Cre:zsgreen;EphB1\textsuperscript{+/-}, Figure~\ref{ESCretinacounts}), confirming that loss of \emph{EphB1} does not affect \emph{SERT} expression in RGCs.
\begin{figure}[hbtp]
    \begin{center}
        \includegraphics{Figures/ESCretinacounts.pdf}
        \caption[SERT expression is unchanged in EphB1 mutant retinae.]
        {SERT expression is unchanged in EphB1 mutant retinae.
		There are no statistically significant differences in cell counts of SERT-Cre:zsgreen\textsuperscript{+} cells in flat-mounted retinae from littermates wild-type, heterozygous, and homozygous knockout for EphB1.
		n=7 EphB1\textsuperscript{+/+}, n=4 EphB1\textsuperscript{+/-}, n=5 EphB1\textsuperscript{-/-}, from 5 litters.
		Data are mean $\pm$ standard error of the mean.
		}
        \label{ESCretinacounts}
    \end{center}
\end{figure}

I next performed anterograde labeling of one eye with DiI in SERT-Cre:zsgreen;EphB1\textsuperscript{+/+} and SERT-Cre:zsgreen;EphB1\textsuperscript{-/-} P0 pups, then sectioned and immunostained for zsgreen.
In this labeling scheme, DiI in the right retina labels ipsi RGC axons in the right hemisphere and contra RGC axons in the left hemisphere, while zsgreen labels SERT\textsuperscript{+} axons - in SERT-Cre:zsgreen;EphB1\textsuperscript{-/-} mutants, this will include both the ipsi axons labeled with DiI as well as aberrantly crossing axons.
It is this second population, labeled in blue in Figure~\ref{EphB1Sertschematic}, that will be most informative, as it will demonstrate whether aberrantly crossing axons associate with remaining ipsi axons or ``true'' contra axons.
\begin{figure}[hbtp]
    \begin{center}
        \includegraphics{file name}
        \caption[Short caption]
        {Long caption}
        \label{file name}
    \end{center}
\end{figure}