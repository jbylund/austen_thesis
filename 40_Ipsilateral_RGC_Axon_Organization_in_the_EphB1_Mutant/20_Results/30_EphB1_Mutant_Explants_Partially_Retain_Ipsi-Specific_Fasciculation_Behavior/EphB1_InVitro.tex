\label{sec:EphB1invitro}
Findings presented in the previous section from SERT-Cre:zsgreen;EphB1\textsuperscript{-/-} samples suggest that aberrantly crossing axons maintain an association with the ``ipsi zone'' of the optic tract.
This suggests that a factor, not EphB1, is still present on EphB1\textsuperscript{-/-} misrouted axons that allows them to be attracted to the lateral optic tract.
An alternative, though not mutually exclusive, possibility is that both misrouted axons and the remaining axons retain a signal that leads to association or fasciculation of the full cohort.
However, even though it appears that the integrity of the ipsilateral cohort (remaining ipsi axons and misrouted axons) is largely preserved in the EphB1\textsuperscript{-/-} mutant, the remaining ipsi axons have a disorganized appearance, with fibers often either oriented out of line with the rest of the tract or straying outside of the bounds of the ``ipsi zone.''
Thus, EphB1 may still be involved in the selective fasciculation of ipsi RGC axons, though likely only as one of a concert of signals.

In order to more directly test whether loss of EphB1 affects the ipsi RGC axon preference to self-fasciculate, I used the \invitro{} bundle width assay from Section~\ref{sec:BundleWidth} to test fasciculation of EphB1\textsuperscript{-/-} VT and DT retinal explants.
The \invivo{} results from the EphB1\textsuperscript{-/-} optic tract discussed in Section~\ref{sec:DiIDiDEphB1} suggested that I would find little to no difference in neurite bundle thickness between EphB1\textsuperscript{-/-} VT and DT explants.
However, results from Section~\ref{sec:EphB1Sert} suggest that an element of self-association is retained by EphB1\textsuperscript{-/-} axons, and therefore that EphB1\textsuperscript{-/-} VT neurites may retain the ability to form thicker bundles than EphB1\textsuperscript{-/-} DT neurites.

I found that, similar to wild-type retinal explants (see Figure~\ref{Figures/BundleWidth}B), EphB1\textsuperscript{-/-} VT neurites form thicker fascicles than EphB1\textsuperscript{-/-} DT neurites (Figure~\ref{EphB1InVitro}B) when grown on 20$\mu$g/ml.
However, the magnitude of the difference between EphB1\textsuperscript{-/-} explant types appears smaller than between wild-type explant types. 
Additionally, similar to wild-type explants, the median bundle width for both DT and VT neurites from EphB1\textsuperscript{-/-} explants is larger at the 500$\mu$m radius compared to the 250$\mu$m, although again, the magnitude of this difference is less than in wild-type explants.
Thus, while EphB1 may be partially involved in the preferential fasciculation of ipsi RGC axon compared to contra axons, it is not necessary for ipsi neurites to fasciculate more than contra neurites.
\begin{figure}[hbtp]
    \begin{center}
        \includegraphics{Figures/EphB1_InVitro}
        \caption[EphB1\textsuperscript{-/-} VT neurites form slightly larger bundles than DT neurites.]
        {EphB1\textsuperscript{-/-} VT neurites form slightly larger bundles than DT neurites.
        A) Representative examples of DT (top, contra) and VT (bottom, ipsi) retinal explants from EphB1\textsuperscript{-/-} E14.5 embryos, grown overnight on 20$\mu$g/ml laminin, as in Figures~\ref{Figures/BundleWidthCartoon} and ~\ref{Figures/BundleWidth}.
        Scale=500$\mu$m.
        B) Cumulative distribution of bundle widths for DT (magenta) and VT (green) neurites intersecting rings at 250 and 500$\mu$m from the explant body.
        n=15 (VT) and 19 (DT), from three experiments.
        }
        \label{EphB1InVitro}
    \end{center}
\end{figure}
