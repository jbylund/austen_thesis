\label{sec:EphB1invitro}
In order to more directly test whether loss of EphB1 affects the ipsi RGC axon preference to self-fasciculate, I used the in vitro bundle width assay from Section~\ref{sec:BundleWidth} to test fasciculation of EphB1\textsuperscript{-/-} VT and DT retinal explants.
Based on the in vivo results from the EphB1\textsuperscript{-/-} optic tract discussed in Section ~\ref{sec:DiIDiDEphB1}, I expected there to be no difference in neurite bundle thickness between EphB1\textsuperscript{-/-} VT and DT explants.
Instead, I found that, similar to wild-type retinal explants (see Figure~\ref{Figures/BundleWidth}B), EphB1\textsuperscript{-/-} VT neurites form thicker fascicles than EphB1\textsuperscript{-/-} DT neurites (Figure~\ref{EphB1InVitro}B).
Additionally, similar to wild-type explants, the median bundle width for both DT and VT neurites from EphB1\textsuperscript{-/-} explants is larger at the 500$\mu$m radius compared to the 250$\mu$m, although again, the magnitude of this difference is less than in wild-type explants.

\begin{figure}[hbtp]
    \begin{center}
        \includegraphics{Figures/EphB1_InVitro}
        \caption[EphB1\textsuperscript{-/-} VT neurites form slightly larger bundles than DT neurites.]
        {EphB1\textsuperscript{-/-} VT neurites form slightly larger bundles than DT neurites.
		A) Representative examples of DT (top, contra) and VT (bottom, ipsi) retinal explants from EphB1\textsuperscript{-/-} E14.5 embryos, grown overhnight on 20$\mg$ml laminin, as in Figures~\ref{BundleWidthCartoon} and ~\ref{BundleWidth}.
		Scale=500$\mu$m.
		B) Cumulative distribution of bundle widths for DT (magenta) and VT (green) neurites intersecting rings at 250 and 500$\mu$m from the explant body. 
		n=15 (VT) and 19 (DT), from three experiments.
		}
        \label{EphB1InVitro}
    \end{center}
\end{figure}