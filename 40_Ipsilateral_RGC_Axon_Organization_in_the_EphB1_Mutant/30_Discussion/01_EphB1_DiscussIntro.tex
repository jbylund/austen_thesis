In this chapter, I used a combination of in vivo anterograde labeling approaches and an in vitro assay of fasciculation to probe the role of EphB1 in eye-specific RGC axon organization in the mouse optic tract.
The motivation to explore axon organization in the EphB1\textsuperscript{-/-} mutant was twofold.
First, it serves as a model with an incorrect midline choice but, broadly speaking, normal targeting.
Thus, examining the organization of the optic tract provides insight into how midline choice, tract order, and targeting are related to each other.
I will discuss this in more detail in Section~\ref{sec:LinkingEphB1}, at the end of this Discussion.
Secondly, Ephs and ephrins have been implicated in fasciculation and axon sorting in other systems.
In this way, they have become a prime example of how one family of axon guidance molecules can have distinct functions in the development of a circuit, acting as midline repellents in one situation and as fasciculation or sorting cues in another.
As such, asking whether EphB1 is involved in the selective fasciculation of ipsi RGC axons was a compelling question to ask and build on the previous work from our lab demonstrating its role in midline choice at the optic chiasm.
I will discuss my findings regarding EphB1 and ipsi RGC axon fasciculation in Section~\ref{sec:EphB1Fascic}.
First, I will summarize and discuss my findings on the organization of the EphB1\textsuperscript{-/-} optic tract.