Our lab previously showed that ipsi RGC axons express EphB1, which mediates their turn away from the midline into the ipsilateral tract via a repulsive interaction with midline ephrin-B2 \cite{williams2003ephrin,petros2009specificity}.
Furthermore, misrouted axons in the EphB1\textsuperscript{-/-} mutant that aberrantly cross the midline target the ipsi-recipient zone in the dLGN \cite{rebsam2009switching}.
Whether loss of \emph{EphB1} affects eye-specific axon organization in the optic tract, however, remained unexamined.

To determine if loss of \emph{EphB1} and the concomitant reduction in the ipsilateral projection affects axon organization in the tract, I used the same anterograde dye tracing approach as in Chapter 2, Section~\ref{sec:DiIDiDWT}, to label ipsi and contra RGC axons in the EphB1\textsuperscript{-/-} optic tract.
The remaining ipsi axons are by and large positioned in the lateral optic tract, where wild-type ipsi RGC axons are situated.
However, compared to ipsi axons in the wild-type tract, ipsi axons in the EphB1\textsuperscript{-/-} tract appear aberrantly fasciculated and disorganized (Figure~\ref{EphB1WTDiIDiD}).
These findings suggest that EphB1 may be involved in ipsi RGC axon fasciculation and/or in ipsi/contra axon segregation, as ipsi fibers often stray outside of the optic tract or into the medial sector of the tract.

Positioning of ipsi axons in the lateral tract of EphB1\textsuperscript{-/-} mutants is grossly normal, suggesting that EphB1 is not the main driver positioning the ipsi cohort within the tract.
However, Ephs and ephrins may be involved in the topographic positioning of axons in the tract (see Chapter 2, Section~\ref{sec:TopogDiscuss}).
Ipsi RGCs also express EphB2, but EphB2 does not appear to be involved in the ipsilateral midline choice \cite{williams2003ephrin,chenaux2011forward}.
Because some ipsi axons in the EphB1\textsuperscript{-/-} optic tract stray farther from the lateral tract region compared to wild-type, a combined EphB1/EphB2 interaction with cues in the tract may be a mechanism positioning the ipsi cohort within the tract.
In this case, EphB2 could partially compensate for loss of EphB1 in ipsi axons, thus maintaining the overall positioning of the cohort, with a fraction of ipsi axons wandering outside the normal constraints of the lateral tract.

In order to test this hypothesis, future studies would need to determine if ephrin or Eph gradients exist within the tract, which could provide either repulsive signals to the ipsi cohort from the medial tract, and/or attractive signals from the lateral tract or pial membrane lining the lateral tract.
Additionally, experiments identifying local translation events in post-commissural axons and their growth cones are needed to better understand the ways in which ipsi and contra RGC axons modulate responsiveness to various cues in the optic tract.
In other words, we need a clearer picture of the growth cone and axon shaft specific expression of EphB1, EphB2, and other candidate molecules that might be involved in tract organization.

Another intriguing aspect of the EphB1\textsuperscript{-/-} mutant is that only a subset of axons are misrouted \cite{williams2003ephrin,rebsam2009switching}.
It is unclear what allows the remaining ipsi axons in the EphB1\textsuperscript{-/-} mutant to correctly turn away from the chiasm and project ipsilaterally.
Our knowledge of ipsi and contra factors in RGCs remains incomplete, and EphB1 is not the only driver of the ipsilateral choice at the midline.
\emph{Zic2} overexpression drives ipsilateral choice to a greater extent than \emph{EphB1} overexpression \cite{garcia2008zic2}, indicating that Zic2 likely drives expression of other ipsi factors that repel axons from the midline.
Other cues, such as the Sonic hedgehog (Shh) receptor Boc, have been identified in participating in the ipsilateral RGC projection \cite{fabre2010segregation}, and other as yet unidentified cues are likely also involved.
All of this raises the question of whether the ipsi RGC axons that remain in the EphB1\textsuperscript{-/-} mutant reflect a specific subpopulation of ipsi RGCs, or, alternatively, are a stochastic subset of the ipsi population.
If they are are specific subset (e.g., Boc\textsuperscript{+} or some other unknown ipsi-specific molecule), then whichever ipsi-specific molecule(s) they express could mediate their lateral positioning in the tract.
Current work in our lab is investigating whether a subset of ipsi RGCs arises from a distinct neurogenic niche in the ciliary margin zone (CMZ), as opposed to the neural retina (F. Marcucci and C. Mason, in press and unpublished).
%add Flor's paper when out
This may shed light on one aspect of the diversity of the ipsilateral RGC population, though, again, there are likely other elements contributing to ipsi RGC heterogeneity.

Finally, disorganization of ipsi axons in the optic tract of EphB1\textsuperscript{-/-} mutants may be a consequence of aberrant interactions with the optic chiasm, rather than a direct result of lack of \emph{EphB1} itself.
The reduction in size of the ipsilateral projection itself may leave the remaining ipsi axons slightly defasciculated and disorganized because the population might need a critical mass of axons to properly associate and maintain appropriate organization.
In this way, neighboring axon-axon interactions would be an important mechanism for establishing and/or maintaining organization within the tract.
Additionally, the EphB1 interaction with ephrin-B2 at the chiasm midline might be a critical step for a subset of ipsi RGC axons, in that the molecular interaction and subsequent repulsion from the midline may be a necessary cue for a post-chiasm local translation event, which in turn is necessary for appropriate interactions with cues within the optic tract (either with tract-specific factors or with other axons).
Again, in order to test these hypotheses, we first need a more thorough understanding of the local translation events occurring within the optic tract, as well as what cues exist within the tract itself.