\label{sec:EphB1Fascic}
Using the bundle width \invitro{} assay described in Chapter 3, I found that EphB1\textsuperscript{-/-} VT neurites fasciculate into thicker bundles than EphB1\textsuperscript{-/-} DT neurites (Figure~\ref{EphB1InVitro}).
While this is similar to the findings in wild-type retinal explants (Figure~\ref{Figures/BundleWidth}B), the magnitude of the difference between EphB1\textsuperscript{-/-} explants appears smaller than between wild-type explants.

The EphB1\textsuperscript{-/-} homozygous knockout strain in our lab, however, is maintained on a 129S1/SvImJ (129) genetic background, while wild-type animals used for experiments in Chapters 2 and 3 are C57BL/6J.
Thus, direct statistical comparisons between results in the bundle width assay from the two genetic backgrounds cannot be conclusively performed.
Most comparisons of C57BL/6J and 129 strains have focused on behavioral differences (e.g., \citenoparens{contet2001faster,smith2007genetic}), but no differences in visual system structure or function have been reported.
Indeed, the first report on the role of EphB1 in mediating RGC axon divergence at the optic chiasm studied mice on an albino background (FVB) \cite{williams2003ephrin}, which are known to have a smaller ipsilateral population compared to pigmented mice \cite{guillery1996albinos}.
Later studies of EphB1 in development of the binocular circuit also used the 129 strain \cite{rebsam2009switching}. 
While it is likely that the two strains are overwhelmingly similar in the genetic and anatomical composition of their visual systems, the use of two genetic backgrounds must be taken into account when comparing results from experiments presented in Chapter 3 (Figure~\ref{Figures/BundleWidth}B) and this chapter (Figure~\ref{EphB1InVitro}).

To more directly test the role of EphB1 in mediating the selective fasciculation ipsi RGC axons, EphB1 function or expression could be manipulated directly in wild-type C57BL/6J retinal explants.
An EphB1 function-blocking antibody added to wild-type explant cultures would demonstrate whether the VT neurite preference to fasciculate more than DT neurites is extinguished in the absence of EphB1 function.
Alternatively, EphB1 could be electroporated into wild-type DT explants in a gain-of-function paradigm, or silenced in VT explants in a loss-of-function paradigm.
In either experiment, the elimination or reduction of differences in bundle width between VT and DT wild-type explants would confirm a role for EphB1 in ipsi axon fasciculation.
Likewise, EphB2 and other candidate molecules can be further tested for a role in fasciculation.

Additionally, loss of \emph{EphB1} may fail to completely eliminate ipsi fasciculation \invitro{} because EphB1 may mediate fasciculation via both inter-axonal and surround repulsion mechanisms.
In order to test its role in potential surround repulsion mediated fasciculation, we first need to identify what ephrins, if any, are expressed by cells within the optic tract.
These could then be added to retinal explant cultures to test their effect on VT neurite fasciculation.
Targeted deletion of ephrins in diencephalic cells \invivo{} could then demonstrate whether they mediate ipsi fasciculation in the tract, and/or ipsi/contra axon sorting in the tract.

If EphB1 is indeed a mediator of ipsi RGC axon fasciculation, it is not the sole factor, as evidenced by the statistically significant, if small in magnitude, difference in bundle widths between EphB1\textsuperscript{-/-} VT and DT neurites (Figure~\ref{EphB1InVitro}).
As discussed in the previous section, EphB2 is also expressed by ipsi RGCs, but does not mediate the midline choice \cite{williams2003ephrin,chenaux2011forward}.
EphB2 is involved in selective fasciculation of sensory and motor axons in the developing limb, via a surround repulsion interaction with ephrin-B1 expressed in the limb bud mesenchyme \cite{luxey2013eph}.
However, the defasciculation phenotype in constitutive knockout animals was more severe than a selective deletion of ephrin-B1 in the mesenchyme, implicating additional inter-axonal Eph/ephrin interactions involved in fasciculation in this system \cite{luxey2013eph}.
EphB2 could play a similar role in ipsi axon fasciculation, either by surround repulsion with as yet unidentified tract-specific cues and/or inter-axonal interactions with other ipsi axons.
Non-Eph molecules may also participate in the concert of cues guiding fasciculation and axon sorting.

Furthermore, ipsi axon self-fasciculation and ipsi/contra axon sorting may be related but distinct phenomena.
That is, EphB1 may mediate ipsi self-fasciculation, but it may also mediate repulsion between ipsi and contra axons via a different interaction or signaling mechanism.
Testing EphB1\textsuperscript{-/-} VT and DT explant pairs (or manipulating EphB1 function in wild-type explants) in the Y assay will provide more insight into whether EphB1 mediates axon sorting in addition to its potential role in ipsi axon fasciculation.