\label{sec:LinkingEphB1}
One of the main motivations for studying axon organization in the optic tract of the EphB1\textsuperscript{-/-} mutant was to better understand the relationship between three steps in circuit formation: midline choice, tract organization, and targeting.
I have approached the EphB1\textsuperscript{-/-} mutant as a model of aberrant midline choice but grossly normal targeting.
As such, understanding RGC axon organization in the optic tract of this mutant provides an indirect way of linking the three axon guidance steps.
If misrouted EphB1\textsuperscript{-/-} axons associate with ``true'' contra axons, then a model in which tract organization is dispensable for targeting would be more likely.
Alternatively, if misrouted EphB1\textsuperscript{-/-} axons maintained an association with the remainder of the ipsi cohort, then a model in which tract organization is important for targeting would be more favorable.

This is, of course, an imperfect way of assessing the relationship between these three developmental phenomena. 
For one, it can only provide correlative evidence regarding the significance of tract order on targeting.
Ideally, we could selectively perturb axon organization in the optic tract and assess the effects of such perturbations on targeting in the dLGN and SC.
However, our current knowledge of tract-specific organizational cues and the developmental time course of RGC axon outgrowth both present significant challenges to direct experimentation to test the hypothesis that tract order is important for early targeting decisions.
While our understanding of midline cues directing ipsi and contra divergence in the optic chiasm is limited, we know almost nothing about the cells and potential cues within the optic tract itself.
This presents an obvious limitation to direct experimental perturbations of axon organization within the tract.

Additionally, in order to directly test whether tract organization affects targeting, independently of midline choice, we need to be able to selectively manipulate only axons that have passed the chiasm and entered the tract, without affecting those that have yet to or are in the midst of navigating the optic chiasm.
This is especially critical when testing molecules that potentially have multiple functions at different parts of the pathway --- e.g., EphB1, which is involved in midline choice, but may be involved separately in fasciculation and axon sorting. 
Therefore, a high degree of experimental precision is needed in order to dissociate effects on targeting that result from midline choice versus those resulting from tract-specific perturbations.
In the mouse visual system, this is particularly challenging because RGCs extend their axons in a continuous wave from $\sim$E12.5 to birth.
Thus, approaches like \inutero{} retinal electroporation affect RGCs with axons at various stages of growth along the pathway, making it difficult to determine if any targeting defects they may have are due to incorrect interactions with the midline, organization in the optic tract, or both.

The fact that ipsi RGC axons in the EphB1\textsuperscript{-/-} mutant incorrectly decussate but still manage to target the correct region in the dLGN suggests that targeting does not depend on correct midline choice (although appropriate refinement in the target may).
My finding that the aberrantly crossing axons preserve an association with the ipsilateral region of the tract is in line with a model in which axonal bundling partners in the tract may be important for axons reaching the appropriate region of their target, independently of midline choice.
The alternative, not mutually exclusive, explanation is that misrouted axons retain enough of their molecular identity as ipsilateral RGCs necessary for navigating local cues within the target.
Indeed, the misrouted axons in the EphB1\textsuperscript{-/-} mutant likely retain a molecular signature that enables them to associate appropriately with the remaining ipsi axons in the contralateral tract.
The putative molecular signatures important for axonal association in the tract and for local targeting decisions may be the same or distinct.

It is very challenging to precisely test the relationship between axon organization in the tract and axon targeting.
Proteomic analysis of ipsi and contra axons in the optic tract will enable molecular dissection of the cues relevant to tract organization, and will hopefully allow us to better differentiate molecular cues important for axon organization versus targeting.
A better understanding of the local molecular expression profiles of axonal cohorts as they navigate different steps of circuit formation, as well as which cues are redundant between the tract and target, are both necessary in order to directly test the effects of axon organizational perturbations on targeting.

Finally, it is worth noting that while the targeting of misrouted axons in the EphB1\textsuperscript{-/-} mutant is grossly normal, it does have defects.
Namely, the misrouted axons fail to undergo normal activity-dependent refinement of their arbors in the dLGN \cite{rebsam2009switching}.
Thus, while they reach the correct general area of the dLGN, the misrouted axons do not behave normally.
It has not yet been tested whether or not the misrouted axons are functionally integrated into the visual circuit, but the fact that they segregate away from both ipsi and other contra axon terminals suggests that they may be functionally abnormal \cite{rebsam2009switching}.

Evidence from a different commissural system supports the hypothesis that the misrouted RGC axons in the EphB1\textsuperscript{-/-} mutant are functionally perturbed, despite grossly normal targeting.
Axons that incorrectly route ipsilaterally instead of contralaterally at the Calyx of Held innervate the correct region in the contralateral target \cite{michalski2013robo3}, presenting an inversely analogous scenario as found in the EphB1\textsuperscript{-/-} mutant optic chiasm.
The aberrantly ipsilateral axons in this case form abnormal synapses, with defects in maturation, transmitter release, and release synchronicity \cite{michalski2013robo3}.
Thus, making the correct midline choice primes axons for appropriate synaptic maturity, which may similarly apply to the misrouted EphB1\textsuperscript{-/-} axons in the contralateral dLGN.
\citetext{michalski2013robo3} did not examine whether axon organization in the system was also perturbed, leaving open the possibility that defects within the axon tract also contribute to the synaptic dysfunction they reported.

Future studies in the EphB1\textsuperscript{-/-} mutant can address several questions that will further our understanding of how different steps of axon guidance are related to each other.
Proteomic analysis of axons in the optic tract of wild-type and EphB1\textsuperscript{-/-} mutants will identify how loss of EphB1 affects the expression of other axon guidance, organization, or targeting molecules within the optic tract.
Understanding the molecular expression of axons within the optic tract will help dissociate tract-specific and target-specific molecular interactions that help build the full circuit.
Thus, further analysis of the EphB1\textsuperscript{-/-} mutant, as well as more detailed molecular dissection in the wild-type optic tract, will help the field connect various steps along an axon's journey that have largely been studied as discrete events.