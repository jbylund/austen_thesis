Tracts and targeting

Many studies, including OB \cite{stjohn2003sorting} or optic tectum removal experiments \cite{reh1983organization}, indicate that pre-target axon organization occurs largely independently of target-derived cues.
However, the functional relevance of pre-target tract organization in circuit formation is still unclear, specifically whether or not tract order is necessary for appropriately specific synaptic connections of terminals in their targets.
Some of the studies discussed above examine the relationship between axon order in tracts and subsequent order in the targets and provide hints that pre-sorting of axons in tracts is in fact necessary for accurate target innervation.

In the olfactory system, selectively perturbing Nrp1 or Sema3A expression in a subset of OSNs disrupts axon order in consistent directions along the central-peripheral axis of the olfactory nerve and produces concomitant shifts in glomerular position in the OB \cite{imai2009pre}.
Similarly, in the CC, constitutive knockout of Sema3A or selective knockout of Nrp1 in the motor cortex both lead to a blurring of dorsal-ventral axon segregation in the CC \cite{zhou2013axon}.
More strikingly, there is a linear relationship between dorsal-ventral position of an axon in the CC and its subsequent medial-lateral position in the target contralateral cortex.
This linear relationship between tract position and targeting is true both in the wt context as well as in the experimental manipulations of CC axon order, leading the authors to conclude that axon position in the tract is a determining factor in establishing correct terminations within the target.
Importantly, postnatal refinement is largely unable to correct mistargeted axons in the contralateral cortex, underscoring the significance of early axon targeting decisions even in systems with known refinement mechanisms at play \cite{zhou2013axon}.

Findings from other systems also support the importance of pre-target axon order for fidelity of axon targeting.
Blocking Robo1 signaling in chick spinocerebellar neurons causes their axons to aberrantly fasciculate with other ascending spinal axons in the medial rather than lateral funiculus of the spinal cord marginal zone.
This incorrect positioning in the tract results in the axons bypassing the cerebellum and misprojecting instead to other hindbrain targets \cite{sakai2012axon}.
In the chick retinotectal system, blocking NCAM expression in focal subsets of RGCs causes affected axons to run in ectopically diffuse positions in the optic nerve and tract.
While roughly 15\% of these ectopic axons are able to make dramatic course corrections to reach the appropriate location once in the tectum, the rest project incorrectly \cite{thanos1984fiber}.
More direct evidence comes from the thalamocortical tract, where many cues within the subpallium organize ascending TCAs.
Selectively perturbing subpallium development blurs the topography of extending TCAs, leading to similarly blurred topography in the somatosensory cortex.
An elegant set of control experiments substantiates the conclusion that the targeting defects are a direct result of defective pre-target axon order, rather than aberrant cues in either the thalamus or cortex \cite{lokmane2013sensory}.

On the other hand, not all studies corroborate the importance of axon pre-sorting for accuracy of target innervation.
For one, plasticity is a hallmark of developing neural circuits, so pre-target errors may be correctable once axons invade the target.
Disrupted HSPG synthesis in the zebrafish optic tract, for instance, leads to axon disorganization in the optic tract, but RGC axons still appear to innervate the optic tectum in topographically correct positions \cite{lee2004axon}.
In the mouse periphery, loss of ephrin-B1 leads to defasciculation of motor and sensory axons, but motor axons innervate the limb in grossly normal patterns \cite{luxey2013eph}.
Further, while much evidence in the thalamocortical system supports a role for axon pre-sorting in establishing accurate connections, cues within the cortex also drive guidance and targeting behavior (reviewed in \cite{garel2014inputs}).
When the arealization of the cortex is perturbed, for example, TCAs retain a normal topography in the subpallium but undergo extensive rearrangements within the cortex in order to innervate the appropriate target region \cite{shimogori2005fibroblast}.
It is worth noting, however, that this study assessed gross targeting outcomes but not maturation of synaptic connections.
As discussed previously, Michalski et al. (2013) demonstrated in the auditory system the degree of subtlety in downstream effects of pathfinding perturbations.
Thus, while the TCAs innervating a disorganized cortex grossly find their correct targets, it is possible that subtle defects persist at the synaptic level.
The field is beginning to piece together the puzzle of how each step of circuit development affects subsequent steps along the pathway.


Conclusions

In 1956, very early in the history of considering pre-target axon sorting, Barnard and Woolsey reflected upon the trajectory of the field, noting a shift from focusing on “origins and terminations of the major fiber tracts,” towards examining “intratract localization of one sort or another.”
They contested the false dichotomy that axons must either maintain highly specific neighbor relationships in order to innervate a target appropriately or else they course along their pathways haphazardly until arriving at their destination \cite{barnard1956study}.
The last several decades of research highlight the complexity and subtlety of many developmental and guidance mechanisms in growing neural circuits.
Taking stock of the current understanding of intratract axon organization, we find that 1) topographic, chronotopic, and typographic order of axons are common modes of pre-target axon sorting (Figure 1) in white matter tracts, and 2) molecular mechanisms mediating these modes of order fall generally into axon-extrinsic and axon-intrinsic categories (Figure 2).
Glia, transient or migrating neuronal populations, and the ECM provide organizational cues to axons within the tracts.
Homotypic and heterotypic axon-axon interactions are proving to be important components in organizing developing tracts, though the detailed mechanisms of these interactions are still somewhat elusive.
Finally, current evidence strongly supports a role for pre-target axon order in creating accurate connections in targets, though plasticity in developing circuits is capable of mending early wiring errors to some degree.


Challenges \& Perspectives

The axon guidance field has historically divided neural circuits into experimentally manageable segments, focusing on the transcriptional regulation of the source neurons, axon guidance decisions at choice points or intermediate targets, and terminal guidance and refinement events within targets.
As our understanding of each segment of the neural pathways improves, so too have the available experimental tools, opening up new avenues of inquiry into the guidance mechanisms within tracts coursing between the segments so far studied.
These new avenues of inquiry of course uncover greater overall complexity in developing neural circuits.
Determining whether pre-target axon organization is a mere consequence of either a neuron’s transcriptional identity or events at choice points, as opposed to an active step in organizing the overall circuit, remains the greatest challenge.
In other words, are guidance events and axon-axon interactions within the tract instructive for subsequent targeting events, or are local guidance cues within the target sufficient for ordering incoming axons, so long as they reach the target?
Additionally, it is unclear if defasciculation in a tract necessarily leads to tract and target disruptions, or if other organizational mechanisms can compensate for disordered fasciculation within tracts.

One particular challenge in answering these questions is the fact that the same molecules may be acting at intermediate targets, between axons (fasciculation and repulsion), and in the target (between axons and target cells).
New approaches will be needed in order to fully understand how each element of the pathway contributes to final targeting events.
In this regard, novel in vitro assays may assist in teasing apart the relative contributions of guidance molecules at choice points and within tracts.
In tracts with less well understood pre-target axon organization, advanced labeling techniques like Brainbow \cite{lu2009interscutularis} or electroporation of fluorescent proteins \cite{saito2001efficient} along with tissue clearing methods to enhance visualization (e.g., \cite{erturk2012three,kuwajima2013cleart,tomer2014advanced} will be particularly useful in more fully detailing modes of order within tracts. %Add new rabies stuff from Reardon?
Additionally, modern molecular genetic techniques now allow for the selective perturbation of gene expression in specific neuronal subpopulations or at particular points along an axon’s course, which will allow us to better comprehend how each part of the pathway contributes to formation of the whole circuit.

Understanding the subtleties of white matter tract formation by studying pre-target axon order in normal development is a necessary step towards unraveling the basis of neurodevelopmental disorders.
Diffusion tensor imaging studies have found white matter tract abnormalities in subjects with autism spectrum disorder \cite{wolff2012differences} and schizophrenia \cite{kubicki2007review}.
A better grasp on the principles of tract development and organization, and the relationship between tracts and targets, is necessary to more effectively study and treat developmental disorders of neural circuitry.
Given the multi-stepwise process of circuit formation, perturbations in pre-target order may not necessarily produce drastic shifts in targeting, and as such could appear relatively inconsequential to the overall order of the circuit.
However, the devil is likely to be in the details, as it were, and the effects such perturbations have on targeting are more likely to be subtle and possibly even undetectable except at the level of individual synapses.
Examining developing axon tracts with a keen eye towards such subtle aberrations in circuit formation and function is central to unlocking underlying mechanisms of complex developmental brain disorders.
