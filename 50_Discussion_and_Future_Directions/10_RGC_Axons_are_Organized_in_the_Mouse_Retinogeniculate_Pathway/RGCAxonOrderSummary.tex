In Chapter 2, I demonstrated that RGC axons are organized by both laterality and topography in the optic tract, and, furthermore, that these two modes of axon organization are largely in register with one another.
Ipsi RGC axons are situated in the lateral optic tract, largely segregated from contra axons.
Topographic RGC organization in the optic tract is evident as well, with the dorsal-ventral axis of the retina mapped to the medial-lateral axis in the tract, and the nasal-temporal axis mapped to the rostral-caudal axis of the tract (summarized in Figure~\ref{Figures/TopographySummary}).
The eye-specific map of axon laterality (i.e., ipsi or contra) is shifted slightly laterally relative to the topographic map in the optic tract.

The relationship between these two modes of axon organization in the optic tract raises interesting questions about multimodal principles of axon organization.
For instance, is there a hierarchy among different organizational modes?
Does topography carry more weight in organizing the tract than typography (or laterality in the case of ipsi and contra RGC axons)?
Or are all modes of organization relatively equivalently enmeshed together in the overal order?

Another question to consider is how the topographic and eye-specific modes of axon organization I have described in this thesis relate to other organizational modes.
Chronotopic organization has been described in the visual system (e.g., \citenoparens{reese1987distributionrat,colello1992observations,colello1998changing}), largely tracing growth cone disposition in electron microscopy (EM) images.
There is still much we don't know about the mechanisms guiding chronotopic order of RGC axons is lacking, particularly in the mouse visual system.
Disruption of radial glia-associated chondroitin sulfate proteoglycans (CSPGs) is known to perturb chronotopic organization of RGC axons in the ferret optic tract \cite{reese1997chronotopic,leung2003enzymatic}, suggesting that extrinsic cues within the optic tract guide age-related order.
However, without a more detailed description of the chronotopic axon organization in the mouse optic tract, it is hard to draw comparisons between chronotopic, topographic, and eye-specific modes of RGC axon organization.

One mode of typographic organization that has not yet been studied in the mouse optic tract is based on genetically identified RGC functional subtypes.
However, especially as our ability to identify and label RGC subtypes improves (e.g., \citenoparens{rivlin2011transgenic,baden2016functional}), it will particularly fascinating to explore the organization of functional subtype specific RGC axons in the optic nerve and tract.
Perhaps all subtypes follow their own organizational principles within the tract, forming distinct subtype-related bundles.
Alternatively, functional subtype identity may be of less consequence for tract organization and targeting, and subtype labeled axons may be scattered throughout the topographic map within the optic tract.
It will be particularly fascinating to compare the organization and distribution in the optic tract of axons from RGC subtypes that are scattered throughout the retina (e.g., DRD4\textsuperscript{+} direction selective RGCs, \citenoparens{huberman2009genetic}) with those that have a more topographically restricted retinal position (e.g., PV\textsuperscript{+} RGCs, \citenoparens{baden2016functional}).
In the latter case, determining whether subtype-specific axons form a distinct bundle within the topographically-defined axon cohort in the tract will provide insight into how topography and typography are related in the mouse retinogeniculate pathway.

In addition to the anterograde labeling I used in Chapter 2 to map the eye-specific and topographic RGC axon organization in the mouse optic tract, I also used SERT-Cre:zsgreen genetic labeling to track ipsi RGC axons in the optic nerve, chiasm, and tract.
This genetic labeling revealed two interesting features of this cohort of axons.
First, there appear to be two separate fascicles of SERT-Cre:zsgreen\textsuperscript{+} ipsi axons in the optic nerve -- a primary bundle in the ventrolateral optic nerve and a secondary, more loosely organized bundle positioned more medially.
The significance of this apparent segregation within the ipsi axon cohort is unclear.
It is possible that these two subsets are destined for different targets -- the primary bundle is perhaps destined for the thalamic targets (the dLGN and SC), while the minor bundle may route to accessory visual targets.
This is purely speculation, though, and further studies are needed to dissect what difference, if any, exists in the identity and destination of these two apparent subpopulations of SERT-Cre:zsgreen\textsuperscript{+} axons in the optic tract.

More relevant and interesting, however, are the apparent dynamic changes in fasciculation of ipsi RGC axons (as labeled by SERT-Cre:zsgreen\textsuperscript{+}) in the nerve, chiasm, and tract.
Zsgreen\textsuperscript{+} axons appear well fasciculated in the proximal optic nerve, only to progressively unravel as they near the chiasm, and subsequently re-form a well segregated bundle in the lateral optic tract (see Figure~\ref{Figures/SertNerveToTract}).
This dynamic fasciculation is in line with reports that topographic order is progressively lost along the proximal to distal optic nerve, is unclear in the chiasm, and is regained in the optic tract \cite{torrealba1982studies,reh1983organization,reese1993reestablishment,chan1994changes,chan1999changes,plas2005pretarget}.
It is interesting to note that two distinct modes of axon organization (topography and laterality) undergo a similar dynamic as axons progress through the optic chiasm.
These observations raise the question of whether there is indeed organization within the chiasm itself that our methods have so far failed to identify.
Additionally, this unbundling of axons near the chiasm and re-bundling in the optic tract strongly suggest the presence of active organizational mechanisms in the tract. 