Chapter 3 presented a series of novel \invitro{} assays to assess selective fasciculation of ipsi and contra RGCs (based on neurite outgrowth from E14.5 VT and DT retinal explants, respectively).
Data from these assays collectively demonstrate that ipsi (i.e., VT) neurites fasciculate more frequently (Figure~\ref{Figures/EnFaceData}B) and to a greater extent (Figure~\ref{Figures/BundleWidth}B) than contra (i.e., DT) neurites.
These data suggest that one of the mechanisms involved in ipsi/contra RGC axon segregation in the optic tract \invivo{} is selective fasciculation of ipsi axons.

\Invivo{}, the ipsi RGC axon cohort appears to become less well fasciculated as it nears the optic chiasm, which seems to be a general principle of all axon cohorts in the retinofugal pathway (see previous section).
Thus, it is somewhat counterintuitive that \invitro{}, the addition of chiasm cues results in increased fasciculation of retinal neurites.
This may be due in part to the limitations of a reductionist \invitro{} approach --- i.e., guidance and fasciculation factors may behave differently \invitro{} than they do \invivo{}.
Additionally, chiasm cells are refractory to neurite outgrowth \invitro{} \cite{wang1995crossed,wang1996chemosuppression}, and increased fasciculation may result from reduced neurite extension in a growth restrictive environment.
In other words, if the growth environment or substrate is less conducive to neurite growth, neurites will preferentially choose other neurites as their growth substrate, rather than the substrate.

In the chiasm co-culture experiments, I used a low density of dissociated chiasm cells so as to minimize the impact of the chemosuppression signals that are refractory to growth.
Neurite extension was still visibly reduced in these cultures, though, so the increased fasciculation may at least partially be a consequence of a growth restrictive environment.
One way of dealing with this limitation would be to assess neurite fasciculation using live imaging.
After establishing a baseline of fasciculation in a chiasm-free culture, dissociated chiasm cells or specific molecules could be added to the culture, and changes in fasciculation could be measured as the neurites respond to the newly introduced cues.
Indeed, as discussed at the end of Chapter 3, live imaging will be an extremely useful tool in furthering our understanding of how neurites interact with one another in different environments and how different cues and neighboring neurites affect fasciculation.

One issue that I have attempted to address, specifically with the Y assay, is the role of selective fasciculation in axon sorting.
Selective self-fasciculation is one mechanism of sorting different axon cohorts and likely occurs by a combination of increased self-fasciculation of one cohort and repulsion between the two cohorts.
\citetext{nishikimi2011segregation} studied axon sorting of medial and and lateral cortical explants \invitro{} using an assay similar to the \emph{en face} assay I used in Chapter 3 (Section~\ref{sec:EnFace}), but with much greater neurite outgrowth than I used to study single fasciculation events.
The assay in \citetext{nishikimi2011segregation} provides a reliable measure of repulsion between like and unlike pairs of cortical explants, and suffers less from the issue of head-to-head growth cone avoidance than my \emph{en face} assay largely because explants were grown very close together in conditions that encouraged prolific neurite outgrowth.
Thus, their assay provides a good system for measuring robust repulsive effects between explant type, but may fail to detect less extreme differences between explants from another system.

It may be worth using the \emph{en face} with greater neurite outgrowth to test VT and DT neurite attraction and avoidance, as \citetext{nishikimi2011segregation} have done with medial and lateral cortical explants. 
However, the Y assay presented in Chapter 3 (Section~\ref{sec:YAssay}) will hopefully provide a more detailed assessment of neurite sorting and, especially when examined by live imaging, neurite-neurite interactions between different neurite types in different growth environments.
Indeed, the differences in fasciculation revealed in the bundle width assay are not extreme in size, though they are statistically significant.
This indicates that selective fasciculation behaviors between ipsi and contra axons are reliably present, but relatively subtle in nature.
Studying such subtle aspects of axon sorting and axon guidance, while challenging, is valuable.
This is especially true as it relates to studying developmental disorders of circuit development, which likely arise from an accumulation of many developmental aberrations.
Such developmental defects may be subtle or small in size, but, as with the accumulation of many small copy number variants in the genome, can collectively amount to profound deficits \cite{rutkowski2016unraveling}.