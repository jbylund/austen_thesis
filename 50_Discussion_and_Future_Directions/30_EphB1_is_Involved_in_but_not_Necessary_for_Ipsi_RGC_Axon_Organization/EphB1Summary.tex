In Chapter 4, I explored eye-specific axon organization in the optic tract of EphB1\textsuperscript{-/-} mutants and the role of EphB1 in selective fasciculation of ipsi axons.
Additionally, I used the EphB1\textsuperscript{-/-} mutant as a model for exploring the relationship between midline choice, tract organization, and targeting.
I found that the remaining ipsi axons in the EphB1\textsuperscript{-/-} optic tract are positioned grossly normally, but are disordered in their fasciculation and organization.
Ipsi axons sometimes stray into the medial optic tract and appear less strictly segregated from contra axons in the mutant tract.
Using the SERT-Cre:zsgreen;EphB1 mouse, I determined that the misrouted axons in the EphB1\textsuperscript{-/-} mutant largely retain their position in the ipsilateral zone of the optic tract.

In sum, these data suggest that while EphB1 is not necessary for the general positioning of ipsi RGC axons in the optic tract, it may be involved in ipsi axon fasciculation and ipsi/contra axon segregation.
Results from the bundle width assay with EphB1\textsuperscript{-/-} DT and VT explants indicate that EphB1 is dispensable for ipsi axon fasciculation, though given the apparent decreased magnitude in effect size compared with wild-type explants, it may be partially involved.
Direct testing of a role for EphB1 in mediating ipsi axon fasciculation can be performed with EphB1 function blocking experiments in wild-type cultures to confirm this hypothesis.
A similar approach can be taken to explore other candidates molecules potentially involved in fasciculation and axon sorting.

Findings from the SERT-Cre:zsgreen;EphB1\textsuperscript{-/-} mouse indicate that the integrity of the ipsilateral cohort -- in this case, both the remaining ipsilaterally projecting axons and the misrouted axons that project contralaterally -- is grossly maintained.
This could be the result of a maintenance of other ipsilateral cues that mediate ipsi axon association and/or position in the optic tract, and also possibly cues that mediate targeting to the ipsi-recipient zone in the dLGN.
Had misrouted axons in the SERT-Cre:zsgreen;EphB1\textsuperscript{-/-} positioned in the medial aspect of the tract or scattered throughout, then it would be less likely that an axon's position in the tract influences its targeting.
However, finding, as I did, that the ipsi cohort largely maintains its self-association in the SERT-Cre:zsgreen;EphB1\textsuperscript{-/-} optic tract, only indirectly support the hypothesis that an axon's bundling partners in the tract influence targeting.
This hypothesis requires significantly more examination.
In order to directly test the hypothesis, we first need a better understanding of the cues present in the tract as well as a more complete understanding not only of the genetic expression profiles of ipsi and contra RGCs, but also local proteomic profiles of ipsi and contra RGC axons at different points along the retinogeniculate pathway.

My analysis presented in Chapter 4 extends our understanding of the EphB1\textsuperscript{-/-} mutant and identifies a potential role for EphB1 in ipsi axon fasciculation.
However, much remains to be explored in terms of the role of Ephs and ephrins in axon organization in the retinogeniculate pathway.
For one, Ephs and ephrins might be involved in organizing axons topographically within the tract, similar to their role in patterning topography in the target.
As is the case in the target \cite{weth2014chemoaffinity}, interactions between axons and their environment as well as axon-axon interactions likely both contribute to establishing and maintaining axon organization in the tract.
One of the first steps toward furthering our understand of the role of Ephs and ephrins in the tract is to identify the expression patterns of Ephs and ephrins both in the tract and on RGC axons inside the tract.
This will allow us to specifically examine Eph/ephrin interactions between different cohorts of axons, and between axons and the tract environment.