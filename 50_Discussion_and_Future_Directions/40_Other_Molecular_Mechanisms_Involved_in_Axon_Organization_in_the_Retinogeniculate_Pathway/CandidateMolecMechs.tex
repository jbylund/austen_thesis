There are several possible molecular mechanisms mediating axon organization in the optic tract, which most likely function in concert to establish and maintain tract order.
Specific axon-axon interactions and selective fasciculation are involved.
My data presented in Chapter 3 show that ipsi axons preferentially self-fasciculate compared with contra axons.
The molecules involved in this selective fasciculation remain largely unknown, but EphB1 appears to be partially involved.
Axon-axon interactions include selective self-fasciculation as well as repulsion between axon cohorts, a phenomena which the \emph{en face} assay suggested and the Y assay may provide further evidence for.
Beyond axon-axon interactions, extrinsic cues within the optic tract are likely involved in organizing RGC axon cohorts along their pathway.
In this section, I will discuss a variety of candidate molecules that could be involved in axon-axon interactions and/or extrinsic cues.

In addition to my examination of EphB1 in Chapter 4, I discussed in Chapter 3 a selection of candidate molecules from our lab's recent microarray screen of ipsi and contra RGCs \cite{wang2016ipsilateral}, which would be interesting to test in the in vitro assays of fasciculation.
In addition to the three adhesion and cytoskeleton-interacting molecules discussed in Section~\ref{sec:InVitroFuture}, several Semaphorins (Semas) were identified as differentially expressed by ipsi and contra RGCs at E16.5 \cite{wang2016ipsilateral}.
Given the broad roles played by Semas in neural development and axon guidance \cite{raper2000semaphorins}, these will be particularly interesting candidates to explore for a role in axon organization.
Specifically, Sema5b and Sema4d are more highly expressed in ipsi RGCs, and Sema3e and Sema7a are both more highly expressed by contra RGCs \cite{wang2016ipsilateral}.
Semas play a role in axon organization in the olfactory nerve \cite{imai2009pre} and the corpus callosum \cite{zhou2013axon}, making the differential expression of these Semas in ipsi and contra RGCs particularly compelling to study further.
A combination of in vitro assessments of fasciculation and axon sorting, along with in vivo analyses of tract order could reveal a role for Semas in eye-specific axon organization in the optic tract.

While the microarray screen provides several new candidates to explore, it cannot be used in isolation to identify interesting candidates for fasciculation and axon organization.
Microarrays have inherent technical limitations, and this one in particular examined a single developmental time point in RGC development (E16.5).
Because of this, EphB1 was not found to be differentially expressed by ipsi and contra RGCs at E16.5 in this study, even though it is known to be necessary and sufficient for mediating the ipsilateral midline choice \cite{williams2003ephrin,petros2009specificity}.
EphB1 expression in the retina is temporally dynamic during RGC development and axon outgrowth, peaking in the VT retina at E14.5-E15.5, before the age that gene profiling was performed by \citetext{wang2016ipsilateral}.
Furthermore, gene profiling of RGCs provides limited information on the local translation events occurring at the growth cone - genes that found to be equally expressed in ipsi and contra RGCs may well be differentially locally regulated in the distal axon and growth cone.
This layer of complexity will require a combination of approaches to examine not only the genetic expression patterns in the RGCs but also local proteomic analyses of axons at different points in the pathway.

One interesting group of candidate genes that was not identified by \citetext{wang2016ipsilateral} is the teneurin family or type II transmembrane glycoproteins.
(Several teneurins were found in the microarray screen, but none with significant differences between ipsi and contra RGC expression levels.)
Teneurins are involved in numerous elements of neural development, including axon guidance, dendritic morphology, fasciculation, targeting, and synaptic development \cite{leamey2014teneurins}.
Teneurins have been implicated in targeting and formation of topographic maps in the visual system (reviewed in \citenoparens{leamey2014teneurins}).
Specifically, \emph{Ten-m2} is required for appropriate guidance of ipsi RGC axons \cite{young2013ten}.
When \emph{Ten-m2} is deleted, the ipsi retinal projection is decreased, similar to the EphB1\textsuperscript{-/-} mutant.
Furthermore, EphB1 itself is downregulated in the Ten-m2 mutant, specifically in the ventral retina, while Zic2 expression was unaltered \cite{young2013ten}.
The details of the molecular interactions between Ten-m2 and EphB1 remain unclear, but given the role of Ten-m2 and the teneurin family generally in development of the binocular circuit, they are compelling candidates to study further in the context of axon organization.

Proteoglycans also present themselves as interesting molecular candidates to explore.
Heparan sulfate proteoglycan (HSPG) function is required for proper retinal axon sorting in the zebrafish \cite{lee2004axon}, and chondroitin sulfate proteoglycan (CSPG) is involved in chronotopic organization of retinal axons in the optic tract \cite{leung2003enzymatic}.
%Thus, it would be interesting to further study both HSPGs and CSPGs in the mouse optic tract and probe their involvement with topographic and eye-specific axon sorting therein.
Furthermore, HSPGs interact with many axon guidance molecules, including the Eph-ephrin, Netrin-DCC, and Slit-Robo pathways (reviewed in \citenoparens{lee2004sugars,masu2016proteoglycans}).
HSPGs could affect distribution of guidance molecules, and may also act as co-receptors, or function as ligands themselves.
Proteoglycan distribution along the retinofugal pathway (e.g., \citenoparens{mcadams1995expression,chung2000expression,ichijo2006restricted}) situates this large family of sugar molecules to provide both guidance and organizational cues to RGC axons, possibly even beyond their known role in mediating chronotopy \cite{leung2003enzymatic}.

The proteoglycan interaction with the Slit-Robo may be particularly interesting to study in the optic tract.
Slit1 and 2 are important for RGC axon navigation through the optic chiasm by creating a repellent border surrounding the chiasm region \cite{plump2002slit1}.
Loss of \emph{Slit1/2} leads to RGC axon defasciculation and misrouting in the chiasm region \cite{plump2002slit1}. 
Furthermore, specific heparan sulfation patterns on RGCs are important for mediating RGC axon repulsion from Slit cues surrounding the chiasm region \cite{pratt2006heparan}.
Similar regulation of sulfation patterns and proteoglycan interactions with Slit cues could be involved in axon fasciculation in the optic tract as well.

The expression of proteoglycans in the extracellular matrix (ECM) along the visual pathway \cite{mcadams1995expression,ichijo2006restricted} raises the interesting question of what other cues are expressed by the ECM and glia within and surround the optic tract.
Distinct changes in the distribution and morphology of glia along the optic nerve, chiasm, and tract have been documented (e.g., \citenoparens{maggs1986glial,guillery1987changing,vanselow1989spatial,sun2009morphology}).
The role of glia in fasciculation, axon guidance, and synaptic development is increasingly appreciated and well understood (e.g., \citenoparens{schafer2012microglia,squarzoni2014microglia,pont2014microglia,molofsky2014astrocyte} and reviewed in \citenoparens{learte2007role,clarke2013emerging,stogsdill2017interplay}).
We have almost no knowledge of what guidance, adhesion, attractive, repulsive, or other organizational molecules are expressed by glia in the tract.
I have conducted preliminary examinations of the expression patterns of astrocytes and microglia in the developing optic tract, which so far suggest that these cell types are well positioned both in and around the optic tract to potentially provide organizational cues to RGC axons.
Specifically, there is a preponderance of microglia in the pia lining the outside of the diencephalon, along the lateral edge of the optic tract -- where ipsi RGC axons are situated.
Thus, a testable hypothesis is that microglia in this region are providing attractive cues to ipsi axons and/or repellent cues to contra axons.

As with EphB1 and other Ephs and ephrins, the challenge in studying the molecules discussed here is their presence at multiple steps along the pathway.
Thus, experimentally teasing apart separate functions of candidate molecules at the midline choice point, within the tract, and in the target will require sophisticated tools and elegant experimental design.
Additionally, functional redundancy in many guidance molecules presents another challenge, such that single gene knockouts may have minimal defects, while double mutants have sever perturbations (as in, for instance, \citenoparens{plump2002slit1}).