In 1956, very early in the history of studies considering pre-target axon sorting, \citetext{barnard1956study} reflected on the trajectory of the field, noting a shift from focusing on ``origins and terminations of the major fiber tracts,'' towards examining ``intratract localization of one sort or another.''
They contested the false dichotomy that axons must either maintain highly specific, one-to-one neighbor relationships in order to innervate a target appropriately, or else course along their pathways haphazardly until arriving at their destination \cite{barnard1956study}.
The last several decades of research highlight the complexity and subtlety of many developmental and guidance mechanisms in growing neural circuits.
Taking stock of the current understanding of intratract axon organization, we find that topographic, chronotopic, and typographic order of axons are common modes of pre-target axon sorting in white matter tracts (Figure~\ref{AxonOrgThreeModes}), and molecular mechanisms mediating these modes of order fall generally into axon-extrinsic and axon-intrinsic categories (Figure~\ref{AxonOrgMechs}).
Glia, transient or migrating neuronal populations, and the ECM provide organizational cues to axons within tracts.
Homotypic and heterotypic axon-axon interactions are important components in organizing developing tracts, though the detailed mechanisms of these interactions are still somewhat elusive.
Finally, current evidence strongly supports a role for pre-target axon order in creating accurate connections in targets, though plasticity in developing circuits is capable of mending early wiring errors to some degree.

The axon guidance field has historically divided neural circuits into experimentally manageable segments, focusing on the transcriptional regulation of the source neurons, axon guidance decisions at choice points or intermediate targets, and terminal guidance and refinement events within targets.
As our understanding of each segment of the neural pathways improves, so too have the available experimental tools, opening up new avenues of inquiry into the guidance mechanisms within tracts coursing between the segments so far studied.
These new avenues of inquiry of course uncover greater overall complexity in developing neural circuits.
Determining whether pre-target axon organization is a mere consequence of either a neuron's transcriptional identity or events at choice points, as opposed to an active step in organizing the overall circuit, remains the greatest challenge.
In other words, are guidance events and axon-axon interactions within the tract instructive for subsequent targeting events, or are local guidance cues within the target sufficient for ordering incoming axons, so long as they reach the target?
Additionally, it is unclear if defasciculation in a tract necessarily leads to tract and target disruptions, or if other organizational mechanisms can compensate for disordered fasciculation within tracts.
For instance, despite disorganization and defasciculation in the EphB1\textsuperscript{-/-} optic tract, axons still grossly target the dLGN normally.
However, they refine improperly and may not be fully functional, hinting at the importance of proper organization for functional innervation.

One particular challenge in answering these questions is the fact that the same molecules often act at intermediate targets, between axons (fasciculation and repulsion), and in the target (between axons and target cells).
New approaches will be needed in order to fully understand how each element of the pathway contributes to final targeting events.
In this regard, novel \invitro{} assays, such as those presented in Chapter 3, may assist in teasing apart the relative contributions of guidance molecules at choice points and within tracts.
In tracts with less well understood pre-target axon organization, advanced labeling techniques such as electroporation of fluorescent proteins \cite{saito2001efficient} or new viral tracing approaches \cite{reardon2016rabies}, along with tissue clearing methods to enhance visualization (e.g., \citenoparens{erturk2012three,kuwajima2013cleart,tomer2014advanced}) will be particularly useful in more fully detailing modes of order within those tracts.
Additionally, modern molecular genetic techniques now allow for the selective perturbation of gene expression in specific neuronal subpopulations or at particular points along an axon's course, which will allow us to better comprehend how each part of the pathway contributes to formation of the whole circuit.

Understanding the subtleties of white matter tract formation by studying pre-target axon order in normal development is a necessary step towards unraveling the basis of neurodevelopmental disorders.
Diffusion tensor imaging studies have found white matter tract abnormalities in subjects with autism spectrum disorder \cite{wolff2012differences} and schizophrenia \cite{kubicki2007review}.
A better grasp on the principles of tract development and organization, and the relationship between tracts and targets, is necessary to more effectively study and treat developmental disorders of neural circuitry.

Given the multi-stepwise process of circuit formation, perturbations in pre-target order may not necessarily produce drastic shifts in targeting, and as such might at first appear relatively inconsequential to the overall order of the circuit.
However, the devil is likely to be in the details, as it were, and the effects such perturbations have on targeting are more likely to be subtle and possibly even undetectable except at the level of individual synapses (e.g., \citenoparens{michalski2013robo3}).
Examining developing axon tracts with a keen eye towards such subtle aberrations in circuit formation and function is key to unlocking underlying mechanisms of complex developmental brain disorders.

The work presented in this thesis contributes to our growing understanding of axon organization and provides evidence for a role of selective fasciculation as one mechanism underlying such order.
Data from the EphB1\textsuperscript{-/-} mutant additionally provide indirect support of a model in which tract organization is important for targeting and overall development of a neural circuit.
Future studies in this system can explore the relationship between different modes of axon organization in the optic tract to identify if there is a hierarchical ordering of chronotopy, topography, and typography (including both laterality and functional subtypes).
Furthermore, there is a great deal left to discover about what cues exist within the tract and how axon interactions with such cues, along with axon-axon interactions give rise to an ordered tract.
Finally, and perhaps most importantly, experiments that selectively and specifically perturb tract organization can eventually demonstrate whether or not axon organization in the tract is instructive or important for early targeting decisions.

In all of these future assessments, an eye to fine grained detail will be necessary.
The development of an accurate, precisely functioning neural circuit is the result of a complex series of axon guidance and navigation steps.
%Each of these steps builds on the previous step to varying degrees, and it is still unclear the degree to which perturbations in one step of an axon's journey can be corrected for in subsequent steps, or if subtle aberrations progressively accumulate along the journey.
%In other words, how crucial is it for each step of the axon's journey to be kept precise -- to what extent does a high degree of accuracy in each part of the pathway set an axon up for success in forming its eventual synaptic connections.
Neural circuits are plastic and adaptable, and small errors in neural development are corrected with relative ease in subsequent steps of circuit formation.
But what are the degrees of freedom in the extent to which a developing circuit can be disordered pre-synaptically and still be successfully corrected and refined at the synaptic level to form a properly functioning circuit?
Perhaps the more orderly axons are along their journey, the better equipped they are to form precise and accurate connections within their target.
And perhaps minor aberrations early in circuit development can accumulate to have detrimental functional consequences.
Alternatively, local cues within the target may be robust enough to overcome such early minor pathfinding errors, rendering tract organization a less important step in circuit formation.
Future studies addressing these questions will enhance our understanding of neural developmental processes and provide much needed insight into complex neurodevelopmental disorders.