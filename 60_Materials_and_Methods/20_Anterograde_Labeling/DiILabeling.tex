Anterograde labeling was performed with DiI (1,1$'$-Dioctadecyl-3,3,3$'$,3$'$-Tetramethylindocarbocyanine Perchlorate (DiIC\textsubscript{18}(3))) and its far red shifted analogue, DiD (1,1$'$-Dioctadecyl-3,3,3$'$,3$'$-Tetramethylindodicarbocyanine, 4-Chlorobenzenesulfonate Salt) (Molecular Probes).
Embryos were harvested from pregnant dams anesthetized with ketamine/xylazine (100 and 10mg/kg, respectively, in 0.9\% saline), transcardially perfused with 4\% paraformaldehyde (PFA), and post-fixed overnight (O/N) at \textcelsius{4}.
Postnatal pups were anesthetized with ketamine/xylazine (100 and 10mg/kg, respectively, in 0.9\% saline), transcardially perfused with 4\% PFA, and post-fixed O/N at \textcelsius{4}.
Perfused heads were rinsed twice in 1X phosphate buffered saline (PBS) before labeling.

For whole eye anterograde labeling, the eyecup was removed and DiI or DiD crystal was placed directly onto the optic nerve head using the fine tip of a pulled glass micropipette.
The tip of the micropipette to crush the DiI or DiD crystals into the optic nerve head, ensuring full label of the retinal axons.
The eyecup was replaced to hold the DiI or DiD in place for the duration of incubation.
For focal anterograde labeling (VT/VT or topographically restricted label), the lens was removed, leaving the rest of the eyecup intact.
The fine tip of a pulled glass micropipette was used to place a small amount of DiI or DiD crystal into the designated peripheral retinal quadrant, ensuring penetration of the RGC layer.
The lens was gently replaced to help keep the DiI or DiD in place for the duration of incubation.
Embryonic samples were incubated at \textcelsius{37} for 6-12 days; postnatal samples were incubated at \textcelsius{37} for 10-14 days.
SERT-Cre:zsgreen;EphB1 samples labeled with DiD were incubated for 14-16 days at room temperature (RT) in order to maintain zsgreen fluorescent signal.

After incubation to allow dye transport, brains were carefully removed and sectioned on the vibratome 75$\mu$m thick in either frontal or horizontal plane, and mounted sequentially for imaging.
Retinae with focal labeling were removed after cutting a slit into the ventral retina for orientation, then flat-mounted to confirm dye placement and extent of labeling.