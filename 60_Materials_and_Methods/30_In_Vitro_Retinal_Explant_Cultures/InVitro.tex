Retinal explant procedures have been described previously by \citetext{wang1996chemosuppression} and \citetext{kuwajima2012optic}.
For all retinal explant experiments, E14.5 embryos were harvested from pregnant dams that were anesthetized with ketamine/xylazine (100 and 10mg/kg, respectively, in 0.9\% saline) and then sacrificed by cervical dislocation.
Embryos were quickly removed and collected into cold DMEM/F12, where they were removed from their individual amniotic sacs and placed into a fresh dish of cold DMEM/F12.
Embryos were then decapitated using forceps and heads were placed into a fresh dish of cold DMEM/F12.
Microscissors were used to dissect each eye of each head, making a small cut into the ventral retina to indicate orientation.
The lens and interior vasculature were removed and the eyecups collected into separate wells of cold DMEM/F12, keeping left and right retinas separate.
After all retinae were collected, each one was then dissected into retinal explants, approximately 300x300$\mu$m, collecting VT and DT explants separately.
The central-most and peripheral-most portions of the retina were removed, and the intermediate region between V and D on the temporal hemiretina was also discarded before collecting VT and DT explants.

Explants were plated in serum free media (SFM) + 0.4\% methylcellulose (Sigma catalog no. M-0512).
SFM is made by vortexing together the first two ingredients listed below, then gently mixing in the second two ingredients:
\begin{itemize}
	\item[-] 50ml DMEM/F12
	\item[-] 0.5g fatty acid free bovine serum albumin (BSA)\textsuperscript{*}
	\item[-] 500$\mu$l ITS supplement (Sigma catalog no. I-1884)
	\item[-] 100$\mu$l Pen/Strep (Invitrogen catalog no. 15140-122)
\end{itemize}
\textsuperscript{*}Note that Sigma catalog no. A8806-5G was used for all experiments and different BSA strongly affects outgrowth and fasciculation.
SFM + 0.4\% methylcellulose solution is left stirring O/N at \textcelsius{4} and filtered using a 0.2$\mu$m filter before using.

Control experiments measuring Zic2 expression in VT and DT cultures in standard culture conditions shows that VT explants retain ipsi-specific molecular expression (Figure~\ref{Zic2invitro}).
Zic2 expression is preserved at 4 and 24 h in vitro, although it largely disappears by 48 h in vitro.
This is due to the fact that Zic2 protein expression is an imperfect measure of ipsi identity, as its expression is very transient in vivo, and is therefore likely similarly downregulated in vitro.
\begin{figure}[hbtp]
    \begin{center}
        \includegraphics{Figures/Zic2_invitro.pdf}
        \caption[Zic2 differential expression between VT and DT retina is preserved in vitro.]
        {Zic2 differential expression between VT and DT retina is preserved in vitro.
		Percent of Islet1/2\textsuperscript{+} that are also Zic2\textsuperscript{+} in VT and DT explants cultured in vitro for 4, 24, and 48 h.
		Differential expression is clearest soon after explants are plated, is well maintained at 24 h, and is gone by 48 h, likely because Zic2 expression is transient in vivo and has been similarly downregulated in vitro.
		Data are shown as means $\pm$ standard error of the mean.
		}
        \label{Zic2invitro}
    \end{center}
\end{figure}

For explants in the \emph{en face} assay, before collecting explants, decapitated embryonic heads were electroporated with GFP or mCherry in either the dorsal or ventral hemiretina using ex vivo electroporation, as described in \citetext{petros2009utero} and schematized in Figure~\ref{Figures/EnFaceAssayDesign}.
After electroporation, retinae were dissected from the head as described above and incubated in SFM O/N at \textcelsius{37} to allow expression of fluorescent proteins.
Before plating explants, glass-bottomed culture dishes were incubated O/N at \textcelsius{4} with poly-DL-ornithine, which was rinsed the following day three times with cold DMEM/F12.
Dishes were then incubated at \textcelsius{37} for at least 2 h with 8$\mu$g/ml laminin (add manufacturer info).
After incubating in SFM O/N, fluorescently labeled VT and DT retinal explants were collected as above and plated in SFM + 0.4\% methylcellulose using fine-pointed forceps to gently but firmly push the explants onto the glass coverslip (in a glass bottomed culture dish prepped as described) with explants pairs placed 1.5mm apart.
Explant pairs were incubated O/N at \textcelsius{37}, and fixed for 15-20 min with warm 4\% PFA the following day, and then immunostained for GFP, mCherry, and NF.

Analysis of the \emph{en face} assay was performed in Neurolucida (v11, MBF Biosciences, Williston, VT, USA).
Intra-explant and inter-explant fasciculation events were traced manually and neurite tips were identified with unique markers manually (see Figure~\ref{EnFaceDataScreenshot} for more detail).
Data for fasciculation event number and length of each type, and neurite tip numbers were exported into .csv files from Neurolucida Explorer (MBF Biosciences, Williston, VT, USA) and then plotted and analyzed using Excel.

For explants in the bundle width assay, VT and DT retinal explants were collected as described.
Glass-bottomed culture dishes were prepared by incubating with poly-DL-ornithine O/N at \textcelsius{4}, rinsed the following day three times with cold DMEM/F12, and then incubated at \textcelsius{37} for at least 2 h with 20$\mu$g/ml laminin.
For chiasm co-cultures, the chiasm region was dissected from the same embryonic heads that provided retinae, and cells were dissociated as in \citenoparens{kuwajima2012optic}.
Dissociated chiasm cells were plated at a density of 3.5\textsuperscript{*}10\textsuperscript{5}cells/ml and allowed to adhere to the laminin-coated dish for at least one hour at \textcelsius{37} before explants were plated.
Explants of the same type were plated three to a dish in SFM + 0.4\% methylcellulose, spaced equally far apart from each other, so that they would not touch another explant during growth.
Any explants that did meet in the culture dish were excluded from analysis.
Plated explants were incubated O/N at \textcelsius{37} and fixed the following day for 15-20 min with warm 4\% PFA, then rinsed with 1X PBS and immunostained for NF.

Analysis of the bundle width assay was performed in Neurolucida.
A low magnification image (5X) was collected of the whole explant and explants with marked asymmetry were excluded from analysis.
The explant body radius was measured and then two rings were placed around the neurite halo, positioned 250$\mu$m and 500$\mu$m radial distance from the edge of the explant body.
High magnification tiled images (40X) were collected around each ring.
The width of each neurite/neurite bundle intersection was manually traced in Neurolucida, and data exported into .csv files from Neurolucida Explorer, to be analyzed with python.

For explants in the Y assay, VT and DT retinal explants were collected as described.
15mm glass coverslips were prepared by plasma cleaning for 5 min, then epoxy-silanized by incubating in a 2\% solution of (3-Glycidyloxypropyl)trimethoxysilane (GPS, Sigma-Aldrich) in 100\% ethanol (EtOH) at RT for 10 min.
Coverslips were then dip-rinsed in 100\% EtOH twice for 5 min each and dried using high purity nitrogen.
PDMS stamps with Y pattern (gift from F. Fiederling and F. Weth) were incubated with 40$\mu$g/ml laminin solution for at least 2 h at \textcelsius{37}, then dipped into distilled H\textsubscript{2}O (dH\textsubscript{2}O) twice and quickly dried under a stream of high purity nitrogen.
Stamps were then placed gently but firmly onto prepared glass coverslip and the location of the Y pattern was marked with a sharpie under a dissecting microscope.
The stamp was then removed (remaining on the coverslip for no more than 2 min), and warm SFM + 0.4\% methylcellulose added to the coverslip.
Explant pairs were positioned into the branches of each Y shape (two Ys per coverslip) and immediately placed into the incubator.
Each coverslip was stamped and plated individually in sequence, as we found that minimizing between stamping and plating, and plating and incubating improved culture outcome.
Fresh warm SFM + 0.4\% methylcellulose was added at 24 h in culture, and fixed with warm 4\% PFA + 0.1\% glutaraldehyde in 10\% sucrose solution in 1X PBS for 15 min, and then immunostained for laminin and NF, mounted, and imaged.