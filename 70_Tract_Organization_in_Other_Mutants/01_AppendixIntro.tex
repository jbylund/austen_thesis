Over the course of my PhD, I examined optic tract organization in other mutants in addition to the EphB1\textsuperscript{-/-} mutant discussed in Chapter 4.
My analyses of these other mutants are presented in this Appendix.
The first of these mutants was included in a publication in 2015, as part of a larger study exploring the role of a member of the leucine-rich repeat superfamily (LRR), \emph{Islr2} (mouse \emph{Linx}) in retinal axon guidance in zebrafish and mouse \cite{panza2015lrr}.
In this collaboration with Dr. Paolo Panza and Dr. Christian S\"ollner, I examined the optic tract phenotype in \emph{Linx} mutant.
The first section of this Appendix is a reproduction of selected portions of the manuscript, focused on the retinal axon guidance phenotype in zebrafish and mice.

The second section is a brief examination of the eye-specific RGC axon organization in the Plexin-A1\textsuperscript{-/-} mouse, which stemmed from ongoing work in the lab by a former postdoctoral fellow, Dr. Alexandra Rebsam.
Dr. Rebsam continues studying the Plexin-A1 in her current position as Research Associate at INSERM, Institut du Fer \`a Moulin, in Paris, France.
The next section presents preliminary findings on the eye-specific projections at the optic chiasm of TAG-1 mutants, and the final section of this Appendix include examinations of the optic chiasm of Protocadherin mutants, a collaborations with Dr. Weisheng Chen in the Maniatis Lab.
Each of these mutants can be a model, in different ways, for examination of variations in axon organization in the optic tract and fasciculation of retinal axons.
Further exploration is warranted in many of these cases, and the data included in this section hopefully serve as compelling preliminary evidence which can extended in the future.