The stereotypical trajectory of developing axons is the outcome of chemotactic guidance.
However, it also substantially depends on axon extension by local substrate adhesion and axon-axon fasciculation, and regulated passage through signaling landmarks \cite{raper2010cellular}.
One of the most clearly defined and best studied choice points encountered \emph{en route} by commissural axons is the embryonic midline.
In the vertebrate visual system, retinal ganglion cell (RGC) axons project into the optic nerve, across the ventral diencephalic midline and into the optic tracts towards their dorsal thalamic and midbrain targets.
In most binocular animals, retinal axons approaching the optic chiasm -- the crossing point at the midline -- diverge into ipsi- and contralateral projections.
While the latter is always predominant in size, ipsilaterally-projecting axons vary in number in different organisms, ranging from zero in zebrafish to ~3\% in rodents and ~45\% in primates \cite{jeffery2005variations}.

\Invitro{} studies using retinal explants showed that the cells of the optic chiasm suppress retinal axon growth, irrespective of their laterality of projection \cite{wizenmann1993differential,wang1995crossed,wang1996chemosuppression,kuwajima2012optic}.
Accordingly, growth cones proceed in a saltatory fashion across the ventral diencephalon and slow down when approaching the midline \cite{harris1987retinal,mason1997growth,hutson2002pathfinding}.
In amphibians and mice, a specific population of midline radial glial cells expresses Ephrin-B2, which repels ventro-temporal retinal
axons \cite{nakagawa2000ephrin,williams2003ephrin}.
As a consequence, growth cones expressing the Ephrin-B2 receptor EphB1 abruptly change morphology to turn and ultimately follow the ipsilateral trajectory.
Conversely, evidence to date argues that retinal axons with a contralateral trajectory cross the midline by overcoming chemosuppression at the chiasm.
In the murine anterior hypothalamus, a cluster of CD44/SSEA1-expressing early neurons is required for retinal axon entry into the chiasm \cite{sretavan1995disruption}, providing the first evidence that the chiasmatic territory is not exclusively refractory to axon growth.
Indeed, the expression of Plexin-A1 in these neurons, in conjunction with NrCAM on radial glia, reverses the inhibitory effect of midline-derived Semaphorin-6D, thereby promoting axon growth \cite{williams2006role,kuwajima2012optic}.
Similarly, a VEGF isoform expressed at the mouse optic chiasm acts as an attractant to support crossing of Neuropilin1- positive retinal fibers \cite{erskine2011vegf}.

In contrast to rodents and primates, fish have an entirely contralateral retinal projection, making it an ideal system in which to dissect mechanisms mediating RGC axon crossing at the midline.
In order to identify new molecular and cellular players in this process, we considered that both axon growth and guidance crucially depend on cell-cell and cell-extracellular matrix (ECM) interactions mediated by cell surface and secreted proteins
\cite{raper2010cellular}.
In this respect, proteins belonging to the leucine-rich repeat (LRR) superfamily – including Slits and Trks – meet important requirements to participate in precise and dynamic processes in neurodevelopment \cite{de2011role}.
These molecules display highly specific and dynamic expression patterns, particularly in the nervous system.
Cell surface LRR proteins can mediate low affinity interactions with their binding partners.
Finally, the number of cell surface LRR proteins is greatly expanded in vertebrates, correlating with the increased complexity of nervous and immune system organization.

Here, by analyzing the spatiotemporal expression patterns of cell surface LRR superfamily members in zebrafish, we identified \emph{islr2} as a LRR receptor-encoding gene expressed in RGCs.
Zebrafish larvae mutant for \emph{islr2} aberrantly display ipsilateral retinal projections, demonstrating a role for Islr2 in RGC axon midline crossing.
In \emph{Islr2} (alias \emph{Linx}) mutant mice, many retinal axons misproject at and distal to the chiasm, confirming the role of Islr2 in ensuring proper axon routing at this choice point.
%We also identify two midline factors expressed in the zebrafish optic pathway, Vasna and Vasnb, that bind to Islr2 and analyze the consequences of their loss on the retinal projection.