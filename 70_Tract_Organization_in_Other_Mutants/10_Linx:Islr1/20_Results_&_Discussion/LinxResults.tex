\emph{Islr2} mutant mice display retinal axon defects at the optic chiasm.
We examined \emph{Islr2} mutant mice (\emph{Linx\textsuperscript{tEGFP}} mice; \cite{mandai2009lig}) to further explore the role of \emph{Islr2} in the decussation decisions of retinal axons navigating the optic chiasm.
Islr2 is expressed by RGCs in the mouse retina (\citenoparens{blackshaw2004genomic} and data not shown), and RGC axons in mutant mice misroute at the optic chiasm and along the optic tract (Figure~\ref{LinxMsFig}).
A greater number of retinal axons aberrantly re-enter the contralateral optic nerve in \emph{Islr2} mutants (Figure~\ref{LinxMsFig}B,C) compared to heterozygous (Figure~\ref{LinxMsFig}A) and wild-type (not shown) littermates.
The mutant optic chiasm also appears widened rostrocaudally relative to the heterozygote chiasm region (Figure~\ref{LinxMsFig}A-C).
Although \emph{Islr2} mutant mice do not directly recapitulate the ipsilateral projection phenotype found in zebrafish, there are marked errors in axon routing in both the proximal and distal optic tract in \emph{Islr2} mutants.
Bundles of axons defasciculate from the rest of the optic tract on both ipsi- and contralateral sides of the chiasm (Figure~\ref{LinxMsFig}B,C, high power of contralateral defasciculation in Figure~\ref{LinxMsFig}B$'$,C$'$), and farther along the tract we detected several axons prematurely leaving the tract and entering the medial thalamus (Figure~\ref{LinxMsFig}A$''$-C$''$).
The common element in the axon routing errors in the mouse and the zebrafish mutants is a lack of coherence of RGC axons, whether at the chiasm or farther along their pathway in the tract.
This array of axon routing defects in \emph{Islr2} mutant mice further supports a role for \emph{Islr2} in fostering axon bundle coherence and proper tract formation, in response to axon guidance cues at the midline and along the optic tract.
\begin{figure}[hbtp]
    \begin{center}
        \includegraphics{Figures/LinxMsFig.pdf}
        \caption[Loss of Islr2 leads to axon routing errors at the optic chiasm and tract in mice.]
        {Loss of Islr2 leads to axon routing errors at the optic chiasm and tract in mice.
		E16.5 \emph{Islr2} mutant mice display expanded chiasms along the antero-posterior axis (brackets in B and C, compared to A).
		In addition, a greater number of axons enter the opposite optic nerve with retinopetal directionality (arrows in B and C).
		Although no clear increase in the number of ipsilaterally-projecting fibers relative to the contralateral projection can be observed, other defects were identified along the post-crossing route of RGC axons.
		Defasciculation effects in the optic tract were observed both at ventral (B$'$ and C$'$)and dorsal locations (B$''$ and C$''$), where many axons stray from their normal course (arrowheads).
		Arrowheads in B$'$ and C$'$ indicate axons departing rostrally from the chiasm, a common phenotype observed in situations when guidance through the chiasm is impaired. 
		I: ipsilateral.
		C: contralateral.
		ON: optic nerve.
		OC: optic chiasm.
		OT: optic tract.
		Scale bars: 500$\mu$m (A-C), 200$\mu$m (A$'$-C$'$ and A$''$-C$''$).
		}
        \label{LinxMsFig}
    \end{center}
\end{figure}