A former postdoctoral fellow in the lab, Dr. Alexandra Rebsam was concluding her time in the Mason Lab when I joined the lab.
At that time, Dr. Rebsam was studying defasciculation and targeting defects in PlexinA1, Sema6D, and NrCAM mutants.
Some of this work developed out of findings from \citetext{kuwajima2012optic}, which identified, among other things, slight defasciculation at the PlexinA1\textsuperscript{-/-} optic chiasm, and more severe defasciculation and an increased ipsilateral projection at the Sema6D\textsuperscript{-/-} optic chiasm and at the chiasm of PlexinA1\textsuperscript{-/-};NrCAM\textsuperscript{-/-} double knockout mice \cite{kuwajima2012optic}.
Dr. Rebsam subsequently found a striking targeting phenotype in Plexin-A1\textsuperscript{-/-} mutants.
In Plexin-A1\textsuperscript{-/-} mutants and to a greater extent in Plexin-A1\textsuperscript{-/-};NrCAM\textsuperscript{-/-} double mutants, ipsilateral axons form ectopic patches in the dLGN and SC.
Additionally, a subset of both ipsi and contra axons cluster in the optic tract just outside of the dLGN in both mutants.
The Sema6D\textsuperscript{-/-} mutant has similar targeting defects as the Plexin-A1\textsuperscript{-/-} mutant, but much milder.
Most intriguingly in the context of my thesis work, however, is the marked defasciculation Dr. Rebsam identified in wholemount views of the optic tract and dLGN of Plexin-A1\textsuperscript{-/-} and Plexin-A1\textsuperscript{-/-};NrCAM\textsuperscript{-/-} double knockout mice.

Thus, early in my training in the Mason Lab, and under the guidance of Dr. Rebsam, I conducted examination of the axon organization in the optic tract of Plexin-A1\textsuperscript{-/-} mutants.
After collecting preliminary data in the Plexin-A1\textsuperscript{-/-} optic tract, I did not continue pursuing an investigation into the Plexin/NrCAM/Sema mutants, in favor of focusing my efforts on the EphB1\textsuperscript{-/-} mutant instead.
However, work has continued in Dr. Rebsam's group on Plexin-A1, and there is much left to learn about fasciculation and axon organization in these mutants.
In particular, as shown by \citetext{kuwajima2012optic}, these mutants are very useful for better understanding the contextual behavior of several guidance cues when they interact in specific combinations at different stages of circuit development.
