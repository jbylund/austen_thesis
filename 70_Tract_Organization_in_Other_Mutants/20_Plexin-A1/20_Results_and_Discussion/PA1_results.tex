As identified by \citetext{kuwajima2012optic}, the Plexin-A1\textsuperscript{-/-} mutants have an increased ipsilateral projection.
This is evident in anterograde labeling of the Plexin-A1\textsuperscript{-/-} optic tract (Figure~\ref{PlexinA1}), where the ipsi RGC axon population is larger than in the wild-type tract (compare with Figure~\ref{Figures/DiIDiDWT}).
Additionally, in both frontal and horizontal sections through the Plexin-A1\textsuperscript{-/-} tract, ipsi RGC axons (green) appear to extend further medially than they do in wild-type samples, although the majority of ipsi axons retain the normal position in the lateral optic tract.
Whether ipsi RGC axons in the Plexin-A1\textsuperscript{-/-} optic tract are less well fasciculated is difficult to conclude from anterograde labeling.
Because there are more ipsi axons in these animals, resulting from misrouting of contra RGC axons into the ipsilateral tract instead, the wider spread in the tract may simply reflect the greater number of ipsi axons.
However, it could be indicative of less coherent self-fasciculation among ipsi axons, allowing them to span a greater swathe of the tract.

Indeed, it is likely that the misrouted axons aberrantly projecting into the ipsi tract do not fasciculate normally with the ``true'' ipsi axons in the Plexin-A1\textsuperscript{-/-} mutant.
Evidence from Chapter 3 indicates that ipsi axons are endowed with an intrinsic proclivity to self-fasciculate more than contra axons, and evidence from Chapter 4 suggests that even when misrouted, aberrantly crossing EphB1\textsuperscript{-/-} axons still largely retain the ability to associate with other ipsi axons.
Similarly, in the Plexin-A1\textsuperscript{-/-} mutant, axons that have lost their ability to cross the midline and instead misroute ipsilaterally likely retain other contra markers and it is unlikely that they would gain ipsi-specific markers that aid ipsi axon self-fasciculation.
Therefore, the axons misrouted into the ipsi tract would more likely associate with contra axons instead of the ``true'' ipsi axons.
Further examination of this mutant, both \invitro{} and \invivo{}, will more clearly elucidate the role of Plexin-A1 in fasciculation, ipsi/contra axon segregation, and also uncover why some axons fail to appropriately enter the dLGN.
\begin{figure}[hbtp]
    \begin{center}
        \includegraphics{Figures/PlexinA1.pdf}
        \caption[Ipsi RGC axons spread more medially in the Plexin-A1\textsuperscript{-/-} optic tract.]
        {Ipsi RGC axons spread more medially in the Plexin-A1\textsuperscript{-/-} optic tract.
		Anterograde labeling of ipsi and contra RGC axon cohorts in P0 Plexin-A1\textsuperscript{-/-} optic tracts.
		Ipsi axons are more abundant than in WT, and extend into the medial optic tract, although they are still largely positioned in the lateral tract (compare to wild-type in Figure~\ref{Figures/DiIDiDWT}). 
		Scale=100$\mu$m (horizontal sections), 200$\mu$m (frontal sections).
		}
        \label{PlexinA1}
    \end{center}
\end{figure}