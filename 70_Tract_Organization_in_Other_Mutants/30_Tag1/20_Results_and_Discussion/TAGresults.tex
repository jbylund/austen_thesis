\citetext{chatzopoulou2008structural} demonstrated interesting axon-glia interactions mediated by TAG-1 in the optic, but they did not directly investigate RGC axon divergence at the optic chiasm or fasciculation in the optic tract.
Therefore, I collected preliminary data on ipsi/contra axon divergence and fasciculation at the optic chiasm (Figure~\ref{TAG1}).
The ipsi/contra RGC axon ratio appears normal in both heterozygous and homozygous mutants for TAG-1.
There is no obvious defasciculation or straying of axons in the chiasm region of the mutants, although it is possible that the chiasm is broader in the caudal direction in mutants compared to wild-type.
Especially because the wild-type sample in this particular litter was poorly filled by DiI/DiD (a technical anomaly), these mutants may warrant further study, especially in the tract itself, which I did not thoroughly examine.
Similar to the Plexin-A1 mutants, I instead focused my attention on a more thorough assessment of the EphB1 mutant.
Given the findings presented by \citetext{chatzopoulou2008structural}, it would be particularly interesting to explore RGC axon TAG-1 mediated interactions with glia along the optic tract.
Perhaps TAG-1 participates in axon-glia interactions that are involved in guidance as well as in myelination.  
\begin{figure}[hbtp]
    \begin{center}
        \includegraphics{Figures/TAG1.pdf}
        \caption[There are no gross defects at the optic chiasm of TAG-1 mutants.]
        {There are no gross defects at the optic chiasm of TAG-1 mutants.
		Anterograde labeling using DiI and DiD in fixed P0 samples of TAG-1 mutants.
		Samples were imaged as wholemount preparations using a fluorescent stereomicroscope (Leica M165).
		The ipsi/contra RGC axon ratio is comparable across genotype.
		Fasciculation at the chiasm is relatively comparable across genotype, although the left-hand homozygous mutant sample appears to have a slightly extended chiasm region in the caudal direction.
		(The wild-type sample is less fully labeled than the mutants, but this a technical aberration.)
		Rostral is up, caudal is down.
		}
        \label{TAG1}
    \end{center}
\end{figure}