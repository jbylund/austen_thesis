Clustered protocadherins (Pcdhs), comprised of Pcdh-\alpha, -\beta, and -\gamma gene clusters, make up the largest subgroup of the cadherin superfamily (reviewed in \citenoparens{yagi2008clustered}).
Similar to DSCAMs in invertebrates, Pcdhs in vertebrates are cell-cell recognition molecules capable of conferring a high degree of specificity in neural circuit development (reviewed in \citenoparens{zipursky2010chemoaffinity}).
DSCAMs and Pcdhs give rise to such specificity via combinatorial expression of isoforms produced by extensive alternative splicing \cite{zipursky2010chemoaffinity}.
Pcdhs are involved in neuron survival, cell-cell recognition important for neurite self-avoidance, and dendritic and axonal patterning (reviewed in \citenoparens{chen2013clustered}).
Given their extensive involvement in cell-cell recognition and homophilic cell adhesion during neural development, the Pcdh family may be involved in axon organization and fasciculation during development of the retinogeniculate pathway.
