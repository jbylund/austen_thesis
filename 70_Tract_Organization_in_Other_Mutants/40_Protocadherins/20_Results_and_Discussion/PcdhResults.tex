Dr. Weisheng Chen in the Maniatis Lab proposed a collaboration exploring whether deletion of specific gene clusters or tri-cluster deletions of Pcdhs would induce aberrant fasciculation or axon sorting at the optic chiasm midline choice point.
I therefore examined several litters of Pcdh mutants produced by Dr. Chen, who was primarily studying their role in serotonergic pathways and the olfactory system.
Clustered Pcdh-$\alpha$ has been implicated in appropriate targeting of RGC terminals in the dLGN \cite{meguro2015impaired}, but Pcdh involvement at the optic chiasm midline or in fasciculation of RGC axons along the pathway had not yet been examined.

I first examined eye-specific axon divergence at the optic chiasm in Pcdh-$\alpha$ mutants, but found no gross defects at the chiasm (data not shown).
Similarly, using whole eye anterograde labeling, I examined axon divergence and fasciculation at the optic chiasm of tri-cluster Pcdh-$\alpha$/$\beta$/$\gamma$ mutants (Figure~\ref{PcdhABGMutantsWholemount}).
Again, however, I found no gross defects at the optic chiasm.
The ratio of ipsi and contra RGC axon projections at the midline are grossly normal and there is no clear defasciculation.
If anything, in fact, the chiasm region of the homozygous mutant appears smaller and more constrained than the heterozygotes (which are used here as controls, as there are no expected differences between heterozygotes and wild-type).
Thus, perhaps loss of triclustered Pcdhs leads to increased fasciculation of retinal axons, although this is a hypothesis in need of more direct testing.
Overall, clustered Pcdhs do not seem to be involved in RGC axon guidance at the midline.
Their function may be more restricted to targeting and synaptic specificity than to earlier development steps like fasciculation and midline choice.
\begin{figure}[hbtp]
    \begin{center}
        \includegraphics{Figures/PcdhABGMutantsWholemount.pdf}
        \caption[There are no gross defects at the optic chiasm of Pcdh-$\alpha$/$\beta$/$\gamma$ tri-cluster mutants.]
        {There are no gross defects at the optic chiasm of Pcdh-$\alpha$/$\beta$/$\gamma$ tri-cluster mutants.
        Anterograde labeling using DiI and DiD in fixed E18.5 samples of Pcdh-$\alpha$/$\beta$/$\gamma$ tri-cluster mutants.
        Pcdh-$\alpha$/$\beta$/$\gamma$ tri-cluster mutants are postnatal lethal, so embryos were collected before birth.
        Samples were imaged as wholemount preparations using a fluorescent stereomicroscope (Leica M165).
        There are no obvious defects in ipsi/contra ratio of RGC axons at the midline, although the optic chiasm region is perhaps more tightly fasciculated in the homozygous mutant compared to heterozygous controls.
        }
        \label{PcdhABGMutantsWholemount}
    \end{center}
\end{figure}
