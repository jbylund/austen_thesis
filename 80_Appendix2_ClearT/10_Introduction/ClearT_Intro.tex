Appreciation of neural circuitry and single-cell morphology has benefited from new labeling methods, including fluorescent tracers and genetically encoded fluorescent proteins \cite{Luo et al., 2008}.
Although these methods produce superb detail of labeled cells and pathways, tissue opacity limits the depth of imaging, necessitating imaging sectioned material in order to attain high microscopic resolution.
However, because images must be reconstructed in three dimensions (3D) post-acquisition, imaging and reconstructing sections is neither as efficient nor as accurate as imaging thicker tissue samples.
New reagents that clear or render tissue transparent include Sca\emph{l}e, benzyl-alcohol and benzyl-benzoate (BABB), and a combination of tetrahydrofuran and BABB, all of which preserve genetically expressed fluorescent signal, allowing deep imaging of neural circuitry in 3D \cite{Dodt et al., 2007; Hama et al., 2011; Ertürk et al., 2012}.
However, these reagents change tissue volume and require several days to weeks to fully clear the tissue \cite{Hama et al., 2011;
Ertürk et al., 2012}.
More importantly, owing to their reliance on detergents or organic solvents, Sca\emph{l}e and BABB disrupt the fluorescent signal of immunohistochemistry, of conventional lipophilic carbocyanine dyes [such as 1,1-dioctadecyl-3,3,3,3-tetramethylindocarbocyanine perchlorate (DiI)] and of fluorescent tracers such as cholera toxin subunit B (CTB).
Here, we describe a rapid clearing method that maintains tissue volume and preserves fluorescent signal from tracers, immunohistochemistry and genetically expressed fluorescent proteins.