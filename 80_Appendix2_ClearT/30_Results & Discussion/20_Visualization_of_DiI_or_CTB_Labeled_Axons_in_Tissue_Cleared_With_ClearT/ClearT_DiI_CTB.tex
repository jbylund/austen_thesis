The projections, connections, and growth cone (GC) morphology of developing axons can be visualized by anterograde labeling with lipophilic dyes \cite{bielle2011emergent,little2009specificity}.
Here, we use the mouse visual system, a classic model for studying neural circuitry development, to demonstrate the advantages of clearing the mouse brain with \emph{Clear\textsuperscript{T}} in preserving lipophilic fluorescent dye labeling.
We anterogradely labeled retinal ganglion cell (RGC) axons in embryonic day (E) 14.5 embryos with DiI and treated embryos with \emph{Clear\textsuperscript{T}}, Sca\emph{l}eA2 or BABB for 1 day.
DiI labeling of retinal axons in the optic nerve and chiasm was preserved after treatment with \emph{Clear\textsuperscript{T}}, but treatment with either Sca\emph{l}eA2 or BABB degraded the fluorescent signal (supplementary material Fig. S2A).
%Fix fig

We then examined DiI-labeled RGC axons in cleared tissue at the optic chiasm at E15.5 (Fig. 2A).
%Fix fig
The DiI-labeled retinal projection was not visible prior to clearing, but could be seen through both dorsal and ventral aspects of the cleared head, with jaw and tongue removed but skin and skull intact (Fig. 2A).
%Fix fig
We examined the resolution of fine morphological detail of DiI-labeled axons and GCs in the proximal ipsilateral optic tract at E14.5 before and after clearing (Fig. 2B).
%Fix fig
The number and resolution of DiI labeled axons and GC processes (e.g. filopodia and lamellopodia) were markedly increased after clearing with \emph{Clear\textsuperscript{T}} (Fig. 2B).
%Fix fig
Furthermore, E18.5 DiI-labeled RGC axons in the thalamus and superior colliculus were not visible before clearing when imaged from the midline of parasagittal hemisections, but the full tract was distinctly visible after clearing with \emph{Clear\textsuperscript{T}}, even through a depth of ~1mm (Fig. 2C).
%Fix fig

CTB is widely used for the analysis of postnatal RGC axon targeting in the dLGN (e.g., \citenoparens{jaubert2005structural,rebsam2009switching}).
To test the compatibility of CTB with \emph{Clear\textsuperscript{T}}, we anterogradely labeled each eye of P5 pups with CTB conjugated to either Alexa Fluor 488 or 594.
CTB labeling was preserved after \emph{Clear\textsuperscript{T}} and BABB treatments, but BABB reduced tissue size by half (before clearing=1\pm0 versus cleared=0.50\pm0.02, P<0.05, n=4 sections), while labeling was diffuse following Sca\emph{l}eA2 treatment (supplementary material Fig. S2B).
%Fix fig
CTB labeling was visible through the entire depth of a 700$\mu$m section of P5 brain treated with \emph{Clear\textsuperscript{T}}, whereas fluorescence could not be seen beyond 250$\mu$m before clearing (Fig. 2D).
%Fix fig
Moreover, it is possible to successively clear, unclear (in PBS) and re-clear DiI- or CTB-labeled samples without compromising tissue or label integrity (supplementary material Fig. S3A,B).
%Fix fig