Our original \emph{Clear\textsuperscript{T}} protocol diminished green fluorescent protein (GFP) intensity in E14.5 \emph{actin}-GFP embryos \cite{} (Ikawa et al., 1995).
Because polyethylene glycol (PEG) stabilizes protein conformation \cite{rawat2010molecular}, we investigated whether PEG would stabilize GFP expression in formamide.
Whereas 50\% formamide failed to clear brains, a 20\% PEG/50\% formamide mixture successfully cleared brain tissue and preserved fluorescence.
This modified method, named \emph{Clear\textsuperscript{T2}}, also requires immersion in a graded series of formamide/PEG solutions (25\% formamide/10\% PEG, then 50\%formamide/20\% PEG) (Table~\ref{ClearT.tgn}, Fig. 3A).
%Fix fig
Although tissue transparency with \emph{Clear\textsuperscript{T2}} was less complete than with \emph{Clear\textsuperscript{T}}, application of \emph{Clear\textsuperscript{T2}} induced robust transparency of thick P0 brain sections without volume changes (before clearing=1.0 versus \emph{Clear\textsuperscript{T2}}=0.98\pm0.02, n=6, not significant) (Fig. 3B).
%Fix fig
Sections treated with \emph{Clear\textsuperscript{T2}} for 1 or 2 days were slightly larger than pre-cleared sections, but these changes were significantly less than with Sca\emph{l}eA2 [1 day, ClearT2=1.30\pm0.02 versus Sca\emph{l}eA2=1.81\pm0.05, P<0.01; 2 days, \emph{Clear\textsuperscript{T2}}=1.30\pm0.01 versus Sca\epmh{l}eA2=1.83\pm0.06, P<0.01, n=6 (\emph{Clear\textsuperscript{T2}}), n=4 sections (Sca\emph{l}eA2)] (supplementary material Fig. S1).
%Fix fig
\emph{Clear\textsuperscript{T2}} also maintained DiI and CTB labeling in axons as successfully as \emph{Clear\textsuperscript{T}} (supplementary material Fig. S2A,B).
%Fix fig

We next examined whether neurons genetically labeled with fluorescent proteins, such as in \emph{Thy1}-GFP (M-line) mice \cite{}(Feng et al., 2000), could be visualized with \emph{Clear\textsuperscript{T2}} (Fig. 3C).
%Fix fig
After clearing thick hippocampal sections with \emph{Clear\textsuperscript{T}}, \emph{Thy1}-GFP\textsuperscript{+} neurons were visible deeper within the granule cell layer (Fig. 3C, top) and details of GFP\textsuperscript{+} pyramidal neuron dendrites in the CA1 region were more distinct than without clearing (Fig. 3C, bottom).
%Fix fig
To determine whether \emph{Clear\textsuperscript{T2}} could be applied to adult or non-neuronal tissue, we used \emph{Tcf/Lef:H2B}-GFP mice, in which reporter expression is detected in neuronal and non-neuronal tissues from early embryonic to adult stages \cite{ferrer2010sensitive}.
\emph{H2B}-GFP nuclear labeling in neurons of the cerebral cortex, cells within the granule cell and molecular layers of the hippocampus, progenitor cells of the small intestine and satellite cells of skeletal muscle were more apparent after clearing with \emph{Clear\textsuperscript{T2}} (supplementary material Fig. S4).
%Fix fig

Immunohistochemistry is used to visualize protein expression, but labeling is usually visible only superficially in thick tissue sections.
To examine whether immunohistochemistry labeling is compatible with tissue clearing, we immunostained E14.5 cryosections through the optic chiasm with an antibody to the radial glia marker RC2 and treated sections with \emph{Clear\textsuperscript{T}}, \emph{Clear\textsuperscript{T2}}, Sca\emph{l}eA2 or BABB (supplementary material Fig. S5A,B).
%Fix fig
\emph{Clear\textsuperscript{T}} and Sca\emph{l}eA2 disrupted RC2 immunolabeling, and BABB maintained fluorescent signal but produced labeling artifacts in bone and cell nuclei that should not express RC2.
As \emph{Clear\textsuperscript{T2}} successfully preserved immunolabeling (supplementary material Fig. S5A-E), we applied it to 200$\mu$m RC2-immunolabeled vibratome sections of the optic chiasm at E14.5 (Fig. 3D).
%Fix fig
RC2\textsuperscript{+} glial processes were visible as deep as ~120$\mu$m in cleared tissue, twice as deep as in pre-cleared tissue (Fig. 3D).
%Fix fig
Finally, \emph{Clear\textsuperscript{T2}} treatment of whole mouse embryos immunostained with an antibody to neurofilament (NF) provided a complete view of axon tracts and arbors in the CNS and PNS in distal appendages (Fig. 3E).
%Fix fig

Finally, we examined whether \emph{Clear\textsuperscript{T2}} is compatible with multiple fluorescent labels.
After applying \emph{Clear\textsuperscript{T2}} to a thick brain section with CTB-labeled dLGN, a biocytin-filled relay neuron was visualized more deeply and at higher resolution than before clearing, with both labels successfully maintained (Fig. 3F).
%Fix fig