We have developed two clearing methods, \emph{Clear\textsuperscript{T}} and \emph{Clear\textsuperscript{T2}}, which aid analysis of fluorescent labeling in embryonic and mature neuronal and non-neuronal tissue.
\emph{Clear\textsuperscript{T2}} clears specimens while effectively maintaining the fluorescent signal of genetically encoded proteins, immunohistochemistry labeling, and dye tracers such as DiI and CTB.
Whereas \emph{Clear\textsuperscript{T}} is incompatible with immunohistochemistry and genetically encoded fluorescence proteins (supplementary material Table S1), transparency of whole brains treated with \emph{Clear\textsuperscript{T}} is better than with \emph{Clear\textsuperscript{T2}} (Fig. 3A).
%Fix fig
Therefore, tissue samples labeled with DiI or CTB alone are best cleared by \emph{Clear\textsuperscript{T}}.

\emph{Clear\textsuperscript{T}} and \emph{Clear\textsuperscript{T2}} provide several advantages over other available clearing methods.
Clearing time for thick sections, whole brains or embryos is significantly faster than with Sca\emph{l}eA2 or BABB.
In addition, \emph{Clear\textsuperscript{T}} and \emph{Clear\textsuperscript{T2}} produce minimal tissue volume changes, significantly less than Sca\emph{l}eA2 or BABB.
Most importantly, our methods maintain DiI- and CTB-labeling in axons, unlike Sca\emph{l}eA2 and BABB (supplementary material Table S1).
%Fix fig
\emph{Clear\textsuperscript{T}} and \emph{Clear\textsuperscript{T2}} successfully clear postnatal and adult brain and other tissues.
\emph{Clear\textsuperscript{T2}} provides a final important advantage over Sca\emph{l}eA2 and BABB, in that it can clear immunolabeled tissue.
Thus, \emph{Clear\textsuperscript{T}} and \emph{Clear\textsuperscript{T2}} provide improved clearing of embryonic and adult neuronal and non-neuronal tissue for viewing fluorescent labeling of cells and fiber tracts by high-resolution optical imaging.