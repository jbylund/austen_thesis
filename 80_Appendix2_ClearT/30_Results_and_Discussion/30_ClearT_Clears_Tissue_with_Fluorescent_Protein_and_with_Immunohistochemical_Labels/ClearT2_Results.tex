Our original \emph{Clear\textsuperscript{T}} protocol diminished green fluorescent protein (GFP) intensity in E14.5 \emph{actin}-GFP embryos \cite{ikawa1995rapid}.
Because polyethylene glycol (PEG) stabilizes protein conformation \cite{rawat2010molecular}, we investigated whether PEG would stabilize GFP expression in formamide.
Whereas 50\% formamide failed to clear brains, a 20\% PEG/50\% formamide mixture successfully cleared brain tissue and preserved fluorescence.
This modified method, named \emph{Clear\textsuperscript{T2}}, also requires immersion in a graded series of formamide/PEG solutions (25\% formamide/10\% PEG, then 50\%formamide/20\% PEG) (Table~\ref{ClearT.tgn}, %Figure~\ref{ClearT\_Fig3}A).
Although tissue transparency with \emph{Clear\textsuperscript{T2}} was less complete than with \emph{Clear\textsuperscript{T}}, application of \emph{Clear\textsuperscript{T2}} induced robust transparency of thick P0 brain sections without volume changes (before clearing=1.0 versus \emph{Clear\textsuperscript{T2}}=0.98$\pm$0.02, n=6, not significant) %(Figure~\ref{ClearT\_Fig3}B).
Sections treated with \emph{Clear\textsuperscript{T2}} for 1 or 2 days were slightly larger than pre-cleared sections, but these changes were significantly less than with Sca\emph{l}eA2 [1 day, ClearT2=1.30$\pm$0.02 versus Sca\emph{l}eA2=1.81$\pm$0.05, P<0.01; 2 days, \emph{Clear\textsuperscript{T2}}=1.30$\pm$0.01 versus Sca\emph{l}eA2=1.83$\pm$0.06, P<0.01, n=6 (\emph{Clear\textsuperscript{T2}}), n=4 sections (Sca\emph{l}eA2)] %(Figure~\ref{ClearT\_SFig1}).
\emph{Clear\textsuperscript{T2}} also maintained DiI and CTB labeling in axons as successfully as \emph{Clear\textsuperscript{T}} %(Figure~\ref{ClearT\_SFig2}A,B).
\begin{figure}[hbtp]
    \begin{center}
        \includegraphics{Figures/ClearT_Fig3.pdf}
        \caption[\emph{Clear\textsuperscript{T2}} clears tissue with fluorescent proteins or immunohistochemistry.]
        {\emph{Clear\textsuperscript{T2}} clears tissue with fluorescent proteins or immunohistochemistry.
        A) \emph{Clear\textsuperscript{T}} cleared E14.5 \emph{actin}-GFP embryos, but reduced GFP fluorescence.
        Formamide (50\%) maintained fluorescence, but failed to adequately clear embryos.
        \emph{Clear\textsuperscript{T2}} cleared embryos and maintained fluorescence.
        B) P0 sections (800$\mu$m) were transparent after \emph{Clear\textsuperscript{T2}}, with no volume change.
        C) P11 \emph{Thy1}-GFP (M-line) hippocampus section (800$\mu$m), before and after clearing with \emph{Clear\textsuperscript{T2}}; 38 images, 20$\mu$m steps (top and middle).
        GFP\textsuperscript{+} pyramidal neurons (arrows) and dendrites (arrowheads) in CA1 region are markedly more visible after clearing; 52 images, 2.5$\mu$m steps (bottom).
        GCL, granule cell layer; ML, molecular layer.
        D) Sections of E14.5 optic chiasm (200$\mu$m), immunolabeled with the radial glial marker RC2, cleared with \emph{Clear\textsuperscript{T2}}; 51 images, 3$\mu$m steps; three optical slices shown.
        RC2\textsuperscript{+} staining was observed deeper in cleared compared with pre-cleared tissue.
        Blue indicates Hoechst staining.
        E) E11.5 whole embryos, immunolabeled with neurofilament antibody (NF) and treated with \emph{Clear\textsuperscript{T2}}.
        NF\textsuperscript{+} axons were much more visible in cleared embryos (top); magnification of trigeminal axons reaching epithelial targets (bottom).
        F) Section (300$\mu$m) of postnatal mouse brain, dLGN anterogradely labeled with CTB conjugated to AlexaFluor 594.
        A single neuron was filled with biocytin and immunostained with streptavidin-AlexaFluor 647.
        Clearing with \emph{Clear\textsuperscript{T2}} enhanced resolution and visibility of the dendritic arbor of the neuron.
        Merged stack, 55 images, 2$\mu$m steps.
        CTB label is in red; biocytin-filled neuron is pseudo-colored green. Scale bars: 1mm in A,B,E; 40$\mu$m in C; 20$\mu$m in D,F.
        }
        \label{ClearT\_Fig3}
    \end{center}
\end{figure}

We next examined whether neurons genetically labeled with fluorescent proteins, such as in \emph{Thy1}-GFP (M-line) mice \cite{feng2000imaging}, could be visualized with \emph{Clear\textsuperscript{T2}} %(Figure~\ref{ClearT\_Fig3}C).
After clearing thick hippocampal sections with \emph{Clear\textsuperscript{T}}, \emph{Thy1}-GFP\textsuperscript{+} neurons were visible deeper within the granule cell layer (Figure~\ref{ClearT\_Fig3}C, top) and details of GFP\textsuperscript{+} pyramidal neuron dendrites in the CA1 region were more distinct than without clearing (Figure~\ref{ClearT\_Fig3}C, bottom).
To determine whether \emph{Clear\textsuperscript{T2}} could be applied to adult or non-neuronal tissue, we used \emph{Tcf/Lef:H2B}-GFP mice, in which reporter expression is detected in neuronal and non-neuronal tissues from early embryonic to adult stages \cite{ferrer2010sensitive}.
\emph{H2B}-GFP nuclear labeling in neurons of the cerebral cortex, cells within the granule cell and molecular layers of the hippocampus, progenitor cells of the small intestine and satellite cells of skeletal muscle were more apparent after clearing with \emph{Clear\textsuperscript{T2}} %(Figure~\ref{ClearT\_Sfig4}).
\begin{figure}[hbtp]
    \begin{center}
        \includegraphics{Figures/ClearT_Sfig4.pdf}
        \caption[\emph{Clear\textsuperscript{T2}} clears adult brain and other tissues with GFP labeling.]
        {\emph{Clear\textsuperscript{T2}} clears adult brain and other tissues with GFP labeling.
        A,B) Cerebral cortex (A) and hippocampus (B) of adult \emph{Tcf/Lef:H2B}-GFP mice were cleared with \emph{Clear\textsuperscript{T2}}.
        In an 800$\mu$m section, 41 images of GFP\textsuperscript{+} nuclei of cells in the cerebral cortex and within the granule cell and molecular layers of the hippocampus were acquired in 20$\mu$m steps at 5X magnification and merged.
        C,D) Small intestine (C) and limb muscle (D) dissected from adult \emph{Tcf/Lef:H2B}-GFP mice were cleared with \emph{Clear\textsuperscript{T2}}.
        Forty-one images of GFP\textsuperscript{+} nuclei in progenitor cells of small intestine and satellite cells of skeletal muscle were acquired in 20$\mu$m steps at 5X magnification.
        GFP\textsuperscript{+} nuclei (arrows) in all of these tissues are significantly more visible after clearing with \emph{Clear\textsuperscript{T2}}.
        Scale bars: 40$\mu$m.
        }
        \label{ClearT\_Sfig4}
    \end{center}
\end{figure}

Immunohistochemistry is used to visualize protein expression, but labeling is usually visible only superficially in thick tissue sections.
To examine whether immunohistochemistry labeling is compatible with tissue clearing, we immunostained E14.5 cryosections through the optic chiasm with an antibody to the radial glia marker RC2 and treated sections with \emph{Clear\textsuperscript{T}}, \emph{Clear\textsuperscript{T2}}, Sca\emph{l}eA2 or BABB %(Figure~\ref{ClearT\_SFig5}A,B).
\emph{Clear\textsuperscript{T}} and Sca\emph{l}eA2 disrupted RC2 immunolabeling, and BABB maintained fluorescent signal but produced labeling artifacts in bone and cell nuclei that should not express RC2.
As \emph{Clear\textsuperscript{T2}} successfully preserved immunolabeling (Figure~\ref{ClearT\_SFig5}A-E), we applied it to 200$\mu$m RC2-immunolabeled vibratome sections of the optic chiasm at E14.5 %(Figure~\ref{ClearT\_Fig3}D).
RC2\textsuperscript{+} glial processes were visible as deep as ~120$\mu$m in cleared tissue, twice as deep as in pre-cleared tissue %(Figure~\ref{ClearT\_Fig3}D).
Finally, \emph{Clear\textsuperscript{T2}} treatment of whole mouse embryos immunostained with an antibody to neurofilament (NF) provided a complete view of axon tracts and arbors in the CNS and PNS in distal appendages %(Figure~\ref{ClearT\_Fig3}E).
\begin{figure}[hbtp]
    \begin{center}
        \includegraphics{Figures/ClearT_SFig5.pdf}
        \caption[Immunolabeling is maintained after prolonged immersion in \emph{Clear\textsuperscript{T2}}.]
        {Immunolabeling is maintained after prolonged immersion in \emph{Clear\textsuperscript{T2}}.
        A) Immunolabeling of RC2 in radial glial cells of the E14.5 optic chiasm was diminished in \emph{Clear\textsuperscript{T}}, but maintained in \emph{Clear\textsuperscript{T2}}, even after immersion for 5 days.
        B) RC2 immunolabeling was diminished by Sca\emph{l}eA2 and showed artifactual globular staining.
        BABB maintained RC2 immunolabeling, but led to staining artifacts in nuclei and bone tissue.
        C) Immunolabeling with both Cy3 and Cy5 are preserved after clearing with \emph{Clear\textsuperscript{T2}}.
        Immunolabeling of the optic chiasm in a 20$\mu$m cryosection with primary antibodies, RC2 for radial glial cells
and neurofilament (NF) for RGC axons, applied together (secondary antibodies conjugated to Cy3 and Cy5, respectively) was
preserved after clearing with \emph{Clear\textsuperscript{T2}}.
        D) Repeated immunostaining of sample after immunostaining and clearing once with \emph{Clear\textsuperscript{T2}}.
        Cryosections (20$\mu$m) were immunostained with primary antibody RC2, secondary antibody Cy3, and cleared with \emph{Clear\textsuperscript{T2}}.
        Sections were then stored in PBS for 30 minutes, immunostained with an antibody to neurofilament (NF) to reveal RGC axons and secondary antibody conjugated to Cy5, and cleared with \emph{Clear\textsuperscript{T2}}.
        Post-cleared sections were reusable for another round of immunostaining and clearing while maintaining RC2-immunolabeling.
        E) Immunostaining of sections cut from whole heads previously cleared with \emph{Clear\textsuperscript{T}} or \emph{Clear\textsuperscript{T2}}.
        E14.5 embryos were cleared with \emph{Clear\textsuperscript{T}} or \emph{Clear\textsuperscript{T2}}, placed in PBS for 30 minutes and in 30\% sucrose overnight.
        Cryosections (20$\mu$m) were immunostained with primary antibodies, RC2 for radial glial cells and neurofilament (NF) for RGC axons, applied together.
        Secondary antibodies were conjugated to Cy3 and Cy5.
        Scale bars: 20$\mu$m.
        }
        \label{ClearT\_SFig5}
    \end{center}
\end{figure}

Finally, we examined whether \emph{Clear\textsuperscript{T2}} is compatible with multiple fluorescent labels.
After applying \emph{Clear\textsuperscript{T2}} to a thick brain section with CTB-labeled dLGN, a biocytin-filled relay neuron was visualized more deeply and at higher resolution than before clearing, with both labels successfully maintained % (Figure~\ref{ClearT\_Fig3}F).
