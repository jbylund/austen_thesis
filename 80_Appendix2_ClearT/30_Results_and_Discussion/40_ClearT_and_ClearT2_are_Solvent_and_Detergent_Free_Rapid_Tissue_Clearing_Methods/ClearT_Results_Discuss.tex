We have developed two clearing methods, \emph{Clear\textsuperscript{T}} and \emph{Clear\textsuperscript{T2}}, which aid analysis of fluorescent labeling in embryonic and mature neuronal and non-neuronal tissue.
\emph{Clear\textsuperscript{T2}} clears specimens while effectively maintaining the fluorescent signal of genetically encoded proteins, immunohistochemistry labeling, and dye tracers such as DiI and CTB.
Whereas \emph{Clear\textsuperscript{T}} is incompatible with immunohistochemistry and genetically encoded fluorescence proteins (supplementary material Table S1), transparency of whole brains treated with \emph{Clear\textsuperscript{T}} is better than with \emph{Clear\textsuperscript{T2}} (Figure~\ref{ClearT_Fig3}A).
Therefore, tissue samples labeled with DiI or CTB alone are best cleared by \emph{Clear\textsuperscript{T}}.

\emph{Clear\textsuperscript{T}} and \emph{Clear\textsuperscript{T2}} provide several advantages over other available clearing methods.
Clearing time for thick sections, whole brains or embryos is significantly faster than with Sca\emph{l}eA2 or BABB.
In addition, \emph{Clear\textsuperscript{T}} and \emph{Clear\textsuperscript{T2}} produce minimal tissue volume changes, significantly less than Sca\emph{l}eA2 or BABB.
Most importantly, our methods maintain DiI- and CTB-labeling in axons, unlike Sca\emph{l}eA2 and BABB (Table~\ref{ClearT_Suppl.tgn}).
\emph{Clear\textsuperscript{T}} and \emph{Clear\textsuperscript{T2}} successfully clear postnatal and adult brain and other tissues.
\emph{Clear\textsuperscript{T2}} provides a final important advantage over Sca\emph{l}eA2 and BABB, in that it can clear immunolabeled tissue.
Thus, \emph{Clear\textsuperscript{T}} and \emph{Clear\textsuperscript{T2}} provide improved clearing of embryonic and adult neuronal and non-neuronal tissue for viewing fluorescent labeling of cells and fiber tracts by high-resolution optical imaging.

\iffalse
\begin{center}
\begin{table}[hbtp]
\caption[Summary of clearing methods.]
{Summary of clearing methods.}
\label{ClearTSuppl.tgn}
\begin{tabular}{lc|c|c|c|}
& \emph\{Clear\textsuperscript\{T\}\} & \emph\{Clear\textsuperscript\{T2\}\} & Sca\emph\{l\}eA2 & BABB \\ \cline{2-5}
\multicolumn{1}{l|}{Composition of solution} & 95\% formamide & 50\% formamide + 20\% PEG & \begin{tabular}[c]{@{}c@{}}Urea \cite\{hama2011scale\}\\ Glycerol\\ Triton-X\end{tabular} & \begin{tabular}[c]{@{}c@{}}Benzyl-alcohol \cite\{dodt2007ultramicroscopy\}\\ Benzyl-benzoate\end{tabular} \\ \hline
\multicolumn{1}{l|}{Transparency} & +++ ($\sim$1 day) & ++ ($\sim$1 day) & \begin{tabular}[c]{@{}c@{}}+/- ($\sim$1 day)\\ +++ (14 days)\end{tabular} & \begin{tabular}[c]{@{}c@{}}+ ($\sim$1 day)\\ +++ (a few days)\end{tabular} \\ \hline
\multicolumn{1}{l|}{Volume change} & Slightly larger (1.2-fold) & Slightly larger (1.3-fold) & Larger (1.8-fold) & Smaller (0.5-fold) \\ \hline
\multicolumn{1}{l|}{Duration of immersion} & $\sim$1 day & $\sim$1 day & $\sim$ a few weeks \cite\{hama2011scale\} & $\sim$ a few days \cite\{dodt2007ultramicroscopy\} \\ \hline
\multicolumn{1}{l|}{DiI labeling} & \checkmark & \checkmark & \times & \times \\ \hline
\multicolumn{1}{l|}{CTB labeling} & \checkmark & \checkmark & \times & \checkmark \\ \hline
\multicolumn{1}{l|}{GFP fluorescence} & \times & \checkmark & \checkmark\cite\{hama2011scale\} & \checkmark\cite\{dodt2007ultramicroscopy\} \\ \hline
\multicolumn{1}{l|}{Immunostaining} & \times & \checkmark & \times & \bigtriangleup * \\ \hline
\multicolumn{1}{l|}{Chamber} & Plastic & Plastic & Plastic & Glass \\ \hline
\end{tabular}
\end{table}
\end{center}
\fi

\iffalse
Summary of clearing methods.
\checkmark=suitable, \bigtriangleup=partial, \times=does not work.
*Immunostaining maintained but with nuclear staining artifacts.
\emph{Clear\textsuperscript{T}} and \emph{Clear\textsuperscript{T2}} were composed of 95\% formamide or 50\% formamide plus 20\% PEG, respectively, and clear tissue more rapidly than Sca\emph{l}eA2 and BABB while also producing less volume change.
Transparency in \emph{Clear\textsuperscript{T}} and \emph{Clear\textsuperscript{T2}} was better than in Sca\emph{l}eA2 or BABB.
Transparency of whole embryos with \emph{Clear\textsuperscript{T2}} was not as extensive as with \emph{Clear\textsuperscript{T}}.
\emph{Clear\textsuperscript{T}} and \emph{Clear\textsuperscript{T2}} are compatible with DiI and CTB labeling.
\emph{Clear\textsuperscript{T2}} is also compatible with visualization of fluorescent proteins and immunolabeled tissues.
Tissue clearing with \emph{Clear\textsuperscript{T}} or \emph{Clear\textsuperscript{T2}} can be carried out in plastic chambers.
\fi
