In 2012, I assisted Dr. Takaaki Kuwajima, a former postdoctoral fellow in the Mason Lab, with his development of a new tissue clearing method.
At the time, the Sca{\it l}e method had recently been published \cite{hama2011scale} and we found ourselves in the still relatively early wave of the tissue clearing trend.
Our motivation, in part, was the fact that existing clearing methods at the time were incompatible with lipophilic dye tracers, which are standard fare not only in our lab but across the neurodevelopmental field.
As such, Dr. Kuwajima set out testing various chemical combinations based on his observation that the hybridization step in the \emph{in situ} hybridization protocol caused thin cryosections to become transparent.
Formamide, one of the ingredients in the hybridization buffer, became the centerpiece of {\it Clear\textsuperscript{T}} and {\it Clear\textsuperscript{T2}}.

One of the other hallmarks of clearing method publications has been the simultaneous development of new long-working-distance objectives and other advanced microscopy methods (e.g. \citenoparens{hama2011scale,tomer2014advanced}).
While such advances are exciting and many labs are putting such microscopy techniques to use, there are a large number of labs with more limited resources, including access to such cutting edge microscopy systems.
With this in mind, one of our goals was to demonstrate the efficacy of our clearing methods using more standard imaging tools.
As such, all of the imaging conducted for this report was with epifluorescence microscopes with standard objective lenses.
I led the imaging efforts for the project, including guiding our approach to tailoring the clearing method and imaging approaches, and performing the majority of the imaging.
The rest of this chapter is a reproduction of our 2013 publication \cite{kuwajima2013cleart} with minor changes (some updated references and additional notes and details.)
