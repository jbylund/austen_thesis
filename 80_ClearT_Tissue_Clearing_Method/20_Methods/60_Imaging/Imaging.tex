Whole brains or sections with DiI, CTB or immunolabeling, or sections of GFP-labeled mice were imaged on a Zeiss AxioImager M2 microscope with Apotome, AxioCam MRm camera, Neurolucida software (v10.40, MBF Biosciences, Williston, VT, USA); with a 5X objective lens (FLUAR, NA=0.25, working distance=12.5mm), a 20X objective lens (PLAN-APOCHROMAT, NA=0.8, working distance=550$\mu$m), or a 40X objective lens (PLAN-NEOFLUAR, NA=0.75, working distance=710$\mu$m) (~\ref{ClearTFig2}B,C,D (bottom); ~\ref{ClearTFig3}C,D,F; ~\ref{ClearTSFig3}B; ~\ref{ClearTSfig4}).
Using the principle of structured illumination, the Apotome provides confocal-like resolution with epifluorescence imaging.
The Apotome improves the signal to noise ratio by acquiring three images of an optical section and subtracting background fluorescence signal.
Imaging of whole heads and brains was performed using a Zeiss dissecting microscope StemiSV11, Axiovision software, AxioCam camera (Figures~\ref{ClearTFig1}; ~\ref{ClearTFig2}A; ~\ref{ClearTFig3}A,B; ~\ref{ClearTSFig1}; ~\ref{ClearTSFig2}A).
Imaging of whole embryos with immunolabeling was performed using Nikon SMZ 1500 zoom stereomicroscope and DS-Qi1Mc camera (Figure~\ref{ClearTFig3}E).
A Zeiss Axioplan 2 microscope with AxioCam camera and Axiovision software was used to image thin brain sections using a 10X objective lens (PLAN-NEOFLUAR, NA=0.3) or a 20X objective lens (PLAN-NEOFLUAR, NA=0.5) (Figures~\ref{ClearTSFig2}B, ~\ref{ClearTSFig3}A, ~\ref{ClearTSFig5}).
Thick samples were imaged using a home-made slide to keep tissue submerged in formamide solutions and covered with a regular glass coverslip: a square rim of plastic or silicone elastomer was super-glued to a regular glass slide creating a ``wall'' surrounding the sample.
