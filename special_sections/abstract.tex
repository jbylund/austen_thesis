Appropriately organized synaptic connections are essential for proper neural circuit function.
Prior to forming and refining those synaptic connections, axons of projection neurons must first navigate over long distances to their appropriate targets.
The field of axon guidance has generated a great deal of knowledge about how axons successfully execute this navigation, including interactions with intermediate choicepoints and formation of initial connections with their synaptic targets.

One area that has been less well studied is whether axons are organized within the lengths of their tracts.
Axons could be highly organized within their tracts, or arranged haphazardly, to be sorted out within their destination target zone.
Findings from several systems indicate that axons are organized in their tracts and, furthermore, that this pre-target organization is important for accurate targeting.
I will discuss these findings and summarize the modes of pre-target axon organization and the mechanisms underlying that organization in Chapter 1.
This thesis presents my work in the mouse retinogeniculate pathway probing three aspects of pre-target axon organization: the organization of cohorts of retinal ganglion cell (RGC) axons in the optic nerve and tract; the role of axon self-association in tract organization; and the relationship between tract order and targeting. 

RGC axons project either ipsi- or contralaterally at the optic chiasm, and in the first thalamic target, the dorsal lateral geniculate nucleus (dLGN), RGC axon terminals are organized retinotopically and in distinct ipsi- and contralateral zones.
Chapter 2 presents my findings on the organization of ipsilateral and contralateral RGC axons in the optic nerve and tract.
I found that ipsilateral RGC axons cluster together in the optic nerve, are less tightly bundled in the optic chiasm, and once in the optic tract, ipsilateral axons again bundle together and segregate away from contralateral axons.
Direct comparisons between topographic axon order and ipsi/contra axon order in the optic tract revealed that the two organizational modes are largely in agreement with each other, though ipsi- and contralateral axons from the same topographic region maintain distinct ipsi/contra segregation in the tract.

Having established two modes of axon organization in the retinogeniculate pathway, Chapter 3 explores one mechanism at play in creating this tract order: differential fasciculation behavior between RGC axon cohorts.
I tested the hypothesis that ipsilateral RGC axons have a greater preference to self-fasciculate than contralateral axons by using an in vitro retinal explant culture system.
Ipsilateral neurites displayed greater self-association/fasciculation than contralateral neurites, indicating an axon-intrinsic mechanism of ipsilateral-specific self-association. 

Chapter 4 examines tract organization and fasciculation in the EphB1 mutant retinogeniculate pathway.
EphB1 is expressed exclusively by ipsilateral RGCs, and loss of EphB1 leads to a reduced ipsilateral projection and increased contralateral projection.
However, aberrantly crossing axons project to the ipsilateral zone in the dLGN.
Given this aberrant decussation phenotype with a grossly normal targeting phenotype, I used this mutant to explore the relationship between midline choice, tract organization, and targeting.
First, I found that remaining ipsilateral axons in the EphB1-/- optic tract largely retain their position in the lateral optic tract, but appeared splayed apart, suggestive of aberrant fasciculation.
In vitro, EphB1-/- ipsilateral neurites still bundled more than EphB1-/- contralateral neurites, although the magnitude of this difference was less striking than in wt retinal explants.
Thus, EphB1 may one factor of many involved in preferential ipsilateral fasciculation.
In vivo, I found that the aberrantly crossing axons in the EphB1 mutant grossly maintained their position in the ipsilateral zone of the optic tract, pointing to a maintenance of ipsilateral integrity in the optic tract.
This is line with a model in which bundling partners in the tract may help guide axons to the correct zone in the target.

These data detail two organizational modes of RGC axons in the developing optic nerve and tract, and suggest that axon-intrinsic factors mediate ipsilateral-specific self-association.
These axon-intrinsic factors likely act in parallel with extrinsic cellular and molecular cues within the developing retinogeniculate pathway to facilitate pre-target axon organization, which may in turn facilitate accurate formation of synaptic connections in the dLGN.