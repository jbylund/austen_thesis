Appropriately organized synaptic connections are essential for proper neural circuit function.
Prior to forming and refining synaptic connections, axons of projection neurons must first navigate long distances to their targets.
Research in the axon guidance field has generated a great deal of knowledge about how axons successfully navigate through intermediate choice points and form initial connections with their synaptic targets.
One aspect of neural circuit development that has been less well studied is whether axons are organized within their tracts.
Axons could be highly ordered, or arranged haphazardly, to be sorted out within their destination target zone.

Findings from several systems indicate that axon tracts are organized and, furthermore, that pre-target organization is important for accurate targeting.
Chapter 1 will survey these findings as an introduction to my thesis.
The remaining chapters present my research in the mouse retinogeniculate pathway, in which I examine three aspects of pre-target axon organization: the organization of cohorts of retinal ganglion cell (RGC) axons in the optic nerve and tract; the role of axon self-association in tract organization; and the relationship between tract order and targeting.

RGC axons project either ipsi- or contralaterally at the optic chiasm.
In the first thalamic target, the dorsal lateral geniculate nucleus (dLGN), RGC axon terminals are organized based on retinotopy and laterality (i.e., into ipsi- and contralateral zones).
Chapter 2 presents my findings on the organization of ipsilateral (ipsi) and contralateral (contra) RGC axons in the optic nerve and tract.
Ipsilateral RGC axons cluster together in the optic nerve, are less tightly bundled in the optic chiasm, and once in the optic tract, again bundle together and are segregated from contralateral axons.
Topographic and ipsi/contra axon order in the optic tract are largely in register, although ipsi- and contralateral axons from the same topographic region maintain distinct ipsi/contra segregation in the tract.

Chapter 3 explores one potential mechanism involved in creating the organization between ipsi and contra RGC axons in the tract: differential fasciculation behavior between RGC axon cohorts.
I used in vitro retinal explant culture systems to test the hypothesis that ipsilateral RGC axons have a greater preference to self-fasciculate than contralateral axons.
Ipsilateral neurites display greater self-association/fasciculation than contralateral neurites, indicating an axon-intrinsic mechanism of ipsilateral-specific self-association. 

Chapter 4 examines tract organization and fasciculation in the EphB1 mutant retinogeniculate pathway.
EphB1 is expressed exclusively by ipsilateral RGCs, and loss of EphB1 leads to a reduced ipsilateral projection and increased contralateral projection.
However, aberrantly crossing axons project to the ipsilateral zone in the dLGN.
Given its combination of an aberrant decussation phenotype with a grossly normal targeting phenotype, I used this mutant to explore the relationship between midline choice, tract organization, and targeting.
First, remaining ipsilateral axons in the EphB1\textsuperscript{-/-} optic tract largely retain their position in the lateral optic tract, but appear splayed apart, suggestive of aberrant fasciculation.
In vitro, EphB1\textsuperscript{-/-} ipsilateral neurites still bundle more than EphB1\textsuperscript{-/-} contralateral neurites, although the magnitude of this difference is less striking than in wild-type retinal explants.
Thus, EphB1 may be involved in preferential ipsilateral RGC axon fasciculation.
In vivo, the aberrantly crossing axons in the EphB1 mutant grossly maintain their position in the ipsilateral zone of the optic tract (i.e., the lateral aspect), indicating a preservation of ipsilateral segregation in the tract.
This is in line with a model in which bundling partners in the tract may help guide axons to the correct zone in the target.

The data presented in this thesis detail two organizational modes of RGC axons in the developing optic nerve and tract, eye-specific (typographic) and topographic, and suggest that axon-intrinsic factors mediate ipsilateral-specific self-association.
Axon-intrinsic factors likely act alongside extrinsic cellular and molecular cues in the developing retinogeniculate pathway to facilitate pre-target axon organization, which may in turn facilitate accurate formation of synaptic connections in the dLGN.